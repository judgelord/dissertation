

% QUESTION 3 
\paragraph{Step 3: Does mass engagement in bureaucratic policymaking affect policy?} 

% THEORY 3
I theorize that the effects of political information on policy depend on the extent to which the strategic environment allows change\footnote{
What social movement scholars call a ``political opportunity'' \citep{Mcadam2017} such as division among elites \citep{Tarrow1994}, in this case, the agency's political principals or business interest groups. Similarly, policy process scholars identify opportunities to align politics with certain identified problems and solutions to create a ``window'' for policy change \citep{Kingdon1984}. All rulemaking processes create opportunities, however small, to shape the new status quo, loosely bounded by the problems the process was initiated to solve, a set of policy solutions considered legitimate, and a constellation of political forces.
}, and how political information is processed, both directly within agencies and indirectly through other actors (e.g. Members of Congress) whose appraisals matter to bureaucrats.
% STRATEGIC ENV 
% MEASUREMENT 3 % FIX THIS 
% it is difficult
\textbf{DVs \& Methods:} The main dependent variable here is changes in rule text. However assessing policy change is difficult. Thus, I also use other measures of agency responses to lobbying efforts. 
%Different inputs may yield different results: 
Agencies may or may not change draft policies or may speed up or delay finalizing them. They write lengthy justifications of their decisions in response to some demands but not others. They may or may not extend the comment period. Measuring actual changes in policy text is more difficult. I aim to use automated methods to systemically identify changes between draft and final rules, parse these textual differences to identify meaningful policy changes, and compare them to demands raised in comments to measure which coalition got their way.\footnote{
Observing policy influence, especially in the final stages of policymaking is difficult. Given the momentum of political agendas and the fact that much is determined before draft rules are made public, changes are often on the margins. But such marginal victories are also the aim of business and other interest groups. 
Additionally, my theory suggests that influence is likely only in cases where mass mobilization is (1) aimed at influencing policy and (2) not accurately anticipated by policymakers. Measuring these will also be difficult.

Observational studies of policy decisions are almost always frustrated by the fact that decisionmakers rationally anticipate the actions of those who would influence them, rendering this influence difficult to observe. Thus I expect to observe larger effects in cases where mobilization or the level of engagement achieved was not anticipated by agency staff. However, 
as long as rulewriters do not perfectly anticipate mass engagement. It should have observable, if depressed, effects. I also hope to leverage small random manipulations in, for example, the specific policy provisions targeted by activist campaigns.

My method of identifying whether a rule seems to move in the direction requested is similar to leading existing methods---\citet{Yackee2006JOP} measure whether commenters requested for more or less regulation---and superior to self-reported influence \citep{Furlong1997}.

As most rules address long-defined problems. They are next steps advancing a policy agenda \citep{West2013} or the first steps in a new, often reverse, policy direction, it is possible that effects of ``going public'' are cumulative in a policy area over time, starting out small, but gaining agenda-setting power with sustained public attention. This may not be possible to measure with my rule-focused research design. However, if sequential rules can be linked to distinct policy agendas, my strategy could be extended to model dynamics over time following \citet{Brookhart2015}.
}  
