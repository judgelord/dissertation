
Formally, political information requires a crucial amendment to existing information-based models of rulemaking. \citet{Libgober2018} posits a utility function for agency $G$ as $u_G(x_F) = \alpha_0 x_f^2 + \sum_{i=1}^N \alpha_i u_i (x_f)$ where $x_f$ is the spatial location of the final policy, $u_i$ is the preference of a member of the public or ``potential commenter'' $i$, and $\alpha$ is a vector of "allocational bias"---i.e. how much the agency cares about its own preferences $\alpha_0$ relative to accommodating the preferences of others $\alpha_{i=1:N}$. Bureaucrats balance their own ideas of their mission against their desire to be responsive. In Libgober's model, $\alpha$ is a fixed ``taste'' for responsiveness to each member of society, so agency decisions simply depend on their answer to the question ``what do people want?'' Including political information requires two additional parameters related to a second question ``why would the agency care?'' 

First, going public (like other lobbying strategies) may shift the strategic environment, leading an agency to shift its allocation in favor of some and away from others. Let this strategic shift be a vector $\alpha_s$. Second, campaigns may directly persuade agencies to adjust their allocational bias, for example by supporting claims about the number of people they represent or the intensity or legitimacy of their policy demands. Let this direct shift be $\alpha_d$ and immutable taste now be $\alpha_t$. Having decomposed an agency's allocative bias into three parts (its fixed tastes, shifting strategic environment, and potential to be convinced), the agency's utility function is now  $u_G(x_F) =  (\alpha_{t0} + \alpha_{s0} + \alpha_{d0}) x_f^2 + \sum_{i=1}^N (\alpha_{ti} + \alpha_{si} + \alpha_{di}) u_i (x_f)$. If, after the comment period, an agency's strategic environment is unchanged and it remains unpersuaded about which segments of society deserve favor, $\alpha_s$ and $\alpha_d$ are 0, and the model collapses to the original information game. This is less plausible when groups go public and expand the scope of conflict. 

Adding these parameters also resolves a problem with Libgober's model. Empirically, rules that receive comments occasionally do not change. This result is impossible in his model. Commenters must either be wrong about an agency's allocative bias or their ability to shift it. Incorporating political information allows change and uncertainty in an agency's biases. 
% While it is possible that commenters greatly misestimate an agency's durable allocative bias towards them (i.e. the agencies taste for them), it is more plausible that they misestimate their ability to influence the agency or the agency's strategic environment in a particular case. 
% Indeed, if taste is the only kind of allocative bias, then the only thing that can explain variance in participation is variation in the location of the proposed rule. The best empirical methods to estimate policy location generally assume that the spatial location of proposed rules written by the same people will be the same. Yet a potential commenter's anticipated ability to affect the agency's strategic environment or beliefs may be likely to vary significantly from rule to rule. It may vary with the level or distribution of economic impact, salience, sympathetic affected populations, timing (e.g. related to elections), recent court victories, proximate legal precedent, or any number of correlates of political context and ammunition for persuasion.
This result also becomes possible if commenters are allowed a strategy of ``going down fighting'' and incentives to do so.

\citet{Libgober2018} asks ``What proportion of commenting activity can be characterized as informing regulators about public preferences versus attempting to attract attention of other political principals?'' (p. 29). Adding political information formalizes this question: under what conditions does the decision to comment depend on estimates of $\alpha_t$ versus estimates of $\alpha_s$? Because they are substitutes in the model, it may be hard to say theoretically, but empirically, we may often be able to infer that the difference in commenting can be attributed to group $i$'s beliefs about $\alpha_{si}$ if other parameters are similar across rules at a given agency.
