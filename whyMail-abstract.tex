Scholars of bureaucratic policymaking have focused on the sophisticated lobbying efforts of powerful interest groups. Yet agencies occasionally receive thousands or even millions of comments from ordinary people. Why? Why do individuals engage when they seemingly have no new information to offer and no power to influence decisions? Who inspires them and to what end? How, if at all, should scholars incorporate mass participation into models of bureaucratic policymaking? 
I argue that mass mobilizing is an attractive and potentially influential tactic because produces political information about the coalition that mobilized it. I measure the scale and intensity of public support for proposed policies and examine alternative explanations that mass mobilization is (1) a conflict expansion tactic, where coalitions with fewer resources leverage public support, or (2) a more conventional lobbying tactic, where groups with superior resources leverage these resources to create an impression of public support. 
To link individual comments to the more sophisticated lobbying efforts they support, I use text reuse and clustering methods to identify formal and informal coalitions. I also classify different types of supporters. Using these new measures of political mobilization and engagement in agency rulemaking, I find that, in contrast to conventional insider lobbying, the vast majority of mass engagement in bureaucratic policymaking is mobilized by public interest group coalitions.
%I identify when mass comment campaigns occur and produce different types of politically-relevant information.