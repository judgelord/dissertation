Scholars of bureaucratic policymaking have focused on the sophisticated lobbying efforts of powerful interest groups. Yet agencies occasionally receive thousands or even millions of comments from ordinary people. Why? Why do individuals comment when they seemingly have no new information to offer and no power to influence decisions? Who inspires them and to what end? How, if at all, should scholars incorporate mass commenting into models of bureaucratic policymaking? I argue that mass commenting produces political information about the coalition that mobilized it. To link individual comments to the more sophisticated lobbying efforts they support, I use text reuse and topic models to identify clusters of similar comments, reflecting formal and informal coalitions. I also classify different types of supporters. Using these new measures of political mobilization and engagement in agency rulemaking, I identify the conditions under which mass comment campaigns occur and produce different types of politically-relevant information.