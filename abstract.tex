This dissertation is about public pressure campaigns that target U.S. federal agency rulemaking, a technocratic policy process in which participation is usually limited to a few policy insiders. Occasionally, however, public pressure campaigns help make agency rules some of the most hotly contested policies of our time. I examine who organizes public pressure campaigns and why, whether these campaigns affect congressional oversight, and whether they affect policy. Answering these questions informs our understanding of bureaucratic politics as well as interest group lobbying, organizing, and mobilizing tactics. Most comments that federal agencies receive on their draft policies come from ordinary people who were inspired by a public pressure campaign. Yet leading theories of bureaucratic policymaking neither explain nor account for these occasional bursts of civic engagement. This dissertation develops and tests theories about the roles of individuals, organizations, coalitions, and social movements in bureaucratic policymaking. I argue that pressure campaigns build movements, attract policymakers' attention, and systematically counter corporate power by generating political information---information about public demands.   I make four main contributions:
  
  First, drawing on scholarship on social movements, interest group behavior, and lobbying, I identify three reasons for lobbying organizations to mobilize ordinary people. Each logic suggests a different observable pattern of public engagement, and I analyze millions of public comments on thousands of agency rules to develop the first systematic measures of public engagement in bureaucratic policymaking. Contrary to other forms of lobbying, I find that mass comment campaigns are almost always a conflict expansion tactic rather than well-resourced groups creating an impression of public support. Most public comments are mobilized by public interest organizations, not by narrow private interests or astroturf campaigns. However, the resources and capacities required to launch a campaign cause a few larger policy advocacy organizations to dominate.
  
  Second, building on theories of political oversight, I theorize that mass engagement in bureaucratic policymaking may alert elected officials to political opportunities and risks, thus affecting oversight behavior. I assess this argument by analyzing correspondence between members of Congress and agency officials on proposed rules with and without mass engagement. I find that public pressure campaigns are correlated with congressional attention and that coalitions with more congressional support are more likely to achieve their policy goals.
  
  Third, I integrate intuitions about outside lobbying and oversight into a broader theory of how public pressure campaigns may affect policy by producing potentially influential political information. I suggest four causal mechanisms by which lobbying may influence bureaucrats and expand leading formal models of interest group influence in bureaucratic policymaking to account for political information. I then test these models using a mix of hand-coding and computational text analysis methods.   
  
  Fourth, I advance the literature on social movement pressure through a case study of the environmental justice movement. I show that pressure campaigns are more than a lobbying tactic; they are also an institutionalized form of contentious politics over the distribution of governmental power. I examine the discursive effects of environmental justice claims both qualitatively and quantitatively, including the role of Native activists and environmental groups in shaping federal policy. I find that agencies frequently ignore environmental justice concerns but are more likely to add language addressing environmental justice in their final rules when public comments raise environmental justice concerns. I conclude with implications for theories of bureaucratic policymaking and future research, a review of dominant ways of thinking about public comment periods, and proposed reforms for public participation in policymaking.