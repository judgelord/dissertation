This dissertation is about ordinary people's input on policies made by bureaucrats. Most new policies in the United States are made by bureaucrats, and most comments that federal agencies receive on their draft policies come from ordinary people who were inspired by an interest-group campaign. Yet leading  theories  of bureaucratic policymaking neither explain nor account for these occasional bursts of civic engagement.
% People may believe that their voices matter, but it is unclear if they do. % or ought to. 
I make three main contributions:

First, drawing on scholarship on social movements, interest group behavior, and lobbying, I identify three distinct reasons for lobbying organizations to mobilize ordinary people. Each logic suggests a different observable pattern of public engagement, and I analyze millions of public comments on thousands of agency rules to develop the first systematic measures of public engagement in bureaucratic policymaking. 

Second, building on theories of political oversight, I theorize that mass engagement in bureaucratic policymaking may alert elected officials to political opportunities and risks, affecting oversight behavior. I assess this argument by analyzing correspondence between Members of Congress and bureaucrats on proposed rules with and without mass engagement.

Third, I integrate these intuitions about outside lobbying and oversight into a broader theory of how mass mobilization may affect policy by producing potentially influential political information. I suggest four causal mechanisms by which lobbying may influence bureaucrats. Thus far, theories of rulemaking have focused on the power of technical information, where insider lobbying is most likely to matter and outside lobbying is least likely to matter. As a result, political scientists have largely overlooked mass engagement. This gap suggests that incorporating theories of social movement influence may advance bureaucratic politics scholarship and that bureaucratic politics may be fertile ground for testing social movement theory. To address this gap, I use my new measures of mass engagement and oversight to assess the effect of political information on rulemaking and rules.% bureaucratic policymaking.

% Finally, two supplemental chapters assess causal processes through a case study of the role of the environmental justice movement in rulemaking and an analysis of rulemakings where organizations randomly select lobbying strategies.

