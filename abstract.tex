%This dissertation will examine the role of mass mobilization in agency policymaking. 
The rise of the administrative state as the main site of policymaking in the United States requires new thinking and study of the practice of democracy. In what ways do agency policy processes advance or undermine democratic ideals? While political scientists focus on accountability through elected officials, administrative procedures also create rights to petition and participate directly in policymaking.  Yet we lack systematic evidence on how civic engagement such as protests, petitions, or large numbers of public comments shape the political environment in which government agencies make policy decisions. Such questions of group influence are central to understanding democracy. While some speculate that mobilizing a large number of people can shift policy, the studies that come closest to addressing this question in the context of agency rulemaking find no such evidence. Theories of bureaucratic policymaking seem to point in both directions and specific theories about how mass-engagement might matter are lacking. Furthermore, the normative appeal of institutions like public comment periods, rooted in ideas of direct democracy, depends on who participates, who is influential, and why---empirical questions that require a study on whether---and if so, why---civic engagement matters. I aim to provide this study.

First, I offer an understanding of agency policymaking focusing on the political information available to policymakers. Occasionally, such information arises from contentious debate and civic mobilization. I draw on scholarship in political science, law, sociology, and public administration and data on hundreds of thousands of rules made in the past 40 years to identify what distinguishes the inputs and outputs of the few policy processes where large numbers of people are engaged.

Second, I aim to identify mechanisms, both direct and indirect, by which civic mobilization may shape an agency's decisionmaking environment. I assess direct mechanisms by measuring the attention and meanings that agency staff give to protests and the results of mass-comment campaigns and indirect mechanisms by measuring variation in other kinds of external attention to rulemaking processes, especially from elected officials.

[Potentially] Third, I develop new methods to study the causal effects of civic mobilization through experimental manipulation of advocacy strategies. This involves partnering with organizations to randomly assign strategies and surveying participating organizations before and after a campaign to compare expert priors with outcomes. 
% Public comment campaigns are not the only way to signal to agencies that large numbers of people are paying attention, but it is one signal that cannot be missed. Activist campaigns may also shape media attention and public discourse. 
%In this design, the number and text of public comments are the treatment and the text of bureaucrats' discussion of public comments and changes in policy are the response. 
Tracing the effects of activist campaigns on agencies' political environments and policy decisions will contribute to our understanding of the roles of advocacy organizations and the people they engage in bureaucratic policymaking. 