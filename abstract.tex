This dissertation is about ordinary people's input on policies made by bureaucrats. 
% People may believe that their voices matter, but it is unclear if they do. % or ought to. 
It makes three main contributions:

First, drawing on scholarship on interest group behavior, social movements, and lobbying, I identify three distinct reasons for groups to mobilize ordinary people. Each logic suggests a different observable pattern of mass public engagement, and I analyze millions of public comments on thousands of agency rules to develop the first systematic measures of mass engagement in bureaucratic policymaking. 

Second, building on theories of political oversight, I theorize that mass public engagement in bureaucratic policymaking may alert elected officials to political opportunities and risks, affecting oversight behavior. I assess this argument by analyzing correspondence between Members of Congress and bureaucrats on proposed rules with and without mass engagement.

Third, I integrate these contributions on interest group lobbying and oversight into a broader theory of how mass mobilization may affect policy by producing potentially influential political information. I argue that there are four broad causal mechanisms by which lobbying may influence bureaucrats. Political scientists have thus far focused on the power of technical information, where insider lobbying is most likely to matter and outside lobbying is least likely to matter, thus largely overlooking mass engagement. This gap suggests that incorporating theories of social movement influence may advance bureaucratic politics scholarship and that bureaucratic politics may be a fruitful empirical ground for exploring social movement theories. To address this gap, I use my new measures of mass engagement and oversight to assess the effect of political information on bureaucratic policymaking.

Finally, two supplemental chapters assess causal processes through a case study of the environmental justice movement and an analysis of rulemakings where organizations randomly select lobbying strategies.

