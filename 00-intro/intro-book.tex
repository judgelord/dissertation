
% \section{Introduction}
%% DEMOCRACY has problems and bureaucratic solutions
%Large democracies face two big problems. First, they are vulnerable to fleeting passions and demagogues. To combat this, many decisions are left to experts who, ideally, exercise judgment loosely guided by the public. Second, everyone cannot vote on every decision. We thus delegate power to representatives (who then delegate it to deputies), create temporary mini-publics, and solicit input from those most affected or moved by a public decision.\footnote{As imagined by \citet{Dahl1989}, mini-publics are representative, selected at random, and deliberative. Besides juries, however, randomly selected deliberative bodies are rare. Instead, citizens more often engage in government decisions when given opportunities to opt-in, such as hearings, petitions, and public comment periods. These mechanisms of engagement generate a different, more contentious flavor of public input than the discourse imagined by scholars who focus on deliberation.} Most policy is then made by bureaucrats, supposedly guided indirectly through elected representatives and directly by limited public input (mostly limited to more contentious policy debates).

% Both of these problems converge in the bureaucracy, run by experts who are deputized by elected officials (or by their deputy's deputy's deputy) and with procedures that create opportunities for public input. It is far from clear how bureaucratic decisions are to balance expertise, accountability to elected officials, and responsiveness to public input in decisionmaking. 


% \subsection{Why study rulemaking?}
% \section{The Importance of Studying Rulemaking}
% Mobilization may increasingly target rulemaking because it is how most policy in the U.S. is now made. 

% rulemaking matters 
With the rise of the administrative state, U.S. federal agencies have become a major site of policymaking and political conflict. By some estimates, upward of 90\% of legally binding U.S. federal policy is now written by agencies. Agency rules are revised much more frequently than statutory law \citep{Wagner2017} and in the years or decades between legislative enactments, federal agencies make legally-binding rules interpreting and reinterpreting old statutes to address emerging issues and priorities. %Ninety percent of new policy that carries the force of law is now made in the bureaucracy rather than in Congress \citep{West2013}.\footnote{I use policy, law, and regulation as nested concepts. My methods generally apply to all policy texts whether they carry the force of law or not. Many public and private organizations, including agencies, have policy statements that are not legally binding. My empirical subject is rules that do carry the force of law based on some authorizing legislation. I use rule (a more technical term) and regulation (a more colloquial term) interchangeably.}
Examples are striking: The effect of the Dodd-Frank Wall Street Reform and Consumer Protection Act was largely unknown until the specific regulations were written, and it continues to change as these rules are revised. 
Congress authorizes billions in farm subsidies and leases for public lands, but who gets them depends on agency policy. In the decades since the last major environmental legislation, agencies have written thousands of pages of new environmental regulations and thousands more changing tack under each new administration. These revisions significantly shape lives and fortunes. For example, in 2006, citing the authority of statutes last amended in the 1950s, the Justice Department's Bureau of Prisons proposed a rule restricting eligibility for parole. In 2016, the Bureau withdrew this rule and announced it would require fewer contracts with private prison companies, precipitating a 50\% loss of industry stock value. Six months later, a new attorney general announced these policies would again be reversed, leading to a 130\% increase in industry stock value. %Like many rulemaking debates, industry and advocacy groups spent millions of dollars lobbying on this issue. Few rulemakings, however, receive this level of public and presidential attention. In the majority of rulemakings, few participate, and we do not know the extent to which participants get what they lobby for.% (but see Yackee and Yackee 2006)
Agency rulemaking matters.

% democracy interbranch relations and autonomy
Less clear, however, is how the new centrality of agency rulemaking fits with democracy. In addition the bureaucracy's complex relationships with the president and Congress, agencies have complex and poorly understood relationships with the public and advocacy groups. Relationships with constituent groups may even provide agencies with a degree of ``autonomy'' from their official principals \citep{Carpenter2001}. % CITE {Yackee2019}.


%\paragraph{Expertise, accountability, and participation.} 
%Debates over the proper roles of bureaucratic expertise, legislative delegation, and public input converge in bureaucratic policymaking. Bureaucratic policymaking involves expert judgment, accountability to elected officials, and be responsive to public input. \footnote{ Notice and comment rulemaking is the main focus of political science scholarship on bureaucratic policymaking in the United States \citep{West2015, Yackee2006JPART, Yackee2010PSQ, Kerwin2011} } 