

% CONCLUSION
This research will add to our understanding of how  bureaucratic policymaking fits with the practice of democracy.
If input solicited from ordinary people has little effect on policy outcomes, directly or indirectly, it may be best understood as providing a veneer of democratic legitimacy on an essentially technocratic and/or elite-driven process.

The legitimacy of bureaucratic policymaking is said to depend on the premise that rulemaking provides an outlet for public voice \citep{Croley2003, Rosenbloom2003}. This is reflected in the ACUS Proposed Recommendation on Public Engagement in Rulemaking begin with this statement: ``The opportunity for public engagement is vital to the rulemaking process, permitting agencies to obtain more comprehensive information, enhance the legitimacy and accountability of their decisions, and enhance public support for their rules'' \citep{ACUS2018}. Yet, it is not just the opportunity to engage, but actual engagement that matters \citep{Herz2018}, and we lack an empirical base necessary to evaluate if this legitimacy is deserved, even if people believe that their comments matter \citep{Yackee2014JPART}.

If public input does shape agency decisions, a new research program will be needed to investigate who exactly these campaigns mobilize and represent.

