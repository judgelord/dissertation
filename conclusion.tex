

% CONCLUSION
%This research will add to our understanding of how  bureaucratic policymaking fits with the practice of democracy.
%If input solicited from ordinary people has little effect on policy outcomes, directly or indirectly, it may be best understood as providing a veneer of democratic legitimacy on an essentially technocratic and/or elite-driven process.
The legitimacy of bureaucratic policymaking is said to depend on the premise that rulemaking provides for public voice \citep{Croley2003, Rosenbloom2003}. Yet we lack an empirical base necessary to evaluate whether any legitimacy the public comment process may provide is deserved. I have made a few initial steps toward better understanding actual public engagement in bureaucratic policymaking.
If mass engagement does shape agency decisions, a new research program will be needed to investigate who exactly these campaigns mobilize and represent.

