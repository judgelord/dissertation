\subsection{Four mechanisms by which groups may influence bureaucratic policymaking}

% How might mass engagement matter?
% A lobbying effort can generate new information, re-frame information, or reshape the political context of a decision. Agency staff may update their beliefs in response to new information or framing. Activists can also reshape agency policymakers’ strategic environment by drawing in or scaring off other actors, especially elected officials. 

Why might mass mobilization matter? The literature on bureaucracy offers two types of explanations rooted in either strategic behavior or organizational processes and norms. Political scientists often focus on strategic context. Public administration and management scholars focus on organizational logics and identities. 
% I begin with the ``indirect'' mechanisms that theories of strategic behavior suggest. 

While scholars often focus on the top right cell of Table \ref{2x2}, the influence of political information is to be found in the other three cells. Political scientists have focused on strategic factors---either on how lobbying provides technical information that directly influences agency decisions or on how oversight indirectly constrains them.  Mass engagement is only likely to affect the later.

New technical information that is relevant to understanding the effects of proposed policy (e.g. inputs to benefit-cost analysis such as the cost or benefits for regulated businesses) may directly affect technocratic decisions. 
New political information that is relevant to the likelihood of political support, sanction, or reversal may indirectly affect bureaucrats' strategic calculations.

Political information may also affect bureaucrats' normative evaluations. These effects may be
direct (e.g. the weight that norms of direct democracy give to limited public input) or 
indirect (the weight that norms of accountability give to elected officials' input).
The strength of norms of direct and indirect accountability to public demands may vary across agencies with levels of political insulation and responsiveness.

% MECHANISMS TABLE 
\begin{table}[h!] \footnotesize
\centering 
  \caption{Mechanisms: How New Information May Influence Bureaucratic Policymakers} 
  \label{2x2} 
\begin{tabular}{@{\extracolsep{5pt}} lcc} 
 & Highlighting Norms in Institutional and Cognitive Processes & Shifting Strategic Incentives  \\ 
\hline \\
Direct    & Select ``public'' opinion & Technical information \\
& (public service/responsiveness/direct democracy) & (e.g. inputs to benefit-cost analysis)\\
 \\
 \hline \\
Indirect & Elected official opinion  & Likely retribution and reward \\ 
& (accountability/representative democracy) & (e.g. future budgets, careers, support) \\
\\
\hline 
\end{tabular}
\end{table}

\subsubsection{Direct effects on strategic calculations:} 
New information may affect agency strategy directly or indirectly. New scientific or legal information spurs revision of calculations about cost and benefits or the likelihood of being reversed in court. New political information spurs bureaucrats to update their beliefs about levels of support among certain populations or their elected representatives and thus the likely political consequences of a decision.

% \subsubsection{Indirect Influence: Signaling a threat of backlash}
% Indirect FIXME
Indirectly, it may alert elected officials to political risks and opportunities, thus reshaping an agency's strategic environment.



% Arnold, Logic Of Congressional Action p 217:
% success depends in poat of the length and complexity of the causal chain connecting a policy instrument with its policy effects. When a causal chain is short and simple, citizens are more likley to know which policy instrument will produceth appropriate effects and are beter able to monitor the performance of their repre3sentatives,. When a causal chain is long and complex, or when a problem in society stems from multile causes, citizens may be incapable of doing the appropriate policy analysis and political anslyss ." 
% p 272 "resoonsiveness to both attentive and inattentive publics avaries depending on the procedures that govern how legislators requd their positions" 
% "The model of citizen's control that I have been discussing is essentially an auditing model. Citizens do not instruct legislators on how to vote, not do thay necessairlily have well-defined policy preferences in advance of cogressional action. Legialators neverthless have strong incentives to consider citizens' potential preferences when they are deciding how to vote for fear that making the wrong choice might triggger and unfavorable audit." 
% 


The White House has several tools to influence agency decisions \citep{Yackee2009a,Simon1954}. These include executive orders \citep{Mayer1999}, appointments \citep{Doherty2014,Lewis2008,Wood1988}, budgets \citep{Whittington2003}, and review of proposed policies \citep{Haeder2015, Acs2013}. 
Congress also has several tools to influence agency decisions. These include the power of the purse \citep{Fenno1986,Bolton2015}, oversight, and new legislation. Some research suggests that this constraint is larger under divided government \citep{Yackee2009RegGov} % CHECK CITE 
and that under divide government Congress tends to divide power among multiple agencies \citep{Farhang2016}.
The anticipation of judicial review makes courts relevant to rulemaking. Some rules are also made under court-imposed settlement or with judicial deadlines. Judicial opinions may also call on Congress to act \citep{Yaver2017}.
Despite these mechanisms and because of conflicts among them, agency staff maintain significant power over agency decisions. For example, Congress is less assured of compliance when power is divided \citep{Yaver2016}.

% more good stuff
%\subsubsection{Legal Scholarship}
Legal scholars' case studies of specific rulemaking process offer an additional relevant body of research. Coglianese (1997) finds that litigation is a common extension of rulemaking. Indeed, unlike legislative lawmaking, rulemaking takes place in the clear and present shadow of judicial review (Rossi 2001). Stakeholders can challenge a rule in court on a variety of procedural grounds and on statutory interpretation. This scholarship suggests those who succeed in rulemaking are those with the resources and experience to succeed in court. Costly mass comment campaigns could be signaling the ability and willingness to spend resources to challenge the rule in court. 

Mass mobilization may signal political risks or benefits of engaging in agency policymaking to members of Congress and the White House. It also may signal to the agency that activists have the capacity to sustain pressure through the policy process \citep{Coglianese2001}, including challenging the policy in court, a constant threat agency policies. Thus, mass mobilization may act as a signal of political power that  reshape rule-writers' beliefs about their strategic context. 


%%%%%%%%%%%%%%%%%%%%%%%%%%%%%%%%%%%%%%%%%%%%%%%%%%%%%%%%%%%%%%%%%%%%%%
\subsubsection{Indirect effects on the strategic calculations}
In the U.S. context, agencies are accountable to the president, Congress, and courts. Many rules receive little attention from these other institutions, but all three significant powers to reward, sanction, or reverse agency actions \citep{Yaver2016}. If mass engagement affects the behavior of these actors, it alters the strategic context in which bureaucrats make decisions.
Mass mobilization may also signal a coalition's \textit{potential} to influence the responses to agency action from the White House, Congress, or courts. 

For example, the number, geographic distribution, size, and proportion of businesses who lobby against a rule, may provide information about how much money and which of their political principals may be invested in attacking a rule. Similarly, the number of people who engage in a rulemaking and the intensity of engagement may provide information about how much support or scrutiny an agency is likely to receive from certain political principals. 

As activist campaigns may be less predictable than business lobbying, civic mobilization may provide even more information about constellations of support or opposition and the intensity of these policy demanders. 
If the information leads bureaucrats to update their understanding of the constellations of interests, their intensity, and the power and resources of each coalition, it may affect their strategic response.

Bureaucrats care about the consequences of their actions, both for themselves for their agency’s mission. Their success and power depend on the support of a political coalition that includes elected officials \citep{Carpenter2001}. \citet{West2004} theorizes that the primary mechanism by which mass-commenting matters is to alert political principals. Members of Congress, especially, may usually be unaware of rulemaking \citep{Nou2016}. Conversely, campaigns may ``scare off'' elected officials who otherwise would have weighed in, threatening consequences, such as legislation that reverses the rule (personal communication with former agency director).
Reshaping strategic incentives may shift how rulewriters weigh commenter demands.

% NORMATIVE FRAMING / INFO PROCESSING 
\subsubsection{Direct effects on information processing and normative evaluations:}
In addition to strategic calculations, mass engagement may shift how information is processed and evaluated, both institutionally and cognitively. % \footnote{
% As Simon observes, a focus on maximizing subjective utility carries a lot of baggage: “In a substantive theory of rationality, there is no place for a variable like focus of attention. But in a procedural theory it may be very important to know under what circumstances certain aspects of reality will be heeded and others ignored” (Simon, 1987: 31).   % Indeed, Newell (1958: 13) argued that any rule of decision-making, comprehensively rational or not, is conditioned on “what information is entered into the system… the judgment law is quite secondary, and amounts to doing the obvious with the information finally selected”.
% }
By focus policymakers attention on certain aspects of policy decisions mass engagement may affect who is involved these decisions and how they are framed.

Institutionally, higher comment volume may engage a larger and more politically-oriented set of staff and consultants. Cognitively, expanding the scope of conflict highlights the political aspects of a decision, perhaps mobilizing cognition focused more on norms of public service or partisan ideology than on strategic or technical rationality. In both cases, campaigns re-frame decisions as political and provide information that is especially relevant if processed through such a frame.

%Because rulemaking is an exercise in relating information about the world to certain norms and policy agendas, how decisions are framed may be decisive. 
%Thus, both new political information and how information is framed may influence policy decisions. 
The source, number, and content of comments all provide political information. Each side may offer frames for interpreting these facts and others. If 
%the opinions of letter-writers, petition-signers, and protesters are
framed as the opinion of the public or as expressing valid public interests, such a frame may shape how officials think about the appropriate course of action for a public servant or a partisan concerned with the popularity of agency decisions.  

Even, perhaps especially, when positions expressed through contentious politics
%through petitions and protests
are not majoritarian, these tactics may communicate political information that is not represented through electoral politics \citep{Gillion2012, Gillion2013}. 
% Evidence that minority petitions and protests affect rulemaking would support theories that focus on policymakers’ limited attention, finite agenda, and satisficing rather than strategic decision making.
% Petitions and protests 
Campaigns may also frame minority groups as deserving of special attention and protection.

To the extent that elected officials' demands guide agency decisionmaking---i.e. to the extent that agency decisions are shaped by norms of accountability in representative democracy---campaigns may be influential by inspiring elected officials to produce new political information. When elected officials take a position publicly or in a private letter to an agency, such political information may have normative force beyond simply simple strategic calculations.

%\paragraph{Indirect influence through elected officials:} 
%Campaigns  do more than reveal latent political information; they mobilize both members of the public and elected officials to take positions on issues they may have never previously considered, thus creating new relevant political information for bureaucrats. 
  %Movements help to shape the political space in which they operate’ (Gamson and Meyer 1996, p. 289).
%The result of thinking differently about a decision may be a shift in how the agency evaluates or weights commenter demands.




% FIXME 
% INFO PROCESSING 
\paragraph{The impact of any kind information on policy decisions depends on the capacity to process this type of information.}
It is possible that many agencies lack the capacity to process political information embedded in mass comments. Some may simply discard this information \citep{Mendelson2011}. The expected influence of mass engagement thus depends on how information is processed, which I expect to vary significantly across agencies. 



% \section{Mechanisms of Influence}



% \paragraph{Mobilizing identities and reputations}

I now turn to the direct-influence pathway: the ability of social movements to mobilize ideas, evaluative frameworks, and claims about what is appropriate and right that may affect bureaucratic decisions. %I use mobilization around the idea of ``environmental justice'' as an example where direct influence may be visible. 

Organizational theory suggests additional mechanisms by which mass mobilization may influence bureaucratic decisions more directly. Here the causal process involves mobilizing norms and ideas right and wrong rooted in individual and institutional identity. Because concepts of mission, reputation, and the validity of claims are intertwined, these mechanisms are difficult to precisely define. Nevertheless, scholars have identified several types of direct influence. One important factor in decision making is personal and institutional reputation \citep{Carpenter2001}. This can take several forms. For example, individuals trained as scientists and agencies that cultivate reputations for producing valid science may be persuaded by rigorous scientific claims. Similarly, individuals who identify strongly as public servants and agencies with reputations for public responsiveness may be persuaded by claims about public or "stakeholder" opinion. In general, claims that resonate with the problems an agency has been tasked with solving and the means it has to solve those problems are likely to be well received.  






%I also look closely at rules that end up before the Supreme Court. These rulemaking processes deserve extra attention for two reasons. First, regardless of how contentious they were at the rulemaking stage, these rules are key to understanding the nature and limits of executive power as a policymaking venue. Second, because all rulemaking is done in the shadow of judicial review, who wins in court and why shapes the political terrain of future rulemaking, empowering some groups with credible threats of litigation and disempowering others. References to court cases and implied threats of litigation are common in interest group comments, but to my knowledge, no study has looked systematically at how they affect rulemaking. Conversely, scholarship has not systematically assessed how mobilization and contestation in a rulemaking process affect judicial review. 