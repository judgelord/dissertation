\paragraph{Causal mechanisms:} How might mass engagement matter?
% A lobbying effort can generate new information, re-frame information, or reshape the political context of a decision. Agency staff may update their beliefs in response to new information or framing. Activists can also reshape agency policymakers’ strategic environment by drawing in or scaring off other actors, especially elected officials. 

\begin{table}[h!] \footnotesize
\centering 
  \caption{Mechanisms: How New Information May Influence Bureaucratic Policymakers} 
  \label{2x2} 
\begin{tabular}{@{\extracolsep{5pt}} lcc} 
 & Highlighting Norms in Institutional and Cognitive Processes & Shifting Strategic Incentives  \\ 
\hline \\
Direct    & Select ``public'' opinion & Scientific or legal facts \\
& (public service/responsiveness/direct democracy) & (e.g. inputs to benefit-cost analysis)\\
 \\
 \hline \\
Indirect & Elected official opinion  & Likely retribution and reward \\ 
& (accountability/representative democracy) & (e.g. future budgets, careers, support) \\
\\
\hline 
\end{tabular}
\end{table}

\textbf{Strategic calculations:} 
New information may affect agency strategy directly or indirectly. New scientific or legal information spurs revision of calculations about cost and benefits or the likelihood of being reversed in court. New political information spurs bureaucrats to update their beliefs about levels of support among certain populations or their elected representatives and thus the likely political consequences of a decision.\footnote{
For example, the number, geographic distribution, size, and proportion of businesses who lobby against a rule, may provide information about how much money and which of their political principals may be invested in attacking a rule. Similarly, the number of people who engage in a rulemaking and the intensity of engagement may provide information about how much support or scrutiny an agency is likely to receive from certain political principals. 
As activist campaigns may be less predictable than business lobbying, civic mobilization may provide even more information about constellations of support or opposition and the intensity of these policy demanders. 
If the information leads bureaucrats to update their understanding of the constellations of interests, their intensity, and the power and resources of each coalition, it may affect their strategic response.

\textbf{Indirect influence through changing the decision environment:} 
Bureaucrats care about the consequences of their actions, both for themselves for their agency’s mission. Their success and power depend on the support of a political coalition that includes elected officials \citep{Carpenter2001}. \citet{West2004} theorizes that the primary mechanism by which mass-commenting matters is to alert political principals. Members of Congress, especially, may usually be unaware of rulemaking \citep{Nou2016}. Conversely, campaigns may ``scare off'' elected officials who otherwise would have weighed in, threatening consequences, such as legislation that reverses the rule (personal communication with former agency director).
}
Reshaping strategic incentives may shift how rulewriters weigh commenter demands.

% NORMATIVE FRAMING / INFO PROCESSING 
\textbf{Information processing and normative evaluations:} 
In addition to strategic calculations, mass engagement may shift how information is processed and evaluated, both institutionally and cognitively.
% \footnote{
% As Simon observes, a focus on maximizing subjective utility carries a lot of baggage: “In a substantive theory of rationality, there is no place for a variable like focus of attention. But in a procedural theory [that is, a behavioral theory] it may be very important to know under what circumstances certain aspects of reality will be heeded and others ignored” (Simon, 1987: 31). % Indeed, Newell (1958: 13) argued that any rule of decision-making, comprehensively rational or not, is conditioned on “what information is entered into the system… the judgment law is quite secondary, and amounts to doing the obvious with the information finally selected”.}
Institutionally, higher comment volume may engage a larger and more politically-oriented set of staff and consultants. Cognitively, expanding the scope of conflict highlights the political aspects of a decision, perhaps mobilizing cognition focused more on norms of public service or partisan ideology than on strategic or technical rationality. In both cases, campaigns re-frame decisions as political and provide information that is especially relevant if processed through such a frame.\footnote{
%Because rulemaking is an exercise in relating information about the world to certain norms and policy agendas, how decisions are framed may be decisive. 
%Thus, both new political information and how information is framed may influence policy decisions. 
The source, number, and content of comments all provide political information. Each side may offer frames for interpreting these facts and others. If 
%the opinions of letter-writers, petition-signers, and protesters are
framed as the opinion of the public or as expressing valid public interests, such a frame may shape how officials think about the appropriate course of action for a public servant or a partisan concerned with the popularity of agency decisions.  

Even, perhaps especially, when positions expressed through contentious politics
%through petitions and protests
are not majoritarian, these tactics may communicate political information that is not represented through electoral politics \citep{Gillion2012, Gillion2013}. 
% Evidence that minority petitions and protests affect rulemaking would support theories that focus on policymakers’ limited attention, finite agenda, and satisficing rather than strategic decision making.
% Petitions and protests 
Campaigns may also frame minority groups as deserving of special attention and protection.
}
The effects of political information on bureaucrats' normative evaluations may be
direct---the weight that norms of direct democracy give to limited public input---or 
indirect---the weight that norms of accountability give to elected officials' input.\footnote{
The strength of norms of direct democracy and accountability may vary across agencies with levels of political insulation and responsiveness.

To the extent that elected officials' demands guide agency decisionmaking---i.e. to the extent that agency decisions are shaped by norms of accountability in representative democracy---campaigns may be influential by inspiring elected officials to produce new political information. When elected officials take a position publicly or in a private letter to an agency, such political information may have normative force beyond simply simple strategic calculations.
}  
%\paragraph{Indirect influence through elected officials:} 
%Campaigns  do more than reveal latent political information; they mobilize both members of the public and elected officials to take positions on issues they may have never previously considered, thus creating new relevant political information for bureaucrats. 
  %Movements help to shape the political space in which they operate’ (Gamson and Meyer 1996, p. 289).
%The result of thinking differently about a decision may be a shift in how the agency evaluates or weights commenter demands.




% FIXME 
% INFO PROCESSING 
It is nevertheless possible that agencies lack the capacity to process political information embedded in mass comments. Some may simply discard this information \citep{Mendelson2011}. My predictions depends on how information is processed, which I expect to vary significantly across agencies. 

% Indirect FIXME
Indirectly, it may alert elected officials to political risks and opportunities, thus reshaping an agency's strategic environment.

% \section{Mechanisms of Influence}

\subsection{Indirect Influence: Signaling a threat of backlash}
% \section{Theories and Case Selection}
Why might mass mobilization matter? The literature on bureaucracy offers two types of explanations rooted in either strategic behavior or organizational norms. Political scientists often focus on strategic context. Public administration and management scholars focus on organizational logics and identities. I begin with the ``indirect'' mechanisms that theories of strategic behavior suggest. 

In the U.S. context, there are three main mechanisms by which mass mobilization could affect an agency action by changing the strategic context. 
Mass mobilization may signal power to influence the responses to agency action from the White House, Congress, or courts. Many rules receive little attention from these other institutions, but all three significant powers to reward, sanction, or reverse agency actions \citep{Yaver2016}. 

% Arnold, Logic Of Congressional Action p 217:
% success depends in poat of the length and complexity of the causal chain connecting a policy instrument with its policy effects. When a causal chain is short and simple, citizens are more likley to know which policy instrument will produceth appropriate effects and are beter able to monitor the performance of their repre3sentatives,. When a causal chain is long and complex, or when a problem in society stems from multile causes, citizens may be incapable of doing the appropriate policy analysis and political anslyss ." 
% p 272 "resoonsiveness to both attentive and inattentive publics avaries depending on the procedures that govern how legislators requd their positions" 
% "The model of citizen's control that I have been discussing is essentially an auditing model. Citizens do not instruct legislators on how to vote, not do thay necessairlily have well-defined policy preferences in advance of cogressional action. Legialators neverthless have strong incentives to consider citizens' potential preferences when they are deciding how to vote for fear that making the wrong choice might triggger and unfavorable audit." 
% 


The White House has several tools to influence agency decisions \citep{Yackee2009a,Simon1954}. These include executive orders \citep{Mayer1999}, appointments \citep{Doherty2014,Lewis2008,Wood1988}, budgets \citep{Whittington2003}, and review of proposed policies \citep{HAEDER2015InfluenceBudget,Acs2013}. 
Congress also has several tools to influence agency decisions. These include the power of the purse \citep{Fenno1986,Bolton2015}, oversight, and new legislation. Some research suggests that this constraint is larger under divided government \citep{Yackee2009b} and that under divide government Congress tends to divide power among multiple agencies \citep{Farhang2016}.
The anticipation of judicial review makes courts relevant to rulemaking. Some rules are also made under court-imposed settlement or with judicial deadlines. Judicial opinions may also call on Congress to act \citep{Yaver2017}.
Despite these mechanisms and because of conflicts among them, agency staff maintain significant power over agency decisions. For example, Congress is less assured of compliance when power is divided \citep{Yaver2016}.

% more good stuff
%\subsubsection{Legal Scholarship}
Legal scholars' case studies of specific rulemaking process offer an additional relevant body of research. Coglianese (1997) finds that litigation is a common extension of rulemaking. Indeed, unlike legislative lawmaking, rulemaking takes place in the clear and present shadow of judicial review (Rossi 2001). Stakeholders can challenge a rule in court on a variety of procedural grounds and on statutory interpretation. This scholarship suggests those who succeed in rulemaking are those with the resources and experience to succeed in court. Costly mass comment campaigns could be signaling the ability and willingness to spend resources to challenge the rule in court. 

Mass mobilization may signal political risks or benefits of engaging in agency policymaking to members of Congress and the White House. It also may signal to the agency that activists have the capacity to sustain pressure through the policy process \citep{Coglianese2001}, including challenging the policy in court, a constant threat agency policies. Thus, mass mobilization may act as a signal of political power that  reshape rule-writers' beliefs about their strategic context. 



\subsection{Direct Influence: Mobilizing identities and reputations}

I now turn to the direct-influence pathway: the ability of social movements to mobilize ideas, evaluative frameworks, and claims about what is appropriate and right that may affect bureaucratic decisions. %I use mobilization around the idea of ``environmental justice'' as an example where direct influence may be visible. 

Organizational theory suggests additional mechanisms by which mass mobilization may influence bureaucratic decisions more directly. Here the causal process involves mobilizing norms and ideas right and wrong rooted in individual and institutional identity. Because concepts of mission, reputation, and the validity of claims are intertwined, these mechanisms are difficult to precisely define. Nevertheless, scholars have identified several types of direct influence. One important factor in decision making is personal and institutional reputation \citep{Carpenter2001}. This can take several forms. For example, individuals trained as scientists and agencies that cultivate reputations for producing valid science may be persuaded by rigorous scientific claims. Similarly, individuals who identify strongly as public servants and agencies with reputations for public responsiveness may be persuaded by claims about public or "stakeholder" opinion. In general, claims that resonate with the problems an agency has been tasked with solving and the means it has to solve those problems are likely to be well received.  

% \subsubsection{ Agencies as Policymaking Venues}
% the good stuff
When political scientists ask whose interests and ideas become law, they have generally focused on the behavior of legislatures, how the executive branch drives legislation, and how the courts review it. Compared to legislative, executive, and judicial institutions, the administrative state is a recent development in American government and theory has not kept pace with the rise in bureaucratic policymaking. 

I argue that theories of bureaucratic policymaking have been characterized by constraining assumptions about what bureaucracies ought to do.  Normative assumptions that \citet{Wilson1967} identified half a century ago, and corresponding scholarly silos, have persisted. This has led to lines of research talking past each other and often failing to engage broader theories of policy change. In particular, I argue that the pervasive implicit assumption that bureaucrats ought to be neutral implementers implies that politics in agency policymaking is inherently undesirable, leading many scholars to focus on compliance with political principals and overlook the role participation and ideas. For example, scholars assume that agencies ought to be engaged in implementing legislation and executive orders. However, most rulemaking takes place many years or decades after its authorizing legislation under a different Congress and with little attention from the White House until the very final draft. Rules that do not follow from contemporary Congressional or executive priorities are often assumed to reflect bureaucrats going rogue or being captured by interest groups. Such studies suffer from a lack of attention to the complex political process of rulemaking. 

Accountability to elected officials has been central to the study of bureaucracy \citep{Epstein1999,Huber2002,McCubbins1984,Wilson1989,Potter2016Slow-RollingRulemaking,Lowande2018PoliticizationAgencies} %add Meier and O’Toole 2006; West 1995; Wood and Waterman 1994
Viewing agencies as \textit{agents} has prevented scholars from incorporating new insights about the endogenous relationship between policy and politics. I suggest rulemaking is better studied in the way that scholars study policymaking in specialized congressional committees than with an unrealistic dichotomy of sincere implementation versus capture or disloyalty. Normatively, accountability to political principals only one of several important concerns. Empirically, it is often unclear what accountability means and there is ample evidence that it may not be the primary driver of bureaucrat behavior.

In contrast to the dominant view of agencies as \textit{agents}, a growing literature in political science draws on scholarship in law and public administration as well as studies of agenda setting and lobbying in legislative policymaking to better understand agencies as policymaking bodies. Public administration and legal scholars have been more attentive to the prominent role of interest groups.  Kerwin (2003) notes that ``Interest groups could find few modes of government decision making better suited to their particular strengths than rulemaking.'' This research finds business groups to be most successful class of commenters in rulemaking \citep{Yackee2006a} especially when lobbying together, often, or unopposed (Nelson and Yackee 2012) and when lobbying across multiple venues \citep{Yackee2015}. Importantly, this literature notes that the currency of lobbying is information (Hall and Deardorf 2006), which includes both science and policy ideas \citep{Jones2005}. Kirilenko (2014) and Yackee and Yackee (2006) both find evidence that comments from sophisticated interest groups like businesses seem to influence rules. These scholars offer one set of answers to the question of who wins: those who succeed in rulemaking tend to be business interests, repeat players, those who lobby together, and those who lobby unopposed. They succeed because they bring in new voices and send unified messages at higher amplitudes, creating perceptions of political consensus.


%There may be an inverse relationship between how responsive agencies are to political principals and to the public \citep{Lewis}.

%Yet public administration and legal scholarship rarely address how interest groups gain political power in the first place. 

% more good stuff
A second major contribution to theory in this area is Carpenter's  research explaining bureaucratic autonomy \citep{Carpenter2001,Carpenter2012}. Rather than asking how bureaucratic practices fit with normative assumptions, he asks how agencies became independent policymaking bodies. Responding to principal-agent literature that has focused on the presidential and Congressional control, Carpenter finds much more complex sets of relationships that explain organizational power and behavior. One of the main tools he gives us for understanding the source of bureaucratic autonomy is the concept institutional reputations. Bureaucrats and the institutions they animate develop reputations for certain competencies: for example, for expertly adjudicating scientific claims, for effectively executing policy aimed at a given goal, or for divining the public interest. Reputations for expertise, effectiveness, or representativeness reflect the mixed roles assigned to the bureaucrats: advisors, implementers, and policymakers. 
% Like Carpenter, I call attention to the fact that agency policy shapes the coalitions that surround and influence it. I depart from Carpenter's narrative in that I do not focus on cases where agencies intend to have these effects. Whereas Carpenter is interested in how bureaucrats intentionally shape lobbying coalitions, I am interested in the endogenous relationship between policy and coalitions, intended or not. While not the focus of his study, Carpenter notes that relationships also evolve in unintended ways. 
%Policy may pro-actively recruit group support, but may also be reacting to political pressure \footnote{For example, an industry may successfully lobby to be reclassified to face lower pollution regulations, perhaps those faced by their competition, thus turning competitors into allies for future policymaking and increasing the size of the coalition for the lower standard.} or be an unintended side effect of action the agency sees as imperative.\footnote{For example, new science on the health hazards of mercury led to pollution controls that differentiated coal power plants and gas power plants reshaping coalitions by turning allies on former air quality policymaking into competitors in future rounds.} %Nevertheless, 
Carpenter and related scholars thus offer a second possible set of predictions for which movements are successful: those who succeed in rulemaking tend to be those with close relationships with the agency, conditional upon (and because of) how those relationships support the agency's reputation for expertise, competence, and representativeness. %Furthermore, lobbying coalitions, their relationship to the agency, and thus their success are functions of past agency policy. 

%Some scholars attempt to estimate the preferences of bureaucrats. Instead, I take 
% Carpenter's findings of close relationships between interest groups and agencies is a potential explanation for why some groups appear to have influence, i.e. because they are aligned with agency ideologies. As my core contribution is to assess how groups are empowered or disempowered rather than how agencies are empowered or disempowered, I focus on discovering which groups' comments are related to changes in rules regardless of whether this is what certain bureaucrats also wanted. 




% \subsubsection{Reputations for accountability, representation, equity, and expertise }

% [How specific organizational identities and reputations drive decisionmaking]



\textbf{Assessing causal mechanisms:}
While it may be impossible to causally identify or attribute effects to normative or strategic mechanisms, 
a focus on political information 
%and the schema of Table \ref{2x2} 
suggests places to look for influence in rulemaking. For example, if Members of Congress are not more likely to voice support for a coalition that goes public, this would be evidence against that indirect mechanism.
% While scholars often focus on the top right cell of Table \ref{2x2}, the influence of political information is to be found in the other three cells.\footnote{Political scientists have focused on strategic factors---either on how lobbying provides technical information that directly influences agency decisions or on how oversight indirectly constrains them.  Mass engagement is only likely to affect the later.}

%I also look closely at rules that end up before the Supreme Court. These rulemaking processes deserve extra attention for two reasons. First, regardless of how contentious they were at the rulemaking stage, these rules are key to understanding the nature and limits of executive power as a policymaking venue. Second, because all rulemaking is done in the shadow of judicial review, who wins in court and why shapes the political terrain of future rulemaking, empowering some groups with credible threats of litigation and disempowering others. References to court cases and implied threats of litigation are common in interest group comments, but to my knowledge, no study has looked systematically at how they affect rulemaking. Conversely, scholarship has not systematically assessed how mobilization and contestation in a rulemaking process affect judicial review. 