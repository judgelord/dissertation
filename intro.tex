
Paticipatory processes like public comment periods, where government agencies must solicit public input on draft policies, are said to provide political oversight opportunities \citep{Balla1998, Mccubbins1984}, democratic legitimacy \citep{Croley2003, Rosenbloom2003}, and new technical information \citep{Yackee2006JPART, Nelson2012}. %\footnote{These various goals are evident in the Proposed Recommendation on Public Engagement in Rulemaking from the Administrative Conference of the United States, which asserts that ``The opportunity for public engagement is vital to the rulemaking process, permitting agencies to obtain more comprehensive information, enhance the legitimacy and accountability of their decisions, and enhance public support for their rules'' \citep{ACUS2018}.}
%Public comment periods are purported to simultaneously produce technical information, accountability to elected officials, and responsiveness to public demands.
While recent scholarship on agency policymaking has shed light on the sophisticated lobbying by businesses and political insiders, we know surprisingly little about the vast majority of public comments which are submitted by ordinary people as part of public pressure campaigns.\footnote{As I show elsewhere \citep{Judge-Lord2019}, most comments submitted to regulations.gov are form comments, more akin to petition signatures than sophisticated lobbying. Indeed, aproximately 40 million out of 50 million (80\%) of these public comments mobilized by just 100 advocacy organizations.}
Activists frequently target agency policymaking with letter-writing campaigns, petitions, protests, and mobilizing people to attend hearings, all classic examples of ``civic engagement'' \citep{Verba1987}. Yet civic engagement remains poorly understood in the context of bureaucratic policymaking.

These occasional bursts of civic engagement in bureaucratic policymaking raise practical and theoretical questions for the practice of democracy.\footnote{In 2018, the Administrative Conference of the United States (ACUS) identified mass commenting as a top issue in administrative law. In their report to ACUS, \citet{SantAmbrogio2018} conclude, ``The `mass comments' occasionally submitted in great volume in highly salient rulemakings are one of the more vexing challenges facing agencies in recent years. Mass comments are typically the result of orchestrated campaigns by advocacy groups to persuade members or other like-minded individuals to express support for or opposition to an agency's proposed rule.'' 
Mass comment campaigns are known to drive significant participation of ordinary people in Environmental Protection Agency rulemaking \citep{Judge-Lord2019, Potter2017, Balla2018}. \citet{Cuellar2005}, who examines public input on three rules, finds that ordinary people made up the majority of commenters demonstrating ``demand among the mass public for a seat at the table in the regulatory process.'' } 
%To date, administrative law scholars have focused on practical and normative questions, much of this analysis depends 
These questions, in turn, hinge on unanswered empirical questions: Do these campaigns affect policy? If so, by what mechanisms? Existing research finds that commenters believe their comments matter \citep{Yackee2015JPART} and that the number of public comments varies across agencies and policy processes \citep{Judge-Lord2019, Libgober2018, Moore2017},
% and policy change is related to the number of comments in sample of nine rules \citep{Shapiro2008}, 
but the relationship between the scale of public engagment and policy change remains untested. 

To address this gap, I assess the relationship between the number of public comments and the amount of change between draft and final policy texts. Next, I assess the relationship between the number of people mobilized by each campaign and whether the campaign acheivied its policy goals. Finally, I theorize and test four mechanisms by which public input may affect bureaucratic policymaking. Each mechanism involves a distinct type of information that pressure campaigns may relay to policymakers: technical information, information about the likelihood of political consequences, information about the preferences of elected officials, or information about the preferences of the attentive public. Because scholarship on bureaucratic policymaking has focused on the power of technical information, where insider lobbying is most likely to matter and where outside strategies are least likely to matter, political scientists have largely overlooked mass mobilization as a tactic.

% To assess the relationship between mass engagement and policy change, I introduce a large new dataset of millions of public comments on agency rules and assess mass comment campaigns' impact on rulemaking processes and outcomes. % To assess the relationship between mass public participation and rule change, I outline four potential mechanisms by which public input may influence bureaucratic decisions. 
I find evidence consistent with the observable implications of mass comment campaigns influencing policymaking through [non-null results] but no evidence that mass engagement affects rulemaking processes or outcomes through [null results].


