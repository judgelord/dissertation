\documentclass{article}

\date{\today} 
 %\setlength{\parskip}{1em}
% \usepackage[round, sort, comma]{natbib}
% \setcitestyle{notesep={, },round,aysep={},yysep={,}}

%\setlength{\parindent}{0pt}

% \usepackage[authordate,strict,backend=bibtex8,babel=other, doi=false, url=false, isbn=false, uniquename=false, maxcitenames = 2, uniquelist=false,%
% bibencoding=inputenc, natbib]{biblatex-chicago}
%\usepackage[authordate, backend=biber, doi=false, url=false, isbn=false, uniquename=false, maxcitenames = 2, uniquelist=false, natbib]{biblatex-chicago}



% ---------------------------------------
%   REFERENCES
% ---------------------------------------
\usepackage{natbib} %[round, sort, comma]{natbib}
  \bibpunct[: ]{(}{)}{;}{a}{}{,}
%\usepackage[nottoc,numbib]{tocbibind} % numbers bib. in table of contents


% ---------------------------------------
%   MARGINS AND SPACING
% ---------------------------------------
\usepackage[margin=1in]{geometry} 
\usepackage{setspace} % line spacing


% ---------------------------------------
%   TABLES AND FIGURES
% ---------------------------------------
\usepackage{graphicx} % input graphics
\graphicspath{ {./Figs/} }

\usepackage{float} % float parameters
\usepackage{placeins} % \FloatBarrier: prevent floats spilling across sections
\usepackage{subcaption} % subfloats with individual captions
\usepackage{multirow} % multicolumn and multirow
\usepackage{booktabs} %? for toprule, midrule etc
\usepackage{dcolumn} % decimal-aligned columns

\usepackage{tikz}
\usetikzlibrary{positioning}
\usetikzlibrary{shapes.geometric}
\usetikzlibrary[patterns]

% ---------------------------------------
%   MATH
% ---------------------------------------
\usepackage{amsmath}
%  Math symbols (depends on whether you customize typeface)
%  \usepackage{amsfonts}
%  \usepackage{mathrsfs}
\usepackage{amssymb}
  %  \newcommand{\E}{\mathrm{E}}
  %  \newcommand{\Var}{\mathrm{Var}}
  %  \newcommand{\Cov}{\mathrm{Cov}}
  %  \newcommand{\plim}{\mathrm{plim}}
  %  \renewcommand{\L}{\mathcal{L}}
  %  \renewcommand{\d}{\mathrm{d}}
  %  \newcommand{\R}{\mathbb{R}}


% ---------------------------------------
%   TYPEFACE AND TEXT STYLES
% ---------------------------------------


% \usepackage{fontenc} 
% advisable to include for nitty-gritty details
% (ligatures, kerning & other typographical things)

% \usepackage[usenames, dvipsnames]{xcolor} % extra colors
\usepackage{hyperref} % hyperlinks
   \hypersetup{
     colorlinks = true, 
     citecolor = black, 
     linkcolor = blue,
     urlcolor = blue}

\usepackage{comment} % provides {comment} environment
\usepackage{enumitem} % allows [nosep] option for lists

\usepackage{endnotes}

\makeatletter
\renewcommand\@makeenmark{%
  \textsuperscript{\normalfont\textcolor{blue}{\@theenmark}}%
}
\newcommand{\uncolormarkers}{%
  \renewcommand\@makeenmark{%
    \textsuperscript{\normalfont\@theenmark}%
  }%
}
\makeatother


\newcommand{\exclude}[1]{\StopSearching ##1\StartSearching}

\usepackage{ragged2e}

\begin{document}
\spacing{2}

Processes like public comment periods, where government agencies must solicit public input on draft policies, are said to produce new technical information \citep{Yackee2006JPART, Nelson2012}, political oversight opportunities \citep{Balla1998, Mccubbins1984}, and democratic legitimacy \citep{Croley2003, Rosenbloom2003}. %There is no normative consensus on how to rank or merge these goals \citep{Wilson1967, Wilson1989, Carrigan2017}. Procedures requiring agencies to solicit public input and the justification of these procedures cite all three aims. For example, 
These various goals are evident in the Proposed Recommendation on Public Engagement in Rulemaking from the Administrative Conference of the United States, which asserts that ``The opportunity for public engagement is vital to the rulemaking process, permitting agencies to obtain more comprehensive information, enhance the legitimacy and accountability of their decisions, and enhance public support for their rules'' \citep{ACUS2018}. %Public comment periods are purported to simultaneously produce technical information, accountability to elected officials, and responsiveness to public demands.

% gap
%Yet, legitimacy, accountability, public support, and, especially, collecting information depend not just on the opportunity to engage but actual engagement \citep{Herz2018}, and 
Yet we know surprisingly little about the vast majority of public comments (i.e., those submitted by ordinary people as part of public pressure campaigns) and the impact this kind of input may play in policymaking. As \citet{SantAmbrogio2018} conclude, ``The `mass comments' occasionally submitted in great volume in highly salient rulemakings are one of the more vexing challenges facing agencies in recent years. These comments are typically the result of orchestrated campaigns by advocacy groups to persuade members or other like-minded individuals to express support for or opposition to an agency's proposed rule.''

% lit 
% DEFINITION
%\paragraph{Defining mass engagement}
Civic engagement includes writing to government officials, signing petitions, attending hearings, attending protests, or donating to a political campaign \citep{Verba1987}. %While donating is more common in electoral politics, a
Activists frequently target agency policymaking with letter-writing campaigns, petitions, protests, and mobilizing people to attend hearings. 
% I suspect that mass commenting is driven by the same privileged populations known to engage in other civic activities. 
Mass comment campaigns are known to drive significant participation of ordinary people in Environmental Protection Agency rulemaking \citet{Potter2017, Balla2018}. \citet{Cuellar2005}, who examines public input on three rules, finds that ordinary people made up the majority of commenters demonstrating ``demand among the mass public for a seat at the table in the regulatory process''. \citet{Moore2017} finds that the number of public comments varies across agencies and policy processes. Does this variation matter for policy outcomes?

Existing research finds that commenters believe their comments matter and that 
\citep{Yackee2015JPART}, % surveys commenters, and find that members of the public believe that their comments matter, %even though powerful groups have more influence;
and a correlation between comments and rule change in certain small samples
\citep{Shapiro2008}, but the relationship between the scale of public participation and policy change remains largely untested. 
%\citep{Shapiro2008} studied nine rules, finding the more complex but lower-salience rules with more comments were also the ones that changed.

In this paper, I introduce a large new dataset of millions of public comments on agency rules and assess mass comment campaigns' impact on the rulemaking processes and outcomes. To assess the relationship between mass public participation and rule change, I outline four potential mechanisms by which public input may influence bureaucratic decisions. I find evidence consistent with the observable implications of mass comment campaigns influencing policymaking through [non-null results] but no evidence that mass engagement affects rulemaking processes or outcomes through [null results].


\singlespace
\small
\bibliographystyle{apsr.bst} 
\bibliography{mendeley.bib}
\end{document}


