\documentclass{article}
\pagestyle{headings}
\date{\today} 

% --- section style -----------------------

% we don't go deeper than subsection (hopefully..?)
\setsecheadstyle{\large \raggedright \bfseries}
\setsubsecheadstyle{\raggedright \bfseries}
\setsubsubsecheadstyle{\raggedright \itshape} 

% numbering depth
\setsecnumdepth{subsubsection}  

 \usepackage{booktabs}
 \usepackage{cleveref}

\renewcommand{\eqref}{\Cref}
\Crefformat{equation}{#2#1#3}

% \usepackage{float}
% \let\origfigure\figure
% \let\endorigfigure\endfigure
% \renewenvironment{figure}[1][2] {
%     \expandafter\origfigure\expandafter[H]
% } {
%     \endorigfigure
% }



% LINKS 
\usepackage{hyperref}
\hypersetup{breaklinks = true,
            bookmarks = true,  
            colorlinks = true,
            citecolor = black,
            urlcolor = black,
            linkcolor = magenta}

% FIG CAPTIONS 
\let\newfloat\relax 
\usepackage{floatrow}
\floatsetup[figure]{capposition=top}
\floatsetup[table]{capposition=top}
\usepackage{setspace}

% inherit spacing names from memoir
%\newcommand{\OnehalfSpacing}{\onehalfspacing}
%\newcommand{\DoubleSpacing}{\doublespacing}

% SPACING FOR BLOCK QUOTES 
\expandafter\def\expandafter\quote\expandafter{\quote\OnehalfSpacing}

% FOR KABLEEXTRA 
\usepackage{booktabs}
\usepackage{longtable}
\usepackage{array}
\usepackage{multirow}
\usepackage{wrapfig}
%\usepackage{float}
\usepackage{colortbl}
\usepackage{pdflscape}
\usepackage{tabu}
\usepackage{threeparttable}
\usepackage{threeparttablex}
\usepackage[normalem]{ulem}
\usepackage{makecell}
\usepackage{xcolor}
\usepackage{ragged2e}

\usepackage{endnotes}

\makeatletter
\renewcommand\@makeenmark{%
  \textsuperscript{\normalfont\textcolor{blue}{\@theenmark}}%
}
\newcommand{\uncolormarkers}{%
  \renewcommand\@makeenmark{%
    \textsuperscript{\normalfont\@theenmark}%
  }%
}
\makeatother


\newcommand{\exclude}[1]{\StopSearching ##1\StartSearching}

\usepackage{ragged2e}

\begin{document}
% \abstract{Do public pressure campaigns affect policies made by unelected bureaucrats? In this article, I develop several measures of lobbying success and corresponding tests of whether mass engagment increases lobbying success. 
I then theorize mechanisms by which mass public engagment may affect policy. Each involves a distinct type of information revealed to decisionmakers. 

}
\spacing{2}

Processes like public comment periods, where government agencies must solicit public input on draft policies, are said to produce new technical information \citep{Yackee2006JPART, Nelson2012}, political oversight opportunities \citep{Balla1998, Mccubbins1984}, and democratic legitimacy \citep{Croley2003, Rosenbloom2003}.%\footnote{These various goals are evident in the Proposed Recommendation on Public Engagement in Rulemaking from the Administrative Conference of the United States, which asserts that ``The opportunity for public engagement is vital to the rulemaking process, permitting agencies to obtain more comprehensive information, enhance the legitimacy and accountability of their decisions, and enhance public support for their rules'' \citep{ACUS2018}.}
%Public comment periods are purported to simultaneously produce technical information, accountability to elected officials, and responsiveness to public demands.
While recent scholarship on agency rulemaking has shed light on the sophisticated lobbying by businesses and political insiders, we know surprisingly little about the vast majority of public comments which are submitted by ordinary people as part of public pressure campaigns. 

These occasional bursts of civic engagement in agency rulemaking raise practical and theoretical questions for democracy.\footnote{In 2018, the Administrative Conference of the United States (ACUS) identified mass commenting as a top issue in administrative law. In their report to ACUS, \citet{SantAmbrogio2018} conclude, ``The `mass comments' occasionally submitted in great volume in highly salient rulemakings are one of the more vexing challenges facing agencies in recent years. Mass comments are typically the result of orchestrated campaigns by advocacy groups to persuade members or other like-minded individuals to express support for or opposition to an agency's proposed rule.'' 
Mass comment campaigns are known to drive significant participation of ordinary people in Environmental Protection Agency rulemaking \citep{Potter2017, Balla2018}. \citet{Cuellar2005}, who examines public input on three rules, finds that ordinary people made up the majority of commenters demonstrating ``demand among the mass public for a seat at the table in the regulatory process.'' Activists frequently target agency policymaking with letter-writing campaigns, petitions, protests, and mobilizing people to attend hearings, all classic examples of ``civic engagement'' \citep{Verba1987}. Yet civic engagement remains poorly understood in the context of bureaucratic policymaking.} 
While administrative law scholars have focused on practical and normative questions, much of this analysis depends on empirical questions: Do these campaigns affect policy? If so, by what mechanisms? Existing research finds that commenters believe their comments matter \citep{Yackee2015JPART} and that the number of public comments varies across agencies and policy processes \citep{Moore2017}, 
% and policy change is related to the number of comments in sample of nine rules \citep{Shapiro2008}, 
but the overall relationship between the scale of public participation and policy change remains untested. 

In this article, I theorize and test four mechanisms by which public comments may affect bureaucratic policymaking. Each mechanism involves a distinct type of information that mass comment campaigns may relay to policymakers: technical information, information about the likelihood of political consequences, information about the preferences of elected officials, or information about the preferences of the attentive public. Thus far, scholarship on bureaucratic policymaking has focused on the power of technical information, where insider lobbying is most likely to matter and where outside lobbying is least likely to matter. As a result, political scientists have largely overlooked mass engagement.

To assess the relationship between mass engagement and policy change, I introduce a large new dataset of millions of public comments on agency rules and assess mass comment campaigns' impact on rulemaking processes and outcomes. % To assess the relationship between mass public participation and rule change, I outline four potential mechanisms by which public input may influence bureaucratic decisions. 
I find evidence consistent with the observable implications of mass comment campaigns influencing policymaking through [non-null results] but no evidence that mass engagement affects rulemaking processes or outcomes through [null results].


\singlespace
\small
\bibliographystyle{apsr.bst} 
\bibliography{mendeley.bib}
\end{document}


