% MORE

% \paragraph{Regulating mercury pollution.}
% \subsection{Puzzle: Why mobilize?}
% \section{Why mobilize?}
% Prior to the 22 million public comments on the Federal Communications Commission's 2017 Open Internet rule,\footnote{It is yet unclear how many of these comments are from real people.} 
In 2011 the Environmental protection agency proposed two regulations to limit mercury emissions from coal and oil-fired power plants. Among other issues, the Environmental Protection Agency (EPA) solicited ``comments on whether there would be a basis for considering area sources to be significantly different from major sources,'' ``on the adequacy of the restrictions associated with bypass conditions regarding maintaining LEE status" and ``on the proposed revisions concerning [equations' 1a and 1b] usefulness in calculating the maximum potential emissions rate from an emissions averaging group'' (EPA 2011). LEE status is not defined in the notice soliciting comments, and equations 1a and 1b are surely inaccessible to most citizens. Yet these two proposed rules received 942,483 comments. 

One comment, from the United States Council of Catholic Bishops, read: ``While we are not experts on air pollution, our general support for a national standard to reduce hazardous air pollution from power plants is guided by Catholic teaching, which calls us to care for God’s creation and protect the common good and the life and dignity of human persons, especially the poor and vulnerable.'' Bishops are not known to closely follow power plant regulations. Their moral authority was mobilized by activists who wanted stricter regulation of mercury and wanted to demonstrate public support of this position. %Groups mobilizing on the mercury rules including environmental and health groups and industry competitors, including the owners of Nuclear, Natural, geothermal power plants. 

In the official, legally-required response to comments, the EPA did not discuss public opinion, obligations to care for the environment, dignity, or the poor. Instead, the EPA asserted that mercury levels are a matter of science, not a matter of justice. But the EPA did implicitly assert a definition of the public good when it used studies of mercury's aggregate public health effects on the U.S. population to set emissions standard.\footnote{As Wagner (1995) %CITE
notes, ``agencies exaggerate the contributions made by science...in order to avoid accountability for the underlying policy decisions. %Although camouflaging controversial policy decisions as science assists the agency in evading various political, legal, and institutional forces, doing so ultimately delays and distorts the standard-setting mission'' (p. 1617).
Wagner goes on to find that  ``While the APA mandates a process for public involvement, it provides almost no protections to ensure that agencies will explain the substantive bases for highly complex or technical rulemakings in a way that the lay public can readily understand and challenge'' (1656) and that ``Mischaracterization of the entire standard-setting endeavor as resolvable by science results in significant obstacles to democratic participation'' (1674). Similarly, Harvey Brooks (1984) %CITE
notes that ``The modern nation risks being no longer recognizable as a democracy, either representative or plebiscitary, if more and more policy areas are excluded from public participation because of the technical complexity.''} 
Then, as required by the Supreme Court, it justified the same standards with cost-benefit analysis in a revised proposed rule, concluding that for every dollar spent to comply with the regulation, the U.S. public receives up to nine dollars in health benefits (EPA 2007). If this is how decisions are made, %why did the EPA receive nearly a million letters? W
why would citizen opinions matter? % MAY NOT MATTER 

\subsubsection{Political information and influence in rulemaking}
% Wagner: A variety of commentators have suggested that agencies may seek increased legitimacy or decreased political accountability by disguising their policy judgments as science. See Majone, supra note 18, at 15 ("Traditionally, government regulators have sought legitimacy for their decisions by wrapping them in a cloak of scientific respectability.");Roberts et al., supra note 26, at 120 ("Too many of the participants [in science-policy decisions] have good reasons not to distinguish scientific evidence from policy preferences, not to analyze carefully the various sources of technical disagreement, and not to accept responsibility for some decisions or judgments."). Beyond these common sense observations scattered at points in articles and books, there has been surprisingly little scholarly discussion of the comprehensive existence of or reasons for a science charade in regulation.

In this section, I integrate the above arguments about interest group lobbying and political oversight to offer a broader model of influence in rulemaking that highlights the political information available to policymakers. As suggested above, such information occasionally arises from contentious debate and civic mobilization and may influence elected officials' oversight behaviors. %I build on studies of interest group influence, social movement influence, and the influence of political principals to offer a model of agency policymaking

% preview 
If we appreciate agency policymaking as a site of contentious politics, mechanisms emerge by which mass mobilization may affect both the strategic environment and ideological perspectives of those who write agency rules. These mechanisms remain under-theorized and untested. I aim to begin to fill this gap by outlining four mechanisms by which political information may influence bureaucratic policymaking. 
Bureaucrats have both strategic and normative reasons for updating policy decisions in light of new political information. The effects of political information on policy thus depend on the strategic context that may or may offer opportunities for influence and cognitive and institutional processes that may or may not incorporate political information.

\paragraph{Political influence in rulemaking}
In contrast to the science-based objectivity suggested by the above quote from the EPA, political scientists, building on law and economics scholarship, offer a different theory of bureaucratic decisionmaking rooted in the policy preferences and strategic behavior of agency leaders and their political principals: Congress, the president, and the courts. They find that political principals do constrain agency action but also leave room for agencies to move policy toward their own ideal point.\footnote{Though political scientists make diverse assumptions about what this ideal is and how to measure it.} Technical information may or may not inform agency decisions, but preferences and the power to realize them in a strategic environment are, these scholars say, are the proximate cause of policy. These scholars would see the Mercury Rules as the result of EPA officials writing a policy as close as possible to their ideal policy given their strategic constraints.

But the strategic model does not explain why activists thought public comments would matter. If the strategic model is sufficient, why did organization write letters to the EPA? The EPA administrator has their preferences and the public has no direct power over their decisions. Why not write to the president or members of Congress who influence EPA's strategic calculations and are more directly accountable to public opinion? In section \ref{whymail-intro} I offered three reasons that organizations may launch mass mobilization campaigns. In section \ref{principals-intro}, I suggested that public comments may have a similar effect on political principals as to mass mail campaigns directly targeting them. In this section, I merge these insights into a broader theory of how mass engagement may affect policy. 

% adding IGs to PA model 
Political scientists, along with scholars of public administration, organizational behavior, and sociology who study interest groups expand on the simple principal-agent model described above that, while less parsimonious, squarely address how the process of soliciting and responding to public comments may influence agency policy (see \citet{Yackee2018} for a review). These scholars find that agency staff develop relationships with those who regularly participate in policy processes: most often businesses but also professional associations and activist organizations. Relationships draw on and reproduce organizational identities and reputations \citep{Carpenter2001, Carpenter2014, carpenter2012}. This scholarship has revealed a good deal about how organized groups lobby agencies and why they succeed, but it has yet to address why these groups sometimes mobilize thousands of citizens to write letters or protest agency decisions. Like most forms of political participation, 
mass public comments on draft agency rules provide no new technical information. 
They lack the authority of elected officials' opinions. 
And the number on each side has no legal import for an agency's response \citep{Kerwin2011}. %, West1995}.
Policymakers may very well pay no attention to them. The theoretical foundations for why mass mobilization may matter is underdeveloped and we lack empirical research on how it may affect agency policymaking. 

% In this chapter, 
%In this paper, 
I address this theoretical and empirical gap in our knowledge on the role of mass mobilization in bureaucratic policymaking. I expand and integrate the above theories to develop testable hypotheses % and analyze rule-related texts %and field experiments
to explore whether mass mobilization matters and, if so, why. 

I argue that if mass mobilization indirectly affects the strategic environment it does so by signaling grass-roots political power to elected officials and if mass mobilization directly affects agency policymaking it does so by evoking norms rooted in organizational identities and reputations. 
Like the vast majority of letter-writers, Catholic Bishops contribute little to the technical aspects of epidemiology, mercury regulations, or cost-benefit analysis. If they influence agency policymaking, it is by signaling a threat of political backlash or by persuading bureaucrats directly that moving policy in a certain direction is the appropriate thing for the agency to do. 
The next two subsections address these indirect and direct mechanisms in more depth. A third discusses why we may still observe mobilization in the absence of influence. 
% Whereas social movement scholars and political scientists have focused the behavior of elected leaders, I focus on the latter pathway of direct persuasion. 




% strategic 
\paragraph{Opportunity.} First, the influence of political information depends on the extent to which the strategic environment allows change. A core concept in social movement scholarship is the ``political opportunity'' or ``opportunity structure'' \citep{Mcadam2017} such as division among elites \citep{Tarrow1994}. In my case, a division among elites could be divergent principals among an agency's political principals or powerful business interest groups. With respect to division among business interests, this aligns with findings from studies of rulemaking that find that consensus among businesses increases their influence in rulemaking \citep{Yackee2006JOP, Nelson2012}. However, the role of divided government is less clear. \citet{Yackee2009RegGov} find that divided government decreases the frequency of rulemaking. However, this does not mean that it reduces the potential for groups to influence those rules that are produced.

Policy process scholars use a similar concept of ``window'' for policy change where advocates have an opportunity to align politics with certain identified problems and solutions  \citep{Kingdon1984}. All rulemaking processes create opportunities, however small, to shape the new status quo, loosely bounded by the problems the process was initiated to solve, a set of policy solutions considered legitimate, and a constellation of political forces.

% normative
\paragraph{Information processing.} Second, influence depends on how political information is processed, both directly within agencies and indirectly through other actors (e.g. Members of Congress) whose appraisals matter to bureaucrats.

%%%%%%%%%%%%%%%%%%%%%%%%%%%%%%%%%%% FIXME
% STRATEGIC ENV 
% \subsubsection{ Agencies as Policymaking Venues}
% the good stuff

% When political scientists ask whose interests and ideas become law, they have generally focused on the behavior of legislatures, how the executive branch drives legislation, and how the courts review it. Compared to legislative, executive, and judicial institutions, the administrative state is a recent development in American government and theory has not kept pace with the rise in bureaucratic policymaking. 

% I argue that theories of bureaucratic policymaking have been characterized by constraining assumptions about what bureaucracies ought to do.  Normative assumptions that \citet{Wilson1967} identified half a century ago, and corresponding scholarly silos, have persisted. This has led to lines of research talking past each other and often failing to engage broader theories of policy change. In particular, I argue that the pervasive implicit assumption that bureaucrats ought to be neutral implementers implies that politics in agency policymaking is inherently undesirable, leading many scholars to focus on compliance with political principals and overlook the role participation and ideas. For example, scholars assume that agencies ought to be engaged in implementing legislation and executive orders. However, most rulemaking takes place many years or decades after its authorizing legislation under a different Congress and with little attention from the White House until the very final draft. Rules that do not follow from contemporary Congressional or executive priorities are often assumed to reflect bureaucrats going rogue or being captured by interest groups. Such studies suffer from a lack of attention to the complex political process of rulemaking. 

%Viewing agencies as \textit{agents} has prevented scholars from incorporating new insights about the endogenous relationship between policy and politics. I suggest rulemaking is better studied in the way that scholars study policymaking in specialized congressional committees than with an unrealistic dichotomy of sincere implementation versus capture or disloyalty. Normatively, accountability to political principals only one of several important concerns. Empirically, it is often unclear what accountability means and there is ample evidence that it may not be the primary driver of bureaucrat behavior.


%There may be an inverse relationship between how responsive agencies are to political principals and to the public \citep{Lewis}.

%Yet public administration and legal scholarship rarely address how interest groups gain political power in the first place. 

% more good stuff
% A second major contribution to theory in this area is work by \citet{Carpenter2001} and \citet{Carpenter2012}  research explaining how bureaucrats realize their own goals and achieve autonomy. %Rather than asking how bureaucratic practices fit with normative assumptions, he asks how agencies became independent policymaking bodies. Responding to principal-agent literature that has focused on the presidential and Congressional control, 

% Carpenter finds much more complex sets of relationships that explain organizational power and behavior. One of the main tools he gives us for understanding the source of bureaucratic autonomy is the concept of institutional reputations. Bureaucrats and the institutions they animate develop reputations for certain competencies: for example, for expertly adjudicating scientific claims, for effectively executing policy aimed at a given goal, or for responding to the public interest. Reputations for expertise, effectiveness, or responsiveness reflect the mixed roles assigned to the bureaucrats: advisers, implementers, and policymakers. 
% Like Carpenter, I call attention to the fact that agency policy shapes the coalitions that surround and influence it. I depart from Carpenter's narrative in that I do not focus on cases where agencies intend to have these effects. Whereas Carpenter is interested in how bureaucrats intentionally shape lobbying coalitions, I am interested in the endogenous relationship between policy and coalitions, intended or not. While not the focus of his study, Carpenter notes that relationships also evolve in unintended ways. 
%Policy may pro-actively recruit group support, but may also be reacting to political pressure \footnote{For example, an industry may successfully lobby to be reclassified to face lower pollution regulations, perhaps those faced by their competition, thus turning competitors into allies for future policymaking and increasing the size of the coalition for the lower standard.} or be an unintended side effect of action the agency sees as imperative.\footnote{For example, new science on the health hazards of mercury led to pollution controls that differentiated coal power plants and gas power plants reshaping coalitions by turning allies on former air quality policymaking into competitors in future rounds.} %Nevertheless, 

% Carpenter and related scholars thus offer a second possible set of predictions for which movements are successful: those who succeed in rulemaking tend to be those with close relationships with the agency, conditional upon (and because of) how those relationships support the agency's reputation for expertise, competence, and representativeness. %Furthermore, lobbying coalitions, their relationship to the agency, and thus their success are functions of past agency policy. 

%Some scholars attempt to estimate the preferences of bureaucrats. Instead, I take 
% Carpenter's findings of close relationships between interest groups and agencies is a potential explanation for why some groups appear to have influence, i.e. because they are aligned with agency ideologies. As my core contribution is to assess how groups are empowered or disempowered rather than how agencies are empowered or disempowered, I focus on discovering which groups' comments are related to changes in rules regardless of whether this is what certain bureaucrats also wanted. 




% \subsubsection{Reputations for accountability, representation, equity, and expertise }

% [How specific organizational identities and reputations drive decisionmaking]

