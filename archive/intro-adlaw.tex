% LEGAL SCHOLARS' DEBATES 
In contrast to political scientists, legal scholars have long debated what to make of mass commenting in rulemaking. Many focus on reforms to help agencies collect more useful information \citep{Farina2011, Farina2014, Rauch2016}. In 2018, ``Public engagement'' was main project of the Administrative Conference of the United States (ACUS) committee on Rulemaking: %\href{https://www.acus.gov/research-projects/public-engagement-rulemaking}
{The project} 
\begin{quote}``explores agency strategies to enhance public engagement prior to and during informal rulemaking. It seeks to ensure that agencies invest resources in a way that maximizes the probability that rulewriters obtain high-quality public information.'' 
\end{quote} 
Among other things, this committee is debating how best to gather ``quality public information,'' how ``to get new people/groups into the real or virtual room'' \citep{Farina2018}, and whether broad engagement is even desirable on all rules \citep{White2018}. 

Administrative law scholars have explored these questions theoretically for decades, but only a few offer empirical analysis. \citet{Mendelson2011} finds that agencies often discard non-technical comments but argues that they should be given more weight. Others worry that mass commenting distracts agencies from good policy and the broader public interest \citep{Coglianese2006}. \citet[p. 112]{Farina2012} claims that ``[Mass] comments typically are neither factually informative nor reliable indicators of citizens’ informed value preferences.'' Some even call them ``spam'' \citep{Balla2018, Noveck2004}. In this prevailing view, ``high-quality'' and ``relevant'' mean novel technical information, not opinions. \citet[p. 208]{Herz2016} concludes ``The goal of e-rulemaking is to more fully capture such credible, specific, and relevant information, not to solicit the views of random, self-nominating members of the public.'' Similarly,  \citet[p. 4]{Epstein2014} dismiss mass comments as ``effectively, votes rather than informational or analytical contributions. Rulemaking agencies are legally required to make policy decisions based on fact-based, reasoned analysis rather than majority sentiment; hence, even hundreds of thousands of such comments have little value in the rulemaking process.''  Notably, the ACUS draft recommendations on ``Mass and Fake Comments in Agency Rulemaking'' suggests that ``effective comments'' give ``reasons rather than just reactions'' \citep[p. 33]{ACUS2018}. If true, most public reactions to proposed rules such as those expressed in mass comments would have no effect in rulemaking. 

Early optimism among legal scholars that the internet would ``change everything'' \citep{Johnson1998} and that ``cyberdemocracy''  would enable more deliberative rulemaking has faded.  
While commenting and mobilizing others to comment has become easier, \citet{Coglianese2006} finds that little else has changed. %\citet{Rossi1997} even suggests that public comment processes should be largely eliminated.
The prediction that the internet would primarily facilitate more of the same kind of engagement among the like-minded (i.e. mass-commenting) \citep{Sunstein2001} has largely been correct. In this sense, the ``quality'' of discourse has not improved.

Even scholars who suggest reforms aimed at ``regulatory democracy'' aim to increase the ``sophistication'' of ordinary peoples' comments \citep{Cuellar2014, Johnson2013}. For example, Beth Simone Noveck \citeyear{Noveck2004} 
is critical of ``notice and spam,'' arguing instead for ``participative practices---methods for `doing democracy' that build the skills and capacity necessary for citizens, experts, and organizations to speak and to be heard. Rulemaking, after all, is a communicative process involving a dialogue between regulators and those affected by regulation" \citet[p. 3]{Noveck2005}.

This scholarship has improved the theory and practice of policy learning in rulemaking. But a focus on sophisticated deliberation and technical information overlooks the potential role of political information.\footnote{But see insights from \citet{Golden1998}, \citet{Nelson2012}, \citet{Rauch2016}, and \citet{Potter2017} on political information, \citet{Cuellar2014} on participation and voice, and \citet{Reich1966} and \citet{Seifter2016UCLA} on representation, which I review elsewhere} Whereas administrative law scholars have focused on ``how technology can connect the expertise of the many to the power of the few'' \citep{Noveck2009}, I ask whether it may also connect the power of the many to the decisions of the few.