\documentclass[
      12pt,
        ]{article}






% --- type and typeface? -----------------------

% input
\usepackage[utf8]{inputenc}

% typography
\usepackage{microtype}


\usepackage[T1]{fontenc}


% text block
\usepackage{setspace}
\usepackage[              left = 1in,top = 1in,right = 1in,bottom = 1in             ]{geometry}

\usepackage{enumitem}
  \setlist{noitemsep}



% decimal numbering for appendix figs and tabs


% Deletes section counters
% \setcounter{secnumdepth}{0}







  \usepackage{longtable, booktabs}









  \usepackage{natbib}
  \bibliographystyle{apa}
  % protect underscores in most circumstances
  \usepackage[strings]{underscore} 


% 

% \newtheorem{hypothesis}{Hypothesis}

\makeatletter
  \@ifpackageloaded{hyperref}{}{%
    \ifxetex
      % page size defined by xetex
      % unicode breaks when used with xetex
      \PassOptionsToPackage{hyphens}{url}\usepackage[setpagesize = false, 
                                                     unicode = false, 
                                                     xetex]{hyperref}
    \else
      \PassOptionsToPackage{hyphens}{url}\usepackage[unicode = true]{hyperref}
    \fi
  }

  \@ifpackageloaded{color}{
    \PassOptionsToPackage{usenames,dvipsnames}{color}
  }{
    \usepackage[usenames,dvipsnames]{color}
  }
\makeatother

\hypersetup{breaklinks = true,
            bookmarks = true,
            pdfauthor = {Devin Judge-Lord (University of Wisconsin-Madison)},
             pdfkeywords  =  {},  
            pdftitle = {Do Public Pressure Campaigns Influence Bureaucratic Policymaking?},
            colorlinks = true,
            citecolor = black,
            urlcolor = blue,
            linkcolor = magenta,
            pdfborder = {0 0 0}}

% \urlstyle{same}  % don't use monospace font for urls


% set default figure placement to htbp
\makeatletter
  \def\fps@figure{hbtp}
\makeatother


% optional footnotes as endnotes


% ----- Pandoc wants this tightlist command ----------
\providecommand{\tightlist}{
  \setlength{\itemsep}{0pt}
  \setlength{\parskip}{0pt}
}





% --- title & section styles -----------------------


% title, author, date
  \title{Do Public Pressure Campaigns Influence Bureaucratic Policymaking?}

  \author{ % author, option footnote, optional affiliation
            Devin Judge-Lord  \\ \emph{University of Wisconsin-Madison} 
            }

% auto-format date?
  \date{\today}


% abstract
\usepackage{abstract}
  \renewcommand{\abstractname}{}    % clear the title
  \renewcommand{\absnamepos}{empty} % originally center

  \newcommand*{\authorfont}{\sffamily\selectfont}


% section titles
\usepackage[small, bf, sc]{titlesec}
  % \titleformat*{\subsection}{\itshape}
  \titleformat*{\subsubsection}{\itshape} 
  \titleformat*{\paragraph}{\itshape} 
  \titleformat*{\subparagraph}{\itshape}




\usepackage{floatrow}
\floatsetup[figure]{capposition=top}
\floatsetup[table]{capposition=top}
\usepackage{multirow}
\usepackage{rotating} 
\usepackage{caption}




\begin{document}
 

% --- PAGE: title and abstract -----------------------

  \maketitle

% \pagenumbering{gobble}



  \begin{abstract}
    \noindent Do public pressure campaigns affect policies made by unelected bureaucrats? In this article, I develop several measures of lobbying success and corresponding tests of whether public pressure campaings increase lobbying success. I then theorize mechanisms by which mass public engagment may affect policy. Each involves a distinct type of information revealed to decisionmakers. 

    
  \end{abstract}



% --- PAGE: contents -----------------------




% --- PAGE: body -----------------------



\noindent 
      \doublespacing 
    \newpage


Paticipatory processes like public comment periods, where government agencies must solicit public input on draft policies, are said to provide political oversight opportunities \citep{Balla1998, Mccubbins1984}, democratic legitimacy \citep{Croley2003, Rosenbloom2003}, and new technical information \citep{Yackee2006JPART, Nelson2012}. %\footnote{These various goals are evident in the Proposed Recommendation on Public Engagement in Rulemaking from the Administrative Conference of the United States, which asserts that ``The opportunity for public engagement is vital to the rulemaking process, permitting agencies to obtain more comprehensive information, enhance the legitimacy and accountability of their decisions, and enhance public support for their rules'' \citep{ACUS2018}.}
%Public comment periods are purported to simultaneously produce technical information, accountability to elected officials, and responsiveness to public demands.
While recent scholarship on agency policymaking has shed light on the sophisticated lobbying by businesses and political insiders, we know surprisingly little about the vast majority of public comments which are submitted by ordinary people as part of public pressure campaigns.\footnote{As I show elsewhere \citep{Judge-Lord2019}, most comments submitted to regulations.gov are form comments, more akin to petition signatures than sophisticated lobbying. Indeed, aproximately 40 million out of 50 million (80\%) of these public comments mobilized by just 100 advocacy organizations.}
Activists frequently target agency policymaking with letter-writing campaigns, petitions, protests, and mobilizing people to attend hearings, all classic examples of ``civic engagement'' \citep{Verba1987}. Yet civic engagement remains poorly understood in the context of bureaucratic policymaking.

These occasional bursts of civic engagement in bureaucratic policymaking raise practical and theoretical questions for the practice of democracy.\footnote{In 2018, the Administrative Conference of the United States (ACUS) identified mass commenting as a top issue in administrative law. In their report to ACUS, \citet{SantAmbrogio2018} conclude, ``The `mass comments' occasionally submitted in great volume in highly salient rulemakings are one of the more vexing challenges facing agencies in recent years. Mass comments are typically the result of orchestrated campaigns by advocacy groups to persuade members or other like-minded individuals to express support for or opposition to an agency's proposed rule.'' 
Mass comment campaigns are known to drive significant participation of ordinary people in Environmental Protection Agency rulemaking \citep{Judge-Lord2019, Potter2017, Balla2018}. \citet{Cuellar2005}, who examines public input on three rules, finds that ordinary people made up the majority of commenters demonstrating ``demand among the mass public for a seat at the table in the regulatory process.'' } 
%To date, administrative law scholars have focused on practical and normative questions, much of this analysis depends 
These questions, in turn, hinge on unanswered empirical questions: Do these campaigns affect policy? If so, by what mechanisms? Existing research finds that commenters believe their comments matter \citep{Yackee2015JPART} and that the number of public comments varies across agencies and policy processes \citep{Judge-Lord2019, Libgober2018, Moore2017},
% and policy change is related to the number of comments in sample of nine rules \citep{Shapiro2008}, 
but the relationship between the scale of public engagment and policy change remains untested. 

To address this gap, I assess the relationship between the number of public comments and the amount of change between draft and final policy texts. Next, I assess the relationship between the number of people mobilized by each campaign and whether the campaign acheivied its policy goals. Finally, I theorize and test four mechanisms by which public input may affect bureaucratic policymaking. Each mechanism involves a distinct type of information that pressure campaigns may relay to policymakers: technical information, information about the likelihood of political consequences, information about the preferences of elected officials, or information about the preferences of the attentive public. Because scholarship on bureaucratic policymaking has focused on the power of technical information, where insider lobbying is most likely to matter and where outside strategies are least likely to matter, political scientists have largely overlooked mass mobilization as a tactic.

% To assess the relationship between mass engagement and policy change, I introduce a large new dataset of millions of public comments on agency rules and assess mass comment campaigns' impact on rulemaking processes and outcomes. % To assess the relationship between mass public participation and rule change, I outline four potential mechanisms by which public input may influence bureaucratic decisions. 
I find evidence consistent with the observable implications of mass comment campaigns influencing policymaking through [non-null results] but no evidence that mass engagement affects rulemaking processes or outcomes through [null results].




\hypertarget{methods}{%
\section{Methods}\label{methods}}

To illustrate how I create these variables, consider two such cases:

\hypertarget{waters-of-the-united-states}{%
\paragraph{Waters of the United States}\label{waters-of-the-united-states}}

In responce to litigation over which waters were protectected by the Clean Water Act, the Environmental Protection Agency and Army Corp of Engineers proposed a rule based on a legal theory articulated by Justice Kennedy, which was more expansive than Justice Scalia's.
An Attorney for the Natural Resources Defense Council submitted a 69-page highly technical comment ``on behalf of the Natural Resources Defense Council\ldots, the Sierra Club, the Conservation Law Foundation, the League of Conservation Voters, Clean Water Action, and Environment America'' supporting the proposed rule:

\begin{quote}
``we strongly support EPA's and the Corps' efforts to clarify which waters are protected by the Clean Water Act. We urge the agencies to strengthen the proposal and move quickly to finalize it\ldots{}''
\end{quote}

I coded this as support for the rule change, specifically not going far enough. I also coded it as requesting speedy publication. NRDC makes four substantive requests: one about retaining language in the proposed rule (``proposed protections for tributaries and adjacent waters\ldots{} must be included in the final rule'') and three proposed changes (``we describe three key aspects of the rule that must be strengthened'').\footnote{These three aspects are: (1) ``The Rule Should Categorically Protect Certain ``Other Waters'' including Vernal Pools, Pocosins, Sinkhole Wetlands, Rainwater Basin Wetlands, Sand Hills Wetlands, Playa Lakes, Interdunal Wetlands, Carolina and Delmarva Bays, and Other Coastal Plain Depressional Wetlands, and Prairie Potholes. Furthermore, ``Other `Isolated' Waters Substantially Affect Interstate Commerce and Should be Categorically Protected Under the Agencies' Commerce Clause Authority.'' (2) ``The Rule Should Not Exempt Ditches Without a Scientific Basis'' (3) ``The Rule Should Limit the Current Exemption for Waste Treatment Systems''} These demands provide specific key words and phrases to search the change in rule text.

A coalition of 15 environmental organizations mobilized over 944,000 comments, over half (518,963) were mobilized by the four above organizations: 2421,641 by Environment America, 108,076 by NRDC, 101,496 by clean water action, and 67,750 by the Sierra Club. Other coalition partners included EarthJustice (99,973 comments) and Organizing for Action (formerly president Obama's campaign organization, 69,369 comments). This case represents the upper end of the distribution. This coalition made sophisticated recomendations and mobilized a million people.

The final rule moved in the the direction requested by this coalition, but to a lesser extent than requested--what I code as ``some desired changs.''" As NRDC et al.~requested, the final rule retained the language protecting tributaries and adjacent waters and added some protections for ``other waters'' like prarie potholes and vernal pools, but EPA did not alter the exemptions for ditches and waste treatement systems.

For this coalition, the dependent variable, \emph{coalitions success} is 1, \emph{coalition size} is 15, \emph{business coalition} is 0, \emph{comment length} is 69/88, 0.78, and \emph{log comments} is log(943,931), 14.

\hypertarget{fine-particle-national-ambient-air-quality-standards}{%
\paragraph{2009 Fine Particle National Ambient Air Quality Standards}\label{fine-particle-national-ambient-air-quality-standards}}

In 2008, the EPA proposed a rule expanding air quality protections. Because measuring small particles of air pollution was once difficult, measurments of large particulates were allowed as a surragate measure for fine particles under EPA's 1977 PM10 Surrogate Policy. EPA proposed eliminating this policy, thus requring regulated entities and state regulators to measure and enforce limits on much finer particles of air pollution.

EPA received 163 comments on the rule, 129 from businesses, business associations such as the American Petrolium Institute and The Chamber of Commerce, and state regulators that opposed the rule, requesting that it be withdrawn or at least delayed. A few state regulators only requested delayed implimation of the rule until they next revised their State Implimentation Plans (SIPs) for Prevention of Significant Deteriouration (PSD). EarthJustice supported the rule but opposed the idea that the cost of measureing fine particles should be a consideration. On behalf of the Sierra Club, the Clean Air Task Force, EarthJustice commented ``We support EPA's proposal to get rid of the policy but reject the line of questioning as to the benefits and costs associated with ending a policy that is illegal.'' The EarthJustice-led coalition also opposed delaying implimentation ``EPA must immediately end any use of the Surrogate Policy -- either by ``grandfathered'' sources or sources in states with SIP‐approved PSD programs -- and may not consider whether some flexibility or transition is warranted by policy considerations."

The final rule did eliminate the Surrate Policy, but allowed states to delay implimentation and enforcement until the next sceduled revision of their Implimentation Plans. I code this as the EarthJustice coalition getting most of what they requested, but not a complete loss for the regulated coalition.

For the EarthJustice coalition, the dependent variable, \emph{coalitions success} is 1, \emph{coalition size} is 3, \emph{business coalition} is 0, \emph{comment length} is 7/85, 0.08, and \emph{log mass comments} is log(0), \ensuremath{-\infty{}}.

For the EarthJustice coalition, the dependent variable, \emph{coalitions success}, is -1, \emph{coalition size} is 129, \emph{business coalition} is 1, \emph{comment length} is and \emph{log mass comments}, \(x_i\), is log(0), \ensuremath{-\infty{}}.
% --- PAGE: endnotes -----------------------
% --- PAGE: refs -----------------------
\newpage
\singlespacing 
                    \renewcommand\refname{Pre-analysis} 
              \bibliography{mendeley.bib} 
  \end{document}
