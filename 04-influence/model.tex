\subsubsection*{Incorporating political information into formal models of rulemaking}

Formally, political information requires several crucial amendments to existing information-based models of rulemaking. In the most sophisticated model of notice-and-comment rulemaking to date, \citet{Libgober2018} posits a utility function for agency $G$ as $u_G(x_F) = \alpha_0 x_f^2 + \sum_{i=1}^N \alpha_i u_i (x_f)$ where $x_f$ is the spatial location of the final policy, $u_i$ is the preference of a member of the public or ``potential commenter'' $i$, and $\alpha$ is a vector of "allocational bias"---i.e. how much the agency cares about its own preferences $\alpha_0$ relative to accommodating the preferences of others $\alpha_{i=1:N}$. Bureaucrats balance their own idea of their mission against their desire to be responsive. In Libgober's model, $\alpha$ is a fixed ``taste'' for responsiveness to each member of society, so agency decisions simply depend on their answer to the question ``what do people want?'' Including political information requires two additional parameters related to a second question ``why would the agency care?''

First, like other lobbying strategies, going public may shift the strategic environment, leading an agency to shift its allocation in favor of some groups and away from others. Let this strategic shift be a vector $\alpha_s$. Second, campaigns may directly persuade agencies to adjust their allocational bias, for example by supporting claims about the number of people the group represents or the intensity or legitimacy of their policy demands. Let this direct shift be $\alpha_d$ and immutable taste now be $\alpha_t$. Having decomposed an agency's allocative bias into three parts (its fixed tastes, shifting strategic environment, and potential to be convinced), the agency's utility function is now  $u_G(x_F) =  (\alpha_{t0} + \alpha_{s0} + \alpha_{d0}) x_f^2 + \sum_{i=1}^N (\alpha_{ti} + \alpha_{si} + \alpha_{di}) u_i (x_f)$. If, after the comment period, an agency's strategic environment is unchanged and it remains unpersuaded about which segments of society deserve favor, $\alpha_s$ and $\alpha_d$ are 0, and the model collapses to the original information game based on fixed taste. This less plausible when groups go public and expand the scope of conflict. 

Incorperating political information allows us to begin formalizing intuitions about mechanisms of influence. For example, \citet{Libgober2018} asks ``What proportion of commenting activity can be characterized as informing regulators about public preferences versus attempting to attract attention of other political principals?'' (p. 29). Adding political information to the model allows us to formalize this question: Under what conditions does the decision to comment depend on an organization's beliefs about $\alpha_t$ versus beliefs about $\alpha_s$? %Because they are substitutes in the model, this may be hard to say theoretically, but e
Empirically, we may often be able to infer that the difference in commenting can be attributed to group $i$'s beliefs about $\alpha_{si}$ if the behavior of political principals varies but other observed parameter values are similar across rules at a given agency.

Rational-choice explanations of why organizations comment on proposed rules build on an intuition that potential commenters will comment only when the benefits exceed the costs of doing so. This intuition ought to apply to other lobbying strategies such as mass mobilization as well. Adding an additional lobbying strategy into the model described above is straightforward. In Libgober's model, a potential commenter has negative quadratic preferences centered on their ideal policy $p_i$ and $u_i = -(x_f - p_i)^2$ where $x_f$ is the final policy chosen by the agency. An organization will comment if the cost of doing so is less than the difference between their utility when the agency selects a policy having been informed about the organization's ideal point $p_i$ versus when the agency selects a policy having made a guess about the organization's ideal point, $z_i$. If $c_i$ is organization $i$'s cost of commenting, then $i$ will comment if it expects to be better off providing information than abstaining $E[u_i | p_i] > E[u_i | z_i] + c_i$. Similarly, an organization will go public when it expects that the cost of running a mass mobilization campaign to be less than the difference in utility when the agency selects a policy having been informed about the intensity of broader public preferences $p_{public}$ versus when the agency selects a policy having made a guess about the intensity of the attentive public's preferences, $z_{public}$. If $c_{campaign, i}$ is organization $i$'s cost of running a mass mobilization campaign, then $i$ will launch a campaign if $E[u_i | p_{public}] > E[u_i | z_{public}] + c_{campaign, i}$. This suggests that mass mobilization with the aim of influencing policy, i.e. going public, should be more common when agencies are either poorly informed or distant from public opinion and potentially influenced by the types of political information created by mass engagement. 

Additionally, an organization may comment or run a mass mobilization campaign if it benefits in ways that are independent of policy outcomes. Strategies such as "going down fighting" can be incorporated by adding exogenous benefit parameters to the utility function of the potential commenter/mobilizer. Let $v_i$ be the benefit of commenting independent of its effect on the policy outcome, such as pleasing members or to reserving the right to sue. Let $w_i$ be the benefit of running a mass mobilization campaign independent of its effect on the outcome of the policy at hand, such as fulfilling expectations of existing members or recruiting new members. An organizations utility function would then be $u_i = -(x - p_i)^2 + v_i + w_i$. 

Adding these parameters also resolves a puzzling result of Libgober's model. Empirically, rules that receive comments do not always change. This result is impossible in a model where bureaucrats only have known fixed tastes and potential commenters only seek changes in policy. For policy seeking organizations to lobby but fail to influence policy requires that they may either be wrong about an agency's allocative bias or their ability to shift it. Incorporating political information allows change and uncertainty in an agency's biases. 
% While it is possible that commenters greatly misestimate an agency's durable allocative bias towards them (i.e. the agencies taste for them), it is more plausible that they misestimate their ability to influence the agency or the agency's strategic environment in a particular case. 
% Indeed, if taste is the only kind of allocative bias, then the only thing that can explain variance in participation is variation in the location of the proposed rule. The best empirical methods to estimate policy location generally assume that the spatial location of proposed rules written by the same people will be the same. Yet a potential commenter's anticipated ability to affect the agency's strategic environment or beliefs may be likely to vary significantly from rule to rule. It may vary with the level or distribution of economic impact, salience, sympathetic affected populations, timing (e.g. related to elections), recent court victories, proximate legal precedent, or any number of correlates of political context and ammunition for persuasion.
The result of commenting without rule change also becomes possible if commenters are allowed a strategy of ``going down fighting'' and incentives to do so. 

