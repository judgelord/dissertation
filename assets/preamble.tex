\pagestyle{headings}
\date{\today} 

 %\setlength{\parskip}{1em}
% \usepackage[round, sort, comma]{natbib}
% \setcitestyle{notesep={, },round,aysep={},yysep={,}}

%\setlength{\parindent}{0pt}

% \usepackage[authordate,strict,backend=bibtex8,babel=other, doi=false, url=false, isbn=false, uniquename=false, maxcitenames = 2, uniquelist=false,%
% bibencoding=inputenc, natbib]{biblatex-chicago}
%\usepackage[authordate, backend=biber, doi=false, url=false, isbn=false, uniquename=false, maxcitenames = 2, uniquelist=false, natbib]{biblatex-chicago}



% ---------------------------------------
%   REFERENCES
% ---------------------------------------
\usepackage{natbib} %[round, sort, comma]{natbib}
  \bibpunct[: ]{(}{)}{;}{a}{}{,}
%\usepackage[nottoc,numbib]{tocbibind} % numbers bib. in table of contents


% ---------------------------------------
%   MARGINS AND SPACING
% ---------------------------------------
% \usepackage[margin=1in]{geometry} % This is done by markdown
\usepackage{setspace} % line spacing


% ---------------------------------------
%   TABLES AND FIGURES
% ---------------------------------------
\usepackage{graphicx} % input graphics
\graphicspath{ {./Figs/} }

\usepackage{float} % float parameters
\usepackage{placeins} % \FloatBarrier: prevent floats spilling across sections
\usepackage{subcaption} % subfloats with individual captions
\usepackage{multirow} % multicolumn and multirow
\usepackage{booktabs} %? for toprule, midrule etc
\usepackage{dcolumn} % decimal-aligned columns

\usepackage{tikz}
\usetikzlibrary{positioning}
\usetikzlibrary{shapes.geometric}
\usetikzlibrary[patterns]

% ---------------------------------------
%   MATH
% ---------------------------------------
\usepackage{amsmath}
%  Math symbols (depends on whether you customize typeface)
%  \usepackage{amsfonts}
%  \usepackage{mathrsfs}
\usepackage{amssymb}
  %  \newcommand{\E}{\mathrm{E}}
  %  \newcommand{\Var}{\mathrm{Var}}
  %  \newcommand{\Cov}{\mathrm{Cov}}
  %  \newcommand{\plim}{\mathrm{plim}}
  %  \renewcommand{\L}{\mathcal{L}}
  %  \renewcommand{\d}{\mathrm{d}}
  %  \newcommand{\R}{\mathbb{R}}

% ---------------------------------------
%   TYPEFACE AND TEXT STYLES
% ---------------------------------------

% Hypotheses
\usepackage{ntheorem}
\theoremseparator{:}
\newtheorem{hyp}{Hypothesis}

\makeatletter
\newcounter{subhyp} 
\let\savedc@hyp\c@hyp
\newenvironment{subhyp}
 {%
  \setcounter{subhyp}{0}%
  \stepcounter{hyp}%
  \edef\saved@hyp{\thehyp}% Save the current value of hyp
  \let\c@hyp\c@subhyp     % Now hyp is subhyp
  \renewcommand{\thehyp}{\saved@hyp\alph{hyp}}%
 }
 {}
\newcommand{\normhyp}{%
  \let\c@hyp\savedc@hyp % revert to the old one
  \renewcommand\thehyp{\arabic{hyp}}%
} 
\makeatother

% \usepackage{fontenc} 
% advisable to include for nitty-gritty details
% (ligatures, kerning & other typographical things)

% \usepackage[usenames, dvipsnames]{xcolor} % extra colors
\usepackage{hyperref} % hyperlinks
   \hypersetup{
     colorlinks = true, 
     citecolor = black, 
     linkcolor = blue,
     urlcolor = blue}

\usepackage{comment} % provides {comment} environment
\usepackage{enumitem} % allows [nosep] option for lists

\usepackage{endnotes}

\makeatletter
\renewcommand\@makeenmark{%
  \textsuperscript{\normalfont\textcolor{blue}{\@theenmark}}%
}
\newcommand{\uncolormarkers}{%
  \renewcommand\@makeenmark{%
    \textsuperscript{\normalfont\@theenmark}%
  }%
}
\makeatother


\newcommand{\exclude}[1]{\StopSearching ##1\StartSearching}

\usepackage{ragged2e}
