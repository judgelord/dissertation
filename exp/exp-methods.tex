

I hope to leverage small random manipulations in, for example, the specific policy provisions targeted by activist campaigns to establish the plausibility that the effects I observe are causal and to assess causal mechanisms. 

\subsection{Outcome Measures}

\subsubsection{Goals Achieved}

%The primary outcome is whether the organization achieved its goals. As priors and goals are stated for both the treatment and control rules in the survey that each organization will be asked to fill out, I can assess the difference between treated and untreated rules and whether the number of comments (dosage) is associated with the number of goals achieved. 

\subsubsection{Attention and issue framing}
%I will use text analysis to measure three outcomes: First, did the agency respond to the concerns raised in the comments? Comparability requires that the mobilizing organization either submit their own comment (which often more technical than mobilized comments) on both treatment and control rules or submit on neither. As agencies are only required to respond to comments that they deem ``substantive,''  If the mobilizing organization submitted their own comment, response to mobilized comments may be measured as the difference in the length of response in treated and untreated rules. 

Second, comparing the preamble of the final rule to that of the draft, did the agency use more of the words and phrases matching those in the comments? Rule preambles are non-binding texts that frame the aims of the rule and the response to comments.  If the preambles of treated rules are more likely to include words or phrases from the organizations messaging and mobilized comments, this may indicate agency attention to their demands.\footnote{Recall that  treated and untreated rules both receive at least one comment with the organization's messaging but only treated rules receive a high dosage of it.}

Third, did the agency adopt language from comments into the text of the policy itself? It is unlikely that language from citizen comments will be sufficiently technical to be adopted into a rule. However, a higher dosage of public comments may make the more technical and legalistic comments of associated organizations more influential.

Each of these could be dummy variables or continuous measures based on the number of words in the agency response and number of new words and length of new phrases adopted from the comments into the preamble or rule. 
\noindent
Existing data on federal rulemaking are rich, but may not be able to identify whether a campaign influenced a rule or a pending change to a rule inspired the campaign.
For example, we may want to know whether public comments affect compliance with E.O.12898 on addressing Environmental Justice issues. As shown in initial correlational analysis in section \ref{ej}
% Figure \ref{ejlogitagencies} (Left) shows that %presents results from a logit model, showing the probability that a final rule will include the phrase ``environmental justice'' when the proposed rule did not but comments did. 
the phrase ``environmental justice" being raised in comments is correlated with it being added to a rule. This correlation is especially clear for the EPA, where a comment mentioning ``environmental justice'' is associated with a 10\% increase in the probability that it will be added in the final rule. However, this may occur for many reasons. Figure \ref{ejlogitagencies} (Right) suggests that this change is not correlated with the number of comments received (at least in this simple model). %Furthermore, I have yet to evidence that proposed rules with many comments were more likely to add ``environmental justice'' to the final rule.

Gathering data in real time and testing different strategies may help disentangle how campaigns shape the rulemaking environment from how aspects of a rulemaking may shape campaigns. For example, did campaigns inspire increased attention to environmental justice? Or did agencies' intentions to address E.O.12898 in their final rules inspire commenters to mention it? %If whether environmental justice concerns were raised in comments or not could be experimentally manipulated, this could offer evidence about the causal effect.
