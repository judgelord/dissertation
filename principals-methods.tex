
% MEASUREMENT  2
To assess the hypothesis that mass engagement affects the engagement of political principals, I examine the relationship between mass commenting and the behavior of Members of Congress, while attempting to control for other reasons that Members of Congress may comment on a proposed rule. The bold arrow in figure \ref{fig:causal-principals-test} indicates the key relationship that I test in this step. I aim to test the relationship between mass public engagement and engagement from Members of Congress, who may receive information about public opinion from mass engagement.

\begin{figure}[h!]
    \centering
    \caption{Modeling the Relationship between Mass Engagement and Political Oversight}
    \label{fig:causal-principals-test}
\tiny
\begin{tikzpicture}[%
    node distance=1.2cm,
    auto,
    text width=1.5cm,
dnode/.style={diamond, align=center, aspect=2, fill=green!5,draw=green!60, very thick, minimum size=2cm},
squarednode/.style={rectangle, align=center, aspect=1, draw=red!60, fill=red!5, very thick, minimum size=1cm},
pnode/.style={ellipse, align=center, aspect=1, draw=black!60, fill=black!5, very thick, minimum size=1cm},
title/.style={rectangle, align=center, aspect=1, minimum size=2cm}
]

% principal nodes 
\node[pnode]        (principaldemands) {Principal Demands};

\node[dnode]      (principaldecides) [right=of principaldemands] {Principal Comments};


% public Nodes
\node[pnode]        (publicdemands) [above=of principaldemands] {Latent Public Demands};

\node[dnode]      (publicdecides) [right=of publicdemands] {Mass\\ Engagement};

% political info
% \node[text centered]      (mobilization) [above=of publicdecides] {Mass\\Mobilization};

% \node[text centered]      (mobilization2) [right=of mobilization] {};

% \node[rectangle, minimum width =3cm, minimum height = 6.8cm, draw=red!60, fill=red!5, very thick]      (politicalinfo) [below=of mobilization2] {};

% \node[text centered]      (politicalinfotext) [below=of mobilization2] {Political Information};

% info
\node[squarednode]      (publicinfo) [right=of publicdecides] {Perceived Public Opinion};
% \node[squarednode]      (principalinfo) [right=of principaldecides] {Perceived Political Consequences};
% \node[squarednode]      (principalinfo2) [below=of principalinfo] {Perceived Principal Opinion};


% \draw[->] (mobilization.south) -- (publicdecides.north);


% public Lines
% \draw[->] (draft.east) -- (publicdemands.west);
\draw[->] (publicdemands.east) -- (publicdecides.west);
\draw[-, line width=2] (publicdecides.east) -- (publicinfo.west);




% principal Lines
% \draw[->] (draft.east) -- (principaldemands.west);
\draw[->] (principaldemands.east) -- (principaldecides.west);
\draw[->, line width=2] (publicinfo.south west) -- (principaldecides.north east);
% \draw[->] (principaldecides.east) -- (principalinfo.west);
% \draw[->] (principaldecides.south east) -- (principalinfo2.north west);
\draw[->] (publicdemands.south east) -- (principaldecides.north west);

% policy 
% \node[dnode]      (policy)       [right=of principalinfo] {Policy Response};
% \draw[->] (politicalinfo.east) -- (policy.west);

\end{tikzpicture}
\end{figure}
\normalsize

I measure the dependent variables, legislator attention and support, several ways. First, I count the number of times Members of Congress engage the agency across rules and before, during, and after comment periods on rules where lobbying organizations did and did not go public. By engaging the agency, I mean that Members of Congress raise a rule in % hearings, committee reports, or 
personal correspondence or comments that members send to the agency.
I then code each contact from the member of congress and each coalition lobbying on the rule on the same three-point scale: are they asking for the rule to go further, be scaled back, or published as is. This is similar to other hand-coding approaches to policy demands.
Next, I use text analysis to compare the sentiment and rhetoric (phrases and word frequencies) used by legislators to that used by each coalition. % Members of Congress may raise agency rules in oversight hearings, reports attached to each agency's budget appropriation, and in personal letters addressed to agency officials. %Using text reuse methods I identify when policy issues raised in draft rules and rule comments attract positive or negative attention from legislators.

%Using texts has several major advantages over other measures of congressional attention and sentiment such as partisanship \citep{Yaver2016, Lewis2008}, changes in budget size, or the length of appropriations reports \citep{Bolton2015}. Unlike partisanship, it is issue-specific and does not require assumptions about agency partisanship. While budget changes may reflect real costs, the many reasons that budgets change make it difficult to attribute changes to particular agency actions. The length of appropriations subcommittee reports may indicate the amount of attention committees pay to an agency but they do not vary significantly over time and do not indicate whether committee attention is positive or negative.

% Similarly, I asses the involvement of presidential appointees and the President's Office of Management and Budget before and after public comment, again comparing rules that were and were not targeted by a campaign (a difference-in-difference). 
% As a validity check, I also look for remarks by elected officials and judges on the level of public engagement.\footnote{
% I expect courts to be more likely to cite the procedural legitimacy of notice comment rulemaking when ruling in favor of public interest group that went public, and less likely to do so when ruling against them, compared to cases where rules received few comments. For example, citing the procedural legitimacy of rulemaking in Vermont Yankee v. NRDC (1978), Justice Rehnquist noted ``More than 40 individuals and organizations representing a wide variety of interests submitted written comments.'' I have collected data, including mentions of public comments, on all Supreme Court cases reviewing agency rules since 1984 and will do the same for a sample of D.C. circuit cases. While I focus on elected officials because they are more likely to respond to mass engagement, courts are also important political principals who explicitly review the legitimacy of rulemaking processes.


\subsubsection{Models testing the relationship between mass engagement and oversight}
There are several ways to test for a relationship between mass engagement and engagement by Members of Congress. The key explanatory variables of interest are the measures of mass engagement created in step 1 (how many and what types of comments). For simplicity, in the equations below, I only include measures related to the number of comments. Thus $\beta_0$ is the estimate for a rule with no public comments.
%\textbf{Dependent variables}

In Model 1, the dependent variable is the total number of comments from Members of Congress on the rule: \textit{$Y$ = Total comments from Congress} $\sim$ zero-inflated negative binomial, with one observation per rule. Let $x$ be the the total number of public comments. 

\begin{align}
    Y \sim \beta_0 + \beta_1x
\end{align}

In Model 2, the dependent variable is the number of comments from Members of Congress on the rule that support the coalition or organizations in the coalition: $Y_i$ = \textit{Total comments from Congress supporting coalition $i$} $\sim$ negative binomial, with one observation per coalition per rule. Let $x_i$ be the total number of public comments supporting coalition $i$.

\begin{align}
    Y_i \sim \beta_0 + \beta_1 x_i
\end{align}

In Model 3, the dependent variable is the share of Congressional comments supporting the coalition: $Y_i$ = \textit{Share of Members of Congress supporting coalition $i$} $\sim$  beta, with one observation per coalition per rule. Let $z_i$ be the \textit{share} of public comments supporting coalition $i$.

\begin{align}
    Y_i \sim \beta_0 + \beta_1 z_i
\end{align}

In Model 4, the dependent variable is rhetorical similarity between comments of each coalition $i$ and each Members of Congress $j$: $Y_{ij}$ = \textit{Text similarity score between legislator comment and coalition $i$ texts}, with one observation per legislator comment per coalition. Let $z_i$ be the share of public comments supporting coalition $i$.

\begin{align}
    Y_{ij} ~ \sim \beta_0 + \beta_1 z_i
\end{align}


\paragraph{Limitations.} One challenge will be controlling for rule salience, which may affect both public and legislator attention (indeed, both are endogenous to rule salience). Another challenge will be controlling for latent public opinion, which may usually, but not exclusively, be revealed to legislators through mass engagement.% In addition to cross-sectional analysis, I use a difference-in-difference design within members on rules where groups do and do not go public.

%I examine the relationship between mass engagement and another key variable in agency decisions, political oversight. % other key features of agencies' decisionmaking environments. 
% Do mass comment campaigns indicate that elected officials will be more involved in a rulemaking? 
% Do they indicate a greater chance of a rule being challenged or overturned in court?
% Dependent variables include political principals' attention, positions, and rhetoric, which I measure several ways across rules and within policy areas before and after mobilization campaigns.
% THEORY 2
% Accountability to Congress, the president, and courts have long been central concerns for bureaucracy scholars \citep{Wilson1989}. 
 % Elected officials, political appointees, and judges may also see it as their job to hold agencies accountable to the public will. On the other hand, elected officials often serve private interests,  such as campaign donors, especially when there is little risk of being held publicly accountable themselves.


% To assess the direct influence pathway, I estimate the extent to which mass mobilization around bureaucratic actions makes members of Congress or White House officials more likely to engage or react and whether such mass mobilizations is salient in subsequent litigation. 

 

% [President and Secretary]

% There are two ways to assess the courts as an indirect pathway where mobilization leads to influence. First, mass mobilization may increase the credibility of the threat of litigation. Second mass mobilization may influence the outcome of subsequent court cases over the rule. Both are difficult to measure. The first I measure with a combination of the litigation history of mobilizing groups and specific references to litigation in the comments. The second I assess by identifying instances where courts reference the number or direction of comments or other forms of protest in their decisions and compare rules to the rules under consideration in those cases. While this is rare, legal scholars have noted that " If the validity of a final regulation is challenged in court, the court's review will be based in significant part on how well the agency responded to the public's comments"  (Wagner 1995).