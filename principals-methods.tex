Measuring indirect influences}
To assess the direct influence pathway, I estimate the extent to which mass mobilization around bureaucratic actions makes members of congress or White House officials more likely to engage or react and whether such mass mobilizations is salient in subsequent litigation. 

Congressional attention to agency actions can be observed in several ways. Members of Congress who sit on oversight committees raise issues in oversight hearings, reports attached to each agency's budget appropriation, and in personal letters addressed to agency officials. Using text reuse methods I will identify when policy issues raised in draft rules and rule comments attract positive or negative attention from legislators. Using texts has several major advantages over previous measures of congressional attention and sentiment such as partisanship \citep{Yaver2016,Lewis2008}, changes in budget size, or the length of appropriations reports. Unlike partisanship, it is issue-specific and does not require assumptions about agency partisanship. While budget changes may reflect real costs, the many reasons that budgets change make it difficult to attribute changes to particular agency actions. The length of appropriations subcommittee reports may indicate the amount of attention committees pay to an agency but they do not vary significantly over time and do not indicate whether committee attention is positive or negative. 

[President and Secretary]

There are two ways to assess the courts as an indirect pathway where mobilization leads to influence. First, mass mobilization may increase the credibility of the threat of litigation. Second mass mobilization may influence the outcome of subsequent court cases over the rule. Both are difficult to measure. The first I measure with a combination of the litigation history of mobilizing groups and specific references to litigation in the comments. The second I assess by identifying instances where courts reference the number or direction of comments or other forms of protest in their decisions and compare rules to the rules under consideration in those cases. While this is rare, legal scholars have noted that " If the validity of a final regulation is challenged in court, the court's review will be based in significant part on how well the agency responded to the public's comments"  (Wagner 1995).