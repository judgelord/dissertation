\documentclass{article}
\pagestyle{headings}
\date{\today} 

% --- section style -----------------------

% we don't go deeper than subsection (hopefully..?)
\setsecheadstyle{\large \raggedright \bfseries}
\setsubsecheadstyle{\raggedright \bfseries}
\setsubsubsecheadstyle{\raggedright \itshape} 

% numbering depth
\setsecnumdepth{subsubsection}  

 \usepackage{booktabs}
 \usepackage{cleveref}

\renewcommand{\eqref}{\Cref}
\Crefformat{equation}{#2#1#3}

% \usepackage{float}
% \let\origfigure\figure
% \let\endorigfigure\endfigure
% \renewenvironment{figure}[1][2] {
%     \expandafter\origfigure\expandafter[H]
% } {
%     \endorigfigure
% }



% LINKS 
\usepackage{hyperref}
\hypersetup{breaklinks = true,
            bookmarks = true,  
            colorlinks = true,
            citecolor = black,
            urlcolor = black,
            linkcolor = magenta}

% FIG CAPTIONS 
\let\newfloat\relax 
\usepackage{floatrow}
\floatsetup[figure]{capposition=top}
\floatsetup[table]{capposition=top}
\usepackage{setspace}

% inherit spacing names from memoir
%\newcommand{\OnehalfSpacing}{\onehalfspacing}
%\newcommand{\DoubleSpacing}{\doublespacing}

% SPACING FOR BLOCK QUOTES 
\expandafter\def\expandafter\quote\expandafter{\quote\OnehalfSpacing}

% FOR KABLEEXTRA 
\usepackage{booktabs}
\usepackage{longtable}
\usepackage{array}
\usepackage{multirow}
\usepackage{wrapfig}
%\usepackage{float}
\usepackage{colortbl}
\usepackage{pdflscape}
\usepackage{tabu}
\usepackage{threeparttable}
\usepackage{threeparttablex}
\usepackage[normalem]{ulem}
\usepackage{makecell}
\usepackage{xcolor}
\usepackage{ragged2e}

\usepackage{endnotes}

\makeatletter
\renewcommand\@makeenmark{%
  \textsuperscript{\normalfont\textcolor{blue}{\@theenmark}}%
}
\newcommand{\uncolormarkers}{%
  \renewcommand\@makeenmark{%
    \textsuperscript{\normalfont\@theenmark}%
  }%
}
\makeatother


\newcommand{\exclude}[1]{\StopSearching ##1\StartSearching}

\usepackage{ragged2e}

% \usepackage{figcaps}
% \setlength{\parskip}{1em}
\title{Why Do Agencies (sometimes) Get So Much Mail? \\
Lobbying coalitions, mass comments, and political information in bureaucratic policymaking}

\author{Devin Judge-Lord \\ JudgeLord@Wisc.edu}

% FOR DAVE: %%%%%%%%%%%%%%%%%%%%%%%%%%%
% \let\footnote=\endnote
% \renewcommand{\footnotesize}{\normal}
%%%%%%%%%%%%%%%%%%%%%%%%%%%%%%%%%%%%%%%
\begin{document}

% FOR DAVE: %%%%%%%%%%%%%%%%%%%%%%%%%%
% \Large
% \RaggedRight
%%%%%%%%%%%%%%%%%%%%%%%%%%%%%%%%%%%%%%
\maketitle

\centering THIS DRAFT WAS PREPARED FOR SPSA 2019. THE MOST RECENT DRAFT IS \href{https://github.com/judgelord/dissertation/raw/master/whyMail.pdf}{HERE}.

\abstract{Scholars of bureaucratic policymaking have focused on the sophisticated lobbying efforts of powerful interest groups. Yet agencies occasionally receive thousands or even millions of comments from ordinary people. Why? Why do individuals engage when they seemingly have no new information to offer and no power to influence decisions? Who inspires them and to what end? How, if at all, should scholars incorporate mass participation into models of bureaucratic policymaking? 
I argue that mass mobilizing is an attractive and potentially influential tactic because produces political information about the coalition that mobilized it. I measure the scale and intensity of public support for proposed policies and examine alternative explanations that mass mobilization is (1) a conflict expansion tactic, where coalitions with fewer resources leverage public support, or (2) a more conventional lobbying tactic, where groups with superior resources leverage these resources to create an impression of public support. 
To link individual comments to the more sophisticated lobbying efforts they support, I use text reuse and clustering methods to identify formal and informal coalitions. I also classify different types of supporters. Using these new measures of political mobilization and engagement in agency rulemaking, I find that, in contrast to conventional insider lobbying, the vast majority of mass engagement in bureaucratic policymaking is mobilized by public interest group coalitions.
%I identify when mass comment campaigns occur and produce different types of politically-relevant information.}


\newpage
\tableofcontents

\newpage
\section*{Note to reader:}
To better understand the role of ordinary people in bureaucratic policymaking, 
I develop theories of why mass engagement occurs and how it may affect policy. To assess these theories, I tackle three related empirical questions: (1) Why does it occur?, (2) How does it affect the oversight behaviors of agencies' political principals?, and %These questions drive two initial empirical chapters.
(3)  Does mass engagement in bureaucratic policymaking affect policy?
% I then use my new measures of the political information that lobbying coalitions create by going public to test whether mass engagement explains variation in agency rulemaking and rules.% But first, I must develop a measure of ``going public.'' % and why it occurs.

\paragraph{Part 1. Why do agencies (occasionally) get so much mail?} %: Lobbying coalitions, mass comments, and political information in bureaucratic policymaking
% Scholars of bureaucratic policymaking have focused on the sophisticated lobbying efforts of powerful interest groups. Yet agencies occasionally receive thousands or even millions of comments from ordinary people. Why? Why do individuals comment when they seemingly have no new information to offer and no power to influence decisions? Who inspires them and to what end? How, if at all, should scholars incorporate mass commenting into models of bureaucratic policymaking? I argue that mass commenting produces political information about the coalition that mobilized it. 
% QUESTION 1\textbf{Puzzle:} 
Why do some rules receive many comments from ordinary people and some do not?
% Why do people comment on draft policies when they seem to have no new information to offer and no power to influence decisions? Who inspires them and to what end? 
% THEORY AND METHODS 1
Answering this question requires a theory explaining variation in mass engagement. Because the vast majority of comments are inspired by interest-group campaigns, finding their cause requires a method to link comments to the lobbying coalitions that mobilized them.  
To link individual comments to the more sophisticated lobbying efforts they support, I use text reuse and Bayesian classifiers to identify clusters of similar comments, reflecting formal and informal coalitions.
%Using new measures of public engagement in agency rulemaking, I identify the conditions under which it occurs and produces different politically-relevant information. 
% The dependent variable is the number of people engaged.
I theorize that lobbying coalitions' resources, opportunities, and public support explain variation in mass engagement %, which I measure in several ways. 
and that it will fit one of three patterns:
(1) Coalitions that perceive an opportunity to influence policy and have sufficient resources to do so will ``go public'' when they are disadvantaged in insider politics but have more public support than opposing coalitions. More public support yields more engagement, more effort per comment, and contagion beyond those mobilized directly. (2) Coalitions with less support may ``counter-mobilize'' with proportionally smaller effects. (3) Finally, coalitions may mobilize for reasons unrelated to the policy at hand, yielding similar mass engagement but with little sophisticated lobbying. 
Measures of mass engagement include 
%(1) total public comments, % $\sim$ zero-inflated negative binomial; 
(1) comments per coalition, % $\sim$ negative binomial; 
(2) effort per comment, % $\sim$ truncated normal; 
(3) share of comments per coalition mobilized indirectly (i.e. the potential for conflict spread).
Next, I test whether variation in engagement explains variation in oversight behavior (part 2) and policy outcomes (part 3).
% (4) type of campaign. % $\sim$ multinomial. 
%Model 1 is one observation per rule. Models 2-4 are one observation per coalition per rule. Explanatory variables include agency alignment with Congress and the president (models 1-4), coalition alignment and unity (models 2-4), whether a coalition is driven more by public or private interests (models 2-3).%, part of the DV in model 4).

%\paragraph{Step 2. Are elected officials more or less likely to engage after mass public engagement?} 
\paragraph{Part 2. Does mass engagement affect political oversight?} The political information signaled by mass engagement may serve as a ``fire alarm,'' altering principals to oversight opportunities or ``warning signs'' altering them to political risks.
When a coalition mobilizes successfully, %especially if it generates a perceived consensus in expressed public sentiments, 
elected officials ought to be more likely to engage on their behalf and less likely to engage against them.
% This suggests an addendum to Hall and Miler's (2008) finding that members are more likely to engage in rulemaking when they have been lobbied by a like-minded interest group.
% When interest groups lobby elected officials to engage in rulemaking, they may be more likely to engage when aligned with most commenters than when opposed.
% If politicians learn from political information, they will be even more likely to engage when lobbied by a coalition that includes a public interest group's with a large mass-comment campaign, and less likely when lobbied by a coalition dominated by private interests opposed by a mass comment campaign. 
% MEASUREMENT  2
To assess these hypotheses, I count the number of times Members of Congress engage the agency before, during, and after comment periods on rules where lobbying organizations did and did not go public. I then use text analysis to compare legislators' sentiments and rhetoric to that used by each coalition.
% Similarly, I asses the involvement of presidential appointees and the President's Office of Management and Budget before and after public comment, again comparing rules that were and were not targeted by a campaign (a difference-in-difference). 
% As a validity check, I also look for remarks by elected officials and judges on the level of public engagement.
Dependent variables include 
(1) the number of comments from Members of Congress on the rule %(total, those mentioning mass comments, and those mentioning organizations in the coalition), %All  $\sim$ zero-inflated negative binomial. 
(2) the share of supportive congressional comments, %  $\sim$  beta. 
(3) the similarity of words in comments from the coalition and Members of Congress. 

\paragraph{Part 3. Does mass engagement affect rulemaking and rules?} 
I theorize that the effects of political information on policy depend on the extent to which the strategic environment allows change and how political information is processed, both directly within agencies and indirectly through other actors (e.g. Members of Congress) whose appraisals matter to bureaucrats.
The main dependent variable is change in the rule text.
%Different inputs may yield different results: 
I systematically identify changes between draft and final rules, parse these differences to identify meaningful policy changes, and compare them to demands raised in comments to measure which coalition got their way. However, assessing policy change is difficult. Thus, I also use other measures of agency responses to lobbying efforts. 

\doublespace
% \section*{Summary}
% 
%\paragraph{Summary:} 
% This dissertation is about ordinary people's input on policies made by bureaucrats. 
% % People may believe that their voices matter, but it is unclear if they do. % or ought to. 
% I make three main contributions. First, drawing on scholarship on interest group behavior, social movements, and lobbying, I suggest three distinct reasons for groups to mobilize ordinary people. Each logic suggests a different pattern of mass engagement, and I analyze millions of public comments on thousands of agency rules to develop the first systematic measures of mass engagement in bureaucratic policymaking. Second, building on theories of political oversight, I theorize that mass public engagement in bureaucratic policymaking may alert elected officials to political opportunities and risks. I assess this argument by analyzing correspondence between Members of Congress and bureaucrats on proposed rules. Third, I integrate these contributions on interest group lobbying and oversight into a broader theory of the potential for mass mobilization to affect policy. I argue that there are four broad causal mechanisms by which lobbying may affect bureaucrats' decisions. Political scientists have thus far focused on the power of technical information where insider lobbying is most likely to matter and outside lobbying is least likely to matter and have thus largely overlooked mass engagement. This gap suggests that incorporating theories of social movement influence may advance bureaucratic politics scholarship and that bureaucratic politics may be a fruitful empirical ground for exploring social movement theories. I use my new measures of mass engagement to assess the effect of political information on bureaucratic policymaking. Finally, two chapters examine the four causal mechanisms I suggest through a case study of the environmental justice movement and a study of rules where organizations randomly select lobbying strategies.

%I theorize that mass engagement may, in limited circumstances, influence bureaucrats by shifting their incentives or evoking powerful norms. Using my new measures to assess these mechanisms, 
%I show how various parts of the U.S. government respond to public input.  %aims to understand the effects of public attention on executive-branch policymaking.

\paragraph{Motivation:} Leading models of influence in bureaucratic policymaking focus on two key political forces: sophisticated interest group lobbying and political oversight. 
As bureaucrats learn about policy problems and balance interest-group demands, public comment processes allow lobbying organizations to provide useful technical information and inform decisionmakers of their preferences on draft policies. 
Agencies may then update policy positions within constraints imposed by their political principals.

While this may describe most cases of bureaucratic policymaking, these models do not explain or account for the contentious politics that occasionally inspire millions of ordinary people to respond to calls for public input on draft agency policies. Mass engagement in bureaucratic policymaking has thus largely been ignored by political scientists, leaving a weak empirical base for normative and prescriptive work. 
Like other forms of mass political participation, such as protests and letter writing campaigns, 
mass public comments on draft agency rules provide no new technical information. %In this sense, they are not useful. 
Nor do they wield any formal authority to reward or sanction bureaucrats, as comments from a Members of Congress might. 
The number on each side, be it ten or ten million, has no legal import for an agency's response. 
% Policymakers may very well pay no attention to them. 
%Instead, scholars focus on the sophisticated lobbying efforts of powerful interest groups, whose role in shaping policy has been theoretically developed and empirically tested.
% If our models and conventional wisdom is correct, why do
Yet agencies occasionally receive input from a large number of ordinary people. %Why? 
How, if at all, should scholars incorporate mass engagement into models of bureaucratic policymaking? 

I argue that mass engagement results from interest groups' strategic choices. When lobbying organizations have an opportunity to shape policy, resources to mobilize, and broader support than their opposition, outside lobbying (``going public'') may produce valuable, politically-relevant information. Depending on how agencies process political information, outside lobbying may %be a plausible strategy for organizations to 
influence policy, both directly and indirectly.
For example, those lobbying in rulemaking often make suspect claims to represent broad segments of the public. Mobilizing a large number of people may directly support such claims.
Indirectly, it may alert elected officials to political risks and opportunities, affecting oversight behavior. % and thus shifting bureaucrats' relationships with their political principals. % It remains to be seen if conditions under which this is plausible ever occur and, if so, if mass engagement does indeed influence policy. 

Does mass engagement in bureaucratic policymaking affect policy? This question drives my project. However, two questions must be answered first: (1) Why does it occur? and (2) How does it affect the oversight behaviors of agencies' political principals? These questions drive two initial empirical chapters.
I then use my new measures of the political information that lobbying coalitions create by going public to test whether mass engagement explains variation agency rulemaking and rules.% But first, I must develop a measure of ``going public.'' % and why it occurs.

\paragraph{Step 1. Why do agencies (occasionally) get so much mail?} %: Lobbying coalitions, mass comments, and political information in bureaucratic policymaking
% Scholars of bureaucratic policymaking have focused on the sophisticated lobbying efforts of powerful interest groups. Yet agencies occasionally receive thousands or even millions of comments from ordinary people. Why? Why do individuals comment when they seemingly have no new information to offer and no power to influence decisions? Who inspires them and to what end? How, if at all, should scholars incorporate mass commenting into models of bureaucratic policymaking? I argue that mass commenting produces political information about the coalition that mobilized it. 
% QUESTION 1\textbf{Puzzle:} 
Why do people comment on draft policies when they seem to have no new information to offer and no power to influence decisions? Who inspires them and to what end? 
% THEORY AND METHODS 1
Answering these questions requires a theory explaining variation in mass engagement and method to link comments to the lobbying coalitions that mobilized them.  
To link individual comments to the more sophisticated lobbying efforts they support, I use text reuse and Bayesian classifiers to identify clusters of similar comments, reflecting formal and informal coalitions.
%Using new measures of public engagement in agency rulemaking, I identify the conditions under which it occurs and produces different politically-relevant information. 
% The dependent variable is the number of people engaged.
I argue that activists' resources, opportunities, and public support explain variation in mass engagement %, which I measure in several ways. 
and that it will fit one of three patterns:
(1) Coalitions will ``go public'' when they are disadvantaged in insider politics but have more support than opposing coalitions. More public support yields more engagement, more effort per comment, and contagion beyond those mobilized directly. (2) Coalitions with less support may ``counter-mobilize'' with smaller effects. (3) Finally, coalitions may mobilize for reasons unrelated to the policy at hand, yielding similar mass engagement but with little sophisticated lobbying. 
Measures of mass engagement include 
%(1) total public comments, % $\sim$ zero-inflated negative binomial; 
(1) comments per coalition, % $\sim$ negative binomial; 
(2) effort per comment, % $\sim$ truncated normal; 
(3) share of comments per coalition mobilized indirectly (i.e. the potential for conflict spread).
Next, I test whether variation in engagement explains variation in oversight behavior (step 2) and policy outcomes (step 3).
% (4) type of campaign. % $\sim$ multinomial. 
%Model 1 is one observation per rule. Models 2-4 are one observation per coalition per rule. Explanatory variables include agency alignment with Congress and the president (models 1-4), coalition alignment and unity (models 2-4), whether a coalition is driven more by public or private interests (models 2-3).%, part of the DV in model 4).

%\paragraph{Step 2. Are elected officials more or less likely to engage after mass public engagement?} 
\paragraph{Step 2. Does mass engagement affect political oversight?} The political information signaled by mass engagement may serve as a ``fire alarm,'' altering principals to oversight opportunities or a ``warning signs'' altering them to political risks.
When a coalition goes public, %especially if it generates a perceived consensus in expressed public sentiments, 
principals ought to be more likely to engage on their behalf and less likely to engage against them.
% This suggests an addendum to Hall and Miler's (2008) finding that members are more likely to engage in rulemaking when they have been lobbied by a like-minded interest group.
% When interest groups lobby elected officials to engage in rulemaking, they may be more likely to engage when aligned with most commenters than when opposed.
% If politicians learn from political information, they will be even more likely to engage when lobbied by a coalition that includes a public interest group's with a large mass-comment campaign, and less likely when lobbied by a coalition dominated by private interests opposed by a mass comment campaign. 
% MEASUREMENT  2
To assess these hypotheses, I count the number of times Members of Congress engage the agency before, during, and after comment periods on rules where lobbying organizations did and did not go public. I then use text analysis to compare legislators' sentiments and rhetoric to that used by each coalition.
% Similarly, I asses the involvement of presidential appointees and the President's Office of Management and Budget before and after public comment, again comparing rules that were and were not targeted by a campaign (a difference-in-difference). 
% As a validity check, I also look for remarks by elected officials and judges on the level of public engagement.
Dependent variables include 
(1) the number of comments from Members of Congress on the rule %(total, those mentioning mass comments, and those mentioning organizations in the coalition), %All  $\sim$ zero-inflated negative binomial. 
(2) the share of supportive congressional comments, %  $\sim$  beta. 
(3) the similarity of words in comments from the coalition and Members of Congress. 
% Models 1 and 2 are one observation per coalition per rule. Model 3 is one observation per comment from a Member of Congress. Explanatory variables of interest are the dependent variables from step 1 (how many and what types of comments--i.e. variation in political information).% In addition to cross-sectional analysis, I use a difference-in-difference design within members on rules where groups do and do not go public.

%I examine the relationship between mass engagement and another key variable in agency decisions, political oversight. % other key features of agencies' decisionmaking environments. 
% Do mass comment campaigns indicate that elected officials will be more involved in a rulemaking? 
% Do they indicate a greater chance of a rule being challenged or overturned in court?
% Dependent variables include political principals' attention, positions, and rhetoric, which I measure several ways across rules and within policy areas before and after mobilization campaigns.
% THEORY 2
% Accountability to Congress, the president, and courts have long been central concerns for bureaucracy scholars \citep{Wilson1989}. 
 % Elected officials, political appointees, and judges may also see it as their job to hold agencies accountable to the public will. On the other hand, elected officials often serve private interests,  such as campaign donors, especially when there is little risk of being held publicly accountable themselves.






% QUESTION 3 
% Are changes in rulemaking and rules more likely after mass engagement?
\paragraph{Step 3. Does mass engagement affect rulemaking and rules?} I theorize that the effects of political information on policy depend on the extent to which the strategic environment allows change and how political information is processed, both directly within agencies and indirectly through other actors (e.g. Members of Congress) whose appraisals matter to bureaucrats.
The main dependent variable is change in the rule text.
%Different inputs may yield different results: 
I will systemically identify changes between draft and final rules, parse these differences to identify meaningful policy changes, and compare them to demands raised in comments to measure which coalition got their way. However, assessing policy change is difficult. Thus, I will also use other measures of agency responses to lobbying efforts. %For example, agencies may speed up or delay finalizing rules. They write lengthy justifications of their decisions in response to some demands but not others. They may or may not extend the comment period.


% \paragraph{Step 4. Causal mechanisms:} How might mass engagement matter?
% A lobbying effort can generate new information, re-frame information, or reshape the political context of a decision. Agency staff may update their beliefs in response to new information or framing. Activists can also reshape agency policymakers’ strategic environment by drawing in or scaring off other actors, especially elected officials. 

% \textbf{Strategic calculations:} 
% New information may affect bureaucrats' decisions directly or indirectly. New scientific or legal information spurs revision of calculations about cost and benefits or the likelihood of being reversed in court. New political information spurs bureaucrats to update their beliefs about levels of support among segments of the public or their elected representatives and thus about the likely political consequences of a decision.
% % Reshaping strategic incentives may shift how rulewriters weigh commenter demands.

% % NORMATIVE FRAMING / INFO PROCESSING 
% \textbf{Information processing and normative evaluations:} 
% In addition to strategic calculations, mass engagement may shift how information is processed and evaluated, both institutionally and cognitively.
% Institutionally, higher comment volume may engage a larger and more politically-oriented set of staff and consultants. Cognitively, expanding the scope of conflict highlights the political aspects of a decision, perhaps mobilizing cognition focused more on norms of public service or partisan ideology than on strategic or technical rationality. In both cases, campaigns re-frame decisions as political and provide information that is especially relevant if processed through such a frame.
% The effects of political information on bureaucrats' normative evaluations may be
% direct---the weight that norms of direct democracy give to limited public input---or 
% indirect---the weight that norms of accountability give to elected officials' input.

% %\textbf{Indirect influence through elected officials:} 
% %Campaigns  do more than reveal latent political information; they mobilize both members of the public and elected officials to take positions on issues they may have never previously considered, thus creating new relevant political information for bureaucrats. 
% %Movements help to shape the political space in which they operate’ (Gamson and Meyer 1996, p. 289).
% %The result of thinking differently about a decision may be a shift in how the agency evaluates or weights commenter demands.

\textbf{Assessing causal mechanisms:} While it may be impossible to causally identify or attribute effects to normative or strategic mechanisms, 
a focus on political information suggests places to look for influence in rulemaking. For example, if Members of Congress are not more likely to voice support for a coalition that goes public, this would be evidence against any indirect mechanism via congressional oversight.

To supplement the core design above, the two additional chapters explore historical and experimental case studies. My historical case is the environmental justice movement, relying on all rules where ``environmental justice'' is raised in the comments and quantitative and qualitative assessment of agency responses. I find that responsiveness varies across agencies. My experimental cases will be rules selected by organizations that have agreed to randomly assign specific targets of their mass comment campaigns. In exchange, I will use the methods outlined above to estimate the effects of their interventions. 

% \paragraph{Conclusion:} This research will add to our understanding of how bureaucratic policymaking fits with the practice of democracy.
% If input solicited from ordinary people has little effect on policy outcomes, directly or indirectly, it may be best understood as providing a veneer of democratic legitimacy on an essentially technocratic and/or elite-driven process.
% If public input does shape agency decisions, a new research program will be needed to investigate who exactly these campaigns mobilize and represent.

% \end{document} % NO SOURCES BEFORE THIS POINT

\newpage
\section{Introduction} \label{intro}
\section{Introduction}
\subsection{Why study rulemaking?}
% \section{The Importance of Studying Rulemaking}
% Mobilization may increasingly target rulemaking because it is how most policy in the U.S. is now made. 
With the rise of the administrative state in the United States, federal agencies have become a major site of policymaking and political contestation. In the years or decades between legislative enactments, federal agencies make legally-binding rules interpreting and reinterpreting old statutes to address emerging issues and priorities. Ninety percent of new policy that carries the force of law is now made in the bureaucracy rather than in Congress \citep{West2013WhoControl}.\footnote{I use policy, law, and regulation as nested concepts. My methods generally apply to all policy texts whether they carry the force of law or not. Many public and private organizations, including agencies, have policy statements that are not legally binding. My empirical subject is rules that do carry the force of law based on some authorizing legislation. I use rule (a more technical term) and regulation (a more colloquial term) interchangeably.}
Examples are striking: %the effect of the Dodd-Frank Wall Street Reform and Consumer Protection Act was largely unknown until the specific regulations were written, and it continues to change as these rules are revised. 
Congress authorizes billions in farm subsidies and leases for public lands, but who gets them depends on agency policy. In the decades since the last major environmental legislation, agencies have written thousands of pages of new environmental regulations and thousands more changing tack under each new administration. This constant revision of administrative rules makes them distinct from legislation \citep{Wagner2017DynamicRulemaking}.
% And these revisions can be significant. In 2006, citing the authority of statutes last amended in the 1950s, the Justice Department's Bureau of Prisons proposed a rule restricting eligibility for parole. In 2016, the Bureau withdrew this rule and announced it would be requiring fewer contracts with private prison companies, precipitating a 50\% loss of industry stock value. Six months later, a new attorney general announced these policies would again be reversed, leading to a 130\% increase in industry stock value. %Like many rulemaking debates, industry and advocacy groups spent millions of dollars lobbying on this issue. Few rulemakings, however, receive this level of public and presidential attention. In the majority of rulemakings, few participate, and we do not really know the extent to which participants get what they lobby for.% (but see Yackee and Yackee 2006)
Rulemaking clearly matters.

Less clear, however, is what the new centrality of agencies and rulemaking means for the practice of American democracy. In addition to the complex relationships agencies have with the president and Congress, agencies have complex and poorly understood relationships with the public and advocacy groups. Relationships with constituents may even provide agencies a degree of ``autonomy'' from their official principals \citep{Carpenter2001}. While some suggest that requirements for agencies to solicit and respond to public comments on proposed rules allows ``civil society'' to provide public oversight, others note that participants in rulemaking often represent elite parts of society \citep{Seifter2016ComplementaryPower} and business interests \citep{Yackee2006a}. Yet agency decisions are also the target of protests and advocacy campaigns.\footnote{For example, along with 50 thousand protesters in Washington D.C., the State Department Received 1.2 million comments on the Environmental Impact Statement for the Keystone Pipeline. Similarly, along with the thousands of protesters supporting the Standing Rock Sioux protest to the Dakota Access Pipeline, the Army Corps of Engineers received hundred of thousands of comments. Along with 22 million comments on the Federal Communications Commission's Open Internet rules, activists are organizing online protest actions. On each of these issues, advocacy activity has been followed by legislative or executive action.} and the notice-and-comment process purports to be an avenue of citizen voice. Big red letters across the top of the Regulations.gov homepage solicit visitors to ``Make a difference. Submit your comments and let your voice be heard.'' A blue "Comment Now!" button accompanies a short description of each draft policy and pending agency action. While most rules receive little attention, the ease of online commenting and mobilizing has created exponential increases in the number of rules where which hundreds of thousands of citizens participate (see figure \ref{fig:comments}). Occasionally, large numbers of citizens are paying attention.

\begin{figure}[!hb]
\caption{Number of Public Comments Total (Left) and Under 1 Million (Right). The most commented on rules have been published by the Federal Communications Commission (FCC, omitted from this plot), the Environmental Protection Agency (EPA), the Department of Interior (DOI), the Bureau of Ocean Energy Management (BOEM), the Consumer Financial Protection Bureau (CFPB), and Fish and Wildlife Service (FWS).}
\includegraphics[width= 3.5in]{number_of_comments.pdf}
\includegraphics[width = 3.5in]{comments_under_1m.pdf}
\label{fig:comments}
\end{figure}

It is even less clear whether actions by average citizens make a difference in agency policymaking. Many may believe that they do, but the mechanisms are not obvious. Indeed letter writing and other forms of mass mobilization do not have a clear place in political scientists' theories of bureaucratic politics. This lack of scholarship may be the result of both a general suspicion, rooted in certain theories of strategic behavior, that mass politics affects unelected career officials as well as a normative assumption that policy ``implementation'' is no place for contentious politics. Neither the bureaucrat who asserts that rules are the result of scientific analysis nor the political scientist who asserts that rules are the result of bureaucrats strategically selecting their most preferred policy within institutional constraints offer an explanation for why an agency would receive millions of public comments or why they would matter.

In this 
dissertation,
% paper,
I argue that if we appreciate agency policymaking as a site of contentious politics, mechanisms emerge by which mass mobilization may affect both the strategic environment and ideological perspectives of those who write agency rules. While the theory that I assemble attempts to describe the relationship between mass mobilization and agency decisionmaking in general, my empirical focus is on the role of  organized campaigns targeting notice-and-comment rulemaking processes, with special attention to environmental 
%and financial 
regulation.



 \subsection{Puzzle: Why mobilize?}
% \section{Why mobilize?}

Prior to the 22 million public comments on the Federal Communications Commission's 2017 Open Internet rule,\footnote{It is yet unclear how many of these comments are from real people.} two of the most commented on rules set standards for mercury emissions from coal and oil-fired power plants. Among other things, the Environmental Protection Agency (EPA) solicited ``comments on whether there would be a basis for considering area sources to be significantly different from major sources,'' ``on the adequacy of the restrictions associated with bypass conditions regarding maintaining LEE status" and ``on the proposed revisions concerning [equations' 1a and 1b] usefulness in calculating the maximum potential emissions rate from an emissions averaging group'' (EPA 2011). LEE status is not defined in the notice soliciting comments, and equations 1a and 1b are surely inaccessible to most citizens. Yet these two proposed rules received 942,483 comments. 

One comment, from the United States Council of Catholic Bishops, read: ``While we are not experts on air pollution, our general support for a national standard to reduce hazardous air pollution from power plants is guided by Catholic teaching, which calls us to care for God’s creation and protect the common good and the life and dignity of human persons, especially the poor and vulnerable.'' Bishops are not known to closely follow power plant regulations. Their moral authority was mobilized by activists who wanted stricter regulation of mercury. Groups mobilizing on the mercury rules including environmental and health groups and industry competitors, including the owners of Nuclear, Natural, geothermal power plants. 

In the official, legally-required response to comments, the EPA did not discuss God's creation, dignity, or the poor.  Indeed, the EPA asserted that mercury levels are a matter of science, not not a matter of justice. But the EPA did implicitly assert a definition of the public good when it used studies of mercury's aggregate public health effects on the U.S. population to set emissions standard.\footnote{As Wagner (1995) %CITE
notes, ``agencies exaggerate the contributions made by science...in order to avoid accountability for the underlying policy decisions. Although camouflaging controversial policy decisions as science assists the agency in evading various political, legal, and institutional forces, doing so ultimately delays and distorts the standard-setting mission'' (p. 1617). She goes on to say that  ``While the APA mandates a process for public involvement, it provides almost no protections to ensure that agencies will explain the substantive bases for highly complex or technical rulemakings in a way that the lay public can readily understand and challenge'' (1656) and that ``Mischaracterization of the entire standard-setting endeavor as resolvable by science results in significant obstacles to democratic participation'' (1674). Similarly, Harvey Brooks (1984) notes that ``The modern nation risks being no longer recognizable as a democracy, either representative or plebiscitary, if more and more policy areas are excluded from public participation because of the technical complexity.''} 
Then, as required by the Supreme Court, it justified the same standards with cost-benefit analysis in a revised proposed rule, concluding that for every dollar spent to comply with the regulation, the U.S. public receives up to nine dollars in health benefits (EPA 2007). If this is how decisions are made, why did the EPA receive nearly a million letters? Why would citizen opinions matter? 

% Wagner: A variety of commentators have suggested that agencies may seek increased legitimacy or decreased political accountability by disguising their policy judgments as science. See Majone, supra note 18, at 15 ("Traditionally, government regulators have sought legitimacy for their decisions by wrapping them in a cloak of scientific respectability.");Roberts et al., supra note 26, at 120 ("Too many of the participants [in science-policy decisions] have good reasons not to distinguish scientific evidence from policy preferences, not to analyze carefully the various sources of technical disagreement, and not to accept responsibility for some decisions or judgments."). Beyond these common sense observations scattered at points in articles and books, there has been surprisingly little scholarly discussion of the comprehensive existence of or reasons for a science charade in regulation.

In contrast to the science-based objectivity presented by the EPA, political scientists, building on law and economics scholarship, offer a different theory of bureaucratic decisionmaking rooted in the policy preferences and strategic behavior of agency leaders and their political principals: Congress, the president, and the courts. They find that political principals do constrain agency action but also leave room for agencies to move policy toward their own ``ideal point.''\footnote{Though political scientists make diverse assumptions about what this ideal is and how to measure it.} Science may or may not inform preferences, but preferences and the power to realize them in a strategic environment are, these scholars say, are the proximate cause of policy. These scholars would see the Mercury Rules as the result of EPA officials writing a policy as close as possible to their ideal policy given their strategic constraints. 

But if the strategic model is correct, why write letters to the EPA? The EPA administrator has their preferences and the public has no direct power over their decisions. Why not write to the president or members of Congress who influence EPA's strategic calculations and are more directly accountable to public opinion? 

Other political scientists, along with scholars of public administration, organizational behavior, and sociology offer alternative theories that, while less parsimonious, squarely address how the process of soliciting and responding to public comments may influence agency policy. They find that agency staff develop relationships with those who regularly participate in policy processes: most often businesses but also professional associations and activist organizations. Relationships draw on and reproduce organizational identities and reputations. This scholarship has revealed a good deal about how organized groups lobby agencies and why they succeed, but it has yet to address why these groups sometimes mobilize thousands of citizens to write letters or protest agency decisions. The theoretical foundations for why mass mobilization may matter is underdeveloped and we lack empirical research on how it may affect agency policymaking. 

In this dissertation, 
%In this paper, 
I address this theoretical and empirical gap in our knowledge on the role of mass mobilization in bureaucratic policymaking. I expand and integrate the above theories to develop testable hypotheses and analyze rule-related texts %and field experiments 
to explore whether mass mobilization matters and, if so, why. 

I argue that if mass mobilization indirectly affects the strategic environment it does so by signaling grass-roots political power to elected officials and if mass mobilization directly affects agency policymaking it does so by evoking organizational identities and reputations. 
Like the vast majority of letter-writers, Catholic Bishops contribute little to the technical aspects of epidemiology, mercury regulations, or cost-benefit analysis. If they influence agency policymaking it is 
by signaling a threat of political backlash or 
by persuading bureaucrats directly that moving policy in a certain direction is the appropriate thing for the agency to do. 
The next two subsections address these indirect and direct mechanisms in more depth. A third discusses why we may still observe mobilization in the absence of influence. 
% Whereas social movement scholars and political scientists have focused the behavior of elected leaders, I focus on the latter pathway of direct persuasion. 

Precisely identifying who participates, how, and the dimensions of disagreement over time is key to any study attempting to discover whose ideas end up in policy. These are descriptive questions but they are not easy ones. In the empirical section, I address three descriptive questions: who participates, who lobbies together, and who wins\footnote{By \textit{who wins?} I mean whose ideas end up in policy. This is distinct from measuring \textit{influence} with respect to a counterfactuals or constellations of ideal points. I measure what people say they want and whether they get it. For example, I measure whether rules where commenters requested consideration of environmental justice issues were more likely to address environmental justice issues in the final draft}. Here, \textit{who wins?} is descriptive rather than causal. While insufficient to infer specific causal influence, policy moving in one's preferred direction may indicate that one is aligned with those who have power in that policy process. %There are many potential causes for policy outcomes matching certain policy demands, and I proposed field experiments to test several of them.
% \section{Mechanisms of Influence}

\subsection{Indirect Influence: Signaling a threat of backlash}
% \section{Theories and Case Selection}
Why might mass mobilization matter? The literature on bureaucracy offers two types of explanations rooted in either strategic behavior or organizational norms. Political scientists often focus on strategic context. Public administration and management scholars focus on organizational logics and identities. I begin with the ``indirect'' mechanisms that theories of strategic behavior suggest. 

In the U.S. context, there are three main mechanisms by which mass mobilization could affect an agency action by changing the strategic context. 
Mass mobilization may signal power to influence the responses to agency action from the White House, Congress, or courts. Many rules receive little attention from these other institutions, but all three significant powers to reward, sanction, or reverse agency actions \citep{Yaver2016}. 

% Arnold, Logic Of Congressional Action p 217:
% success depends in poat of the length and complexity of the causal chain connecting a policy instrument with its policy effects. When a causal chain is short and simple, citizens are more likley to know which policy instrument will produceth appropriate effects and are beter able to monitor the performance of their repre3sentatives,. When a causal chain is long and complex, or when a problem in society stems from multile causes, citizens may be incapable of doing the appropriate policy analysis and political anslyss ." 
% p 272 "resoonsiveness to both attentive and inattentive publics avaries depending on the procedures that govern how legislators requd their positions" 
% "The model of citizen's control that I have been discussing is essentially an auditing model. Citizens do not instruct legislators on how to vote, not do thay necessairlily have well-defined policy preferences in advance of cogressional action. Legialators neverthless have strong incentives to consider citizens' potential preferences when they are deciding how to vote for fear that making the wrong choice might triggger and unfavorable audit." 
% 


The White House has several tools to influence agency decisions \citep{Yackee2009a,Simon1954}. These executive orders \citep{Mayer1999}, appointments \citep{Doherty2014,Lewis2008,Wood1988}, budgets \citep{Whittington2003}, and review of proposed policies \citep{HAEDER2015InfluenceBudget,Acs2013}. 
Congress also has several tools to influence agency decisions. These include the power of the purse \citep{Fenno1986,Bolton2015}, oversight, and new legislation. Some research suggests that this constraint is larger under divided government \citep{Yackee2009b} and that under divide government Congress tends to divide power among multiple agencies \citep{Farhang2016}.
The anticipation of judicial review makes courts relevant to rulemaking. Some rules are also made under court-imposed settlement or with judicial deadlines. Judicial opinions may also call on Congress to act \citep{Yaver2017}.
Despite these mechanisms and because of conflicts among them, agency staff maintain significant power over agency decisions. For example, Congress is less assured of compliance when power is divided \citep{Yaver2016}.

% more good stuff
%\subsubsection{Legal Scholarship}
Legal scholars' case studies of specific rulemaking process offer an additional relevant body of research. Coglianese (1997) finds that litigation is a common extension of rulemaking. Indeed, unlike legislative lawmaking, rulemaking takes place in the clear and present shadow of judicial review (Rossi 2001). Stakeholders can challenge a rule in court on a variety of procedural grounds and on statutory interpretation. This scholarship suggests those who succeed in rulemaking are those with the resources and experience to succeed in court. Costly mass comment campaigns could be signaling the ability and willingness to spend resources to challenge the rule in court. 

Mass mobilization may signal political risks or benefits of engaging in agency policymaking to members of Congress and the White House. It also may signal to the agency that activists have the capacity to sustain pressure through the policy process \citep{Coglianese2001}, including challenging the policy in court, a constant threat agency policies. Thus, mass mobilization may act as a signal of political power that  reshape rule-writers' beliefs about their strategic context. 



\subsection{Direct Influence: Mobilizing identities and reputations}

I now turn to the direct-influence pathway: the ability of social movements to mobilize ideas, evaluative frameworks, and claims about what is appropriate and right that may affect bureaucratic decisions. %I use mobilization around the idea of ``environmental justice'' as an example where direct influence may be visible. 

Organizational theory suggests additional mechanisms by which mass mobilization may influence bureaucratic decisions more directly. Here the causal process involves mobilizing norms and ideas right and wrong rooted in individual and institutional identity. Because concepts of mission, reputation, and the validity of claims are intertwined, these mechanisms are difficult to precisely define. Nevertheless, scholars have identified several types of direct influence. One important factor in decision making is personal and institutional reputation \citep{Carpenter2001}. This can take several forms. For example, individuals trained as scientists and agencies that cultivate reputations for producing valid science may be persuaded by rigorous scientific claims. Similarly, individuals who identify strongly as public servants and agencies with reputations for public responsiveness may be persuaded by claims about public or "stakeholder" opinion. In general, claims that resonate with the problems an agency has been tasked with solving and the means it has to solve those problems are likely to be well received.  

% \subsubsection{ Agencies as Policymaking Venues}
% the good stuff
When political scientists ask whose interests and ideas become law, they have generally focused on the behavior of legislatures, how the executive branch drives legislation, and how the courts review it. Compared to legislative, executive, and judicial institutions, the administrative state is a recent development in American government and theory has not kept pace with the rise in bureaucratic policymaking. 

I argue that theories of bureaucratic policymaking have been characterized by constraining assumptions about what bureaucracies ought to do.  Normative assumptions that \citet{Wilson1967} identified half a century ago, and corresponding scholarly silos, have persisted. This has led to lines of research talking past each other and often failing to engage broader theories of policy change. In particular, I argue that the pervasive implicit assumption that bureaucrats ought to be neutral implementers implies that politics in agency policymaking is inherently undesirable, leading many scholars to focus on compliance with political principals and overlook the role participation and ideas. For example, scholars assume that agencies ought to be engaged in implementing legislation and executive orders. However, most rulemaking takes place many years or decades after its authorizing legislation under a different Congress and with little attention from the White House until the very final draft. Rules that do not follow from contemporary Congressional or executive priorities are often assumed to reflect bureaucrats going rogue or being captured by interest groups. Such studies suffer from a lack of attention to the complex political process of rulemaking. 

Accountability to elected officials has been central to the study of bureaucracy \citep{Epstein1999,Huber2002,McCubbins1984,Wilson1989,Potter2016Slow-RollingRulemaking,Lowande2018PoliticizationAgencies} %add Meier and O’Toole 2006; West 1995; Wood and Waterman 1994
Viewing agencies as \textit{agents} has prevented scholars from incorporating new insights about the endogenous relationship between policy and politics. I suggest rulemaking is better studied in the way that scholars study policymaking in specialized congressional committees than with an unrealistic dichotomy of sincere implementation versus capture or disloyalty. Normatively, accountability to political principals only one of several important concerns. Empirically, it is often unclear what accountability means and there is ample evidence that it may not be the primary driver of bureaucrat behavior.

In contrast to the dominant view of agencies as \textit{agents}, a growing literature in political science draws on scholarship in law and public administration as well as studies of agenda setting and lobbying in legislative policymaking to better understand agencies as policymaking bodies. Public administration and legal scholars have been more attentive to the prominent role of interest groups.  Kerwin (2003) notes that ``Interest groups could find few modes of government decision making better suited to their particular strengths than rulemaking.'' This research finds business groups to be most successful class of commenters in rulemaking \citep{Yackee2006a} especially when lobbying together, often, or unopposed (Nelson and Yackee 2012) and when lobbying across multiple venues \citep{Yackee2015}. Importantly, this literature notes that the currency of lobbying is information (Hall and Deardorf 2006), which includes both science and policy ideas \citep{Jones2005}. Kirilenko (2014) and Yackee and Yackee (2006) both find evidence that comments from sophisticated interest groups like businesses seem to influence rules. These scholars offer one set of answers to the question of who wins: those who succeed in rulemaking tend to be business interests, repeat players, those who lobby together, and those who lobby unopposed. They succeed because they bring in new voices and send unified messages at higher amplitudes, creating perceptions of political consensus.


%There may be an inverse relationship between how responsive agencies are to political principals and to the public \citep{Lewis}.

%Yet public administration and legal scholarship rarely address how interest groups gain political power in the first place. 

% more good stuff
A second major contribution to theory in this area is Carpenter's  research explaining bureaucratic autonomy \citep{Carpenter2001,Carpenter2012}. Rather than asking how bureaucratic practices fit with normative assumptions, he asks how agencies became independent policymaking bodies. Responding to principal-agent literature that has focused on the presidential and Congressional control, Carpenter finds much more complex sets of relationships that explain organizational power and behavior. One of the main tools he gives us for understanding the source of bureaucratic autonomy is the concept institutional reputations. Bureaucrats and the institutions they animate develop reputations for certain competencies: for example, for expertly adjudicating scientific claims, for effectively executing policy aimed at a given goal, or for divining the public interest. Reputations for expertise, effectiveness, or representativeness reflect the mixed roles assigned to the bureaucrats: advisors, implementers, and policymakers. 
% Like Carpenter, I call attention to the fact that agency policy shapes the coalitions that surround and influence it. I depart from Carpenter's narrative in that I do not focus on cases where agencies intend to have these effects. Whereas Carpenter is interested in how bureaucrats intentionally shape lobbying coalitions, I am interested in the endogenous relationship between policy and coalitions, intended or not. While not the focus of his study, Carpenter notes that relationships also evolve in unintended ways. 
%Policy may pro-actively recruit group support, but may also be reacting to political pressure \footnote{For example, an industry may successfully lobby to be reclassified to face lower pollution regulations, perhaps those faced by their competition, thus turning competitors into allies for future policymaking and increasing the size of the coalition for the lower standard.} or be an unintended side effect of action the agency sees as imperative.\footnote{For example, new science on the health hazards of mercury led to pollution controls that differentiated coal power plants and gas power plants reshaping coalitions by turning allies on former air quality policymaking into competitors in future rounds.} %Nevertheless, 
Carpenter and related scholars thus offer a second possible set of predictions for which movements are successful: those who succeed in rulemaking tend to be those with close relationships with the agency, conditional upon (and because of) how those relationships support the agency's reputation for expertise, competence, and representativeness. %Furthermore, lobbying coalitions, their relationship to the agency, and thus their success are functions of past agency policy. 

%Some scholars attempt to estimate the preferences of bureaucrats. Instead, I take 
% Carpenter's findings of close relationships between interest groups and agencies is a potential explanation for why some groups appear to have influence, i.e. because they are aligned with agency ideologies. As my core contribution is to assess how groups are empowered or disempowered rather than how agencies are empowered or disempowered, I focus on discovering which groups' comments are related to changes in rules regardless of whether this is what certain bureaucrats also wanted. 




% \subsubsection{Reputations for accountability, representation, equity, and expertise }

% [How specific organizational identities and reputations drive decisionmaking]

\subsection{Mobilizing for Recruitment}

A third possibility is that mobilization around bureaucratic decisions is unrelated to the possibility of affecting policy and primarily a way to recruit and engage members or raise the profile of the movement. If this is the case, behaviors like protesting and mass commenting on rules are largely epiphenomena to unrelated kinds of politics. Organizers may know that mobilization has minimal effects, but lead members to engage as means to other ends. Many of the mobilized themselves may doubt their efficacy but still take advantage of the opportunity to protest. 

% The remainder of this paper presents a case study and an empirical test of whether comments influences rulemaking.



[Elaborate on org behavior]

\subsection{Advancing Theories of Bureaucratic Policymaking}

Quantitative studies of bureaucratic policymaking in political science tend to collapse the time dimension and rarely consider the historical context in which each rule is made. For example, scholarship exploring how political context affects timing and delay in rulemaking models rules as if they are independent of each other and independent of the date they began (see Potter 2017). These studies also tend to focus on the degree to which agency policymaking reflects presidential, congressional priorities and, occasionally, interest group priorities. Political scientists most often ask if agencies are doing what the president wants, what congress wants, or something else. They find significant amounts of ``something else,'' but theory inconsistent on what it is and where it comes from. 

%bad stuff


\section{Methods}
This project makes two core contributions. First, I introduce new methods to 
%test theories about
measure the formation of lobbying coalitions, their demands, and whether they got what they asked for. 
% and of specific actors within coalitions
Second, I employ field experiments to test mechanisms by which mass mobilization may influence bureaucratic policymaking. 
%Rulemaking gives specific meaning to legislation and thus governmental force to political ideas. Like other policies, regulations also shape the terrain for future politics. 

\section{Measuring indirect influences}
To assess the direct influence pathway, I estimate the extent to which mass mobilization around bureaucratic actions makes members of congress or White House officials more likely to engage or react and whether such mass mobilizations is salient in subsequent litigation. 

Congressional attention to agency actions can be observed in several ways. Members of Congress who sit on oversight committees raise issues in oversight hearings, reports attached to each agency's budget appropriation, and in personal letters addressed to agency officials. Using text reuse methods I will identify when policy issues raised in draft rules and rule comments attract positive or negative attention from legislators. Using texts has several major advantages over previous measures of congressional attention and sentiment such as partisanship \citep{Yaver2016,Lewis2008}, changes in budget size, or the length of appropriations reports. Unlike partisanship, it is issue-specific and does not require assumptions about agency partisanship. While budget changes may reflect real costs, the many reasons that budgets change make it difficult to attribute changes to particular agency actions. The length of appropriations subcommittee reports may indicate the amount of attention committees pay to an agency but they do not vary significantly over time and do not indicate whether committee attention is positive or negative. 

[President and Secretary]

There are two ways to assess the courts as an indirect pathway where mobilization leads to influence. First, mass mobilization may increase the credibility of the threat of litigation. Second mass mobilization may influence the outcome of subsequent court cases over the rule. Both are difficult to measure. The first I measure with a combination of the litigation history of mobilizing groups and specific references to litigation in the comments. The second I assess by identifying instances where courts reference the number or direction of comments or other forms of protest in their decisions and compare rules to the rules under consideration in those cases. While this is rare, legal scholars have noted that " If the validity of a final regulation is challenged in court, the court's review will be based in significant part on how well the agency responded to the public's comments"  (Wagner 1995).

\subsection{Measuring Policy Change}
Policies may shape and be shaped by many forces, including the collective action of citizens, expert opinions, and businesses interests. Yet the drivers and consequences of policymaking are difficult to disentangle. Business groups may fund scientists or advocacy campaigns to preempt or undo costly regulations. Experts and policymakers may inspire broader civic mobilization, and citizen mobilization may, in turn, shape the priorities of experts and policymakers. Some policy debates divide along lines of citizen and corporate interest or expert and popular opinion, but many entail various clusters of claims regarding the public interest, expertise, and business interests: claims about the public good, scientific truths, and the proper role of government. Inferring policy demands from identity alone and assuming static coalitions may miss much of the story. 



\subsection{Data}
I focus on bureaucratic policymaking because, due to its sheer volume, it is both rich in opportunities to see different types of political mobilization, organization, and power at work and incompletely understood by political scientists. Specifically, I focus on agency rulemaking, a key part of U.S. policymaking that offers analytical leverage. Rulemaking is a process where agencies must solicit and respond to public comments on regulations (rules) before they carry the force of law (see Figure 1). Draft rules are published in a Notice of Proposed Rulemaking (NPRM). Occasionally, comments are also solicited before the draft rule is published through an Advanced Notice of Proposed Rulemaking (ANPRM). Intriguingly, while this process originally aimed to promote direct democracy and citizen voice, it is now generally seen as a mechanism to engage expertise (Coglianese 2006). Furthermore, research finds that the comment process actually favors business interests (Yackee and Yackee 2006). %Despite a large number of case studies, largely from legal scholars, our systematic understanding of the politics of rulemaking is thin.

% diogram of rulemaking  and commenting 
\begin{figure}[h!]
\label{inputs}
\caption{The Textual Record of Agency Rulemaking}
%\begin{table}
\begin{tabular}{@{\extracolsep{5pt}}cccccc}
 &  & &  \\
 & &\multicolumn{3}{c}{(Public Comments)}\\
 & & &$ \downarrow $& \\
\fbox{Inputs} & $\longrightarrow$ & \fbox{Proposal Text} &$\longrightarrow$ & \fbox{Outcome Text}\\
 & & & \\
List of Statutory Authorities &  & Proposed Rule & & Final Rule\\
(Advanced Notice)  &   &  & &   (and Response to Comments)\\
(Comments)  &   &  & &   \\
\end{tabular}
%\end{table}
\end{figure}
 

Rich data on several decades of rulemaking are available but have yet to be fully utilized by scholars.  Agencies publish draft rules, and comments received by interest groups, experts, and citizens. This offers leverage to identify the players, winners, and losers and to track those participating in the policy process over time. Rulemaking records often also cite the statutes, executive orders, and court cases that form a rule's historical institutional context. Some of this information, along with draft and final rule publication or withdrawal dates, is summarized since 1981 in the Unified Agenda of Regulatory and Deregulatory Actions (reginfo.gov). From 1994 onward, the text of most proposed draft rules, final rules, and summaries of comments received are published in the Federal Register (federalregister.gov). The result is the text of more than 70 thousand rules and . Finally, I collected text of over 7 million comments from 2002 onward via regulations.gov's API.

With the text of over 70 thousand regulations published since 1981 and over 7 million of the public comments on regulations since 2002, the second chapter of this dissertation will sketch the broad outlines of rulemaking in the American political context: who participates, how often rules are contested, whose ideas and interests are reflected in the text of rules, and who wins with different patterns of mobilization and contestation (or non-contestation). 

To make the project reasonable, the remaining chapters focus on [three] policy areas that have seen the highest levels of mass mobilization: [environmental, financial services, and communications technology]. To identify environmental rules, I select all rules made by the environmental protection agency and rules made by other agencies that cite president Clinton's executive order on environmental justice or president Obama's executive order on climate adaption. This allows me to consider how the same environmental problems may be addressed by different agencies. Financial services regulations are those that cite the Dodd-Frank act. Communications technology regulations are those proposed by the Federal Communications Commission.

%I also look closely at rules that end up before the Supreme Court. These rulemaking processes deserve extra attention for two reasons. First, regardless of how contentious they were at the rulemaking stage, these rules are key to understanding the nature and limits of executive power as a policymaking venue. Second, because all rulemaking is done in the shadow of judicial review, who wins in court and why shapes the political terrain of future rulemaking, empowering some groups with credible threats of litigation and disempowering others. References to court cases and implied threats of litigation are common in interest group comments, but to my knowledge, no study has looked systematically at how they affect rulemaking. Conversely, scholarship has not systematically assessed how mobilization and contestation in a rulemaking process affect judicial review. 

%%%%%%%%%%%%%%%%%%%%%%%%%%%%%%%%%%%%%%%%%%%%%%%%%%%%%%%%%%%%%%
\section{Theory} 
% MAIN QUESTION 
\paragraph{Incorporating mass engagement into theories of bureaucratic policymaking.}
How, if at all, should scholars incorporate mass engagement into models of bureaucratic policymaking? 
I argue that mass engagement produces potentially valuable political information about the coalition that mobilized it.
Thus, depending on how agencies process political information, ``going public'' may occasionally be an effective strategy for organizations to influence policy, both directly and indirectly. However, influencing policy may not be the only reason to mobilize. 

% % PREVIEW 
The next section builds a causal theory. I theorize that activists' opportunities and strategies and latent public opinion drive engagement.

The following section outlines methodologies to assess my three overarching questions and their component parts. Theses methods rely on analysis of comment and policy texts as large-n observational data. 


\subsection{Why mobilize?} \label{whymail-intro}
%: Lobbying coalitions, mass comments, and political information in bureaucratic policymaking

Scholars of bureaucratic policymaking have focused on the sophisticated lobbying efforts of powerful interest groups. 

[INSERT - INFORMATION IS THE CURRENCY OF INTEREST GROUP LOBBYING]

\begin{figure}
    \centering
    \caption{The Classic Model of Interest Group Lobbying in Bureaucratic Policymaking}
    \label{fig:causal-classic-lobbying}
\tiny
\begin{tikzpicture}[%
    node distance=1.2cm,
    auto,
    text width=1.5cm,
dnode/.style={diamond, align=center, aspect=2, fill=green!5,draw=green!60, very thick, minimum size=2cm},
squarednode/.style={rectangle, align=center, aspect=1, draw=red!60, fill=red!5, very thick, minimum size=1cm},
pnode/.style={ellipse, align=center, aspect=1, draw=black!60, fill=black!5, very thick, minimum size=1cm},
title/.style={rectangle, align=center, aspect=1, minimum size=2cm},
]
% Draft 
\node[dnode]      (draft)                     {Draft Policy};



% Group Nodes
\node[pnode]      (groupdemands) [right=of draft] {Group Demands};
\node[dnode]        (groupdecides) [right=of groupdemands] {Lobbying};
\node[squarednode]      (groupinfo) [right=of groupdecides] {Technical Information};

% policy 
\node[dnode]      (policy)       [right=of groupinfo] {Policy Response};
\draw[->] (groupinfo.east) -- (policy.west);
% \draw[->] (publicinfo.east) -- (policy.west);
% \draw[->] (principalinfo.east) -- (policy.south);
% \draw[->] (principalinfo2.east) -- (policy.south);

% Group Lines
\draw[->] (draft.east) -- (groupdemands.west);
\draw[->] (groupdemands.east) -- (groupdecides.west);
\draw[->] (groupdecides.east) -- (groupinfo.west);

% Titles
% \node[title]      (1) [above=of draft] {Policy};
%\node[title]      (2) [above=of groupdemands] {Preferences};
%\node[title]      (4) [above=of groupinfo] {Information/ Signal};
%\node[title]      (3) [above=of groupdecides] {Observed Behavior};


\end{tikzpicture}
\end{figure}
\normalsize

Yet agencies occasionally receive thousands or even millions of comments from ordinary people. 
%Why? Why do individuals comment when they seemingly have no new information to offer and no power to influence decisions? Who inspires them and to what end? How, if at all, should scholars incorporate mass commenting into models of bureaucratic policymaking? I argue that mass commenting produces political information about the coalition that mobilized it. 
% QUESTION 1
% subsubsection{Puzzle} 
Why do people comment on draft policies when they seem to have no new information to offer and no power to influence decisions? Who inspires them and to what end? 
% THEORY AND METHODS 1
Answering these questions requires a method to link comments to coalitions and a theory explaining variation in mass engagement.  

This section offers a definition of mass engagement, an argument that it is best understood as an outside lobbying tactic, and theory predicting different patterns of engagement depending which of three reasons organizations may have for launching a mobilization campaign. In the next section, I develop methods to measure these patterns.



% DEFINITION
\subsubsection{Defining mass engagement}
Political scientists often define civic engagement as writing to government officials, signing petitions, attending hearings, attending protests, or donate to a political campaign. While donating is more common in electoral politics, activists frequently attempt to influence agency policymaking through letter-writing, petitions, hearings, and protests. 
% I suspect that mass commenting is driven by the same privileged populations known to engage in other civic activities. 
% Does it work? If so, by what mechanisms?

Following the conventional terms ``mass comment campaign'' and ``public engagement,'' I call the general phenomenon ``mass engagement'' resulting from ``mass mobilization'' in order to distinguish the magnitude of civic engagement.
By mass engagement, I mean that thousands of people beyond professional policy influencers engage. Specifically, I define mass engagement as more than 1000 public comments or 100 identical comments, plausibly reflecting a mobilization effort.  Contrary to the common assumption that this emerges organically, it is almost always mobilized by an organization that also engages in sophisticated lobbying. %\footnote{
As \citet{SantAmbrogio2018} conclude ``The `mass comments' occasionally submitted in great volume in highly salient
rulemakings are one of the more vexing challenges facing agencies in recent years. These comments are typically the result of orchestrated campaigns by advocacy groups to persuade members or other like-minded individuals to express support or opposition for an agency's proposed rule.'' 

We lack systematic analysis of public comments. Political scientists have thus far focused on sophisticated lobbying efforts. However, as \citet{Cuellar2005} finds in his study of several rules, ``contrary to conventional wisdom, comments from the lay public make up the vast majority of total comments about some regulations. This shows at least some potential demand among the mass public for a seat at the table in the regulatory process.'' Having collected over 70 million comments on over 300,000 proposed rules, I am able to offer a much more systematic analysis. As figure \ref{fig:mass-comments} shows, not only do comments from ordinary people make up the vast majority of comments across all rule, most comments are, in fact mobilized by mass commenting campaigns. 

\begin{figure}
    \centering
    \includegraphics{}
    \caption{Unique vs Form-letter Comments Posted to Regulations.gov 2006-2018}
    \label{fig:mass-comments}
\end{figure}


\subsubsection{Understanding mass mobilization as a tactic}
% MY THEORY 
% Representation 
When lobbying during rulemaking, groups often make suspect claims to represent broad segments of the public \citep{Seifter2016UCLA}. Mobilizing a large number of people may support such claims.
% public private interests
Appeals to government are almost always couched in the language of public interest, even when true motivations obviously private \citep{Schattschneider1975}. Theorists may debate whether effectively signing a petition of support without having a role in crafting the appeal is meaningful voice and whether petitions effectively channel public interests, but, at a minimum, engaging a large number of supporters helps distinguish narrower interests from broader ones. It suggests the organization is not ``memberless'' \citep{Skocpol2003} in the sense that they are able to demonstrate some public support.

% tactic
Mass mobilization is a strategy. When successful, mass engagement is the result. An organization's ability to expand the scope of conflict by mobilizing members of the public is a political resource. 
In contrast to scholars who focus on the deliberative potential of public comment processes, I focus on public engagement as a tactic aimed at gaining power, either by leveraging powerful ideas or engaging actors with the institutional power to shape decisions.
Scholars who do understand mobilization as a tactic \citep{Furlong1997, Kerwin2011} have thus far focused organizations mobilizing their membership. %In contrast, 
I expand this to include a campaign's broader audience and its potential to grow, more akin to the concept of an attentive public \citep{Key1961} or issue public \citep{Converse1964}. If organizations claim to represent people beyond their official members, 
reforms requiring groups to disclose information about their funding and membership \citep{Seifter2016UCLA} only go part way to assess groups' claims to represent these broader segments of the public. Indeed, if advocacy group decisions are largely made by D.C. professionals, these advocates themselves may be unsure how broadly their claims resonate until potentially-attentive publics are actually engaged.

% three insights 
Here I build on three insights. First, \citet{Kerwin2011} and \citet{Furlong1997} identify mobilization as a tactic. In their survey, organizations report that forming coalitions and mobilizing large numbers of people are among the most effective lobbying tactics. Second, \citet{Nelson2012} identify political information a potentially influential result of lobbying by different business coalitions. While they focus on mobilizing experts, \citet{Nelson2012} describe a dynamic that can be extended to mass commenting: 
``strategic recruitment, we theorize, mobilizes new actors to participate in the policymaking process, bringing with them novel technical and political information. In other words, when an expanded strategy is employed, leaders activate individuals and organizations to participate in the policymaking process who, without the coordinating efforts of the leaders, would otherwise not lobby. This activation is important because it implies that coalition lobbying can generate new information and new actors---beyond simply the `usual suspects'---relevant to policy decision makers. Thus, we theorize consensus, coalition size, and composition matter to policy change.'' 
I argue that, with respect to political information, this logic extends to non-experts. 
Third, \citet{Furlong1998}, \citet{Yackee2006JPART}, and others distinguish direct and indirect forms interest group influence in rulemaking. I argue that mass mobilization is a tactic aimed at producing political information that may have direct and indirect influence. 

% PUBLIC OPINION and INFORMATION 
\citet{Rauch2016} suggests that agencies reform the public comment process to include opinion polls. I build from a similar intuition that mass comment campaigns currently function like a poll or, more accurately, a petition, capturing the intensity of preferences among a segment of the public---i.e. how many people are willing to take the time to engage. Self-selection may not be ideal for representation, but opt-in participation---whether voting, attending a hearing, or writing a comment---still provides political information. 
Mobilizing citizens and generating new political information are key functions of interest groups in a democracy \citep{Mansbridge1992, Mahoney2007}. The information generated by mass mobilization campaigns is explicitly political and more complex than an opinion poll. Activists aim to convince people which issues are important and how to think about them---mapping new issues and debates to familiar ones, thereby shifting the political landscape. 

Importantly, rule-specific campaigns inform agencies about the distribution and intensity of opinions that are often too nuanced to estimate a priori. Many rules may lack analogous public opinion polling questions, making mass commenting a unique source of political information. Indeed, most members of the public and their elected representatives may only learn about the issue as a result of an mobilization campaign. I thus consider public demands to be a latent factor in my model of policymaking. Public demands shape the decisions of groups who lobby in rulemaking. If they beleive the attentive pubic is on their side, groups may attempt to reveal this political information to policymakers by launching a mass mobilization campaign. The public response to the campaign depends on extent that the attentive public is passionate about the issue.

\input{causal-whymail.tex}






% TYPES OF CAMPAIGNS AND COALITIONS
\paragraph{Types of campaigns:} The mix of types of supporters depends, in part, on the aims of a campaign. Campaigns may have one of three distinct aims: (1) to win concessions by going public, (2) to disrupt a perceived consensus, or (3) to go down fighting. 

Coalitions ``go public'' when they believe that expanding the scope of conflict gives them an advantage.\footnote{
Going public (or an ``outside strategy'') is used by Presidents \citep{Kernell2007}, Members of Congress \citep{Malecha2012}, interest groups \citep{Walker1991, Dur2013}, Lawyers, and Judges (Davis 2011). 
% Sophisticated organizations also use phone banks, targeting strategies, and direct-mail techniques to drum-up and channel public support (see Cooper 1985:2036).
This strategy is likely to be used by those disadvantaged (those \citet{Schattschneider1975} calls the `losers') with less public attention.
Rulemaking with little public attention is the norm. Nearly all scholarship on rulemaking in political science thus focuses on interest-group and inter-branch bargaining, ignoring public opinion and social movements. 
}
As these are the coalitions that believe they have more intense public support, many people may be inspired indirectly and to engage with more effort. In these cases, mass engagement will likely skew heavily toward this side. This is important because a perceived consensus may be especially influential political information.\footnote{
For example, consensus among interest groups \citep{Golden1998, Yackee2006JPART}, especially business unity \citep{Yackee2006JOP, Haeder2015}, predicts policy change, though it is not clear if this is a result of strategic calculation, a perceived obligation due to the normative power of consensus (e.g. following a majoritarian \citep{Mendelson2011}), or simply that the information is easier to process.
}

Second, because the perception of consensus is powerful, when a coalition goes public, an opposing coalition may countermobilize. As this is likely a coalition with less intense public support and its aim is merely to break a perceived consensus, I expect such campaigns to engage fewer people, less effort per person, and yield a smaller portion of indirect engagement. 

Finally, campaigns may target supporters rather than policymakers. Sometimes organizations ``go down fighting'' to fulfill supporters' expectations.\footnote{
I use ``going down fighting'' as shorthand for campaigns aimed at only at fulfilling supporter (e.g. donor, membership) expectations and related logics that are internal to the organization (e.g. fundraising, member retention or recruitment, or satisfying a board of directors).} While such campaigns may engage many people, they are unlikely to affect policy or to inspire countermobilization. I expect such campaigns to occur on rules that have high partisan salience (e.g. rules following major legislation passed on a narrow vote), propose large shifts on policy issues dear to well-funded public interest groups, and occur after presidential transitions when executive-branch agendas shift more quickly than public opinion.


% \subsection{Mobilizing for Recruitment}
% A third possibility is that mobilization around bureaucratic decisions is unrelated to the possibility of affecting policy and primarily a way to recruit and engage members or raise the profile of the movement. If this is the case, behaviors like protesting and mass commenting on rules are largely epiphenomena to unrelated kinds of politics. Organizers may know that mobilization has minimal effects, but lead members to engage as means to other ends. Many of the mobilized themselves may doubt their efficacy but still take advantage of the opportunity to protest. 

While the coalitions may form around various material and ideological conflicts, those most likely to be advantaged by going public or going down fighting are public interest groups---organizations primarily serving an idea of the public good rather than the material interests of their members.\footnote{
One exception may be the few types of membership organizations that are both broad and focused on material outcomes such as labor unions.} Thus, I theorize that mass mobilization is most likely to occur in conflicts of public versus private interests or public versus public interests (i.e. between coalitions led by groups with distinct ideas of the public good), but only ones with sufficient resources to run a campaign.\footnote{
If true, one implication is that mass mobilization will systematically run counter to concentrated business interests where they conflict with the values of organized, privileged groups.
}
To assess these propositions, I classify coalitions as primarily driven by public or private interests and roughly estimate each coalition's resources. 



% We do not really know who engages in mass commenting. Some assume that people who engage in mass commenting belong to membership organizations. Others imply that they are people who happen to have an opinion. RAUCH discusses both "members" and  _______
% % The people who engage in mass commenting are often assumed to be 

% Engaging a broader audience and thus changing the scope of conflict is a basic political strategy. Presidents, supreme court justices, and others "go public" when doing so alters their opponents' calculations. 

% Which campaigns engage a broader audience and which do not? 


\paragraph{Types of engagement:} I classify supporters into three types using the texts of their comment to infer how they were mobilized.  Comments that are exact copies of a form letter are akin to petition signatures from supporters who were engaged by a campaign to comment with minimal effort. Commenters that repeat text but also take time to add their own text indicate more intense preferences. Finally, commenters who express solidarity in similar but distinct phrases indicate they were engaged indirectly
, perhaps by a news story or a social media post about the campaign, 
as campaign messages spread beyond those originally targeted. 

The size of each of each group thus offers political information to policymakers, including coalition resources, intensity of sentiment, and potential for conflict to spread.
The first two types signal two kinds of intensity or resolve. First, they show the mobilizers' willingness to commit resources to the issue. Second, costly actions show the intensity of opinions among the mobilized segment of the public \citep{Dunleavy1991}. The number of people engaged by a campaign is not strictly proportional to an organizations investment. The less people care, the more it costs to mobilize them. If agency staff do not trust organizations' representational claims, engaging actual people may be one of the few credible signals of a broad base of support. The third type indicates potential contagion. Indications that messages spread beyond those originally targeted be especially effective \citep{Kollman1998}. Information about organizational resolve, intensity of preference, and contagiousness are thus produced, but will only influence decisions if mass comments are processed in a way that captures this information and relays it to decisionmakers. These organizational processes may vary significantly across agencies.





%%%%%%%%%%%%%%%%%%%%%%%%%%%%%%%%%%%%%%%%%%%%%%%%%%%%%%%%%%%%
\section{Methods}

\subsection{Measuring mass engagement and political information}
\label{whyMail-methods}

\subsection{Measuring Coalitions}



\begin{figure}[h!]
    \centering
    \caption{Step 1: Explaining Mass Mobilization and Mass Engagement} 
    \label{fig:causal-whymail-test}
\tiny
\begin{tikzpicture}[%
    node distance=1.2cm,
    auto,
    text width=1.5cm,
dnode/.style={diamond, align=center, aspect=2, fill=green!5,draw=green!60, very thick, minimum size=2cm},
squarednode/.style={rectangle, align=center, aspect=1, draw=red!60, fill=red!5, thick, minimum size=1cm},
pnode/.style={ellipse, align=center, aspect=1, draw=black!60, fill=black!5, thick, minimum size=1cm},
title/.style={rectangle, align=center, aspect=1, minimum size=2cm},
]

% Group Nodes
\node[pnode]      (groupdemands) {Group Demands};
\node[dnode]        (groupdecides) [right=of groupdemands] {Lobbying Strategy};
\node[squarednode]      (groupinfo) [right=of groupdecides] {Technical Information};


% Group Lines
\draw[->] (groupdemands.east) -- (groupdecides.west);
\draw[->] (groupdecides.east) -- (groupinfo.west);

% Titles
% \node[title]      (1) [above=of draft] {Policy};
% \node[title]      (2) [above=of groupdemands] {Preferences};
% \node[title]      (4) [above=of groupinfo] {Information/ Signal};
% \node[title]      (3) [above=of groupdecides] {Observed Behavior};
% \node[title]      (5) [above=of policy] {Policy'};

\node[text centered]      (mobilizing) [below=of groupdecides] {Mass\\ Mobilization};

% political info
\node[rectangle, minimum width =2cm, minimum height = 3cm, draw=red!60, fill=red!5,  thick]      (politicalinfo) [below=of groupinfo] {};

\node[text centered]      (politicalinfotext) [below=of groupinfo] {Political Information};

\node[text centered]      (mobilizing) [below=of groupdecides] {Mass\\ Mobilization};

\node[squarednode]      (publicinfo) [below=of politicalinfotext] {Perceived Public Opinion};
\node[dnode]      (publicdecides) [left=of publicinfo] {Mass\\ Engagement};
\node[pnode]        (publicdemands) [left=of publicdecides] {Latent Public Demands};


% public Lines
\draw[->, line width=2] (publicdemands.east) -- (publicdecides.west);
\draw[->, line width=2] (publicdemands.north east) -- (groupdecides.south west);
\draw[-, line width=2] (groupdecides.south) -- (mobilizing.north);
\draw[->, line width=2] (mobilizing.south) -- (publicdecides.north);
\draw[->] (publicdecides.east) -- (publicinfo.west);



\end{tikzpicture}
\end{figure}
\normalsize


\subsubsection{Identifying participants}
Previous studies of rulemaking stress the importance of coalitions \citep{Yackee2006a}. Scholars have measured coalitions of organized groups but have yet to be able to attribute citizen comments to the coalition that mobilized them.
Metadata on participants in rulemaking including the date and author of comments (including the type of author, i.e. business, business group, citizen, public interest group, etc.)% and briefs
allows me to track and compare relative alignment  across venues and over time to assess whose ideas and interests are reflected at each stage of policymaking and in policy processes over time. % and review, for example from a statute or executive order, to the agency rule(s), to review by the White House, to court opinions. 
Unfortunately, metadata regarding the authors of comments and court briefs are often inconsistent and incomplete. As this information is key to assessing who influential actors are, improving these data is a significant data-organization task. After cleaning the corpus of comment texts, simple text search matching organization and individual names across texts, especially those named as comment authors will help systematically link individuals and groups who may participate in different coalitions and under different names over time. This help to identify formal coalitions of organizations that sign onto the same comment as well as experts and citizens mobilized by advocacy campaigns to submit separate comments.

\subsubsection{Identifying coalitions }

Having identified who is participating in rulemaking over time, the next step is to identify who is lobbying together. 

When actors sign onto the same comment, it is clear that they are lobbying together. This generally takes two forms. Businesses and groups representing allied industries often co-sign carefully crafted suggestions that reflect their common interest. We expect this to occur when the benefits of coordination outweigh the costs (Yackee and Yackee 2006). The other form this take is public campaigns that ask citizens to submit a form letter, often alongside other actions such as protests. These occasional bursts of civic participation may affect rulemaking (Coglianese 2001), but this is yet to be tested. %In the first form, many of the businesses are repeat players and I record them individually. In the second form, the advocacy groups are repeat players, and I recorded their participation, but it would be citizens who participate are likely not and I record the number of these comments as an amplitude parameter for the text they signed and I attribute form-letter texts to the advocacy groups promoting them.

Various businesses, advocacy groups, and citizens often comment separately even when they aligned. The comment process is open to anyone and it is often not worthwhile for all actors to coordinate their messages. There may be many dimensions of demands and it is unclear to which coalition many comments belong.

Classifying comments into common groups is a task well suited for a single membership topic model.\footnote{This is in contrast to the mixture model I use to estimate the distribution of multiple topics in each document and each coalition} This model clusters documents by the frequency they use different words. Being classified together does not mean that the documents all address exactly the same distribution of substantive issues, just that how issues are discussed is similar relative to the full set of documents.

% Identifying when commenters change coalitions is key for testing policy feedback theories about how policies reorganize political coalitions. I do this by indexing rules over time and adding a parameter for the probability that an actor switches from one coalition to another at each point in time. This allows the model to achieve a better fit by reclassifying an actor after some point in time. These actor- and coalition- specific points in time are a key output of this approach required to test theories of how policies reorganize political coalitions. 

Bounding the scope of this model (i.e. the policy system) is a challenge. On the one hand, each agency deals with many issues of interest to different coalitions. On the other hand, many lobby across multiple agencies. I opt to model coalition advocacy at a fine scale based on Office of Management and Budget agency sub-function codes, but I may try to link across related issues and agencies. 


%%%%%%%%%%%%%%%%%%%%%%%%%%%%%%%%%%%%%%%%%%%%%%%%%%%%%%%%%%%%
% RESULTS
\section{Findings}

In this section, I present the results of applying the classification methods described above.

%Political scientists have thus far focused on sophisticated lobbying efforts. However, as 
\paragraph{Data.}
In his case-study of several rules, \citet{Cuellar2005} finds that ``contrary to conventional wisdom, comments from the lay public make up the vast majority of total comments about some regulations. This shows at least some potential demand among the mass public for a seat at the table in the regulatory process.'' 
With over 70 million comments on over 300,000 proposed rules, I am able to offer a much more systematic analysis. 

\paragraph{Most comments result from mass-comment campaigns.}
Figure \ref{fig:comments-mass} shows all comments posted on regulations.gov over time by whether they are exact copies of another comment or not. This highly restrictive definition of what counts as mass engagement captures comments that were certainly mobilized by a campaign. As \ref{fig:comments-mass} shows the vast majority of comments are mobilized by mass commenting campaigns. In other words, most comments are from ordinary people.

\begin{figure}[h!]
    \centering
        \caption{Unique vs Form-letter Comments Posted to Regulations.gov 2006-2018}
    \includegraphics[height =2in]{Figs/comments-mass-vs-unique-1.png}
    \includegraphics[height =2in]{Figs/comments-mass-support-vs-oppose-1.png}

    \label{fig:comments-mass}
\end{figure}

The right pane of \ref{fig:comments-mass} shows results from a sample of several million comments for which I have digitized texts. Many of these comments appear to support proposed agency rules, as was the case with both the do not call and mercury rule examples. A rough measure of support (whether the comment text includes `` support '' or `` oppose '') shows that many more comments mention support, until 2018, when there is a fairly dramatic reversal in the share of comments mentioning `` support '' compared to those mentioning ` `oppose '' (figure \ref{fig:comments-mass}). This may be a function of the changing regulatory agenda due to the change in presidential administration. 



\paragraph{Most comments occur on a small number of salient rules.} Approximately a third of public comments posted to regulations.gov were received on just ten dockets.


\begin{figure}
    \centering
    \includegraphics[width = 6in]{Figs/topdockets.png}
    \caption{Top 10 Dockets Receiving the Most Comments on regulations.gov and the 20 Mobilizers}
    \label{fig:topdockets}
\end{figure}


\paragraph{A coalition of public-interest organizations mobilize most comments.} As figures \ref{fig:topdockets}% and \ref{fig:toporgs} 
show, the most prolific mobilizers are environmental groups. In part, this is because the Environmental Protection Agency produces a large share of substantive rules posted to regulations.gov. However, it is notable that, on the top ten dockets, 19 of the top 20 mobilizers generally lobby together. America's Energy Cooperatives, an industry association, stands out as the lone mobilizer on behalf of material interest for its members. Notably, it only mobilized significantly on the Clean Power Plan. 

\section{Conclusion}


% CONCLUSION
%This research will add to our understanding of how  bureaucratic policymaking fits with the practice of democracy.
If input solicited from ordinary people has little effect on policy outcomes, directly or indirectly, it may be best understood as providing a veneer of democratic legitimacy on an essentially technocratic and/or elite-driven process.
The legitimacy of bureaucratic policymaking is said to depend on the premise that rulemaking provides an outlet for public voice \citep{Croley2003, Rosenbloom2003}. Yet, it is not just the opportunity to engage, but actual engagement that matters \citep{Herz2018}, and we lack an empirical base necessary to evaluate whether any legitimacy the public comment process may provide is deserved, even if people believe that their comments matter \citep{Yackee2014JPART}.
If mass engagement does shape agency decisions, a new research program will be needed to investigate who exactly these campaigns mobilize and represent.



\newpage
\section{Appendix}

\begin{figure}
    \centering
    \caption{The Role of Mass Commenting and Political Information in Bureaucratic Policymaking}
    \label{fig:causal-full}
\tiny
\begin{tikzpicture}[%
    node distance=1.2cm,
    auto,
    text width=1.5cm,
dnode/.style={diamond, align=center, aspect=2, fill=green!5,draw=green!60, very thick, minimum size=2cm},
squarednode/.style={rectangle, align=center, aspect=1, draw=red!60, fill=red!5, very thick, minimum size=1cm},
pnode/.style={ellipse, align=center, aspect=1, draw=black!60, fill=black!5, very thick, minimum size=1cm},
title/.style={rectangle, align=center, aspect=1, minimum size=2cm}
]
% Draft 
\node[dnode]      (draft)                     {Draft Policy};



% Group Nodes
\node[pnode]      (groupdemands) [right=of draft] {Group Demands};
\node[dnode]        (groupdecides) [right=of groupdemands] {Lobbying Strategy};
\node[squarednode]      (groupinfo) [right=of groupdecides] {Technical Information};

% policy 
\node[dnode]      (policy)       [right=of groupinfo] {Policy Response};
\draw[->] (groupinfo.east) -- (policy.west);
% \draw[->] (publicinfo.east) -- (policy.west);
% \draw[->] (principalinfo.east) -- (policy.south);
% \draw[->] (principalinfo2.east) -- (policy.south);

% Group Lines
\draw[->] (draft.east) -- (groupdemands.west);
\draw[->] (groupdemands.east) -- (groupdecides.west);
\draw[->] (groupdecides.east) -- (groupinfo.west);

% Titles
% \node[title]      (1) [above=of draft] {Policy};
\node[title]      (2) [above=of groupdemands] {Preferences};
\node[title]      (4) [above=of groupinfo] {Information/ Signal};
\node[title]      (3) [above=of groupdecides] {Observed Behavior};
% \node[title]      (5) [above=of policy] {Policy'};

% political info
\node[rectangle, minimum width =2cm, minimum height = 7.5cm, draw=red!60, fill=red!5, very thick]      (politicalinfo) [below=of groupinfo] {};
\node[text centered]      (politicalinfotext) [below=of groupinfo] {Political Information};
\node[text centered]      (mobilizing) [below=of groupdecides] {Mass\\ Mobilization};
\draw[->] (politicalinfo.north east) -- (policy.south west);

% public Nodes
\node[squarednode]      (publicinfo) [below=of politicalinfotext] {Perceived Public Opinion};
\node[dnode]      (publicdecides) [left=of publicinfo] {Mass\\ Engagement};
\node[pnode]        (publicdemands) [left=of publicdecides] {Latent Public Demands};

% public Lines
% \draw[->] (draft.east) -- (publicdemands.west);
\draw[->] (publicdemands.east) -- (publicdecides.west);
\draw[->] (publicdemands.north east) -- (groupdecides.south west);
\draw[-] (groupdecides.south) -- (mobilizing.north);
\draw[->] (mobilizing.south) -- (publicdecides.north);
\draw[->] (publicdecides.east) -- (publicinfo.west);


% principal Nodes
\node[squarednode]      (principalinfo) [below=of publicinfo] {Perceived Political Consequences};
\node[dnode]      (principaldecides) [left=of principalinfo] {Principal Comments};
\node[pnode]        (principaldemands) [left=of principaldecides] {Principal Demands};
\node[squarednode]      (principalinfo2) [below=of principalinfo] {Perceived Principal Opinion};


% principal Lines
\draw[->] (draft.south east) -- (principaldemands.north west);
\draw[->] (principaldemands.east) -- (principaldecides.west);
\draw[->] (publicinfo.south west) -- (principaldecides.north east);
\draw[->] (principaldecides.east) -- (principalinfo.west);
\draw[->] (principaldecides.south east) -- (principalinfo2.north west);
\draw[->] (publicdemands.south east) -- (principaldecides.north west);

\end{tikzpicture}
\end{figure}
\normalsize

\begin{figure}[h!]
    \centering
        \caption{Rules ranked by number of comments posted to regulations.gov}
    \includegraphics[width = 6.5in]{Figs/rules-ranked-comments-per-year-1.png}
    \label{fig:rules-ranked}
\end{figure}

\begin{figure}[p!]
    \centering
        \caption{Major and non-major rules on regulations.gov}
    \includegraphics[width = 7in]{Figs/major-comments-density-1.png}
    \label{fig:rules-major}
\end{figure}

\begin{figure}[p!]
    \centering
        \caption{Rules on regulations.gov by priority level}
    \includegraphics[width = 7in]{Figs/priority-comment-density-1.png}
    \label{fig:rules-priority}
\end{figure}
\clearpage 



%\theendnotes
\singlespace
\small
\bibliographystyle{apsr.bst} 
\bibliography{mendeley.bib}
\end{document}

