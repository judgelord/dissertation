\documentclass{article}
\pagestyle{headings}
\date{\today} 

% --- section style -----------------------

% we don't go deeper than subsection (hopefully..?)
\setsecheadstyle{\large \raggedright \bfseries}
\setsubsecheadstyle{\raggedright \bfseries}
\setsubsubsecheadstyle{\raggedright \itshape} 

% numbering depth
\setsecnumdepth{subsubsection}  

 \usepackage{booktabs}
 \usepackage{cleveref}

\renewcommand{\eqref}{\Cref}
\Crefformat{equation}{#2#1#3}

% \usepackage{float}
% \let\origfigure\figure
% \let\endorigfigure\endfigure
% \renewenvironment{figure}[1][2] {
%     \expandafter\origfigure\expandafter[H]
% } {
%     \endorigfigure
% }



% LINKS 
\usepackage{hyperref}
\hypersetup{breaklinks = true,
            bookmarks = true,  
            colorlinks = true,
            citecolor = black,
            urlcolor = black,
            linkcolor = magenta}

% FIG CAPTIONS 
\let\newfloat\relax 
\usepackage{floatrow}
\floatsetup[figure]{capposition=top}
\floatsetup[table]{capposition=top}
\usepackage{setspace}

% inherit spacing names from memoir
%\newcommand{\OnehalfSpacing}{\onehalfspacing}
%\newcommand{\DoubleSpacing}{\doublespacing}

% SPACING FOR BLOCK QUOTES 
\expandafter\def\expandafter\quote\expandafter{\quote\OnehalfSpacing}

% FOR KABLEEXTRA 
\usepackage{booktabs}
\usepackage{longtable}
\usepackage{array}
\usepackage{multirow}
\usepackage{wrapfig}
%\usepackage{float}
\usepackage{colortbl}
\usepackage{pdflscape}
\usepackage{tabu}
\usepackage{threeparttable}
\usepackage{threeparttablex}
\usepackage[normalem]{ulem}
\usepackage{makecell}
\usepackage{xcolor}
\usepackage{ragged2e}

\usepackage{endnotes}

\makeatletter
\renewcommand\@makeenmark{%
  \textsuperscript{\normalfont\textcolor{blue}{\@theenmark}}%
}
\newcommand{\uncolormarkers}{%
  \renewcommand\@makeenmark{%
    \textsuperscript{\normalfont\@theenmark}%
  }%
}
\makeatother


\newcommand{\exclude}[1]{\StopSearching ##1\StartSearching}

\usepackage{ragged2e}

% \setlength{\parskip}{1em}
\title{Political Information in Bureaucratic Policymaking}
\author{Devin Judge-Lord} % \\ JudgeLord@Wisc.edu}

% FOR DAVE: %%%%%%%%%%%%%%%%%%%%%%%%%%%
% \let\footnote=\endnote
% \renewcommand{\footnotesize}{\normal}
%%%%%%%%%%%%%%%%%%%%%%%%%%%%%%%%%%%%%%%
\begin{document}
\maketitle
\abstract{This dissertation is about ordinary people's input on policies made by bureaucrats. 
% People may believe that their voices matter, but it is unclear if they do. % or ought to. 
It makes three main contributions:

First, drawing on scholarship on interest group behavior, social movements, and lobbying, I identify three distinct reasons for groups to mobilize ordinary people. Each logic suggests a different observable pattern of mass public engagement, and I analyze millions of public comments on thousands of agency rules to develop the first systematic measures of mass engagement in bureaucratic policymaking. 

Second, building on theories of political oversight, I theorize that mass public engagement in bureaucratic policymaking may alert elected officials to political opportunities and risks, affecting oversight behavior. I assess this argument by analyzing correspondence between Members of Congress and bureaucrats on proposed rules with and without mass engagement.

Third, I integrate these contributions on interest group lobbying and oversight into a broader theory of how mass mobilization may affect policy by producing potentially influential political information. I argue that there are four broad causal mechanisms by which lobbying may influence bureaucrats. Political scientists have thus far focused on the power of technical information, where insider lobbying is most likely to matter and outside lobbying is least likely to matter, thus largely overlooking mass engagement. This gap suggests that incorporating theories of social movement influence may advance bureaucratic politics scholarship and that bureaucratic politics may be a fruitful empirical ground for exploring social movement theories. To address this gap, I use my new measures of mass engagement and oversight to assess the effect of political information on bureaucratic policymaking.

Finally, two supplemental chapters assess causal processes through a case study of the environmental justice movement and an analysis of rulemakings where organizations randomly select lobbying strategies.

}
% FOR DAVE: %%%%%%%%%%%%%%%%%%%%%%%%%%
% \large
% \RaggedRight
%%%%%%%%%%%%%%%%%%%%%%%%%%%%%%%%%%%%%%
\doublespace


% TO DO: 

% --Yackee Annual Review 

% --Cueller article 

% -- Dave's Book

% --Discuss causal models 

% -- Hypotheses

% -- Discuss level, intensity, and contagion 


\newpage
\tableofcontents

\newpage
\section*{Summary}
To better understand the role of ordinary people in bureaucratic policymaking, 
I develop theories of why mass engagement occurs and how it may affect policy. To assess these theories, I tackle three related empirical questions: (1) Why does it occur?, (2) How does it affect the oversight behaviors of agencies' political principals?, and %These questions drive two initial empirical chapters.
(3)  Does mass engagement in bureaucratic policymaking affect policy?
% I then use my new measures of the political information that lobbying coalitions create by going public to test whether mass engagement explains variation in agency rulemaking and rules.% But first, I must develop a measure of ``going public.'' % and why it occurs.

\paragraph{Part 1. Why do agencies (occasionally) get so much mail?} %: Lobbying coalitions, mass comments, and political information in bureaucratic policymaking
% Scholars of bureaucratic policymaking have focused on the sophisticated lobbying efforts of powerful interest groups. Yet agencies occasionally receive thousands or even millions of comments from ordinary people. Why? Why do individuals comment when they seemingly have no new information to offer and no power to influence decisions? Who inspires them and to what end? How, if at all, should scholars incorporate mass commenting into models of bureaucratic policymaking? I argue that mass commenting produces political information about the coalition that mobilized it. 
% QUESTION 1\textbf{Puzzle:} 
Why do some rules receive many comments from ordinary people and some do not?
% Why do people comment on draft policies when they seem to have no new information to offer and no power to influence decisions? Who inspires them and to what end? 
% THEORY AND METHODS 1
Answering this question requires a theory explaining variation in mass engagement. Because the vast majority of comments are inspired by interest-group campaigns, finding their cause requires a method to link comments to the lobbying coalitions that mobilized them.  
To link individual comments to the more sophisticated lobbying efforts they support, I use text reuse and Bayesian classifiers to identify clusters of similar comments, reflecting formal and informal coalitions.
%Using new measures of public engagement in agency rulemaking, I identify the conditions under which it occurs and produces different politically-relevant information. 
% The dependent variable is the number of people engaged.
I theorize that lobbying coalitions' resources, opportunities, and public support explain variation in mass engagement %, which I measure in several ways. 
and that it will fit one of three patterns:
(1) Coalitions that perceive an opportunity to influence policy and have sufficient resources to do so will ``go public'' when they are disadvantaged in insider politics but have more public support than opposing coalitions. More public support yields more engagement, more effort per comment, and contagion beyond those mobilized directly. (2) Coalitions with less support may ``counter-mobilize'' with proportionally smaller effects. (3) Finally, coalitions may mobilize for reasons unrelated to the policy at hand, yielding similar mass engagement but with little sophisticated lobbying. 
Measures of mass engagement include 
%(1) total public comments, % $\sim$ zero-inflated negative binomial; 
(1) comments per coalition, % $\sim$ negative binomial; 
(2) effort per comment, % $\sim$ truncated normal; 
(3) share of comments per coalition mobilized indirectly (i.e. the potential for conflict spread).
Next, I test whether variation in engagement explains variation in oversight behavior (part 2) and policy outcomes (part 3).
% (4) type of campaign. % $\sim$ multinomial. 
%Model 1 is one observation per rule. Models 2-4 are one observation per coalition per rule. Explanatory variables include agency alignment with Congress and the president (models 1-4), coalition alignment and unity (models 2-4), whether a coalition is driven more by public or private interests (models 2-3).%, part of the DV in model 4).

%\paragraph{Step 2. Are elected officials more or less likely to engage after mass public engagement?} 
\paragraph{Part 2. Does mass engagement affect political oversight?} The political information signaled by mass engagement may serve as a ``fire alarm,'' altering principals to oversight opportunities or ``warning signs'' altering them to political risks.
When a coalition mobilizes successfully, %especially if it generates a perceived consensus in expressed public sentiments, 
elected officials ought to be more likely to engage on their behalf and less likely to engage against them.
% This suggests an addendum to Hall and Miler's (2008) finding that members are more likely to engage in rulemaking when they have been lobbied by a like-minded interest group.
% When interest groups lobby elected officials to engage in rulemaking, they may be more likely to engage when aligned with most commenters than when opposed.
% If politicians learn from political information, they will be even more likely to engage when lobbied by a coalition that includes a public interest group's with a large mass-comment campaign, and less likely when lobbied by a coalition dominated by private interests opposed by a mass comment campaign. 
% MEASUREMENT  2
To assess these hypotheses, I count the number of times Members of Congress engage the agency before, during, and after comment periods on rules where lobbying organizations did and did not go public. I then use text analysis to compare legislators' sentiments and rhetoric to that used by each coalition.
% Similarly, I asses the involvement of presidential appointees and the President's Office of Management and Budget before and after public comment, again comparing rules that were and were not targeted by a campaign (a difference-in-difference). 
% As a validity check, I also look for remarks by elected officials and judges on the level of public engagement.
Dependent variables include 
(1) the number of comments from Members of Congress on the rule %(total, those mentioning mass comments, and those mentioning organizations in the coalition), %All  $\sim$ zero-inflated negative binomial. 
(2) the share of supportive congressional comments, %  $\sim$  beta. 
(3) the similarity of words in comments from the coalition and Members of Congress. 

\paragraph{Part 3. Does mass engagement affect rulemaking and rules?} 
I theorize that the effects of political information on policy depend on the extent to which the strategic environment allows change and how political information is processed, both directly within agencies and indirectly through other actors (e.g. Members of Congress) whose appraisals matter to bureaucrats.
The main dependent variable is change in the rule text.
%Different inputs may yield different results: 
I systematically identify changes between draft and final rules, parse these differences to identify meaningful policy changes, and compare them to demands raised in comments to measure which coalition got their way. However, assessing policy change is difficult. Thus, I also use other measures of agency responses to lobbying efforts. 
% \end{document} % NO SOURCES BEFORE THIS POINT

\newpage
\section{Introduction} \label{intro}
\section{Introduction}
\subsection{Why study rulemaking?}
% \section{The Importance of Studying Rulemaking}
% Mobilization may increasingly target rulemaking because it is how most policy in the U.S. is now made. 
With the rise of the administrative state in the United States, federal agencies have become a major site of policymaking and political contestation. In the years or decades between legislative enactments, federal agencies make legally-binding rules interpreting and reinterpreting old statutes to address emerging issues and priorities. Ninety percent of new policy that carries the force of law is now made in the bureaucracy rather than in Congress \citep{West2013WhoControl}.\footnote{I use policy, law, and regulation as nested concepts. My methods generally apply to all policy texts whether they carry the force of law or not. Many public and private organizations, including agencies, have policy statements that are not legally binding. My empirical subject is rules that do carry the force of law based on some authorizing legislation. I use rule (a more technical term) and regulation (a more colloquial term) interchangeably.}
Examples are striking: %the effect of the Dodd-Frank Wall Street Reform and Consumer Protection Act was largely unknown until the specific regulations were written, and it continues to change as these rules are revised. 
Congress authorizes billions in farm subsidies and leases for public lands, but who gets them depends on agency policy. In the decades since the last major environmental legislation, agencies have written thousands of pages of new environmental regulations and thousands more changing tack under each new administration. This constant revision of administrative rules makes them distinct from legislation \citep{Wagner2017DynamicRulemaking}.
% And these revisions can be significant. In 2006, citing the authority of statutes last amended in the 1950s, the Justice Department's Bureau of Prisons proposed a rule restricting eligibility for parole. In 2016, the Bureau withdrew this rule and announced it would be requiring fewer contracts with private prison companies, precipitating a 50\% loss of industry stock value. Six months later, a new attorney general announced these policies would again be reversed, leading to a 130\% increase in industry stock value. %Like many rulemaking debates, industry and advocacy groups spent millions of dollars lobbying on this issue. Few rulemakings, however, receive this level of public and presidential attention. In the majority of rulemakings, few participate, and we do not really know the extent to which participants get what they lobby for.% (but see Yackee and Yackee 2006)
Rulemaking clearly matters.

Less clear, however, is what the new centrality of agencies and rulemaking means for the practice of American democracy. In addition to the complex relationships agencies have with the president and Congress, agencies have complex and poorly understood relationships with the public and advocacy groups. Relationships with constituents may even provide agencies a degree of ``autonomy'' from their official principals \citep{Carpenter2001}. While some suggest that requirements for agencies to solicit and respond to public comments on proposed rules allows ``civil society'' to provide public oversight, others note that participants in rulemaking often represent elite parts of society \citep{Seifter2016ComplementaryPower} and business interests \citep{Yackee2006a}. Yet agency decisions are also the target of protests and advocacy campaigns.\footnote{For example, along with 50 thousand protesters in Washington D.C., the State Department Received 1.2 million comments on the Environmental Impact Statement for the Keystone Pipeline. Similarly, along with the thousands of protesters supporting the Standing Rock Sioux protest to the Dakota Access Pipeline, the Army Corps of Engineers received hundred of thousands of comments. Along with 22 million comments on the Federal Communications Commission's Open Internet rules, activists are organizing online protest actions. On each of these issues, advocacy activity has been followed by legislative or executive action.} and the notice-and-comment process purports to be an avenue of citizen voice. Big red letters across the top of the Regulations.gov homepage solicit visitors to ``Make a difference. Submit your comments and let your voice be heard.'' A blue "Comment Now!" button accompanies a short description of each draft policy and pending agency action. While most rules receive little attention, the ease of online commenting and mobilizing has created exponential increases in the number of rules where which hundreds of thousands of citizens participate (see figure \ref{fig:comments}). Occasionally, large numbers of citizens are paying attention.

\begin{figure}[!hb]
\caption{Number of Public Comments Total (Left) and Under 1 Million (Right). The most commented on rules have been published by the Federal Communications Commission (FCC, omitted from this plot), the Environmental Protection Agency (EPA), the Department of Interior (DOI), the Bureau of Ocean Energy Management (BOEM), the Consumer Financial Protection Bureau (CFPB), and Fish and Wildlife Service (FWS).}
\includegraphics[width= 3.5in]{number_of_comments.pdf}
\includegraphics[width = 3.5in]{comments_under_1m.pdf}
\label{fig:comments}
\end{figure}

It is even less clear whether actions by average citizens make a difference in agency policymaking. Many may believe that they do, but the mechanisms are not obvious. Indeed letter writing and other forms of mass mobilization do not have a clear place in political scientists' theories of bureaucratic politics. This lack of scholarship may be the result of both a general suspicion, rooted in certain theories of strategic behavior, that mass politics affects unelected career officials as well as a normative assumption that policy ``implementation'' is no place for contentious politics. Neither the bureaucrat who asserts that rules are the result of scientific analysis nor the political scientist who asserts that rules are the result of bureaucrats strategically selecting their most preferred policy within institutional constraints offer an explanation for why an agency would receive millions of public comments or why they would matter.

In this 
dissertation,
% paper,
I argue that if we appreciate agency policymaking as a site of contentious politics, mechanisms emerge by which mass mobilization may affect both the strategic environment and ideological perspectives of those who write agency rules. While the theory that I assemble attempts to describe the relationship between mass mobilization and agency decisionmaking in general, my empirical focus is on the role of  organized campaigns targeting notice-and-comment rulemaking processes, with special attention to environmental 
%and financial 
regulation.



 \subsection{Puzzle: Why mobilize?}
% \section{Why mobilize?}

Prior to the 22 million public comments on the Federal Communications Commission's 2017 Open Internet rule,\footnote{It is yet unclear how many of these comments are from real people.} two of the most commented on rules set standards for mercury emissions from coal and oil-fired power plants. Among other things, the Environmental Protection Agency (EPA) solicited ``comments on whether there would be a basis for considering area sources to be significantly different from major sources,'' ``on the adequacy of the restrictions associated with bypass conditions regarding maintaining LEE status" and ``on the proposed revisions concerning [equations' 1a and 1b] usefulness in calculating the maximum potential emissions rate from an emissions averaging group'' (EPA 2011). LEE status is not defined in the notice soliciting comments, and equations 1a and 1b are surely inaccessible to most citizens. Yet these two proposed rules received 942,483 comments. 

One comment, from the United States Council of Catholic Bishops, read: ``While we are not experts on air pollution, our general support for a national standard to reduce hazardous air pollution from power plants is guided by Catholic teaching, which calls us to care for God’s creation and protect the common good and the life and dignity of human persons, especially the poor and vulnerable.'' Bishops are not known to closely follow power plant regulations. Their moral authority was mobilized by activists who wanted stricter regulation of mercury. Groups mobilizing on the mercury rules including environmental and health groups and industry competitors, including the owners of Nuclear, Natural, geothermal power plants. 

In the official, legally-required response to comments, the EPA did not discuss God's creation, dignity, or the poor.  Indeed, the EPA asserted that mercury levels are a matter of science, not not a matter of justice. But the EPA did implicitly assert a definition of the public good when it used studies of mercury's aggregate public health effects on the U.S. population to set emissions standard.\footnote{As Wagner (1995) %CITE
notes, ``agencies exaggerate the contributions made by science...in order to avoid accountability for the underlying policy decisions. Although camouflaging controversial policy decisions as science assists the agency in evading various political, legal, and institutional forces, doing so ultimately delays and distorts the standard-setting mission'' (p. 1617). She goes on to say that  ``While the APA mandates a process for public involvement, it provides almost no protections to ensure that agencies will explain the substantive bases for highly complex or technical rulemakings in a way that the lay public can readily understand and challenge'' (1656) and that ``Mischaracterization of the entire standard-setting endeavor as resolvable by science results in significant obstacles to democratic participation'' (1674). Similarly, Harvey Brooks (1984) notes that ``The modern nation risks being no longer recognizable as a democracy, either representative or plebiscitary, if more and more policy areas are excluded from public participation because of the technical complexity.''} 
Then, as required by the Supreme Court, it justified the same standards with cost-benefit analysis in a revised proposed rule, concluding that for every dollar spent to comply with the regulation, the U.S. public receives up to nine dollars in health benefits (EPA 2007). If this is how decisions are made, why did the EPA receive nearly a million letters? Why would citizen opinions matter? 

% Wagner: A variety of commentators have suggested that agencies may seek increased legitimacy or decreased political accountability by disguising their policy judgments as science. See Majone, supra note 18, at 15 ("Traditionally, government regulators have sought legitimacy for their decisions by wrapping them in a cloak of scientific respectability.");Roberts et al., supra note 26, at 120 ("Too many of the participants [in science-policy decisions] have good reasons not to distinguish scientific evidence from policy preferences, not to analyze carefully the various sources of technical disagreement, and not to accept responsibility for some decisions or judgments."). Beyond these common sense observations scattered at points in articles and books, there has been surprisingly little scholarly discussion of the comprehensive existence of or reasons for a science charade in regulation.

In contrast to the science-based objectivity presented by the EPA, political scientists, building on law and economics scholarship, offer a different theory of bureaucratic decisionmaking rooted in the policy preferences and strategic behavior of agency leaders and their political principals: Congress, the president, and the courts. They find that political principals do constrain agency action but also leave room for agencies to move policy toward their own ``ideal point.''\footnote{Though political scientists make diverse assumptions about what this ideal is and how to measure it.} Science may or may not inform preferences, but preferences and the power to realize them in a strategic environment are, these scholars say, are the proximate cause of policy. These scholars would see the Mercury Rules as the result of EPA officials writing a policy as close as possible to their ideal policy given their strategic constraints. 

But if the strategic model is correct, why write letters to the EPA? The EPA administrator has their preferences and the public has no direct power over their decisions. Why not write to the president or members of Congress who influence EPA's strategic calculations and are more directly accountable to public opinion? 

Other political scientists, along with scholars of public administration, organizational behavior, and sociology offer alternative theories that, while less parsimonious, squarely address how the process of soliciting and responding to public comments may influence agency policy. They find that agency staff develop relationships with those who regularly participate in policy processes: most often businesses but also professional associations and activist organizations. Relationships draw on and reproduce organizational identities and reputations. This scholarship has revealed a good deal about how organized groups lobby agencies and why they succeed, but it has yet to address why these groups sometimes mobilize thousands of citizens to write letters or protest agency decisions. The theoretical foundations for why mass mobilization may matter is underdeveloped and we lack empirical research on how it may affect agency policymaking. 

In this dissertation, 
%In this paper, 
I address this theoretical and empirical gap in our knowledge on the role of mass mobilization in bureaucratic policymaking. I expand and integrate the above theories to develop testable hypotheses and analyze rule-related texts %and field experiments 
to explore whether mass mobilization matters and, if so, why. 

I argue that if mass mobilization indirectly affects the strategic environment it does so by signaling grass-roots political power to elected officials and if mass mobilization directly affects agency policymaking it does so by evoking organizational identities and reputations. 
Like the vast majority of letter-writers, Catholic Bishops contribute little to the technical aspects of epidemiology, mercury regulations, or cost-benefit analysis. If they influence agency policymaking it is 
by signaling a threat of political backlash or 
by persuading bureaucrats directly that moving policy in a certain direction is the appropriate thing for the agency to do. 
The next two subsections address these indirect and direct mechanisms in more depth. A third discusses why we may still observe mobilization in the absence of influence. 
% Whereas social movement scholars and political scientists have focused the behavior of elected leaders, I focus on the latter pathway of direct persuasion. 

Precisely identifying who participates, how, and the dimensions of disagreement over time is key to any study attempting to discover whose ideas end up in policy. These are descriptive questions but they are not easy ones. In the empirical section, I address three descriptive questions: who participates, who lobbies together, and who wins\footnote{By \textit{who wins?} I mean whose ideas end up in policy. This is distinct from measuring \textit{influence} with respect to a counterfactuals or constellations of ideal points. I measure what people say they want and whether they get it. For example, I measure whether rules where commenters requested consideration of environmental justice issues were more likely to address environmental justice issues in the final draft}. Here, \textit{who wins?} is descriptive rather than causal. While insufficient to infer specific causal influence, policy moving in one's preferred direction may indicate that one is aligned with those who have power in that policy process. %There are many potential causes for policy outcomes matching certain policy demands, and I proposed field experiments to test several of them.
% \section{Mechanisms of Influence}

\subsection{Indirect Influence: Signaling a threat of backlash}
% \section{Theories and Case Selection}
Why might mass mobilization matter? The literature on bureaucracy offers two types of explanations rooted in either strategic behavior or organizational norms. Political scientists often focus on strategic context. Public administration and management scholars focus on organizational logics and identities. I begin with the ``indirect'' mechanisms that theories of strategic behavior suggest. 

In the U.S. context, there are three main mechanisms by which mass mobilization could affect an agency action by changing the strategic context. 
Mass mobilization may signal power to influence the responses to agency action from the White House, Congress, or courts. Many rules receive little attention from these other institutions, but all three significant powers to reward, sanction, or reverse agency actions \citep{Yaver2016}. 

% Arnold, Logic Of Congressional Action p 217:
% success depends in poat of the length and complexity of the causal chain connecting a policy instrument with its policy effects. When a causal chain is short and simple, citizens are more likley to know which policy instrument will produceth appropriate effects and are beter able to monitor the performance of their repre3sentatives,. When a causal chain is long and complex, or when a problem in society stems from multile causes, citizens may be incapable of doing the appropriate policy analysis and political anslyss ." 
% p 272 "resoonsiveness to both attentive and inattentive publics avaries depending on the procedures that govern how legislators requd their positions" 
% "The model of citizen's control that I have been discussing is essentially an auditing model. Citizens do not instruct legislators on how to vote, not do thay necessairlily have well-defined policy preferences in advance of cogressional action. Legialators neverthless have strong incentives to consider citizens' potential preferences when they are deciding how to vote for fear that making the wrong choice might triggger and unfavorable audit." 
% 


The White House has several tools to influence agency decisions \citep{Yackee2009a,Simon1954}. These executive orders \citep{Mayer1999}, appointments \citep{Doherty2014,Lewis2008,Wood1988}, budgets \citep{Whittington2003}, and review of proposed policies \citep{HAEDER2015InfluenceBudget,Acs2013}. 
Congress also has several tools to influence agency decisions. These include the power of the purse \citep{Fenno1986,Bolton2015}, oversight, and new legislation. Some research suggests that this constraint is larger under divided government \citep{Yackee2009b} and that under divide government Congress tends to divide power among multiple agencies \citep{Farhang2016}.
The anticipation of judicial review makes courts relevant to rulemaking. Some rules are also made under court-imposed settlement or with judicial deadlines. Judicial opinions may also call on Congress to act \citep{Yaver2017}.
Despite these mechanisms and because of conflicts among them, agency staff maintain significant power over agency decisions. For example, Congress is less assured of compliance when power is divided \citep{Yaver2016}.

% more good stuff
%\subsubsection{Legal Scholarship}
Legal scholars' case studies of specific rulemaking process offer an additional relevant body of research. Coglianese (1997) finds that litigation is a common extension of rulemaking. Indeed, unlike legislative lawmaking, rulemaking takes place in the clear and present shadow of judicial review (Rossi 2001). Stakeholders can challenge a rule in court on a variety of procedural grounds and on statutory interpretation. This scholarship suggests those who succeed in rulemaking are those with the resources and experience to succeed in court. Costly mass comment campaigns could be signaling the ability and willingness to spend resources to challenge the rule in court. 

Mass mobilization may signal political risks or benefits of engaging in agency policymaking to members of Congress and the White House. It also may signal to the agency that activists have the capacity to sustain pressure through the policy process \citep{Coglianese2001}, including challenging the policy in court, a constant threat agency policies. Thus, mass mobilization may act as a signal of political power that  reshape rule-writers' beliefs about their strategic context. 



\subsection{Direct Influence: Mobilizing identities and reputations}

I now turn to the direct-influence pathway: the ability of social movements to mobilize ideas, evaluative frameworks, and claims about what is appropriate and right that may affect bureaucratic decisions. %I use mobilization around the idea of ``environmental justice'' as an example where direct influence may be visible. 

Organizational theory suggests additional mechanisms by which mass mobilization may influence bureaucratic decisions more directly. Here the causal process involves mobilizing norms and ideas right and wrong rooted in individual and institutional identity. Because concepts of mission, reputation, and the validity of claims are intertwined, these mechanisms are difficult to precisely define. Nevertheless, scholars have identified several types of direct influence. One important factor in decision making is personal and institutional reputation \citep{Carpenter2001}. This can take several forms. For example, individuals trained as scientists and agencies that cultivate reputations for producing valid science may be persuaded by rigorous scientific claims. Similarly, individuals who identify strongly as public servants and agencies with reputations for public responsiveness may be persuaded by claims about public or "stakeholder" opinion. In general, claims that resonate with the problems an agency has been tasked with solving and the means it has to solve those problems are likely to be well received.  

% \subsubsection{ Agencies as Policymaking Venues}
% the good stuff
When political scientists ask whose interests and ideas become law, they have generally focused on the behavior of legislatures, how the executive branch drives legislation, and how the courts review it. Compared to legislative, executive, and judicial institutions, the administrative state is a recent development in American government and theory has not kept pace with the rise in bureaucratic policymaking. 

I argue that theories of bureaucratic policymaking have been characterized by constraining assumptions about what bureaucracies ought to do.  Normative assumptions that \citet{Wilson1967} identified half a century ago, and corresponding scholarly silos, have persisted. This has led to lines of research talking past each other and often failing to engage broader theories of policy change. In particular, I argue that the pervasive implicit assumption that bureaucrats ought to be neutral implementers implies that politics in agency policymaking is inherently undesirable, leading many scholars to focus on compliance with political principals and overlook the role participation and ideas. For example, scholars assume that agencies ought to be engaged in implementing legislation and executive orders. However, most rulemaking takes place many years or decades after its authorizing legislation under a different Congress and with little attention from the White House until the very final draft. Rules that do not follow from contemporary Congressional or executive priorities are often assumed to reflect bureaucrats going rogue or being captured by interest groups. Such studies suffer from a lack of attention to the complex political process of rulemaking. 

Accountability to elected officials has been central to the study of bureaucracy \citep{Epstein1999,Huber2002,McCubbins1984,Wilson1989,Potter2016Slow-RollingRulemaking,Lowande2018PoliticizationAgencies} %add Meier and O’Toole 2006; West 1995; Wood and Waterman 1994
Viewing agencies as \textit{agents} has prevented scholars from incorporating new insights about the endogenous relationship between policy and politics. I suggest rulemaking is better studied in the way that scholars study policymaking in specialized congressional committees than with an unrealistic dichotomy of sincere implementation versus capture or disloyalty. Normatively, accountability to political principals only one of several important concerns. Empirically, it is often unclear what accountability means and there is ample evidence that it may not be the primary driver of bureaucrat behavior.

In contrast to the dominant view of agencies as \textit{agents}, a growing literature in political science draws on scholarship in law and public administration as well as studies of agenda setting and lobbying in legislative policymaking to better understand agencies as policymaking bodies. Public administration and legal scholars have been more attentive to the prominent role of interest groups.  Kerwin (2003) notes that ``Interest groups could find few modes of government decision making better suited to their particular strengths than rulemaking.'' This research finds business groups to be most successful class of commenters in rulemaking \citep{Yackee2006a} especially when lobbying together, often, or unopposed (Nelson and Yackee 2012) and when lobbying across multiple venues \citep{Yackee2015}. Importantly, this literature notes that the currency of lobbying is information (Hall and Deardorf 2006), which includes both science and policy ideas \citep{Jones2005}. Kirilenko (2014) and Yackee and Yackee (2006) both find evidence that comments from sophisticated interest groups like businesses seem to influence rules. These scholars offer one set of answers to the question of who wins: those who succeed in rulemaking tend to be business interests, repeat players, those who lobby together, and those who lobby unopposed. They succeed because they bring in new voices and send unified messages at higher amplitudes, creating perceptions of political consensus.


%There may be an inverse relationship between how responsive agencies are to political principals and to the public \citep{Lewis}.

%Yet public administration and legal scholarship rarely address how interest groups gain political power in the first place. 

% more good stuff
A second major contribution to theory in this area is Carpenter's  research explaining bureaucratic autonomy \citep{Carpenter2001,Carpenter2012}. Rather than asking how bureaucratic practices fit with normative assumptions, he asks how agencies became independent policymaking bodies. Responding to principal-agent literature that has focused on the presidential and Congressional control, Carpenter finds much more complex sets of relationships that explain organizational power and behavior. One of the main tools he gives us for understanding the source of bureaucratic autonomy is the concept institutional reputations. Bureaucrats and the institutions they animate develop reputations for certain competencies: for example, for expertly adjudicating scientific claims, for effectively executing policy aimed at a given goal, or for divining the public interest. Reputations for expertise, effectiveness, or representativeness reflect the mixed roles assigned to the bureaucrats: advisors, implementers, and policymakers. 
% Like Carpenter, I call attention to the fact that agency policy shapes the coalitions that surround and influence it. I depart from Carpenter's narrative in that I do not focus on cases where agencies intend to have these effects. Whereas Carpenter is interested in how bureaucrats intentionally shape lobbying coalitions, I am interested in the endogenous relationship between policy and coalitions, intended or not. While not the focus of his study, Carpenter notes that relationships also evolve in unintended ways. 
%Policy may pro-actively recruit group support, but may also be reacting to political pressure \footnote{For example, an industry may successfully lobby to be reclassified to face lower pollution regulations, perhaps those faced by their competition, thus turning competitors into allies for future policymaking and increasing the size of the coalition for the lower standard.} or be an unintended side effect of action the agency sees as imperative.\footnote{For example, new science on the health hazards of mercury led to pollution controls that differentiated coal power plants and gas power plants reshaping coalitions by turning allies on former air quality policymaking into competitors in future rounds.} %Nevertheless, 
Carpenter and related scholars thus offer a second possible set of predictions for which movements are successful: those who succeed in rulemaking tend to be those with close relationships with the agency, conditional upon (and because of) how those relationships support the agency's reputation for expertise, competence, and representativeness. %Furthermore, lobbying coalitions, their relationship to the agency, and thus their success are functions of past agency policy. 

%Some scholars attempt to estimate the preferences of bureaucrats. Instead, I take 
% Carpenter's findings of close relationships between interest groups and agencies is a potential explanation for why some groups appear to have influence, i.e. because they are aligned with agency ideologies. As my core contribution is to assess how groups are empowered or disempowered rather than how agencies are empowered or disempowered, I focus on discovering which groups' comments are related to changes in rules regardless of whether this is what certain bureaucrats also wanted. 




% \subsubsection{Reputations for accountability, representation, equity, and expertise }

% [How specific organizational identities and reputations drive decisionmaking]

\subsection{Mobilizing for Recruitment}

A third possibility is that mobilization around bureaucratic decisions is unrelated to the possibility of affecting policy and primarily a way to recruit and engage members or raise the profile of the movement. If this is the case, behaviors like protesting and mass commenting on rules are largely epiphenomena to unrelated kinds of politics. Organizers may know that mobilization has minimal effects, but lead members to engage as means to other ends. Many of the mobilized themselves may doubt their efficacy but still take advantage of the opportunity to protest. 

% The remainder of this paper presents a case study and an empirical test of whether comments influences rulemaking.



[Elaborate on org behavior]

\subsection{Advancing Theories of Bureaucratic Policymaking}

Quantitative studies of bureaucratic policymaking in political science tend to collapse the time dimension and rarely consider the historical context in which each rule is made. For example, scholarship exploring how political context affects timing and delay in rulemaking models rules as if they are independent of each other and independent of the date they began (see Potter 2017). These studies also tend to focus on the degree to which agency policymaking reflects presidential, congressional priorities and, occasionally, interest group priorities. Political scientists most often ask if agencies are doing what the president wants, what congress wants, or something else. They find significant amounts of ``something else,'' but theory inconsistent on what it is and where it comes from. 

%bad stuff


\section{Methods}
This project makes two core contributions. First, I introduce new methods to 
%test theories about
measure the formation of lobbying coalitions, their demands, and whether they got what they asked for. 
% and of specific actors within coalitions
Second, I employ field experiments to test mechanisms by which mass mobilization may influence bureaucratic policymaking. 
%Rulemaking gives specific meaning to legislation and thus governmental force to political ideas. Like other policies, regulations also shape the terrain for future politics. 

\section{Measuring indirect influences}
To assess the direct influence pathway, I estimate the extent to which mass mobilization around bureaucratic actions makes members of congress or White House officials more likely to engage or react and whether such mass mobilizations is salient in subsequent litigation. 

Congressional attention to agency actions can be observed in several ways. Members of Congress who sit on oversight committees raise issues in oversight hearings, reports attached to each agency's budget appropriation, and in personal letters addressed to agency officials. Using text reuse methods I will identify when policy issues raised in draft rules and rule comments attract positive or negative attention from legislators. Using texts has several major advantages over previous measures of congressional attention and sentiment such as partisanship \citep{Yaver2016,Lewis2008}, changes in budget size, or the length of appropriations reports. Unlike partisanship, it is issue-specific and does not require assumptions about agency partisanship. While budget changes may reflect real costs, the many reasons that budgets change make it difficult to attribute changes to particular agency actions. The length of appropriations subcommittee reports may indicate the amount of attention committees pay to an agency but they do not vary significantly over time and do not indicate whether committee attention is positive or negative. 

[President and Secretary]

There are two ways to assess the courts as an indirect pathway where mobilization leads to influence. First, mass mobilization may increase the credibility of the threat of litigation. Second mass mobilization may influence the outcome of subsequent court cases over the rule. Both are difficult to measure. The first I measure with a combination of the litigation history of mobilizing groups and specific references to litigation in the comments. The second I assess by identifying instances where courts reference the number or direction of comments or other forms of protest in their decisions and compare rules to the rules under consideration in those cases. While this is rare, legal scholars have noted that " If the validity of a final regulation is challenged in court, the court's review will be based in significant part on how well the agency responded to the public's comments"  (Wagner 1995).

\subsection{Measuring Policy Change}
Policies may shape and be shaped by many forces, including the collective action of citizens, expert opinions, and businesses interests. Yet the drivers and consequences of policymaking are difficult to disentangle. Business groups may fund scientists or advocacy campaigns to preempt or undo costly regulations. Experts and policymakers may inspire broader civic mobilization, and citizen mobilization may, in turn, shape the priorities of experts and policymakers. Some policy debates divide along lines of citizen and corporate interest or expert and popular opinion, but many entail various clusters of claims regarding the public interest, expertise, and business interests: claims about the public good, scientific truths, and the proper role of government. Inferring policy demands from identity alone and assuming static coalitions may miss much of the story. 



\subsection{Data}
I focus on bureaucratic policymaking because, due to its sheer volume, it is both rich in opportunities to see different types of political mobilization, organization, and power at work and incompletely understood by political scientists. Specifically, I focus on agency rulemaking, a key part of U.S. policymaking that offers analytical leverage. Rulemaking is a process where agencies must solicit and respond to public comments on regulations (rules) before they carry the force of law (see Figure 1). Draft rules are published in a Notice of Proposed Rulemaking (NPRM). Occasionally, comments are also solicited before the draft rule is published through an Advanced Notice of Proposed Rulemaking (ANPRM). Intriguingly, while this process originally aimed to promote direct democracy and citizen voice, it is now generally seen as a mechanism to engage expertise (Coglianese 2006). Furthermore, research finds that the comment process actually favors business interests (Yackee and Yackee 2006). %Despite a large number of case studies, largely from legal scholars, our systematic understanding of the politics of rulemaking is thin.

% diogram of rulemaking  and commenting 
\begin{figure}[h!]
\label{inputs}
\caption{The Textual Record of Agency Rulemaking}
%\begin{table}
\begin{tabular}{@{\extracolsep{5pt}}cccccc}
 &  & &  \\
 & &\multicolumn{3}{c}{(Public Comments)}\\
 & & &$ \downarrow $& \\
\fbox{Inputs} & $\longrightarrow$ & \fbox{Proposal Text} &$\longrightarrow$ & \fbox{Outcome Text}\\
 & & & \\
List of Statutory Authorities &  & Proposed Rule & & Final Rule\\
(Advanced Notice)  &   &  & &   (and Response to Comments)\\
(Comments)  &   &  & &   \\
\end{tabular}
%\end{table}
\end{figure}
 

Rich data on several decades of rulemaking are available but have yet to be fully utilized by scholars.  Agencies publish draft rules, and comments received by interest groups, experts, and citizens. This offers leverage to identify the players, winners, and losers and to track those participating in the policy process over time. Rulemaking records often also cite the statutes, executive orders, and court cases that form a rule's historical institutional context. Some of this information, along with draft and final rule publication or withdrawal dates, is summarized since 1981 in the Unified Agenda of Regulatory and Deregulatory Actions (reginfo.gov). From 1994 onward, the text of most proposed draft rules, final rules, and summaries of comments received are published in the Federal Register (federalregister.gov). The result is the text of more than 70 thousand rules and . Finally, I collected text of over 7 million comments from 2002 onward via regulations.gov's API.

With the text of over 70 thousand regulations published since 1981 and over 7 million of the public comments on regulations since 2002, the second chapter of this dissertation will sketch the broad outlines of rulemaking in the American political context: who participates, how often rules are contested, whose ideas and interests are reflected in the text of rules, and who wins with different patterns of mobilization and contestation (or non-contestation). 

To make the project reasonable, the remaining chapters focus on [three] policy areas that have seen the highest levels of mass mobilization: [environmental, financial services, and communications technology]. To identify environmental rules, I select all rules made by the environmental protection agency and rules made by other agencies that cite president Clinton's executive order on environmental justice or president Obama's executive order on climate adaption. This allows me to consider how the same environmental problems may be addressed by different agencies. Financial services regulations are those that cite the Dodd-Frank act. Communications technology regulations are those proposed by the Federal Communications Commission.

%I also look closely at rules that end up before the Supreme Court. These rulemaking processes deserve extra attention for two reasons. First, regardless of how contentious they were at the rulemaking stage, these rules are key to understanding the nature and limits of executive power as a policymaking venue. Second, because all rulemaking is done in the shadow of judicial review, who wins in court and why shapes the political terrain of future rulemaking, empowering some groups with credible threats of litigation and disempowering others. References to court cases and implied threats of litigation are common in interest group comments, but to my knowledge, no study has looked systematically at how they affect rulemaking. Conversely, scholarship has not systematically assessed how mobilization and contestation in a rulemaking process affect judicial review. 

%%%%%%%%%%%%%%%%%%%%%%%%%%%%%%%%%%%%%%%%%%%%%%%%%%%%%%%%%%%%%%
\section{Theory} 
% MAIN QUESTION 
\paragraph{Incorporating mass engagement into theories of bureaucratic policymaking.}
How, if at all, should scholars incorporate mass engagement into models of bureaucratic policymaking? 
I argue that mass engagement produces potentially valuable political information about the coalition that mobilized it.
Thus, depending on how agencies process political information, ``going public'' may occasionally be an effective strategy for organizations to influence policy, both directly and indirectly.

% % PREVIEW 
\paragraph{Dissertation outline.}
Does mass engagement in bureaucratic policymaking affect policy? This question drives the my project. However, two questions must be answered first: (1) Why does it occur? and (2) How does it affect political oversight? These questions drive two initial empirical chapters. Thus, my analysis has three steps.

The next section builds a causal theory for each step. First, I theorize that activists' opportunities and strategies and latent public opinion drive engagement. Second I theorize that variation in mass engagement will lead to variation in political oversight. Third, I theorize that mass engagement may affect rulemaking both directly, and indirectly via its affect on political oversight.

The following section outlines methodologies to assess my three overarching questions and their component parts. Theses methods rely on analysis of comment and policy texts as large-n observational data. 

To explore causal arguments, the last two chapters of the dissertation will explore historical and experimental case studies. My historical case is the environmental justice movement, relying on all rules where ``environmental justice'' is raised in the comments and quantitative and qualitative assessment of agency responses. This prospectus includes preliminary analysis of the environmental justice case study. I find that responsiveness to activist comments varies in predictable ways across agencies but I find no evidence that the total number of comments affects rules. My experimental cases are not included in this prospectus. They will be rules selected by organizations that have agreed to randomly assign specific targets of their mass comment campaigns. While these few cases will not provide necessary power for statistical tests, the responses of the public, elected officials, and rulewriters may help illuminate causal mechanisms.


\subsection{Step 1: Why agencies (occasionally) get so much mail} \label{whymail-intro}
%: Lobbying coalitions, mass comments, and political information in bureaucratic policymaking

Scholars of bureaucratic policymaking have focused on the sophisticated lobbying efforts of powerful interest groups. 

[INSERT - INFORMATION IS THE CURRENCY OF INTEREST GROUP LOBBYING]

\begin{figure}
    \centering
    \caption{The Classic Model of Interest Group Lobbying in Bureaucratic Policymaking}
    \label{fig:causal-classic-lobbying}
\tiny
\begin{tikzpicture}[%
    node distance=1.2cm,
    auto,
    text width=1.5cm,
dnode/.style={diamond, align=center, aspect=2, fill=green!5,draw=green!60, very thick, minimum size=2cm},
squarednode/.style={rectangle, align=center, aspect=1, draw=red!60, fill=red!5, very thick, minimum size=1cm},
pnode/.style={ellipse, align=center, aspect=1, draw=black!60, fill=black!5, very thick, minimum size=1cm},
title/.style={rectangle, align=center, aspect=1, minimum size=2cm},
]
% Draft 
\node[dnode]      (draft)                     {Draft Policy};



% Group Nodes
\node[pnode]      (groupdemands) [right=of draft] {Group Demands};
\node[dnode]        (groupdecides) [right=of groupdemands] {Lobbying};
\node[squarednode]      (groupinfo) [right=of groupdecides] {Technical Information};

% policy 
\node[dnode]      (policy)       [right=of groupinfo] {Policy Response};
\draw[->] (groupinfo.east) -- (policy.west);
% \draw[->] (publicinfo.east) -- (policy.west);
% \draw[->] (principalinfo.east) -- (policy.south);
% \draw[->] (principalinfo2.east) -- (policy.south);

% Group Lines
\draw[->] (draft.east) -- (groupdemands.west);
\draw[->] (groupdemands.east) -- (groupdecides.west);
\draw[->] (groupdecides.east) -- (groupinfo.west);

% Titles
% \node[title]      (1) [above=of draft] {Policy};
%\node[title]      (2) [above=of groupdemands] {Preferences};
%\node[title]      (4) [above=of groupinfo] {Information/ Signal};
%\node[title]      (3) [above=of groupdecides] {Observed Behavior};


\end{tikzpicture}
\end{figure}
\normalsize

Yet agencies occasionally receive thousands or even millions of comments from ordinary people. 
%Why? Why do individuals comment when they seemingly have no new information to offer and no power to influence decisions? Who inspires them and to what end? How, if at all, should scholars incorporate mass commenting into models of bureaucratic policymaking? I argue that mass commenting produces political information about the coalition that mobilized it. 
% QUESTION 1
% subsubsection{Puzzle} 
Why do people comment on draft policies when they seem to have no new information to offer and no power to influence decisions? Who inspires them and to what end? 
% THEORY AND METHODS 1
Answering these questions requires a method to link comments to coalitions and a theory explaining variation in mass engagement.  

This section offers a definition of mass engagement, an argument that it is best understood as an outside lobbying tactic, and theory predicting different patterns of engagement depending which of three reasons organizations may have for launching a mobilization campaign. In the next section, I develop methods to measure these patterns.



% DEFINITION
\subsubsection{Defining mass engagement}
Political scientists often define civic engagement as writing to government officials, signing petitions, attending hearings, attending protests, or donate to a political campaign. While donating is more common in electoral politics, activists frequently attempt to influence agency policymaking through letter-writing, petitions, hearings, and protests. 
% I suspect that mass commenting is driven by the same privileged populations known to engage in other civic activities. 
% Does it work? If so, by what mechanisms?

Following the conventional terms ``mass comment campaign'' and ``public engagement,'' I call the general phenomenon ``mass engagement'' resulting from ``mass mobilization'' in order to distinguish the magnitude of civic engagement.
By mass engagement, I mean that thousands of people beyond professional policy influencers engage. Specifically, I define mass engagement as more than 1000 public comments or 100 identical comments, plausibly reflecting a mobilization effort.  Contrary to the common assumption that this emerges organically, it is almost always mobilized by an organization that also engages in sophisticated lobbying. %\footnote{
As \citet{SantAmbrogio2018} conclude ``The `mass comments' occasionally submitted in great volume in highly salient
rulemakings are one of the more vexing challenges facing agencies in recent years. These comments are typically the result of orchestrated campaigns by advocacy groups to persuade members or other like-minded individuals to express support or opposition for an agency's proposed rule.'' 

We lack systematic analysis of public comments. Political scientists have thus far focused on sophisticated lobbying efforts. However, as \citet{Cuellar2005} finds in his study of several rules, ``contrary to conventional wisdom, comments from the lay public make up the vast majority of total comments about some regulations. This shows at least some potential demand among the mass public for a seat at the table in the regulatory process.'' Having collected over 70 million comments on over 300,000 proposed rules, I am able to offer a much more systematic analysis. As figure \ref{fig:mass-comments} shows, not only do comments from ordinary people make up the vast majority of comments across all rule, most comments are, in fact mobilized by mass commenting campaigns. 

\begin{figure}
    \centering
    \includegraphics{}
    \caption{Unique vs Form-letter Comments Posted to Regulations.gov 2006-2018}
    \label{fig:mass-comments}
\end{figure}


\subsubsection{Understanding mass mobilization as a tactic}
% MY THEORY 
% Representation 
When lobbying during rulemaking, groups often make suspect claims to represent broad segments of the public \citep{Seifter2016UCLA}. Mobilizing a large number of people may support such claims.
% public private interests
Appeals to government are almost always couched in the language of public interest, even when true motivations obviously private \citep{Schattschneider1975}. Theorists may debate whether effectively signing a petition of support without having a role in crafting the appeal is meaningful voice and whether petitions effectively channel public interests, but, at a minimum, engaging a large number of supporters helps distinguish narrower interests from broader ones. It suggests the organization is not ``memberless'' \citep{Skocpol2003} in the sense that they are able to demonstrate some public support.

% tactic
Mass mobilization is a strategy. When successful, mass engagement is the result. An organization's ability to expand the scope of conflict by mobilizing members of the public is a political resource. 
In contrast to scholars who focus on the deliberative potential of public comment processes, I focus on public engagement as a tactic aimed at gaining power, either by leveraging powerful ideas or engaging actors with the institutional power to shape decisions.
Scholars who do understand mobilization as a tactic \citep{Furlong1997, Kerwin2011} have thus far focused organizations mobilizing their membership. %In contrast, 
I expand this to include a campaign's broader audience and its potential to grow, more akin to the concept of an attentive public \citep{Key1961} or issue public \citep{Converse1964}. If organizations claim to represent people beyond their official members, 
reforms requiring groups to disclose information about their funding and membership \citep{Seifter2016UCLA} only go part way to assess groups' claims to represent these broader segments of the public. Indeed, if advocacy group decisions are largely made by D.C. professionals, these advocates themselves may be unsure how broadly their claims resonate until potentially-attentive publics are actually engaged.

% three insights 
Here I build on three insights. First, \citet{Kerwin2011} and \citet{Furlong1997} identify mobilization as a tactic. In their survey, organizations report that forming coalitions and mobilizing large numbers of people are among the most effective lobbying tactics. Second, \citet{Nelson2012} identify political information a potentially influential result of lobbying by different business coalitions. While they focus on mobilizing experts, \citet{Nelson2012} describe a dynamic that can be extended to mass commenting: 
``strategic recruitment, we theorize, mobilizes new actors to participate in the policymaking process, bringing with them novel technical and political information. In other words, when an expanded strategy is employed, leaders activate individuals and organizations to participate in the policymaking process who, without the coordinating efforts of the leaders, would otherwise not lobby. This activation is important because it implies that coalition lobbying can generate new information and new actors---beyond simply the `usual suspects'---relevant to policy decision makers. Thus, we theorize consensus, coalition size, and composition matter to policy change.'' 
I argue that, with respect to political information, this logic extends to non-experts. 
Third, \citet{Furlong1998}, \citet{Yackee2006JPART}, and others distinguish direct and indirect forms interest group influence in rulemaking. I argue that mass mobilization is a tactic aimed at producing political information that may have direct and indirect influence. 

% PUBLIC OPINION and INFORMATION 
\citet{Rauch2016} suggests that agencies reform the public comment process to include opinion polls. I build from a similar intuition that mass comment campaigns currently function like a poll or, more accurately, a petition, capturing the intensity of preferences among a segment of the public---i.e. how many people are willing to take the time to engage. Self-selection may not be ideal for representation, but opt-in participation---whether voting, attending a hearing, or writing a comment---still provides political information. 
Mobilizing citizens and generating new political information are key functions of interest groups in a democracy \citep{Mansbridge1992, Mahoney2007}. The information generated by mass mobilization campaigns is explicitly political and more complex than an opinion poll. Activists aim to convince people which issues are important and how to think about them---mapping new issues and debates to familiar ones, thereby shifting the political landscape. 

Importantly, rule-specific campaigns inform agencies about the distribution and intensity of opinions that are often too nuanced to estimate a priori. Many rules may lack analogous public opinion polling questions, making mass commenting a unique source of political information. Indeed, most members of the public and their elected representatives may only learn about the issue as a result of an mobilization campaign. I thus consider public demands to be a latent factor in my model of policymaking. Public demands shape the decisions of groups who lobby in rulemaking. If they beleive the attentive pubic is on their side, groups may attempt to reveal this political information to policymakers by launching a mass mobilization campaign. The public response to the campaign depends on extent that the attentive public is passionate about the issue.

\input{causal-whymail.tex}






% TYPES OF CAMPAIGNS AND COALITIONS
\paragraph{Types of campaigns:} The mix of types of supporters depends, in part, on the aims of a campaign. Campaigns may have one of three distinct aims: (1) to win concessions by going public, (2) to disrupt a perceived consensus, or (3) to go down fighting. 

Coalitions ``go public'' when they believe that expanding the scope of conflict gives them an advantage.\footnote{
Going public (or an ``outside strategy'') is used by Presidents \citep{Kernell2007}, Members of Congress \citep{Malecha2012}, interest groups \citep{Walker1991, Dur2013}, Lawyers, and Judges (Davis 2011). 
% Sophisticated organizations also use phone banks, targeting strategies, and direct-mail techniques to drum-up and channel public support (see Cooper 1985:2036).
This strategy is likely to be used by those disadvantaged (those \citet{Schattschneider1975} calls the `losers') with less public attention.
Rulemaking with little public attention is the norm. Nearly all scholarship on rulemaking in political science thus focuses on interest-group and inter-branch bargaining, ignoring public opinion and social movements. 
}
As these are the coalitions that believe they have more intense public support, many people may be inspired indirectly and to engage with more effort. In these cases, mass engagement will likely skew heavily toward this side. This is important because a perceived consensus may be especially influential political information.\footnote{
For example, consensus among interest groups \citep{Golden1998, Yackee2006JPART}, especially business unity \citep{Yackee2006JOP, Haeder2015}, predicts policy change, though it is not clear if this is a result of strategic calculation, a perceived obligation due to the normative power of consensus (e.g. following a majoritarian \citep{Mendelson2011}), or simply that the information is easier to process.
}

Second, because the perception of consensus is powerful, when a coalition goes public, an opposing coalition may countermobilize. As this is likely a coalition with less intense public support and its aim is merely to break a perceived consensus, I expect such campaigns to engage fewer people, less effort per person, and yield a smaller portion of indirect engagement. 

Finally, campaigns may target supporters rather than policymakers. Sometimes organizations ``go down fighting'' to fulfill supporters' expectations.\footnote{
I use ``going down fighting'' as shorthand for campaigns aimed at only at fulfilling supporter (e.g. donor, membership) expectations and related logics that are internal to the organization (e.g. fundraising, member retention or recruitment, or satisfying a board of directors).} While such campaigns may engage many people, they are unlikely to affect policy or to inspire countermobilization. I expect such campaigns to occur on rules that have high partisan salience (e.g. rules following major legislation passed on a narrow vote), propose large shifts on policy issues dear to well-funded public interest groups, and occur after presidential transitions when executive-branch agendas shift more quickly than public opinion.


% \subsection{Mobilizing for Recruitment}
% A third possibility is that mobilization around bureaucratic decisions is unrelated to the possibility of affecting policy and primarily a way to recruit and engage members or raise the profile of the movement. If this is the case, behaviors like protesting and mass commenting on rules are largely epiphenomena to unrelated kinds of politics. Organizers may know that mobilization has minimal effects, but lead members to engage as means to other ends. Many of the mobilized themselves may doubt their efficacy but still take advantage of the opportunity to protest. 

While the coalitions may form around various material and ideological conflicts, those most likely to be advantaged by going public or going down fighting are public interest groups---organizations primarily serving an idea of the public good rather than the material interests of their members.\footnote{
One exception may be the few types of membership organizations that are both broad and focused on material outcomes such as labor unions.} Thus, I theorize that mass mobilization is most likely to occur in conflicts of public versus private interests or public versus public interests (i.e. between coalitions led by groups with distinct ideas of the public good), but only ones with sufficient resources to run a campaign.\footnote{
If true, one implication is that mass mobilization will systematically run counter to concentrated business interests where they conflict with the values of organized, privileged groups.
}
To assess these propositions, I classify coalitions as primarily driven by public or private interests and roughly estimate each coalition's resources. 



% We do not really know who engages in mass commenting. Some assume that people who engage in mass commenting belong to membership organizations. Others imply that they are people who happen to have an opinion. RAUCH discusses both "members" and  _______
% % The people who engage in mass commenting are often assumed to be 

% Engaging a broader audience and thus changing the scope of conflict is a basic political strategy. Presidents, supreme court justices, and others "go public" when doing so alters their opponents' calculations. 

% Which campaigns engage a broader audience and which do not? 


\paragraph{Types of engagement:} I classify supporters into three types using the texts of their comment to infer how they were mobilized.  Comments that are exact copies of a form letter are akin to petition signatures from supporters who were engaged by a campaign to comment with minimal effort. Commenters that repeat text but also take time to add their own text indicate more intense preferences. Finally, commenters who express solidarity in similar but distinct phrases indicate they were engaged indirectly
, perhaps by a news story or a social media post about the campaign, 
as campaign messages spread beyond those originally targeted. 

The size of each of each group thus offers political information to policymakers, including coalition resources, intensity of sentiment, and potential for conflict to spread.
The first two types signal two kinds of intensity or resolve. First, they show the mobilizers' willingness to commit resources to the issue. Second, costly actions show the intensity of opinions among the mobilized segment of the public \citep{Dunleavy1991}. The number of people engaged by a campaign is not strictly proportional to an organizations investment. The less people care, the more it costs to mobilize them. If agency staff do not trust organizations' representational claims, engaging actual people may be one of the few credible signals of a broad base of support. The third type indicates potential contagion. Indications that messages spread beyond those originally targeted be especially effective \citep{Kollman1998}. Information about organizational resolve, intensity of preference, and contagiousness are thus produced, but will only influence decisions if mass comments are processed in a way that captures this information and relays it to decisionmakers. These organizational processes may vary significantly across agencies.



\subsection{Step 2: Why mass engagement may affect political oversight} \label{principals-intro}
% intro do not call story 
When George W. Bush replaced Bill Clinton as president, career bureaucrats at the Federal Trade Commission knew that this meant a change in policy priorities. Many rulemaking projects initiated under the Clinton administration were likely to be withdrawn or put on hold. They also knew that the new administration wanted to be perceived as advancing a new policy agenda, not merely undoing Clinton-era regulations. Entrepreneurs within the agency saw a political window of opportunity to initiate a new regulatory agenda aimed at curbing a growing volume of telemarketing calls. This initiative seemed likely to be popular with voters but, even with a supportive president, would be difficult to advance over the objections of the telemarketing industry whose campaign donations had earned them many powerful allies in Congress. Agency officials report being pessimistic about the FTC's telemarketing effort succeeding over opposition from Congress.

When the draft ``Telemarketing Sales Rule'' (also known as the ``Do Not Call'' rule) was published, however, public support and engagement were overwhelming. The rule received thousands of supportive comments from frustrated members of the public who were encouraged to comment by advocacy groups like the Consumer Federation of America. Agency officials report that the volume of public response not only encouraged the agency and the administration but, more importantly, ``scared off'' Members of Congress who the industry was relying on to kill or reverse the rule. Once it became clear that the public was paying attention and sufficiently mobilized to act on the issue, elected officials became much less willing to take unpopular positions supporting industry donors. Instead, Congress ended up codifying the agency's authority to implement the Do Not Call regulations with legislation the following year.

The story of the Do Not Call rule suggests that public engagement in rulemaking may occasionally be influential because it affects the behavior of elected officials who have the power to provide key support or opposition to a proposed rule.


% oversight affects policy outcomes 
\paragraph{Political oversight.}
Political oversight of bureaucracies has long concerned both practitioners and theorists \citep{Wilson1989Epstein1999, Huber2002, McCubbins1984, Wilson1989, Potter2016, Lowande2018}. Political scientists often model the relationship between elected officials and bureaucrats as a principal-agent problem. For example, an agency may have a preferred policy but, upon observing the preferences of principals, may change the rule or delay its publication to avoid being reversed \citep{Potter2016}. While it is widely accepted that agency officials must take the positions of their principals into account, the mechanisms by which this occurs and the empirical conditions for political control are debated.

%Accountability to elected officials has been central to the study of bureaucracy \citep{Epstein1999, Huber2002, McCubbins1984, Wilson1989, Potter2016, Lowande2018} %add Meier and O’Toole 2006; West 1995; Wood and Waterman 1994

% classic model 
Because I focus on influence in the period between publication of draft and final rules, I focus on information about principals' preferences revealed to the agency in this period. Oversight during rulemaking is a form of ex-post control \citep{Epstein1994}. Figure \ref{fig:causal-classic-principals} shows a version of the classic model of principal influence in rulemaking. Upon learning of content of a draft rule, an official with power over the agency may choose to signal their demands to the agency. There is an ongoing debate among scholars over how political oversight operates--i.e. how the observable behaviors of principals inform agency decisions. 

\begin{figure}
    \centering
    \caption{The Classic Model of Principal-Agent Oversight in Bureaucratic Policymaking}
    \label{fig:causal-classic-principals}
\tiny
\begin{tikzpicture}[%
    node distance=1.2cm,
    auto,
    text width=1.5cm,
dnode/.style={diamond, align=center, aspect=2, fill=green!5,draw=green!60, very thick, minimum size=2cm},
squarednode/.style={rectangle, align=center, aspect=1, draw=red!60, fill=red!5, very thick, minimum size=1cm},
pnode/.style={ellipse, align=center, aspect=1, draw=black!60, fill=black!5, very thick, minimum size=1cm},
title/.style={rectangle, align=center, aspect=1, minimum size=2cm},
]
% Draft 
\node[dnode]      (draft)                     {Draft Policy};


% principal Nodes
\node[pnode]        (principaldemands) [right=of draft] {Principal Demands};

\node[dnode]      (principaldecides) [right=of principaldemands] {?};

\node[squarednode]      (principalinfo) [right=of principaldecides] {Perceived Political Consequences};


% \node[squarednode]      (principalinfo2) [below=of principalinfo] {Perceived Principal Opinion};


% principal Lines
\draw[->] (draft.east) -- (principaldemands.west);
\draw[->] (principaldemands.east) -- (principaldecides.west);
\draw[->] (principaldecides.east) -- (principalinfo.west);

% policy 
\node[dnode]      (policy)       [right=of principalinfo] {Policy Response};
\draw[->] (principalinfo.east) -- (policy.west);


% Titles
% \node[title]      (1) [above=of draft] {Policy};
%\node[title]      (2) [above=of principaldemands] {Preferences};
%\node[title]      (4) [above=of principalinfo] {Information/ Signal};
%\node[title]      (3) [above=of principaldecides] {Observed Behavior};
% \node[title]      (5) [above=of policy] {Policy'};

\end{tikzpicture}
\end{figure}
\normalsize

\citet{McCubbins1987} suggest two oversight mechanisms. Principals may proactively attend to agency activities, like a ``police patrol'' or they may rely on bureaucrats' fear of sanction when attentive interest groups alert principals about agency activities, more like a ``fire alarm.'' Administrative procedures like notice and Comment remaking thus offer opportunities for direct oversight and to be alerted to oversight opportunities.



\subsubsection{Incorporating political information into models of political oversight.}
In addition to interest groups directly alerting elected officials to oversight opportunities as in the ``fire alarm'' model,
the political information signaled by mass engagement may serve as a serve as ``warning sign,''--altering elected officials to political risks
% More precisely, political information may alert elected officials of opportunities to rein in an agency (the concept of ``fire alarm'' oversight discussed by \citep{Mccubbins1984}) 
\emph{or} to, conversely, to encourage the agency to hold course (what might better be described as a ``beacon'' attracting positive attention and credit claiming opportunities. In the case of the FTC's ``Do Not Call'' rule and subsequent legislation, mass engagement functioned more as a ``warning'' and a ``beacon,'' effectively enabling and empowering rather than restraining the agency as classical "fire alarm" concept suggests.

Mass engagement in bureaucratic policymaking may affect the behavior of an agency's principals because the shadow of public sanction hangs over elected officials \citep{Arnold1979, Mayhew2000}. When the public is more attentive, it is more important for officials to take popular positions and avoid unpopular ones.
Thus, when a coalition goes public, especially if it generates a perceived consensus in expressed public sentiments, elected officials ought to be more likely to intervene on their behalf and less likely to intervene against them.  

\begin{hyp} \label{hyp:principals}
Elected officials' advocacy is moderated by mass engagement. Elected officials are more likely to engage in rulemaking in support of positions supported by the majority of comments and less likely to engage in opposition to the majority of comments.
\end{hyp}

This suggests an addendum to Hall and Miler's (2008) finding that legislators are more likely to engage in rulemaking when they have been lobbied by a like-minded interest group.
When interest groups lobby elected officials to engage in rulemaking, they may be more likely to engage when aligned with the majority of commenters than when opposed to them (Hypothesis \ref{hyp:principals}.
If elected officials learn from political information, they will be even more likely to engage when lobbied by a coalition that includes public interest groups running a mass-comment campaign, and less likely when lobbied by a coalition dominated by private interests opposed by a large mass comment campaign.\footnote{Of course, if Members of Congress receive signals about the distribution of comments from their districts, the distribution of opinions in their district constituency may be more important. Figure \ref{fig:sierra}, shows that the Sierra Club requires zip code information from commenters, so mass-mobilizers may often send such signals to elected officials.}

\begin{figure}
    \centering
    \caption{Incorporating Mass Engagement and Political Information into Models of Political Oversight}
    \label{fig:causal-principals}
\tiny
\begin{tikzpicture}[%
    node distance=1.2cm,
    auto,
    text width=1.5cm,
dnode/.style={diamond, align=center, aspect=2, fill=green!5,draw=green!60, very thick, minimum size=2cm},
squarednode/.style={rectangle, align=center, aspect=1, draw=red!60, fill=red!5, very thick, minimum size=1cm},
pnode/.style={ellipse, align=center, aspect=1, draw=black!60, fill=black!5, very thick, minimum size=1cm},
title/.style={rectangle, align=center, aspect=1, minimum size=2cm}
]
% Draft 
\node[dnode]      (draft)                     {Draft Policy};



% \draw[->] (publicinfo.east) -- (policy.west);
% \draw[->] (principalinfo.east) -- (policy.south);
% \draw[->] (principalinfo2.east) -- (policy.south);

% Titles
% \node[title]      (1) [above=of draft] {Policy};
%\node[title]      (2) [above=of groupdemands] {Preferences};
%\node[title]      (4) [above=of groupinfo] {Information/ Signal};
%\node[title]      (3) [above=of groupdecides] {Observed Behavior};
% \node[title]      (5) [above=of policy] {Policy'};

% principal Nodes

\node[dnode]      (principaldecides) [right=of principaldemands] {Principal Comments};
\node[pnode]        (principaldemands) [right=of draft] {Principal Demands};


% public Nodes
\node[pnode]        (publicdemands) [above=of principaldemands] {Latent Public Demands};
\node[dnode]      (publicdecides) [right=of publicdemands] {Mass\\ Engagement};

% political info
\node[text centered]      (mobilization) [above=of publicdecides] {};

\node[text centered]      (mobilization2) [right=of mobilization] {};

\node[rectangle, minimum width =3cm, minimum height = 6.5cm, draw=red!60, fill=red!5, very thick]      (politicalinfo) [below=of mobilization2] {};

\node[text centered]      (politicalinfotext) [below=of mobilization2] {Political Information};

% info
\node[squarednode]      (publicinfo) [right=of publicdecides] {Perceived Public Opinion};
\node[squarednode]      (principalinfo) [right=of principaldecides] {Perceived Political Consequences};
\node[squarednode]      (principalinfo2) [below=of principalinfo] {Perceived Principal Opinion};


%\draw[->] (mobilization.south) -- (publicdecides.north);


% public Lines
% \draw[->] (draft.east) -- (publicdemands.west);
\draw[->] (publicdemands.east) -- (publicdecides.west);
\draw[->] (publicdecides.east) -- (publicinfo.west);




% principal Lines
\draw[->] (draft.east) -- (principaldemands.west);
\draw[->] (principaldemands.east) -- (principaldecides.west);
\draw[->] (publicinfo.south west) -- (principaldecides.north east);
\draw[->] (principaldecides.east) -- (principalinfo.west);
\draw[->] (principaldecides.south east) -- (principalinfo2.north west);
\draw[->] (publicdemands.south east) -- (principaldecides.north west);

% policy 
\node[dnode]      (policy)       [right=of principalinfo] {Policy Response};
\draw[->] (politicalinfo.east) -- (policy.west);

\end{tikzpicture}
\end{figure}
\normalsize

Figure \ref{fig:causal-principals} builds on the classic model of political oversight in two ways. First, it suggests that the public and private comments by elected officials are a particularly relevant oversight behavior and a mechanism by which bureaucrats learn and update beliefs about their principals demands. Second, it suggests that such oversight behaviors may be affected by mass engagement because of the impressions of public opinion (i.e. the political information) it creates.

\subsection{Step 3: Why mass engagement may affect rulemaking and rules} \label{influence-intro}
% MORE

% \paragraph{Regulating mercury pollution.}
% \subsection{Puzzle: Why mobilize?}
% \section{Why mobilize?}
% Prior to the 22 million public comments on the Federal Communications Commission's 2017 Open Internet rule,\footnote{It is yet unclear how many of these comments are from real people.} 
In 2011 the Environmental protection agency proposed two regulations to limit mercury emissions from coal and oil-fired power plants. Among other issues, the Environmental Protection Agency (EPA) solicited ``comments on whether there would be a basis for considering area sources to be significantly different from major sources,'' ``on the adequacy of the restrictions associated with bypass conditions regarding maintaining LEE status" and ``on the proposed revisions concerning [equations' 1a and 1b] usefulness in calculating the maximum potential emissions rate from an emissions averaging group'' (EPA 2011). LEE status is not defined in the notice soliciting comments, and equations 1a and 1b are surely inaccessible to most citizens. Yet these two proposed rules received 942,483 comments. 

One comment, from the United States Council of Catholic Bishops, read: ``While we are not experts on air pollution, our general support for a national standard to reduce hazardous air pollution from power plants is guided by Catholic teaching, which calls us to care for God’s creation and protect the common good and the life and dignity of human persons, especially the poor and vulnerable.'' Bishops are not known to closely follow power plant regulations. Their moral authority was mobilized by activists who wanted stricter regulation of mercury and wanted to demonstrate public support of this position. %Groups mobilizing on the mercury rules including environmental and health groups and industry competitors, including the owners of Nuclear, Natural, geothermal power plants. 

In the official, legally-required response to comments, the EPA did not discuss public opinion, obligations to care for the environment, dignity, or the poor. Instead, the EPA asserted that mercury levels are a matter of science, not a matter of justice. But the EPA did implicitly assert a definition of the public good when it used studies of mercury's aggregate public health effects on the U.S. population to set emissions standards.\footnote{As Wagner (1995) %CITE
notes, ``agencies exaggerate the contributions made by science...in order to avoid accountability for the underlying policy decisions. %Although camouflaging controversial policy decisions as science assists the agency in evading various political, legal, and institutional forces, doing so ultimately delays and distorts the standard-setting mission'' (p. 1617).
Wagner goes on to find that  ``While the APA mandates a process for public involvement, it provides almost no protections to ensure that agencies will explain the substantive bases for highly complex or technical rulemakings in a way that the lay public can readily understand and challenge'' (1656) and that ``Mischaracterization of the entire standard-setting endeavor as resolvable by science results in significant obstacles to democratic participation'' (1674). Similarly, Harvey Brooks (1984) %CITE
notes that ``The modern nation risks being no longer recognizable as a democracy, either representative or plebiscitary, if more and more policy areas are excluded from public participation because of the technical complexity.''} 
Then, as required by the Supreme Court, it justified the same standards with cost-benefit analysis in a revised proposed rule, concluding that for every dollar spent to comply with the regulation, the U.S. public receives up to nine dollars in health benefits (EPA 2007). If this is how decisions are made, %why did the EPA receive nearly a million letters? W
why would citizen opinions matter? % MAY NOT MATTER 

\subsubsection{Political information and influence in rulemaking}
% Wagner: A variety of commentators have suggested that agencies may seek increased legitimacy or decreased political accountability by disguising their policy judgments as science. See Majone, supra note 18, at 15 ("Traditionally, government regulators have sought legitimacy for their decisions by wrapping them in a cloak of scientific respectability.");Roberts et al., supra note 26, at 120 ("Too many of the participants [in science-policy decisions] have good reasons not to distinguish scientific evidence from policy preferences, not to analyze carefully the various sources of technical disagreement, and not to accept responsibility for some decisions or judgments."). Beyond these common sense observations scattered at points in articles and books, there has been surprisingly little scholarly discussion of the comprehensive existence of or reasons for a science charade in regulation.

In this section, I integrate the above arguments about interest group lobbying and political oversight to offer a broader model of influence in rulemaking that highlights the political information available to policymakers. As suggested above, such information occasionally arises from contentious debate and civic mobilization and may influence elected officials' oversight behaviors. %I build on studies of interest group influence, social movement influence, and the influence of political principals to offer a model of agency policymaking

% preview 
If we appreciate agency policymaking as a site of contentious politics, mechanisms emerge by which mass mobilization may affect both the strategic environment and ideological perspectives of those who write agency rules. These mechanisms remain under-theorized and untested. I aim to begin to fill this gap by outlining four mechanisms by which political information may influence bureaucratic policymaking. 
Bureaucrats have both strategic and normative reasons for updating policy decisions in light of new political information. The effects of political information on policy thus depend on the strategic context that may or may not offer opportunities for influence. It also depends on cognitive and institutional processes that may or may not incorporate political information.

% In contrast to the science-based objectivity suggested by the above quote from the EPA, political scientists, building on law and economics scholarship, offer a different theory of bureaucratic decisionmaking rooted in the policy preferences and strategic behavior of agency leaders and their political principals: Congress, the president, and the courts. They find that political principals do constrain agency action but also leave room for agencies to move policy toward their own ideal point.\footnote{Though political scientists make diverse assumptions about what this ideal is and how to measure it.} Technical information may or may not inform agency decisions, but preferences and the power to realize them in a strategic environment are, these scholars say, the proximate cause of policy. These scholars would see the Mercury Rules as the result of EPA officials writing a policy as close as possible to their ideal policy given their strategic constraints.

%But the strategic model does not explain why activists thought public comments would matter. If the strategic model is sufficient, why did organizations write letters to the EPA? The EPA administrator has their preferences and the public has no direct power over their decisions. Why not write to the president or members of Congress who influence EPA's strategic calculations and are more directly accountable to public opinion? 

%In section \ref{whymail-intro} I offered three reasons that organizations may launch mass mobilization campaigns. In section \ref{principals-intro}, I suggested that public comments may have a similar effect on political principals as to mass mail campaigns directly targeting them. In this section, I merge these insights into a broader theory of how mass engagement may affect policy. 

% adding IGs to PA model 
%Political scientists, along with scholars of public administration, organizational behavior, and sociology who study interest groups expand on the simple principal-agent model described above that, while less parsimonious, squarely address how the process of soliciting and responding to public comments may influence agency policy (see \citet{Yackee2018} for a review). These scholars find that agency staff develop relationships with those who regularly participate in policy processes: most often businesses but also professional associations and activist organizations. Relationships draw on and reproduce organizational identities and reputations \citep{Carpenter2001, Carpenter2014, carpenter2012}. This scholarship has revealed a good deal about how organized groups lobby agencies and why they succeed, but it has yet to address why these groups sometimes mobilize thousands of citizens to write letters or protest agency decisions. Like most forms of political participation, mass public comments on draft agency rules provide no new technical information. They lack the authority of elected officials' opinions. And the number on each side has no legal import for an agency's response \citep{Kerwin2011}. %, West1995}.
% Policymakers may very well pay no attention to them. The theoretical foundations for why mass mobilization may matter is underdeveloped and we lack empirical research on how it may affect agency policymaking. 

% In this chapter, 
%In this paper, 
% I address this theoretical and empirical gap in our knowledge on the role of mass mobilization in bureaucratic policymaking. I expand and integrate the above theories to develop testable hypotheses % and analyze rule-related texts %and field experiments
% to explore whether mass mobilization matters and, if so, why. 

I argue that if mass mobilization indirectly affects the strategic environment it does so by signaling grass-roots political power to elected officials and if mass mobilization directly affects agency policymaking it does so by evoking norms rooted in organizational identities and reputations. 
Like the vast majority of letter-writers, Catholic Bishops contribute little to the technical aspects of epidemiology, mercury regulations, or cost-benefit analysis. If they influence agency policymaking, it is by signaling a threat of political backlash or by persuading bureaucrats directly that moving policy in a certain direction is the appropriate thing for the agency to do. 
The next two subsections address these indirect and direct mechanisms in more depth. A third discusses why we may still observe mobilization in the absence of influence. 
% Whereas social movement scholars and political scientists have focused the behavior of elected leaders, I focus on the latter pathway of direct persuasion. 




% strategic 
\paragraph{Opportunity.} First, the influence of political information depends on the extent to which the strategic environment allows change. A core concept in social movement scholarship is the ``political opportunity'' or ``opportunity structure'' \citep{Mcadam2017} such as division among elites \citep{Tarrow1994}. In my case, a division among elites could be divergent principals among an agency's political principals or powerful business interest groups. With respect to division among business interests, this aligns with findings from studies of rulemaking that find that consensus among businesses increases their influence in rulemaking \citep{Yackee2006JOP, Nelson2012}. However, the role of divided government is less clear. \citet{Yackee2009RegGov} find that divided government decreases the frequency of rulemaking. However, this does not mean that it reduces the potential for groups to influence those rules that are produced.

Policy process scholars use a similar concept of ``window'' for policy change where advocates have an opportunity to align politics with certain identified problems and solutions  \citep{Kingdon1984}. All rulemaking processes create opportunities, however small, to shape the new status quo, loosely bounded by the problems the process was initiated to solve, a set of policy solutions considered legitimate, and a constellation of political forces.

% normative
\paragraph{Information processing.} Second, influence depends on how political information is processed, both directly within agencies and indirectly through other actors (e.g. Members of Congress) whose appraisals matter to bureaucrats.

%%%%%%%%%%%%%%%%%%%%%%%%%%%%%%%%%%% FIXME
% STRATEGIC ENV 
% \subsubsection{ Agencies as Policymaking Venues}
% the good stuff

% When political scientists ask whose interests and ideas become law, they have generally focused on the behavior of legislatures, how the executive branch drives legislation, and how the courts review it. Compared to legislative, executive, and judicial institutions, the administrative state is a recent development in American government and theory has not kept pace with the rise in bureaucratic policymaking. 

% I argue that theories of bureaucratic policymaking have been characterized by constraining assumptions about what bureaucracies ought to do.  Normative assumptions that \citet{Wilson1967} identified half a century ago, and corresponding scholarly silos, have persisted. This has led to lines of research talking past each other and often failing to engage broader theories of policy change. In particular, I argue that the pervasive implicit assumption that bureaucrats ought to be neutral implementers implies that politics in agency policymaking is inherently undesirable, leading many scholars to focus on compliance with political principals and overlook the role participation and ideas. For example, scholars assume that agencies ought to be engaged in implementing legislation and executive orders. However, most rulemaking takes place many years or decades after its authorizing legislation under a different Congress and with little attention from the White House until the very final draft. Rules that do not follow from contemporary Congressional or executive priorities are often assumed to reflect bureaucrats going rogue or being captured by interest groups. Such studies suffer from a lack of attention to the complex political process of rulemaking. 

%Viewing agencies as \textit{agents} has prevented scholars from incorporating new insights about the endogenous relationship between policy and politics. I suggest rulemaking is better studied in the way that scholars study policymaking in specialized congressional committees than with an unrealistic dichotomy of sincere implementation versus capture or disloyalty. Normatively, accountability to political principals only one of several important concerns. Empirically, it is often unclear what accountability means and there is ample evidence that it may not be the primary driver of bureaucrat behavior.


%There may be an inverse relationship between how responsive agencies are to political principals and to the public \citep{Lewis}.

%Yet public administration and legal scholarship rarely address how interest groups gain political power in the first place. 

% more good stuff
% A second major contribution to theory in this area is work by \citet{Carpenter2001} and \citet{Carpenter2012}  research explaining how bureaucrats realize their own goals and achieve autonomy. %Rather than asking how bureaucratic practices fit with normative assumptions, he asks how agencies became independent policymaking bodies. Responding to principal-agent literature that has focused on the presidential and Congressional control, 

% Carpenter finds much more complex sets of relationships that explain organizational power and behavior. One of the main tools he gives us for understanding the source of bureaucratic autonomy is the concept of institutional reputations. Bureaucrats and the institutions they animate develop reputations for certain competencies: for example, for expertly adjudicating scientific claims, for effectively executing policy aimed at a given goal, or for responding to the public interest. Reputations for expertise, effectiveness, or responsiveness reflect the mixed roles assigned to the bureaucrats: advisers, implementers, and policymakers. 
% Like Carpenter, I call attention to the fact that agency policy shapes the coalitions that surround and influence it. I depart from Carpenter's narrative in that I do not focus on cases where agencies intend to have these effects. Whereas Carpenter is interested in how bureaucrats intentionally shape lobbying coalitions, I am interested in the endogenous relationship between policy and coalitions, intended or not. While not the focus of his study, Carpenter notes that relationships also evolve in unintended ways. 
%Policy may pro-actively recruit group support, but may also be reacting to political pressure \footnote{For example, an industry may successfully lobby to be reclassified to face lower pollution regulations, perhaps those faced by their competition, thus turning competitors into allies for future policymaking and increasing the size of the coalition for the lower standard.} or be an unintended side effect of action the agency sees as imperative.\footnote{For example, new science on the health hazards of mercury led to pollution controls that differentiated coal power plants and gas power plants reshaping coalitions by turning allies on former air quality policymaking into competitors in future rounds.} %Nevertheless, 

% Carpenter and related scholars thus offer a second possible set of predictions for which movements are successful: those who succeed in rulemaking tend to be those with close relationships with the agency, conditional upon (and because of) how those relationships support the agency's reputation for expertise, competence, and representativeness. %Furthermore, lobbying coalitions, their relationship to the agency, and thus their success are functions of past agency policy. 

%Some scholars attempt to estimate the preferences of bureaucrats. Instead, I take 
% Carpenter's findings of close relationships between interest groups and agencies is a potential explanation for why some groups appear to have influence, i.e. because they are aligned with agency ideologies. As my core contribution is to assess how groups are empowered or disempowered rather than how agencies are empowered or disempowered, I focus on discovering which groups' comments are related to changes in rules regardless of whether this is what certain bureaucrats also wanted. 




% \subsubsection{Reputations for accountability, representation, equity, and expertise }

% [How specific organizational identities and reputations drive decisionmaking]


\paragraph{Causal mechanisms:} How might mass engagement matter?
% A lobbying effort can generate new information, re-frame information, or reshape the political context of a decision. Agency staff may update their beliefs in response to new information or framing. Activists can also reshape agency policymakers’ strategic environment by drawing in or scaring off other actors, especially elected officials. 

Why might mass mobilization matter? The literature on bureaucracy offers two types of explanations rooted in either strategic behavior or organizational norms. Political scientists often focus on strategic context. Public administration and management scholars focus on organizational logics and identities. I begin with the ``indirect'' mechanisms that theories of strategic behavior suggest. 

In the U.S. context, there are three main mechanisms by which mass mobilization could affect an agency action by changing the strategic context. 
Mass mobilization may signal power to influence the responses to agency action from the White House, Congress, or courts. Many rules receive little attention from these other institutions, but all three significant powers to reward, sanction, or reverse agency actions \citep{Yaver2016}. 

% MECHANISMS TABLE 
\begin{table}[h!] \footnotesize
\centering 
  \caption{Mechanisms: How New Information May Influence Bureaucratic Policymakers} 
  \label{2x2} 
\begin{tabular}{@{\extracolsep{5pt}} lcc} 
 & Highlighting Norms in Institutional and Cognitive Processes & Shifting Strategic Incentives  \\ 
\hline \\
Direct    & Select ``public'' opinion & Technical information \\
& (public service/responsiveness/direct democracy) & (e.g. inputs to benefit-cost analysis)\\
 \\
 \hline \\
Indirect & Elected official opinion  & Likely retribution and reward \\ 
& (accountability/representative democracy) & (e.g. future budgets, careers, support) \\
\\
\hline 
\end{tabular}
\end{table}

\textbf{Strategic calculations:} 
New information may affect agency strategy directly or indirectly. New scientific or legal information spurs revision of calculations about cost and benefits or the likelihood of being reversed in court. New political information spurs bureaucrats to update their beliefs about levels of support among certain populations or their elected representatives and thus the likely political consequences of a decision.

For example, the number, geographic distribution, size, and proportion of businesses who lobby against a rule, may provide information about how much money and which of their political principals may be invested in attacking a rule. Similarly, the number of people who engage in a rulemaking and the intensity of engagement may provide information about how much support or scrutiny an agency is likely to receive from certain political principals. 

As activist campaigns may be less predictable than business lobbying, civic mobilization may provide even more information about constellations of support or opposition and the intensity of these policy demanders. 
If the information leads bureaucrats to update their understanding of the constellations of interests, their intensity, and the power and resources of each coalition, it may affect their strategic response.

\textbf{Indirect influence through changing the decision environment:} 
Bureaucrats care about the consequences of their actions, both for themselves for their agency’s mission. Their success and power depend on the support of a political coalition that includes elected officials \citep{Carpenter2001}. \citet{West2004} theorizes that the primary mechanism by which mass-commenting matters is to alert political principals. Members of Congress, especially, may usually be unaware of rulemaking \citep{Nou2016}. Conversely, campaigns may ``scare off'' elected officials who otherwise would have weighed in, threatening consequences, such as legislation that reverses the rule (personal communication with former agency director).

Reshaping strategic incentives may shift how rulewriters weigh commenter demands.

% NORMATIVE FRAMING / INFO PROCESSING 
\textbf{Information processing and normative evaluations:} 
In addition to strategic calculations, mass engagement may shift how information is processed and evaluated, both institutionally and cognitively.\footnote{
As Simon observes, a focus on maximizing subjective utility carries a lot of baggage: “In a substantive theory of rationality, there is no place for a variable like focus of attention. But in a procedural theory it may be very important to know under what circumstances certain aspects of reality will be heeded and others ignored” (Simon, 1987: 31).   % Indeed, Newell (1958: 13) argued that any rule of decision-making, comprehensively rational or not, is conditioned on “what information is entered into the system… the judgment law is quite secondary, and amounts to doing the obvious with the information finally selected”.
}
By focus policymakers attention on certain aspects of policy decisions mass engagement may affect who is involved these decisions and how they are framed.

Institutionally, higher comment volume may engage a larger and more politically-oriented set of staff and consultants. Cognitively, expanding the scope of conflict highlights the political aspects of a decision, perhaps mobilizing cognition focused more on norms of public service or partisan ideology than on strategic or technical rationality. In both cases, campaigns re-frame decisions as political and provide information that is especially relevant if processed through such a frame.

%Because rulemaking is an exercise in relating information about the world to certain norms and policy agendas, how decisions are framed may be decisive. 
%Thus, both new political information and how information is framed may influence policy decisions. 
The source, number, and content of comments all provide political information. Each side may offer frames for interpreting these facts and others. If 
%the opinions of letter-writers, petition-signers, and protesters are
framed as the opinion of the public or as expressing valid public interests, such a frame may shape how officials think about the appropriate course of action for a public servant or a partisan concerned with the popularity of agency decisions.  

Even, perhaps especially, when positions expressed through contentious politics
%through petitions and protests
are not majoritarian, these tactics may communicate political information that is not represented through electoral politics \citep{Gillion2012, Gillion2013}. 
% Evidence that minority petitions and protests affect rulemaking would support theories that focus on policymakers’ limited attention, finite agenda, and satisficing rather than strategic decision making.
% Petitions and protests 
Campaigns may also frame minority groups as deserving of special attention and protection.

The effects of political information on bureaucrats' normative evaluations may be
direct---the weight that norms of direct democracy give to limited public input---or 
indirect---the weight that norms of accountability give to elected officials' input.
The strength of norms of direct democracy and accountability may vary across agencies with levels of political insulation and responsiveness.

To the extent that elected officials' demands guide agency decisionmaking---i.e. to the extent that agency decisions are shaped by norms of accountability in representative democracy---campaigns may be influential by inspiring elected officials to produce new political information. When elected officials take a position publicly or in a private letter to an agency, such political information may have normative force beyond simply simple strategic calculations.

%\paragraph{Indirect influence through elected officials:} 
%Campaigns  do more than reveal latent political information; they mobilize both members of the public and elected officials to take positions on issues they may have never previously considered, thus creating new relevant political information for bureaucrats. 
  %Movements help to shape the political space in which they operate’ (Gamson and Meyer 1996, p. 289).
%The result of thinking differently about a decision may be a shift in how the agency evaluates or weights commenter demands.




% FIXME 
% INFO PROCESSING 
\paragraph{The impact of any kind information on policy decisions depends on the capacity to process this type of information.}
It is possible that many agencies lack the capacity to process political information embedded in mass comments. Some may simply discard this information \citep{Mendelson2011}. The expected influence of mass engagement thus depends on how information is processed, which I expect to vary significantly across agencies. 



% \section{Mechanisms of Influence}

\subsection{Indirect Influence: Signaling a threat of backlash}
% Indirect FIXME
Indirectly, it may alert elected officials to political risks and opportunities, thus reshaping an agency's strategic environment.



% Arnold, Logic Of Congressional Action p 217:
% success depends in poat of the length and complexity of the causal chain connecting a policy instrument with its policy effects. When a causal chain is short and simple, citizens are more likley to know which policy instrument will produceth appropriate effects and are beter able to monitor the performance of their repre3sentatives,. When a causal chain is long and complex, or when a problem in society stems from multile causes, citizens may be incapable of doing the appropriate policy analysis and political anslyss ." 
% p 272 "resoonsiveness to both attentive and inattentive publics avaries depending on the procedures that govern how legislators requd their positions" 
% "The model of citizen's control that I have been discussing is essentially an auditing model. Citizens do not instruct legislators on how to vote, not do thay necessairlily have well-defined policy preferences in advance of cogressional action. Legialators neverthless have strong incentives to consider citizens' potential preferences when they are deciding how to vote for fear that making the wrong choice might triggger and unfavorable audit." 
% 


The White House has several tools to influence agency decisions \citep{Yackee2009a,Simon1954}. These include executive orders \citep{Mayer1999}, appointments \citep{Doherty2014,Lewis2008,Wood1988}, budgets \citep{Whittington2003}, and review of proposed policies \citep{HAEDER2015InfluenceBudget,Acs2013}. 
Congress also has several tools to influence agency decisions. These include the power of the purse \citep{Fenno1986,Bolton2015}, oversight, and new legislation. Some research suggests that this constraint is larger under divided government \citep{Yackee2009b} and that under divide government Congress tends to divide power among multiple agencies \citep{Farhang2016}.
The anticipation of judicial review makes courts relevant to rulemaking. Some rules are also made under court-imposed settlement or with judicial deadlines. Judicial opinions may also call on Congress to act \citep{Yaver2017}.
Despite these mechanisms and because of conflicts among them, agency staff maintain significant power over agency decisions. For example, Congress is less assured of compliance when power is divided \citep{Yaver2016}.

% more good stuff
%\subsubsection{Legal Scholarship}
Legal scholars' case studies of specific rulemaking process offer an additional relevant body of research. Coglianese (1997) finds that litigation is a common extension of rulemaking. Indeed, unlike legislative lawmaking, rulemaking takes place in the clear and present shadow of judicial review (Rossi 2001). Stakeholders can challenge a rule in court on a variety of procedural grounds and on statutory interpretation. This scholarship suggests those who succeed in rulemaking are those with the resources and experience to succeed in court. Costly mass comment campaigns could be signaling the ability and willingness to spend resources to challenge the rule in court. 

Mass mobilization may signal political risks or benefits of engaging in agency policymaking to members of Congress and the White House. It also may signal to the agency that activists have the capacity to sustain pressure through the policy process \citep{Coglianese2001}, including challenging the policy in court, a constant threat agency policies. Thus, mass mobilization may act as a signal of political power that  reshape rule-writers' beliefs about their strategic context. 



\subsection{Direct Influence: Mobilizing identities and reputations}

I now turn to the direct-influence pathway: the ability of social movements to mobilize ideas, evaluative frameworks, and claims about what is appropriate and right that may affect bureaucratic decisions. %I use mobilization around the idea of ``environmental justice'' as an example where direct influence may be visible. 

Organizational theory suggests additional mechanisms by which mass mobilization may influence bureaucratic decisions more directly. Here the causal process involves mobilizing norms and ideas right and wrong rooted in individual and institutional identity. Because concepts of mission, reputation, and the validity of claims are intertwined, these mechanisms are difficult to precisely define. Nevertheless, scholars have identified several types of direct influence. One important factor in decision making is personal and institutional reputation \citep{Carpenter2001}. This can take several forms. For example, individuals trained as scientists and agencies that cultivate reputations for producing valid science may be persuaded by rigorous scientific claims. Similarly, individuals who identify strongly as public servants and agencies with reputations for public responsiveness may be persuaded by claims about public or "stakeholder" opinion. In general, claims that resonate with the problems an agency has been tasked with solving and the means it has to solve those problems are likely to be well received.  

% \subsubsection{ Agencies as Policymaking Venues}
% the good stuff
When political scientists ask whose interests and ideas become law, they have generally focused on the behavior of legislatures, how the executive branch drives legislation, and how the courts review it. Compared to legislative, executive, and judicial institutions, the administrative state is a recent development in American government and theory has not kept pace with the rise in bureaucratic policymaking. 

I argue that theories of bureaucratic policymaking have been characterized by constraining assumptions about what bureaucracies ought to do.  Normative assumptions that \citet{Wilson1967} identified half a century ago, and corresponding scholarly silos, have persisted. This has led to lines of research talking past each other and often failing to engage broader theories of policy change. In particular, I argue that the pervasive implicit assumption that bureaucrats ought to be neutral implementers implies that politics in agency policymaking is inherently undesirable, leading many scholars to focus on compliance with political principals and overlook the role participation and ideas. For example, scholars assume that agencies ought to be engaged in implementing legislation and executive orders. However, most rulemaking takes place many years or decades after its authorizing legislation under a different Congress and with little attention from the White House until the very final draft. Rules that do not follow from contemporary Congressional or executive priorities are often assumed to reflect bureaucrats going rogue or being captured by interest groups. Such studies suffer from a lack of attention to the complex political process of rulemaking. 

Accountability to elected officials has been central to the study of bureaucracy \citep{Epstein1999,Huber2002,McCubbins1984,Wilson1989,Potter2016Slow-RollingRulemaking,Lowande2018PoliticizationAgencies} %add Meier and O’Toole 2006; West 1995; Wood and Waterman 1994
Viewing agencies as \textit{agents} has prevented scholars from incorporating new insights about the endogenous relationship between policy and politics. I suggest rulemaking is better studied in the way that scholars study policymaking in specialized congressional committees than with an unrealistic dichotomy of sincere implementation versus capture or disloyalty. Normatively, accountability to political principals only one of several important concerns. Empirically, it is often unclear what accountability means and there is ample evidence that it may not be the primary driver of bureaucrat behavior.

In contrast to the dominant view of agencies as \textit{agents}, a growing literature in political science draws on scholarship in law and public administration as well as studies of agenda setting and lobbying in legislative policymaking to better understand agencies as policymaking bodies. Public administration and legal scholars have been more attentive to the prominent role of interest groups.  Kerwin (2003) notes that ``Interest groups could find few modes of government decision making better suited to their particular strengths than rulemaking.'' This research finds business groups to be most successful class of commenters in rulemaking \citep{Yackee2006a} especially when lobbying together, often, or unopposed (Nelson and Yackee 2012) and when lobbying across multiple venues \citep{Yackee2015}. Importantly, this literature notes that the currency of lobbying is information (Hall and Deardorf 2006), which includes both science and policy ideas \citep{Jones2005}. Kirilenko (2014) and Yackee and Yackee (2006) both find evidence that comments from sophisticated interest groups like businesses seem to influence rules. These scholars offer one set of answers to the question of who wins: those who succeed in rulemaking tend to be business interests, repeat players, those who lobby together, and those who lobby unopposed. They succeed because they bring in new voices and send unified messages at higher amplitudes, creating perceptions of political consensus.


%There may be an inverse relationship between how responsive agencies are to political principals and to the public \citep{Lewis}.

%Yet public administration and legal scholarship rarely address how interest groups gain political power in the first place. 

% more good stuff
A second major contribution to theory in this area is Carpenter's  research explaining bureaucratic autonomy \citep{Carpenter2001,Carpenter2012}. Rather than asking how bureaucratic practices fit with normative assumptions, he asks how agencies became independent policymaking bodies. Responding to principal-agent literature that has focused on the presidential and Congressional control, Carpenter finds much more complex sets of relationships that explain organizational power and behavior. One of the main tools he gives us for understanding the source of bureaucratic autonomy is the concept institutional reputations. Bureaucrats and the institutions they animate develop reputations for certain competencies: for example, for expertly adjudicating scientific claims, for effectively executing policy aimed at a given goal, or for divining the public interest. Reputations for expertise, effectiveness, or representativeness reflect the mixed roles assigned to the bureaucrats: advisors, implementers, and policymakers. 
% Like Carpenter, I call attention to the fact that agency policy shapes the coalitions that surround and influence it. I depart from Carpenter's narrative in that I do not focus on cases where agencies intend to have these effects. Whereas Carpenter is interested in how bureaucrats intentionally shape lobbying coalitions, I am interested in the endogenous relationship between policy and coalitions, intended or not. While not the focus of his study, Carpenter notes that relationships also evolve in unintended ways. 
%Policy may pro-actively recruit group support, but may also be reacting to political pressure \footnote{For example, an industry may successfully lobby to be reclassified to face lower pollution regulations, perhaps those faced by their competition, thus turning competitors into allies for future policymaking and increasing the size of the coalition for the lower standard.} or be an unintended side effect of action the agency sees as imperative.\footnote{For example, new science on the health hazards of mercury led to pollution controls that differentiated coal power plants and gas power plants reshaping coalitions by turning allies on former air quality policymaking into competitors in future rounds.} %Nevertheless, 
Carpenter and related scholars thus offer a second possible set of predictions for which movements are successful: those who succeed in rulemaking tend to be those with close relationships with the agency, conditional upon (and because of) how those relationships support the agency's reputation for expertise, competence, and representativeness. %Furthermore, lobbying coalitions, their relationship to the agency, and thus their success are functions of past agency policy. 

%Some scholars attempt to estimate the preferences of bureaucrats. Instead, I take 
% Carpenter's findings of close relationships between interest groups and agencies is a potential explanation for why some groups appear to have influence, i.e. because they are aligned with agency ideologies. As my core contribution is to assess how groups are empowered or disempowered rather than how agencies are empowered or disempowered, I focus on discovering which groups' comments are related to changes in rules regardless of whether this is what certain bureaucrats also wanted. 




% \subsubsection{Reputations for accountability, representation, equity, and expertise }

% [How specific organizational identities and reputations drive decisionmaking]



\textbf{Assessing causal mechanisms:}
While it may be impossible to causally identify or attribute effects to normative or strategic mechanisms, 
a focus on political information 
%and the schema of Table \ref{2x2} 
suggests places to look for influence in rulemaking. For example, if Members of Congress are not more likely to voice support for a coalition that goes public, this would be evidence against that indirect mechanism.
% While scholars often focus on the top right cell of Table \ref{2x2}, the influence of political information is to be found in the other three cells.\footnote{Political scientists have focused on strategic factors---either on how lobbying provides technical information that directly influences agency decisions or on how oversight indirectly constrains them.  Mass engagement is only likely to affect the later.}

%I also look closely at rules that end up before the Supreme Court. These rulemaking processes deserve extra attention for two reasons. First, regardless of how contentious they were at the rulemaking stage, these rules are key to understanding the nature and limits of executive power as a policymaking venue. Second, because all rulemaking is done in the shadow of judicial review, who wins in court and why shapes the political terrain of future rulemaking, empowering some groups with credible threats of litigation and disempowering others. References to court cases and implied threats of litigation are common in interest group comments, but to my knowledge, no study has looked systematically at how they affect rulemaking. Conversely, scholarship has not systematically assessed how mobilization and contestation in a rulemaking process affect judicial review. 
\subsection{Incorporating Political Information into Formal Models}

Formally, political information requires several crucial amendments to existing information-based models of rulemaking. In the most sophisticated model of notice-and-comment rulemaking to date, \citet{Libgober2018} posits a utility function for agency $G$ as $u_G(x_F) = \alpha_0 x_f^2 + \sum_{i=1}^N \alpha_i u_i (x_f)$ where $x_f$ is the spatial location of the final policy, $u_i$ is the preference of a member of the public or ``potential commenter'' $i$, and $\alpha$ is a vector of "allocational bias"---i.e. how much the agency cares about its own preferences $\alpha_0$ relative to accommodating the preferences of others $\alpha_{i=1:N}$. Bureaucrats balance their own idea of their mission against their desire to be responsive. In Libgober's model, $\alpha$ is a fixed ``taste'' for responsiveness to each member of society, so agency decisions simply depend on their answer to the question ``what do people want?'' Including political information requires two additional parameters related to a second question ``why would the agency care?''

First, going public (like other lobbying strategies) may shift the strategic environment, leading an agency to shift its allocation in favor of some groups and away from others. Let this strategic shift be a vector $\alpha_s$. Second, campaigns may directly persuade agencies to adjust their allocational bias, for example by supporting claims about the number of people the group represents or the intensity or legitimacy of their policy demands. Let this direct shift be $\alpha_d$ and immutable taste now be $\alpha_t$. Having decomposed an agency's allocative bias into three parts (its fixed tastes, shifting strategic environment, and potential to be convinced), the agency's utility function is now  $u_G(x_F) =  (\alpha_{t0} + \alpha_{s0} + \alpha_{d0}) x_f^2 + \sum_{i=1}^N (\alpha_{ti} + \alpha_{si} + \alpha_{di}) u_i (x_f)$. If, after the comment period, an agency's strategic environment is unchanged and it remains unpersuaded about which segments of society deserve favor, $\alpha_s$ and $\alpha_d$ are 0, and the model collapses to the original information game based on fixed taste. This less plausible when groups go public and expand the scope of conflict. 

Incorperating political information allows us to begin formalizing intuitions about mechanisms of influence. For example, \citet{Libgober2018} asks ``What proportion of commenting activity can be characterized as informing regulators about public preferences versus attempting to attract attention of other political principals?'' (p. 29). Adding political information to the model allows us to formalize this question: Under what conditions does the decision to comment depend on an organization's beliefs about $\alpha_t$ versus beliefs about $\alpha_s$? %Because they are substitutes in the model, this may be hard to say theoretically, but e
Empirically, we may often be able to infer that the difference in commenting can be attributed to group $i$'s beliefs about $\alpha_{si}$ if the behavior of political principals varies but other observed parameter values are similar across rules at a given agency.

Rational-choice explanations of why organizations comment on proposed rules build on an intuition that potential commenters will comment only when the benefits exceed the costs of doing so. This intuition ought to apply to other lobbying strategies such as mass mobilization as well. Adding an additional lobbying strategy into the model described above is straightforward. In Libgober's model, a potential commenter has negative quadratic preferences centered on their ideal policy $p_i$ and $u_i = -(x_f - p_i)^2$ where $x_f$ is the final policy chosen by the agency. An organization will comment if the cost of doing so is less than the difference between their utility when the agency selects a policy having been informed about the organization's ideal point $p_i$ versus when the agency selects a policy having made a guess about the organization's ideal point, $z_i$. If $c_i$ is organization $i$'s cost of commenting, then $i$ will comment if it expects to be better off providing information than abstaining $E[u_i | p_i] > E[u_i | z_i] + c_i$. Similarly, an organization will go public when it expects that the cost of running a mass mobilization campaign to be less than the difference in utility when the agency selects a policy having been informed about the intensity of broader public preferences $p_{public}$ versus when the agency selects a policy having made a guess about the intensity of the attentive public's preferences, $z_{public}$. If $c_{campaign, i}$ is organization $i$'s cost of running a mass mobilization campaign, then $i$ will launch a campaign if $E[u_i | p_{public}] > E[u_i | z_{public}] + c_{campaign, i}$. This suggests that mass mobilization with the aim of influencing policy, i.e. going public, should be more common when agencies are either poorly informed or distant from public opinion and potentially influenced by the types of political information created by mass engagement. 

Additionally, an organization may comment or run a mass mobilization campaign if it benefits in ways that are independent of policy outcomes. Strategies such as "going down fighting" can be incorporated by adding exogenous benefit parameters to the utility function of the potential commenter/mobilizer. Let $v_i$ be the benefit of commenting independent of its effect on the policy outcome, such as pleasing members or to reserving the right to sue. Let $w_i$ be the benefit of running a mass mobilization campaign independent of its effect on the outcome of the policy at hand, such as fulfilling expectations of existing members or recruiting new members. An organizations utility function would then be $u_i = -(x - p_i)^2 + v_i + w_i$. 

Adding these parameters also resolves a puzzling result of Libgober's model. Empirically, rules that receive comments do not always change. This result is impossible in a model where bureaucrats only have known fixed tastes and potential commenters only seek changes in policy. For policy seeking organizations to lobby but fail to influence policy requires that they may either be wrong about an agency's allocative bias or their ability to shift it. Incorporating political information allows change and uncertainty in an agency's biases. 
% While it is possible that commenters greatly misestimate an agency's durable allocative bias towards them (i.e. the agencies taste for them), it is more plausible that they misestimate their ability to influence the agency or the agency's strategic environment in a particular case. 
% Indeed, if taste is the only kind of allocative bias, then the only thing that can explain variance in participation is variation in the location of the proposed rule. The best empirical methods to estimate policy location generally assume that the spatial location of proposed rules written by the same people will be the same. Yet a potential commenter's anticipated ability to affect the agency's strategic environment or beliefs may be likely to vary significantly from rule to rule. It may vary with the level or distribution of economic impact, salience, sympathetic affected populations, timing (e.g. related to elections), recent court victories, proximate legal precedent, or any number of correlates of political context and ammunition for persuasion.
The result of commenting without rule change also becomes possible if commenters are allowed a strategy of ``going down fighting'' and incentives to do so. 



%%%%%%%%%%%%%%%%%%%%%%%%%%%%%%%%%%%%%%%%%%%%%%%%%%%%%%%%%%%%
\section{Methods}

\subsection{Measuring mass engagement and political information}
\label{whyMail-methods}

\subsection{Measuring Coalitions}



\begin{figure}[h!]
    \centering
    \caption{Step 1: Explaining Mass Mobilization and Mass Engagement} 
    \label{fig:causal-whymail-test}
\tiny
\begin{tikzpicture}[%
    node distance=1.2cm,
    auto,
    text width=1.5cm,
dnode/.style={diamond, align=center, aspect=2, fill=green!5,draw=green!60, very thick, minimum size=2cm},
squarednode/.style={rectangle, align=center, aspect=1, draw=red!60, fill=red!5, thick, minimum size=1cm},
pnode/.style={ellipse, align=center, aspect=1, draw=black!60, fill=black!5, thick, minimum size=1cm},
title/.style={rectangle, align=center, aspect=1, minimum size=2cm},
]

% Group Nodes
\node[pnode]      (groupdemands) {Group Demands};
\node[dnode]        (groupdecides) [right=of groupdemands] {Lobbying Strategy};
\node[squarednode]      (groupinfo) [right=of groupdecides] {Technical Information};


% Group Lines
\draw[->] (groupdemands.east) -- (groupdecides.west);
\draw[->] (groupdecides.east) -- (groupinfo.west);

% Titles
% \node[title]      (1) [above=of draft] {Policy};
% \node[title]      (2) [above=of groupdemands] {Preferences};
% \node[title]      (4) [above=of groupinfo] {Information/ Signal};
% \node[title]      (3) [above=of groupdecides] {Observed Behavior};
% \node[title]      (5) [above=of policy] {Policy'};

\node[text centered]      (mobilizing) [below=of groupdecides] {Mass\\ Mobilization};

% political info
\node[rectangle, minimum width =2cm, minimum height = 3cm, draw=red!60, fill=red!5,  thick]      (politicalinfo) [below=of groupinfo] {};

\node[text centered]      (politicalinfotext) [below=of groupinfo] {Political Information};

\node[text centered]      (mobilizing) [below=of groupdecides] {Mass\\ Mobilization};

\node[squarednode]      (publicinfo) [below=of politicalinfotext] {Perceived Public Opinion};
\node[dnode]      (publicdecides) [left=of publicinfo] {Mass\\ Engagement};
\node[pnode]        (publicdemands) [left=of publicdecides] {Latent Public Demands};


% public Lines
\draw[->, line width=2] (publicdemands.east) -- (publicdecides.west);
\draw[->, line width=2] (publicdemands.north east) -- (groupdecides.south west);
\draw[-, line width=2] (groupdecides.south) -- (mobilizing.north);
\draw[->, line width=2] (mobilizing.south) -- (publicdecides.north);
\draw[->] (publicdecides.east) -- (publicinfo.west);



\end{tikzpicture}
\end{figure}
\normalsize


\subsubsection{Identifying participants}
Previous studies of rulemaking stress the importance of coalitions \citep{Yackee2006a}. Scholars have measured coalitions of organized groups but have yet to be able to attribute citizen comments to the coalition that mobilized them.
Metadata on participants in rulemaking including the date and author of comments (including the type of author, i.e. business, business group, citizen, public interest group, etc.)% and briefs
allows me to track and compare relative alignment  across venues and over time to assess whose ideas and interests are reflected at each stage of policymaking and in policy processes over time. % and review, for example from a statute or executive order, to the agency rule(s), to review by the White House, to court opinions. 
Unfortunately, metadata regarding the authors of comments and court briefs are often inconsistent and incomplete. As this information is key to assessing who influential actors are, improving these data is a significant data-organization task. After cleaning the corpus of comment texts, simple text search matching organization and individual names across texts, especially those named as comment authors will help systematically link individuals and groups who may participate in different coalitions and under different names over time. This help to identify formal coalitions of organizations that sign onto the same comment as well as experts and citizens mobilized by advocacy campaigns to submit separate comments.

\subsubsection{Identifying coalitions }

Having identified who is participating in rulemaking over time, the next step is to identify who is lobbying together. 

When actors sign onto the same comment, it is clear that they are lobbying together. This generally takes two forms. Businesses and groups representing allied industries often co-sign carefully crafted suggestions that reflect their common interest. We expect this to occur when the benefits of coordination outweigh the costs (Yackee and Yackee 2006). The other form this take is public campaigns that ask citizens to submit a form letter, often alongside other actions such as protests. These occasional bursts of civic participation may affect rulemaking (Coglianese 2001), but this is yet to be tested. %In the first form, many of the businesses are repeat players and I record them individually. In the second form, the advocacy groups are repeat players, and I recorded their participation, but it would be citizens who participate are likely not and I record the number of these comments as an amplitude parameter for the text they signed and I attribute form-letter texts to the advocacy groups promoting them.

Various businesses, advocacy groups, and citizens often comment separately even when they aligned. The comment process is open to anyone and it is often not worthwhile for all actors to coordinate their messages. There may be many dimensions of demands and it is unclear to which coalition many comments belong.

Classifying comments into common groups is a task well suited for a single membership topic model.\footnote{This is in contrast to the mixture model I use to estimate the distribution of multiple topics in each document and each coalition} This model clusters documents by the frequency they use different words. Being classified together does not mean that the documents all address exactly the same distribution of substantive issues, just that how issues are discussed is similar relative to the full set of documents.

% Identifying when commenters change coalitions is key for testing policy feedback theories about how policies reorganize political coalitions. I do this by indexing rules over time and adding a parameter for the probability that an actor switches from one coalition to another at each point in time. This allows the model to achieve a better fit by reclassifying an actor after some point in time. These actor- and coalition- specific points in time are a key output of this approach required to test theories of how policies reorganize political coalitions. 

Bounding the scope of this model (i.e. the policy system) is a challenge. On the one hand, each agency deals with many issues of interest to different coalitions. On the other hand, many lobby across multiple agencies. I opt to model coalition advocacy at a fine scale based on Office of Management and Budget agency sub-function codes, but I may try to link across related issues and agencies. 

\subsection{Assessing effects on elected officials' oversight behavior} \label{principals-methods}
Measuring indirect influences}
To assess the direct influence pathway, I estimate the extent to which mass mobilization around bureaucratic actions makes members of congress or White House officials more likely to engage or react and whether such mass mobilizations is salient in subsequent litigation. 

Congressional attention to agency actions can be observed in several ways. Members of Congress who sit on oversight committees raise issues in oversight hearings, reports attached to each agency's budget appropriation, and in personal letters addressed to agency officials. Using text reuse methods I will identify when policy issues raised in draft rules and rule comments attract positive or negative attention from legislators. Using texts has several major advantages over previous measures of congressional attention and sentiment such as partisanship \citep{Yaver2016,Lewis2008}, changes in budget size, or the length of appropriations reports. Unlike partisanship, it is issue-specific and does not require assumptions about agency partisanship. While budget changes may reflect real costs, the many reasons that budgets change make it difficult to attribute changes to particular agency actions. The length of appropriations subcommittee reports may indicate the amount of attention committees pay to an agency but they do not vary significantly over time and do not indicate whether committee attention is positive or negative. 

[President and Secretary]

There are two ways to assess the courts as an indirect pathway where mobilization leads to influence. First, mass mobilization may increase the credibility of the threat of litigation. Second mass mobilization may influence the outcome of subsequent court cases over the rule. Both are difficult to measure. The first I measure with a combination of the litigation history of mobilizing groups and specific references to litigation in the comments. The second I assess by identifying instances where courts reference the number or direction of comments or other forms of protest in their decisions and compare rules to the rules under consideration in those cases. While this is rare, legal scholars have noted that " If the validity of a final regulation is challenged in court, the court's review will be based in significant part on how well the agency responded to the public's comments"  (Wagner 1995).

\subsection{Assessing effects on rulemaking and rules}
\label{influence-methods}


% MEASUREMENT 3 % FIX THIS 
% main measure
The main dependent variable here is whether a coalition got their way. I measure this in three ways. First, on a sample of rules, I hand-coded lobbying success for each lobbying coalition, comparing the change between the draft and final rule to each comment's demands on a five-point scale from "mostly as requested" to "significantly different/opposite than requested." Second, I use text-reuse methods underlying plagarism detection algorithms to identify changes between draft and final rules and count the number of new 10-word phrases that appear in the comment and final rule, but not the draft rule. Finally, I model assess the similarity in word distributions between comments and changes made to the rule.
% Alt measures
% However, policy change is a high bar for influence. Thus, I also use other measures of agency responses to lobbying efforts. 
In addition to changing policy texts, agencies may speed up or delay finalizing a rule, extend the comment period, or delay date at which the rule goes into effect. Indeed, commentors often request speedy or delayed rule finalization, comment period extensions, or a delayed effective dates. Additionally, agencies write lengthy justifications of their decisions in response to some comments but not others. This measure of attention to commentor demands will be captured by the text similarity measure. % Thus, additional dependent variables include (1) whether the comment period is extended, (2) days until rule finalization, (3) days until rule effective date, and (4) length of the agency's response to a coalition's demands. 



% \paragraph{Measuring Policy Change}
%\subsection{What Policy Texts Do?}

I measure the extent to which the text of a policy becomes more similar or less similar to the text of each public comment. 

One way to think about this is that this change represents an increase in utility for those lobbying for the change. Purposeful actors got what they asked for and presumably reap the rewards. All dimensions of disagreement collapse to the latent dimension of utility. %Actors participate and form coalitions to the extent that the expected benefits exceed the costs of doing so. Each new layer of law affects politics by altering actors' utility functions. 
Taking a broader view of politics offers a less parsimonious, but more direct interpretation. Changes in law may deliver utility, but more precisely they reflect ideas. These may be ideas about ``who gets what, when, and how,'' but also about identities, aspirations, possibilities, whose opinions matter, and who constitutes the political community, which may be difficult to reduce to a single dimension. Focusing on costs and benefits alone risks overlooking much of the ideological and interpretive work lawmaking does in constructing political communities, possibilities, and norms. Public comments in rulemaking, like other forms of policymaking, may often be about more than self-interest. 

% Indeed ``who succeeds?'' is a constraining question laden with certain ideas of what politics really is or ought to be. A better question may be ``How do certain ideas end up in law and how has this shaped politics and law over time?'' This latter formulation recognizes that ideas may end up in law as accidental to other efforts and that politics is not only who gets what but, more fundamentally, a process of deciding who we are and what we want to do together. 

% In the case of rulemaking, each notice-and-comment process creates an implicit political community based on who participates and whose interests are claimed to be represented. Regulations generally prescribe concrete rights, prohibitions, and governmental actions, making clear what was decided to be done. They also establish norms, issue frames, scientific and legal standards, and goals that inspire and constrain politics in other policy processes, especially future rulemakings on the same issue.

% Importantly, few observers would see any final rule as the final word. Rules are frequently revisited at regular intervals, challenged in court, rewritten under future administrations, rebuked by a new Congress, and occasionally swept aside by new legislation. Most often, the questions in rulemaking will be revisited by the same agency and many of the same participants, but they are not the same; both have been transformed and engage the questions on terrain reshaped by added layers of law. This means that it is essential to consider the historically contingent evolution of policy and lobbying coalitions over time. Quantitative scholarship on rulemaking rarely considers the date rules are made, much less how the politics of rulemaking may be historically contingent. I aim to address this gap. 

% One way to think about the kind of political influence I aim to study is to ...

%Recognizing this, I try to...
% \paragraph{Measuring change in policy texts}

% If available and reliably reflecting what actors want, policy positions expressed as texts have great advantages over those expressed in votes.

% GOOD BRING BACK 
% Policy disagreements are disagreements about words that give governmental force to ideas. Information is an important currency for those trying to influence policy and that rhetoric and framing can affect perceptions of facts and policies. Policy learning can be seen as an updating of where one stands, an updating of beliefs about the what is true about the world and, most specifically, an updating about which words (and thus ideas) ought to be law. Importantly, policymakers are constantly learning about new problems, facts, and policy ideas about which they had no prior position. Thus new dimensions of disagreement are created every time claims are advanced about new problem definitions, new facts, and new ideas for what the law ought to say.

% GOOD BRING BACK 
% Estimating spatial ideal points is a leading way of estimating what actors want, especially when we only observe a series of votes. The quantity of interest is what kind of policy actors ideally want, and because voting only tells us whether one is for or against a policy text, we need multiple observations in which an actor (or others who plausibly share their position) falls on each side to begin to narrow down their ideological distance from any given policy. To get multiple observations, we must assume that a number of proposed policies can be placed at different points on the same underlying dimension of disagreement. When it is persuasively argued that a number of these underlying issue dimensions more or less collapse to an even more general dimension, we further increase our observations and thus the information we have about each actor regarding that more general dimension. 

% GOOD BRING BACK 
% When, rather than up or down votes, we have the text of what each actor wanted, we have much more information about where they stand in relation to a policy text. Indeed, instead of needing to uncover more general latent dimensions and estimate actors' positions on them, we directly observe the quantity of interest: the substance, direction, and magnitude of the disagreement in each case. We can cluster these disagreements to make more general statements about the broader nature of disagreements and relative policy potions, but, unlike with voting data, this reduction is not necessary to know the extent to which actors are getting what they want as we can directly observe how much outcome texts incorporate various actors' expressed ideal language. Flattening proposed changes in texts to \textit{n} dimensions where \textit{n} is fewer than the number of unique demands made by all actors may help descriptively, but is not required analytically. For example, we may reduce textual policy demands to left and right ideologies or preferences for more or less government, and doing so can be descriptively useful, but it is neither necessary nor helpful for answering the question of whose ideas end up in law.

Importantly, my project does not attempt to answer what actors want in general, \textit{a priori} of any policy proposal. Without some reference point (existing policy, for example), what actors want is difficult to define. I assess what commenters want \textit{given} a proposed policy. What commenters request may not be a sincere representation of their ideal policy, but it is plausibly what they really want given what they believe is possible. While this may be insufficient for estimating their ideal policy, it is sufficient for measuring who gets what they ask for. 

%If the primary aim is to identify policy actors' ideological proximity to each other, the analysis can be reduced to the similarity and difference in their ideal policy texts, summed across all areas or within broad policy areas. If alternatively, we ask whose ideas end up in policy, we want to know how similar the policy outcome was to the specific suggestions of each actor and the frequency of changes in their proposed direction. % regardless of whether they are advocating for ideologically consistent, orthogonal, or even opposite positions in different cases. In practice, the unique dimensions of disagreement in each policy process are rarely perfectly parallel or orthogonal. Mapping ideological distance requires reducing to some tractable number of dimensions. Measuring rates of getting one's way do not.
%This is one great advantage of textual data compared to votes. Each observation is far richer in content and analysis requires fewer assumptions to say where actors really stand on a proposed policy.

\paragraph{Limitations.} 

Observing policy influence, especially in the final stages of policymaking is difficult. Given the momentum of political agendas and the fact that much is determined before draft rules are made public, changes are often on the margins. But such marginal victories are also the aim of business and other interest groups that are not part of a mass mobilization campaign. 

Additionally, my theory suggests that influence is likely only in cases where mass mobilization is (1) aimed at influencing policy and (2) not accurately anticipated by policymakers. Identifying these conditions these will also be difficult.
Observational studies of policy decisions are almost always frustrated by the fact that decisionmakers rationally anticipate the actions of those who would influence them, rendering this influence difficult to observe. Thus I expect to observe larger effects in cases where mobilization or the level of engagement achieved was not anticipated by agency staff. However, as long as rulewriters do not perfectly anticipate mass engagement, it should have observable, if depressed, effects. Thus, without accounting for anticipation, any effects I uncover are likely underestimated, making the methods described above conservative tests.






%---indirect-strategic, direct-normative, indirect-normative---
% by which political information may influence agency decisions.

% Most rules address long-defined problems. They are next steps advancing a policy agenda \citep{West2013} or the first steps in a new, often reverse, policy direction, it is possible that effects of ``going public'' are cumulative in a policy area over time, starting out small, but gaining agenda-setting power with sustained public attention. This may not be possible to measure with the rule-focused research design outlined above. However, if sequential rules can be linked to distinct policy agendas, my strategy could be extended to model dynamics over time following \citet{Brookhart2015}.



% Measuring Policy Change}
% Policies may shape and be shaped by many forces, including the collective action of citizens, expert opinions, and businesses interests. Yet the drivers and consequences of policymaking are difficult to disentangle. Business groups may fund scientists or advocacy campaigns to preempt or undo costly regulations. Experts and policymakers may inspire broader civic mobilization, and citizen mobilization may, in turn, shape the priorities of experts and policymakers. Some policy debates divide along lines of citizen and corporate interest or expert and popular opinion, but many entail various clusters of claims regarding the public interest, expertise, and business interests: claims about the public good, scientific truths, and the proper role of government. Inferring policy demands from identity alone and assuming static coalitions may miss much of the story. 

Participants may ask for thee general kinds of things: (1) specific changes to identified parts of the text (2) broad shifts in emphasis, what \citet{Jones2005} call a policy image, and (3) a general direction of policy change. For example, on the same Clean Power Plan rule, some may ask the Environmental Protection Agency to make specific changes to two sentences having to do with the classification of power plants. % and the division of federal and state enforcement authority. 
Others may ask for broad re-framing to focus less on economic costs and more on environmental equity and the effects of pollution on children \citep{Rinfret2011}.
Nearly all commenters will at least say whether they think the policy goes too far, not far enough, or is about right. Many comments do all of the above. 

To capture specific and broad alignment between commenter requsts and rule changes, I leverage two key pieces of information from the rulemaking record: The text comments and the change from the draft rule to the final. %I use each type of information in a different kind of analysis, one targeting specific demands and one targeting broad demands. 
Both approaches require the same initial steps. To identify how exactly the rule changed from draft to final, I use text reuse methods to identify text that remained the same and text that was added or subtracted. 
%Similarly, I identify text in comments that is not copied from the draft rule.\footnote{I may also need to exclude other kinds of text such as narratives introducing the commenter which commonly precede policy demands.} 
% The result of these two steps is the changes requested by commenters and textual change in the rule. 





% Existing measures of who gets their way in rulemaking are blunt---hand-coding texts on a few pre-defined and simplified categories such as ``pro- or anti-regulation.'' This is well motivated by theory, but in practice, it is often unclear whether a policy, on its face, increases or decreases government regulation.%, is liberal or conservative, or maps on to any other such latent dimension. 
% Another drawback of the hand-coding approach is that one must often read each comment and compare it to the rule change to identify influential groups. %If groups get their way when they lobby in many venues over time or on obscure or uncontested rules, studies focusing on highly visible and contested policy processes may miss much. Conversely, if mass mobilization in high-profile and hotly-contested rules matters, studies of rulemaking that discard the hundreds of thousands of form letters and other evidence of occasional bursts of civic engagement are unable to assess if mass participation matters. Legal scholarship suggests that mass mobilization may be important (Coglianese 2001). This limited and potentially biased empirical focus is largely due to the cost of hand-coding methods.

 % Political scientists have only begun to leverage text-analysis tools to measure political relationships (see Grimmer 2013 on priorities, Kl\"uver and Mahoney 2015 on framing, Wilkerson et al. 2015 on tracing policy ideas). I expand analysis of rulemaking from thousands of documents to hundreds of thousands and from a few variables to many. %Specifically, text-analysis tools can do two things: (1) rapidly code large amounts of text for pre-established theoretically-informed variables and (2) identify new dimensions of variation (Grimmer and King 2011). A key advantage of text-analysis methods is that they can identify new dimensions of variation one had no prior reason to suspect, suggesting new hypotheses. For example, in addition to our prior notions of who ought to be influential, we may expect that the topics, priorities, arguments, or issue frames on which comments cluster may identify new dimensions distinguishing winning and losing coalitions in different contexts.  %The literature suggests three broad rival hypotheses: \textit{The change between the draft and final rule reflects the interests of ($H_1$) underrepresented groups, ($H_2$) regulated businesses, ($H_3$) no particular class of commenters.} 
% regs.gov
% regulation.gov - sometimes they are coming in after the time period

%I refine tools to measure influence in policymaking with two aims: (1) to estimate the relative influence of different actors or texts and (2) to identify new factors that predict influence. These methods will allow new tests to advance core debates about how and why influence occurs and may benefit a wide range of scholarship. Active debates over bureaucratic autonomy and capture (Carpenter 2001, 2010, Carpenter and Moss 2014) and the influence of Congress (Clinton et al. 2014, Farhang and Yaver 2015), the courts (Lauderdale and Clark 2014), the president (Carrigan and Krazdin 2015), and public opinion (Dunleavy 2013) will all benefit from using text-analysis to measure influence. 

%Generally speaking, my method of measuring influence in each rulemaking case has two steps, each targeting a conceptually distinct type of influence. First I identify the major dimensions of variation in comments and the direction in which the policy moved in relation to those dimensions, i.e. which side(s) won. Second, I identify particularly influential texts within the winning coalition(s). 

% More specifically, to address my three descriptive questions, I propose to uses an ensemble of five text-analysis methods to identify participants, track coalitions, and quantify the relationship between various input texts and outcome text. This ensemble combines (1) citations(finding texts that mention the same organization and individual names across comments) to identify participants (2) single member topic modeling to identify coalitions, (3) Smith-Waterman alignment (text reuse) identify text fragments copied from each individual texts, (4) mixed-member topic models to identify the distribution of topics in comments and changes to rule texts, and (5) sentiment analysis to identify positive and negative positions on each topic. These are described in more detail below.

%This ensemble method has several applications in this project. First, it allows me to identify clusters (possible coalitions) of actors commenting on draft regulations and estimate the relative alignment of each of these clusters (and individual documents within them) to the draft rule and final rule. Similarly, it allows me to identify clusters of actors submitting briefs to courts reviewing these rules and their alignment with the court opinion. 

%In the remainder of this section I describe what is measured by each component of the ensemble, and then discuss how these improved measures of who participate, who lobbies together, and whose ideas end up in policy through rulemaking can provide leverage to test mechanisms of policy feedback. 

\subsubsection{Measuring lobbying success as changes in rule texts.}

Having identified coalitions by the textual similarity in comments (having removed all sentences quoting the agency's draft rule and call for comments), I identify general and specific policy demands and whether a policy changes in the direction requested by each coalition. The result is several measures of lobbying success. I then model the relationship between my measures of coalition size (i.e. comment volume), intensity, and contagion with lobbying success.


I measure whether organizations lobbying in rulemaking got what they asked for in three ways. Each captures a different dimension of success. 

% hand coding 
\paragraph{Measure 1: General direction of change.}
First, each new rule moves policy in a general direction. For a subset of rules that received mass comment campaigns\footnote{Approximately 2000 of 14,000 draft rules for agencies participating in regulations.gov)} and a matched sample of other rules, I will hand-code each rule and each coalition on a simple three-point scale: did the comment say the rule goes too far, not far enough, or is about right and did the final rule (in general) go further, retrench, or stay the same as the draft. This is similar to the coding scheme used by \citep{Potter2017} and others who code comments as requesting that rules be ``published as is, strengthened, weakened, or withdrawn. This method of identifying whether a rule seems to move in the direction requested is also similar to leading  methods of assessing influence in rulemaking---\citet{Yackee2006JOP} measure whether commenters requested for more or less regulation---and superior to self-reported influence \citep{Furlong1997}.

% text reuse - specific 
\paragraph{Measure 2: Specific changes in policy text.} Second, when specific changes are made between a draft and final rule, I use text reuse methods (like plagiarism detection) to identify comments that suggested changes in language that match observed rule changes. %To identify the adoption of specific demands, I use the same text reuse methods to identify any matches between textual changes in the rule and the changes requested by commenters. 
If final rules add the specific phrases suggested in comments or revise phrases identified in comments, this is evidence that these commenters may have received some of the changes they requested.\footnote{The significance of this kind of relationship between texts could be measured by how many words were copied, weighted by the forcefulness of these words. For example, I could create a dictionary of legally-significant words such as ``shall,’’ ``must,’’ ``enforcement,’’ and ``standard.’’ and weight textual alignment scores accordingly.} Text reuse can be measured for individual commenters and averaged over coalitions.

\paragraph{Measure 3: General change in policy text.} Third, I assess the relative similarity in word use (i.e. frequency) between each coalitions' comments and changes in rule text and preambles. 
To identify the adoption of demands for broader shifts in policy image and emphasis, I propose a relational topic modeling approach. In contrast to the classifiers I use to classify commenters into coalitions, this approach assumes that each text is a mixture over a number of topics. Each word token in a document is assigned to exactly one topic. Words and thus documents have distributions over topics. The extent to which distribution of topics that changed from the draft to the final is similar to the distribution in comments may be seen as a measure of whether the commenter got what the kind of change in policy emphasis they asked for.\footnote{
% anprm footnote 
Some rulemaking processes also have a commenting period before the draft policy is published. In these cases, commenters respond to an Advanced Notice of Proposed Rulemaking (ANPRM). A similar approach can be used in these processes with the key difference that similarities between comments and the draft rule (now the outcome text), either in specific text fragments or general topic distribution, take on a different meaning. Instead of representing changes to a policy text, it may represent common understandings of what policy already was or had to be on this topic. Changes in the final rule more plausibly represent differences in what policy could be. With respect to the ANPRM and proposed rule, it is more difficult to infer that the same result would not have occurred without their comments. While such counterfactual inference is not my purpose and both measure the same core phenomena of the words actors want becoming policy, interpretation of what this means must attend to this difference.}


I combine topic modeling approaches with text reuse methods, allowing better measurement not just of what is discussed but the topic distributions of what is being added, cut, copied, or otherwise receiving special attention. 
%I call this a relational topic modeling approach. 
Of course, all topic models focus on the relationship between text, but by making some of the text units themselves a relationship between texts with text reuse methods, the topic model takes a ``difference in difference'' form (i.e. the relationship between comment texts and text added or deleted from the final rule versus text that remained the same).  % or ``difference in similarity'' (e.g. what was copied). 
Much of the rule content is retained from one version to the next, but some content changes. This method measures how these changes relate to the changes proposed by sophisticated comments. 

I focus on changes from draft to final rule by selecting only the text that was added or subtracted. This can be thought of as a versioning problem where the agency updates the rule. To focus on what changed, I excluded sentences that appear (approximately) verbatim in both the draft and final. %I use the Smith-Waterman alignment algorithm (developed for identifying DNA matches and commonly used in plagiarism software) to identify sections of text that are close matches. 
\citet{Wilkerson2015} successfully employ this approach to identify content copied from various bills in the legislative processes leading to the Affordable Care Act.

The result is a quantitative measure of the alignment between suggestions made in comments and text added or subtracted from the draft to final rule. %A similar approach can estimate the relationship between ANPRM comments and the draft rule text, omitting the draft-to-final text reuse step. Credible intervals for these comment and topic-specific alignment scores can be calculated from posterior distributions. 

A finding that the words or topics that one actor suggested were twice as likely to appear in the final policy compared to the draft is a powerful and intuitive description of where the power to shape policy resides.\footnote{This is, perhaps, even more powerful than saying that the policy tended to shift toward their ideal point on some latent dimension, where the exact content anchoring the ideal point and the dimension is at least slightly ambiguous.}

\subsubsection{Models of lobbying success in rulemaking}

The three measures of lobbying success (general direction, specific language, general language) allow three distinct tests for hypotheses \ref{hyp:ds}-\ref{hyp:is}:
Let $a_i$ be the demands of sophisticated comments of a coalition $i$, $b_i$ be the number of comments supporting a coalition $i$, $c_i$ be the number of comments from all members of Congress supporting coalition $i$, and $d_i$ be the number of comments from members of Congress with formal oversight powers supporting coalition $i$.

The general model for all three dependent variables is approximately\footnote{As these variables are correlated, a more sophisticated modeling strategy will be needed.} 

\begin{align}
Y_i \sim \beta_0  + 
\beta_1a_i + \beta_2a_i*b_i + \beta_3a_i*c_i + \beta_4a_i*d_i
\end{align}
%\textbf{Dependent variables}

Model 1 is an ordered logit, with one observation per coalition per rule. 
The Dependent variable is the absolute distance between general change in rule change direction and the direction requested by the coalition $\in$ \{less/withdrawn, about the same, more\}: 

$Y_i$ =
\textit{$-|$Direction - Direction requested by coalition $ i |$}.

Model 2 uses a negative binomial link function.
The dependent variable is the textual alignment between specific policy suggestions and specific policy changes: 

$Y_i$ = \textit{number of words requested to be changed that were changed.}

Model 3 uses beta regression. The dependent variable is the similarity in overall word use between the comments of each coalition and changes in the rule and/or rule preamble:

$Y_i$ = \textit{$-|$Proportion of topic in rule and/or preamble change - proportion of topic in coalition $i's$ comment(s) $|$}.


% \subsection{Measuring the Role of Agency Missions and Reputations in Decision making}

% \subsubsection{Measuring agency reputations}

% \subsubsection{Linking comments and reputations

%%%%%%%%%%%%%%%%%%%%%%%%%%%%%%%%%%%%%%%%%%%%%%%%%%%%%%%%%%%%
% RESULTS
% \section{Why So Much Mail?}


% \section{Influence on Principals}


% \section{Influence on Agency Policymaking}

\section{Supplemental case study: The environmental justice movement} \label{ej}
\documentclass[
      12pt,
        ]{article}






% --- type and typeface? -----------------------

% input
\usepackage[utf8]{inputenc}

% typography
\usepackage{microtype}


\usepackage[T1]{fontenc}


% text block
\usepackage{setspace}
\usepackage[ 
              left = 1in,top = 1in,right = 1in,bottom = 1in 
            ]{geometry}

\usepackage{enumitem}
  \setlist{noitemsep}



% decimal numbering for appendix figs and tabs


% Deletes section counters
% \setcounter{secnumdepth}{0}







  \usepackage{longtable, booktabs}









  \usepackage{natbib}
  \bibliographystyle{/Users/devin/dissertation/assets/apsr.bst}
  % protect underscores in most circumstances
  \usepackage[strings]{underscore} 


% 

% \newtheorem{hypothesis}{Hypothesis}

\makeatletter
  \@ifpackageloaded{hyperref}{}{%
    \ifxetex
      % page size defined by xetex
      % unicode breaks when used with xetex
      \PassOptionsToPackage{hyphens}{url}\usepackage[setpagesize = false, 
                                                     unicode = false, 
                                                     xetex]{hyperref}
    \else
      \PassOptionsToPackage{hyphens}{url}\usepackage[unicode = true]{hyperref}
    \fi
  }

  \@ifpackageloaded{color}{
    \PassOptionsToPackage{usenames,dvipsnames}{color}
  }{
    \usepackage[usenames,dvipsnames]{color}
  }
\makeatother

\hypersetup{breaklinks = true,
            bookmarks = true,
            pdfauthor = {Devin Judge-Lord ()},
             pdfkeywords  =  {},  
            pdftitle = {The Environmental Justice Movement's Impact on Technocratic Policymaking},
            colorlinks = true,
            citecolor = black,
            urlcolor = blue,
            linkcolor = magenta,
            pdfborder = {0 0 0}}

% \urlstyle{same}  % don't use monospace font for urls


% set default figure placement to htbp
\makeatletter
  \def\fps@figure{hbtp}
\makeatother


% optional footnotes as endnotes


% ----- Pandoc wants this tightlist command ----------
\providecommand{\tightlist}{
  \setlength{\itemsep}{0pt}
  \setlength{\parskip}{0pt}
}





% --- title & section styles -----------------------


% title, author, date
  \title{The Environmental Justice Movement's Impact on Technocratic Policymaking}
 

  \author{ % author, option footnote, optional affiliation
            Devin Judge-Lord\footnote{University of Wisconsin-Madison, \href{mailto:judgelord@wisc.edu}{\nolinkurl{judgelord@wisc.edu}}. Slides and the most recent draft are available at \url{https://judgelord.github.io/research/ej}. Additional tables, figures, and replication code are available at \url{https://judgelord.github.io/dissertation/ej-appendix.html}} 
            }

% auto-format date?
  \date{2021-02-13}


% abstract
\usepackage{abstract}
  \renewcommand{\abstractname}{}    % clear the title
  \renewcommand{\absnamepos}{empty} % originally center

  \newcommand*{\authorfont}{\sffamily\selectfont}


% section titles
\usepackage[small, bf, sc]{titlesec}
  % \titleformat*{\subsection}{\itshape}
  \titleformat*{\subsubsection}{\itshape} 
  \titleformat*{\paragraph}{\itshape} 
  \titleformat*{\subparagraph}{\itshape}



%\usepackage{float}
%\floatstyle{plaintop}
%\restylefloat{table}
\usepackage{floatrow}
\floatsetup[figure]{capposition=top}
\floatsetup[table]{capposition=top}
\usepackage{multirow}
\usepackage{rotating} 
\usepackage{caption}






















\begin{document}
 

% --- PAGE: title and abstract -----------------------

  \maketitle

% \pagenumbering{gobble}
  \pagenumbering{gobble}



  \begin{abstract}
    \noindent THIS DRAFT WAS PREPARED FOR THE AMERICAN POLITICS WORKSHOP

THE MOST RECENT DRAFT IS \href{https://judgelord.github.io/research/ej/}{HERE}

\bigskip

I explore the role of public comments in rulemaking by focusing on their role in the environmental justice movement. Environmental justice concerns focus on unequal access to healthy environments and protection from harms caused by things like pollution and climate change. The ways in which agencies consider environmental justice highlights how rulemaking has distributive consequences, how the public comment process creates a political community, and how claims raised by activists are addressed. Examining thousands of rulemaking processes at agencies known to address environmental justice concerns, I find that when public comments raise environmental justice concerns, these concerns are more likely to be addressed in the final rule. However, baseline rates of addressing environmental justice in rulemaking are so low that even as the probability that agencies will address environmental justice significantly increases when commenters raise these issues, in most rules, even those where commenters raise environmental justice concerns, there is no explicit attention to environmental justice. Furthermore, even when agencies do address environmental justice concerns, they often do not make the substantive policy changes that activists demand. While the number of comments raising environmental justice concerns is positively correlated with change in policy texts, the effect of the general level of public attention is mixed. Rules with more comments are more likely to address environmental justice when they did not address it in the draft rule, but rules with more comments are less likely to change how they addressed environmental justice if they did address it in the draft rule. These results suggest that the politics of rulemaking differs when there is more public attention. Patterns also vary across agencies, possibly due to the alignment of environmental justice aims with agency missions. 

    

  \end{abstract}



% --- PAGE: contents -----------------------





% --- PAGE: body -----------------------


  \newpage
  \pagenumbering{arabic}

\noindent 
 
   
    \onehalfspacing 
   
   
\thispagestyle{empty}

\singlespacing

\centerline{\textbf{\underline{NOTE TO READER}}}

The following chapter is part of a dissertation exploring the effects of public pressure on agency rulemaking, a technocratic policy process where ``public participation'' is usually limited to sophisticated lobbying but occasionally includes millions of people mobilized by public pressure campaigns. Public comment periods on proposed policies purport to provide democratic accountability. Yet theories of bureaucratic policymaking largely ignore the occasional bursts of civic engagement that generate the vast majority of public comments on proposed rules. To fill this gap, I build and test theories about the role of public pressure in policymaking. I collect and analyze millions of public comments to develop the first systematic measures of civic engagement and influence in bureaucratic policymaking. The outline of the dissertation is as follows:

\textbf{Chapter 1 ``Agency Rulemaking in American Politics''} situates agency rulemaking in the context of American politics. Tracing broad trends over the past 40 years, I show that rulemaking has become a major site of policymaking and political conflict.

\textbf{Chapter 2 ``Why Do Agencies (Sometimes) Get So Much Mail?''} addresses who participates in public pressure campaigns and why. Are public pressure campaigns, like other lobbying tactics, primarily used by well-resourced groups to create an ``astroturf'' impression of public support? Or are they better understood as conflict expansion tactics used by less-resourced ``grassroots'' groups? I find that mass comment campaigns are almost always a conflict expansion tactic. Furthermore, I find no evidence of negativity bias in public comments. Indeed, from 2005 to 2017, most comments supported proposed rules. This is because public comments tend to support Democratic policies and oppose Republican policies, reflecting the asymmetry in mobilizing groups.

\textbf{Chapter 3 ``Do Public Pressure Campaigns Influence Congressional Oversight?''} examines the effect of public pressure campaigns on whether legislators are more likely to engage in rulemaking. This involves collecting and coding thousands of comments from Members of Congress on proposed rules with and without public pressure campaigns. These data also allow me to assess congressional oversight as a mediator in policy influence, i.e., the extent to which public pressure campaigns affect policy indirectly through their effects on legislators' oversight behaviors.

\textbf{Chapter 4 ``Do Public Pressure Campaigns Influence Policy?''} leverages a mix of hand-coding and computational text analysis methods to assess whether public pressure campaigns increase lobbying success. To measure lobbying success, I develop computational methods to identify lobbying coalitions and estimate their effect on each rule posted for comment on regulations.gov. I then validate these methods against a random sample of 100 rules with a mass-comment campaign and 100 rules without a mass-comment campaign, hand-coded for whether each coalition got the policy outcome they sought. Finally, I assess potential mechanisms by which mass public engagement may affect policy.

\textbf{Chapter 5 ``The Environmental Justice Movement and Technocratic Policymaking''} examines the discursive effects of environmental justice claims both qualitatively and quantitatively. I write about the role of Native activists and environmental groups in shaping federal environmental regulations. Looking across over 20,000 draft regulations that failed to address environmental justice issues, I find that agencies are more likely to add language addressing environmental justice in their final rules when public comments raise environmental justice concerns.

\newpage

\onehalfspacing

\setcounter{page}{1}

\hypertarget{introduction}{%
\section{Introduction}\label{introduction}}

When do activist movements change public policy? Over the last year, few topics have dominated politics more than political protest. Yet, we lack systematic evidence about the impact of social movements on modern policymaking. I examine how social movements affect policymaking by assessing the environmental justice movement's impact on policymaking in 40 U.S. federal agencies from 1992 to 2020. Environmental justice
concerns focus on unequal access to healthy environments and
protection from harms caused by things like pollution and climate
change. How government agencies consider (or fail to consider) environmental justice in policy documents allows empirical tests of theories about when institutions will address claims raised by activists.

The environmental justice movement's successes and failures in rulemaking illustrate how activists attempt to
inject ideas directly into the policymaking process. I focus on the
environmental justice movement because it offers a broad but tractable
scope for analysis and illuminates what is at stake in the politics of
rulemaking. Executive-branch policymaking has distributive consequences. How policy documents address distributive issues highlights policy processes construct communities of ``relevant'' stakeholders and ``appropriate'' criteria to evaluate policy consequences.
Mobilizing for environmental justice is an example of how social
movements mobilize norms and evaluative frameworks that are
connected to organizational identities, mission, and reputations and
thus have implications for policy decisions \citep{Carpenter2001}.

Tracing ideas like environmental justice through policy processes like rulemaking offers one way to study the mechanisms by which social movements succeed or fail to influence policy. If draft rules
do not mention EJ concerns, but
EJ concerns are raised in the public comments and
then appear in the final policy, this may be evidence that public pressure mattered. Likewise, when proposed rules do include an EJ analysis, if groups comment on it and the agency changes its discussion of EJ in the final rule, this may be evidence that public pressure mattered.

Tracing the evolution of environmental justice (EJ) analyses through several of these rulemaking processes shows that the concept is hotly contested and rarely addressed by agencies in ways that activists find acceptable. Activist pressure affected how rules address EJ in some cases but failed to affect others.

Examining all rules published by 40 agencies to regulations.gov between 1992 and 2020, I find that Activist mobilization affected policy discourse even under administrations that were explicitly hostile to their cause. When public comments raise
EJ concerns, these concerns are more likely to be
addressed in policy documents. Specifically, the number of comments mobilized (both overall and by EJ advocates specifically) is related to agencies adding language addressing EJ to final rules where the draft did not. In contrast, change in draft rules that did address EJ is inversely related to the overall level of public attention while still positively correlated with comments raising EJ concerns.
While we cannot infer that public comments cause agencies to address EJ concerns, comments may be a good proxy for activist pressure in general. Furthermore, the correlation between EJ activist mobilization and policy changes is largest and most significant in agencies with missions focused on ``environmental'' and distributional policy---the kinds of policymakers who we may expect to have
institutional and cognitive processes primed to be most responsive to
EJ concerns.

\hypertarget{environmental-justice-as-a-boundary-drawing-tool}{%
\subsection{``Environmental Justice'' as a Boundary-drawing Tool}\label{environmental-justice-as-a-boundary-drawing-tool}}

The politics of environmental justice has several convenient properties for studying the policy impact of social movements. First, discourse around policies framed
as ``environmental'' issues are, unlike issues like civil
rights and immigration, inconsistently racialized and, unlike issues
like taxes and spending, inconsistently focused on \emph{distributions} of
costs and benefits. This means that policies may or may not be framed in environmental justice terms. Despite policy almost always having disparate impacts, an ``environmental'' frame often creates a human-environment distinction and
shifts attention to non-human objects such as air, water, food, or
landscapes and away from the distribution of access to these objects or
protection from them when they are contaminated. By focusing on distributions of costs and benefits, fights over EJ analyses differ from more traditional utilitarian or preservationist analyses.

Second, compared to
other ideas around which people mobilize, ``environmental justice'' is a
fairly distinctive phrase. Most people who use this phrase share a
general definitional foundation. Even attempts to reframe the term (e.g., to focus on class rather than race or jobs rather than health) come about as dialectical moves related to the term's historical uses. Thus, when ``environmental justice'' appears in a text, it is rarely a coincidence of words; its appearance is a result of the movement or reactions to it.

Third, this phrase appears frequently
when the idea is discussed, i.e., there are few synonyms. Groups raising equity concerns on ``environmental'' issues commonly use the phrase ``environmental justice.'' Those who use narrower, related terms--including the older concept of
``environmental racism'' and the newer concept of ``climate justice''--almost always use ``environmental justice'' in their advocacy as well.

Finally, the term is relevant to rulemaking records in
particular because Executive Order 12898 issued in 1994 by President
Clinton ``Federal Actions to Address Environmental Justice in Minority
Populations and Low-Income Populations'' requires all agency
actions and policies to consider EJ implications. Executive Orders from Presidents Obama and Biden and statements from agency heads in every administration have since interpreted and reinterpreted parts of this Order, all with direct implications for rulemaking.
This does not mean that all draft or final rules address EJ, but when they
do, they tend to cite Executive Order 12898 and explicitly discuss
environmental justice. For the same reason, commenters who critique draft rules also cite this
Executive Order and use this language.

Advocates may even sue agencies for failing to satisfy the procedural requirements of E.O. 12898, giving agencies an incentive to explain how their policies and justifications address the Executive Order and the EJ concerns of commenters.
Courts may strike down rules for failing to comply with procedural requirements of the Administrative Procedures Act (APA) if the agency fails to ``examine the relevant data'' or ``consider an important aspect of the problem'' (\emph{Motor Vehicle Mfrs. Ass'n v. State Farm Mut. Auto. Ins. Co.}, 1983). This can include an agency's 12898 EJ analysis: ``environmental justice analysis can be reviewed under NEPA and the APA'' (\emph{Communities Against Runway Expansion, Inc.~v. FAA}, 2004).

\hypertarget{theory}{%
\section{Theory}\label{theory}}

Participatory processes like public comment periods, where government
agencies must solicit public input on draft policies, are said to provide democratic legitimacy \citep{Croley2003, Rosenbloom2003}, new technical information \citep{Yackee2006JPART, Nelson2012}, and political oversight opportunities \citep{Balla1998, Mccubbins1984}. While recent scholarship on agency policymaking has shed light on sophisticated lobbying by businesses, we know surprisingly little about the vast majority of public comments, which are submitted as part of public pressure campaigns.\footnote{
  As shown in \citet{judgelord2019SPSA}, most comments submitted to
  regulations.gov are part of organized campaigns, more akin to petition signatures than ``deliberative'' participation or sophisticated lobbying. Indeed, approximately 40 million out of
  50 million (80\%) of these public comments on rulemaking dockets between 2004 and 2019 were mobilized by just 100
  advocacy organizations.}
Activists frequently target agency policymaking with letter-writing campaigns, petitions, protests,
and mobilizing people to attend hearings, all classic examples of ``civic engagement'' \citep{Verba1987}. Yet civic engagement remains poorly understood in the context of bureaucratic policymaking.
While practitioners and administrative law scholars have long pondered
what to make of activists' mass comment campaigns, political scientists have had
surprisingly little to say about this kind of civic participation.

\hypertarget{technical-information-as-the-currency-of-lobbying}{%
\subsection{Technical Information as the Currency of Lobbying}\label{technical-information-as-the-currency-of-lobbying}}

Dominant theories of bureaucratic policymaking focus on how agencies learn about policy problems \citep{Kerwin2011}. Leading formal models are information-based models where comments matter by revealing information to the agency \citep{Gailmard2017, Libgober2018}, and empirical studies support the conclusion that information is the currency of lobbying in rulemaking \citep{Yackee2012, Cook2017, Gordon2018, Walters2019}.

Rulemaking is a highly technocratic and legalistic form of policymaking that explicitly privileges scientific and legal facts as the appropriate basis for decisions. Procedural requirements to consider relevant information create incentives for lobbying groups to overwhelm agencies with complex technical information, making rulemaking obscure to all but the most well-informed insiders \citep{Wagner2010}.
As \citet{Yackee2019} notes:

\begin{quote}
``to be influential during rulemaking,
commenters may require resources and technical expertise.
As Epstein, Heidt, and Farina (2014) suggest, agency rule-writers--who are often chosen because
of their technical or policy-specific expertise--privilege the type of data-driven
arguments and reasoning that are not common to citizen comments.'' (p.~10)
\end{quote}

The result is that rulemaking is dominated by sophisticated and well-resourced interest groups capable of providing new technical or legal information. Empirical scholarship finds that economic elites and business groups
dominate American politics in general \citetext{\citealp{Jacobs2005}; \citealp{Soss2007}; \citealp[Hertel-Fernandez2019;][]{Hacker2003}; \citealp{Gilens2014}} and rulemaking in
particular. While some are optimistic that requirements for agencies to
solicit and respond to public comments on proposed rules allow ``civil
society'' to provide public oversight \citep{Michaels2015, Metzger2010}, most
studies find that participants in rulemaking often represent elites and
business interests \citep{Seifter2016UCLA, Crow2015, Wagner2011, West2009, Yackee2006JOP, Yackee2006JPART, Golden1998, Haeder2015, Cook2017, LibgoberCarpenter2018}. To the extent that scholars address public pressure campaigns, both
existing theory and empirical scholarship suggest skepticism that it
matters \citep{Balla2018}.

\hypertarget{political-information}{%
\subsection{Political Information}\label{political-information}}

\citet{Nelson2012}
identify political information as a potentially influential result of
lobbying by different business coalitions. While they focus on
mobilizing experts, \citet{Nelson2012} describe a dynamic that can be extended
to mobilizing public pressure:

\begin{quote}
``strategic recruitment, we theorize, mobilizes new actors to
participate in the policymaking process, bringing with them novel
technical and political information. In other words, when an expanded
strategy is employed, leaders activate individuals and organizations
to participate in the policymaking process who, without the
coordinating efforts of the leaders, would otherwise not lobby. This
activation is important because it implies that coalition lobbying can
generate new information and new actors---beyond simply the `usual
suspects' ---relevant to policy decisionmakers.''
\end{quote}

I argue that, concerning political information, this logic extends to
non-experts in at least two ways.

\textbf{1. Information about a policy's disparate effects:}
First, while specific \emph{data} on disparate impacts of policy may require expertise, anyone can highlight a community of concern and potential distributional effects of a policy. Just as \citet{Nelson2012} found regarding the mobilizing of diverse experts, mobilizing diverse communities affected by a policy may introduce new claims from new actors about how the communities benefited or harmed by a policy should be constructed. Instead of bolstering
\emph{scientific} claims, such comments bolster \emph{political}
claims about who counts and even \emph{who exists} as a distinct, potentially affected group that deserves policymakers' attention. While bureaucratic policymaking in the United States is dominated by cost-benefit analysis that must abstract away from the distribution of costs and benefits, agencies have many reasons to consider the distributional effects of policy and often do.

\textbf{2. Public pressure as a political resource: }
Second, The number of supporters may
matter because it indicates support among relevant communities or the broader public. Again, instead of bolstering
\emph{scientific} claims, perceived public support bolsters \emph{political}
claims.

Like other forms of political participation, such as protests and letter-writing campaigns,
public pressure campaigns provide no new technical information.
Nor do they wield any formal authority to reward or sanction bureaucrats.
The number on each side, be it ten or ten million, has no legal import for an agency's response.

However, an organization's ability to expand the scope of conflict by mobilizing
a large number of people can be a valuable political resource \citep{Schattschneider1975}. \citet{Furlong1997} and \citet{Kerwin2011}
identify mobilization as a tactic. The organizations they surveyed
believed that forming coalitions and mobilizing large numbers of people
were among the most effective lobbying tactics. While \citet{Furlong1997} and \citet{Kerwin2011} focused on how
organizations mobilize their members, I expand on this understanding of mobilization as a lobbying tactic to include a campaign's broader audience, more akin to the concept of
an attentive public \citep{Key1961} or issue public \citep{Converse1964}.

Regardless of the specific claims of commenters, expanding the scope of conflict by mobilizing public attention to rulemaking may shift policymakers' attention away from the technical information provided by the ``usual suspects'' and toward the distributive effects of policy.

\hypertarget{hypotheses}{%
\subsection{Hypotheses}\label{hypotheses}}

The existing literature on bureaucratic policymaking in general---and EJ advocacy in particular---presents competing intuitions about the effect of EJ activists and the broader public in rulemaking. Below, I posit hypotheses in the direction that these advocacy groups do affect rulemaking while also noting equally plausible intuitions for the opposite conclusions. Because of the general skepticism and empirical work that has found that advocacy groups and public pressure campaigns have little to no effect on rulemaking, I set the empirical bar low: do EJ advocates and public pressure campaigns have \emph{any} effect at all on policy documents.

\hypertarget{information}{%
\subsubsection{Information}\label{information}}

\begin{quote}
\emph{Distributive Information Hypothesis}: Policymakers are more likely to change whether or how policies address distributional justice when commenters raise distributional justice concerns.
\end{quote}

As discussed above, agency policymakers have incentives to address distributive concerns, especially environmental justice, due to judicial review of compliance with the Administrative Procedures Act and E.O. 12898. By raising EJ concerns, commenters draw attention to the distribution of policy impacts--who a policy may affect. Asserting definitions and categories of stakeholders and affected groups is one type of policy-relevant information.

\begin{quote}
\emph{Repeated Information Hypothesis}: Policymakers are more likely to change whether or how policies address concerns when more commenters raise them.
\end{quote}

Scholarship on lobbying in rulemaking emphasizes the value of repeated information and coalition size \citep{Nelson2012}. This implies that the more unique comments raising EJ concerns, the more likely the coalition will influence policy.

Competing intuitions and other prior studies oppose both the \emph{Distributive and Repeated Information Hypotheses}. Scholarship on lobbying in rulemaking emphasizes the importance of novel science and technical information--things unknown to agency experts \citep{Wagner2010}. Furthermore, scholarship finds business commenters are influential, and public interest groups are not \citep[\citet{Haeder2015}]{Yackee2006JOP}. Furthermore, policymakers may be more likely to anticipate EJ concerns when they are more salient to interest groups and the public. This would mean that rules where commenters raise EJ concerns may be the \emph{least} likely to change whether or how EJ is addressed because policymakers are more likely to have considered the issues and stated their final position in the draft rule.

\begin{quote}
\emph{Policy Receptivity Hypothesis}: Policymakers that more frequently address concerns like environmental justice will be more responsive to commenters raising those concerns.
\end{quote}

Bureaucracies are specialized institutions built to make and implement certain kinds of policies based on certain goals and facts. Each agency's distinct norms and epistemic community determine whether policymakers see issues as ``environmental'' and whether they have disparate impacts that demand consideration of distributive ``justice.'' Some policymakers may see their policy area as more related to environmental justice than others and thus be more receptive to commenter concerns.

The competing intuition to the \emph{Policy Receptivity Hypothesis} is that policymakers familiar with EJ concerns are the \emph{least} likely to respond to EJ concerns because they anticipate these concerns--they are not novel. If so, agencies that rarely consider EJ may be more easily influenced by commenters who present somewhat novel information and concerns. These agencies may be less likely to have preempted critiques.

\hypertarget{public-pressure}{%
\subsubsection{Public Pressure}\label{public-pressure}}

\begin{quote}
\emph{General Pressure Hypothesis}: Policies are more likely to change when they receive more public attention (e.g., more public comments).
\end{quote}

The competing intuition against the \emph{General Pressure Hypothesis} is again that large numbers of comments indicate policy processes that were already salient before the public pressure campaign. Policymakers anticipate public scrutiny and are thus more likely to have stated their final position in the draft rule.

\begin{quote}
\emph{Specific Pressure Hypothesis}: Policies are more likely to address an issue when they receive more public attention (e.g., more public comments) \emph{and} some comments raise that issue.
\end{quote}

This hypothesis asserts that the overall level of public attention will condition policy responses to specific issues--it is the interaction between the number of total public comments and at least some of those comments raising EJ concerns that makes policy more likely to address EJ.

The competing intuition against the \emph{Specific Pressure Hypothesis} is again that large numbers of comments indicate high-salience rulemakings where the agency anticipates public scrutiny, including how it did or did not address specific issues like environmental justice. Policymakers anticipate public scrutiny and are thus more likely to have stated their final position in the draft rule.

\hypertarget{data}{%
\section{Data}\label{data}}

In order to examine whether EJ advocates or public pressure campaigns shape the discourse around policies,
I use the text of draft rules, public comments, and final rules retrieved from regulations.gov. I select rulemaking documents from agencies that published at least one rule explicitly addressing EJ from 1992 to 2020. This yields over 25,000 rulemaking dockets from 40 agencies.

Despite E.O. 12898, most rules do not address EJ. Figure \ref{fig:ej-data} shows that most draft and final rules (about 90\%) do not address EJ. Interestingly, the total number of final rules and the percent of the total addressing EJ have remained fairly stable for the time period where regulations.gov data are complete (after 2005). Every year from 2006 to 2020, these agencies published between 2000 and 3000 final rules, of which between 200 and 300 addressed EJ.

\begin{figure}

{\centering \includegraphics[width=0.99\linewidth]{/Users/devin/dissertation/Figs/ej-data-ejpr-1} \includegraphics[width=0.99\linewidth]{/Users/devin/dissertation/Figs/ej-data-ejfr-1} 

}

\caption{Number of Proposed and Final Rules Addressing Environmental Justice.}\label{fig:ej-data}
\end{figure}

Even at the Environmental Protection Agency (EPA), where most policies are clearly framed as ``environmental'' issues, a consistent minority of rules address EJ. Many agencies that almost never address EJ make policy with clear EJ effects, including Housing and Urban Development (HUD), the Nuclear Regulatory Commission (NRC), and the Office of Surface Mining (OSM). A majority of rules addressed EJ only in a few years at a few agencies that publish relatively few rules, such as the Council on Environmental Quality (CEQ), Federal Emergency Management Agency (FEMA), and several Department of Transportation (DOT) agencies (the Federal High Way Administration, Federal Railroad Administration (FRA)). Figure \ref{fig:ej-data-agencies100} shows the number of rulemaking projects at each agency over time by whether they ultimately addressed EJ.

\begin{figure}

{\centering \includegraphics[width=0.99\linewidth]{/Users/devin/dissertation/Figs/ej-data-agencies100-1} 

}

\caption{Number of Proposed and Final Rules Addressing Environmental Justice at the Environmental Protection Agency (EPA), Department of Housing and Urban Development (HUD), the Nuclear Regulatory Commission (NRC), Office of Surface Mining (OSM), Council on Environmental Quality (CEQ), Federal Emergency Management Agency (FEMA), Federal High Way Administration, Federal Railroad Administration (FRA), National Highway Transportation Saftey Administration (NHTSA), and Fish and Wildlife Service (FWS)}\label{fig:ej-data-agencies100}
\end{figure}

\hypertarget{comments}{%
\subsection{Comments}\label{comments}}

Figure \ref{fig:ej-comments} shows the number of comments on each proposed rule published between 1992 and 2020. Red circles indicate rules where no commenters raised EJ concerns. Blue Triangles indicate rules where they did. The bottom row of plots shows the subset of rules where ``environmental justice'' appeared in neither the draft nor the final rule. The middle row of plots show rules where ``environmental justice'' appeared in the final but not the draft. My first analysis compares these two rows. The top row of plots shows rules where ``environmental justice'' appeared in both the draft and final rule. My second analysis compares rules in this first row. Predictably, commenters most often raised EJ concerns on rules in the first row.

\begin{figure}

{\centering \includegraphics[width=1\linewidth]{/Users/devin/dissertation/Figs/ej-comments-1} 

}

\caption{Number of Comments on Proposed and Final Rules and Whether Comments Raised Environmental Justice Concerns}\label{fig:ej-comments}
\end{figure}

\hypertarget{interest-groups-and-second-order-representation}{%
\subsection{Interest Groups and Second-order Representation}\label{interest-groups-and-second-order-representation}}

When lobbying during rulemaking, groups often
make dubious claims to represent broad segments of the public \citep{Seifter2016UCLA}. It is thus insufficient to know which groups participate. We
also need to know who these groups claim to represent and whether those people are actually involved in the decisions of the organization.

I investigate who is raising EJ concerns in two ways.
First, I identify the top organizational commenters such as tribes,
businesses, and nonprofits that are using EJ language
and investigate whom these groups represent. Second, for comments where a
citizen signed their name, I compare surnames to their racial and ethnic
identity propensities with respect to the U.S. Census. Together these
two pieces of information allow me to comment on ``second-order'' representation, i.e., the extent to which public comments are
representative of the groups they claim to represent \citep{Seifter2016UCLA}.

\hypertarget{organizations-raising-ej-concerns-on-the-most-rulemaking-dockets}{%
\subsubsection{Organizations Raising EJ Concerns on the Most Rulemaking Dockets}\label{organizations-raising-ej-concerns-on-the-most-rulemaking-dockets}}

The top mobilizer of comments mentioning ``environmental justice'' between 1992 and 2020 was the Sierra Club, with over 340,000 comments on dozens of rules. While it is a membership organization whose members pay dues, elect the leaders of local chapters, and have some say in local advocacy efforts, its policy work is directed by a more traditional national advocacy organization funded by donations, including over \$174 million from Bloomberg Philanthropies that funded several of the public pressure campaigns in these data. Like other national advocacy groups, the Sierra Club advocates on behalf of ``EJ and Frontline'' communities, but those individuals have little formal say in the national organization's lobbying decisions. The Sierra Club does have a major program arm dedicated to Environmental Justice and many local programs. As a federated organization with many local efforts, it is difficult to generalize about second-order representation, which likely varies across its campaigns. The National Board of Directors adopted a statement on social justice in 1993 and principles on environmental justice in 2001. The national website does contain regular Spanish language content.

The second most prolific organizer of EJ comments was Earthjustice, with over 175,000 comments on many of the same rules that the Sierra Club lobbied on. Earthjustice is primarily engaged in litigation on
behalf of environmental causes. Their website boasts 2.2 million
supporters, but it is not clear who they are or if they play any role in
the advocacy strategy. A search on the website returns 360 results for
"Environmental Justice," with the top results from staff biographies
who work on more local or targeted campaigns, such as environmental conditions
for the incarcerated, but the EJ language used on the
main page is relatively vague. For example, ``We are fighting for a future
where children can breathe clean air, no matter where they live.''
\citep{Earthjustice2017}. The website does contain Spanish language content.

The Natural Resources Defense Council is similar to Earthjustice--a
national nonprofit funded by donations and focused on litigation--but
they also lobby and organize public pressure campaigns, including over 160,000 comments mentioning environmental justice.

CREDO Action and MoveOn are more generic progressive
mobilizers who lack a systematic focus on EJ issues,
but occasionally leverage their very large membership lists to support
EJ campaigns led by others
\citep{MoveOn.org2017, CREDO2017}.

The Alliance for Climate Protection is more of an elite political group founded by former Vice President Al
Gore.

We Act and Communities for a Better Environment both have environmental
justice in their central mission statement. We Act was founded by
community leaders in Harlem, NY, to fight environmental racism and
advocate for better air quality \citep{WEACT2017}. Communities for a Better Environment has projects throughout California but is particularly
active in Oakland \citep{CBECAL2017}. Much of the
content of their website is in both English and Spanish. Both
organizations focus primarily on ``low-income communities of color'' and
thus frame their work primarily in terms of race and class. While both
organizations participated in national policymaking We Act is more
focused on communities in Harlem and New York, whereas Communities for a
Better Environment casts a wider frame: "CBE's vision of environmental
justice is global--that's why the organization continues to participate
in such international efforts as the Indigenous Environmental Network
and the Global Week of Action for Climate Justice" \citep{CBECAL2017}.

While not a large portion of EJ comments, companies repeatedly raise research about the unequal impacts of policy in order to frame these issues as a legitimate but unresolved scientific debate that is not yet conclusive enough to base regulations
on. For example, in one comment, the Southern Company wrote:

\begin{quote}
``People with lower SES are exposed to almost an order of magnitude
more traffic near their homes (Reynolds et al., 2001), and live closer
to large industrial sites and are exposed to more industrial air
pollution (Jerrett et al., 2001). Legitimate health concerns must be
addressed. But adopting standards with a scientific basis so uncertain
that health improvement cannot be assured is not sound public health
policy.''
\end{quote}

Like many companies, they claim to represent their customers:
"electric generating companies and their customers are expected to bear
much of the burden" of regulations \citep{Hobson2004}.

With respect to second-order representation, it appears that the groups
most often using the language of environmental justice may do so
sincerely but do not themselves represent affected communities. Several
groups representing local communities and led by community leaders have
participated, but not nearly as often or with the same intensity as the
``big greens.'' This highlights the importance of resources as a condition
for mobilizing. Not all groups who may benefit from political
information are able to leverage it because they lack the resources to
invest in a campaign. However, it may be the case that smaller, more
member-driven groups join coalitions with groups with more resources who
mobilize on their behalf.
Finally, a third, much less common type of commenter raises EJ issues
as a way to re-frame them as ongoing debates and thus undermine their
urgency. I call this reason for engaging as ``breaking a perceived
consensus.'' In a way, the fact that an energy company felt compelled to
acknowledge and question EJ concerns suggests their
importance for policy outcomes.

\hypertarget{commenter-race}{%
\subsubsection{Commenter Race}\label{commenter-race}}

To estimate the racial distribution of those who comment
using EJ language, I select commenters who
signed their full name on their comment with a
surname appearing in census records. Figure
\ref{fig:ejcommentsbyrace} shows a probabilistic racial
distribution of commenters who raise ``environmental justice'' concerns in
their comments based on the distribution of self-reported racial identities
associated with surnames as recorded in the 2010 census\footnote{I recode ``Hispanic'' as ``Latinx''}. This distribution is estimated using the proportion of
people with a given surname identified as belonging to each racial
category (from this limited set of options). It does not
assign specific individuals to racial categories, but instead represents each commenter as a set of probabilities adding up to 1 and the estimated racial distribution of the sample as a sum of individual probabilities.

\begin{figure}

{\centering \includegraphics[width=0.49\linewidth]{/Users/devin/dissertation/Figs/race-prob} 

}

\caption{Estimated Racial Distribution from Census Surnames of Commenters raising "Environmental Justice" Concerns in Rulemaking}\label{fig:ejcommentsbyrace}
\end{figure}

Compared to the overall distribution in the 2010 census, this sample of commenters
appears to be disproportionately Black and less than proportionately
Latinx or Asian, with just slightly fewer Whites relative to the
national population. This makes sense given that environmental justice
theorizing and activism have been led by African Americans
\citep{Bullard1993}.

\hypertarget{tracing-ideas-through-rulemaking}{%
\section{Tracing Ideas Through Rulemaking}\label{tracing-ideas-through-rulemaking}}

\hypertarget{environmental-justice-as-a-contested-concept}{%
\subsection{``Environmental Justice'' as a Contested Concept}\label{environmental-justice-as-a-contested-concept}}

The use of an environmental justice frame does not always imply the same
communities of concern. Environmental justice emerged out of movements
against environmental racism, especially the disposal of toxic
substances in predominantly-Black neighborhoods \citep{Bullard1993}. However, the term
quickly took on a wider array of meanings, encompassing various marginalized groups. President Clinton's 1994 Executive Order on
Environmental Justice required all parts of the federal government
to make ``addressing disproportionately high and adverse human health or
environmental effects of programs, policies, and activities on minority
populations and low-income populations'' a core aspect of their mission.
This meant considering disproportionate effects during rulemaking.

In 2005, Environmental Protection Agency (EPA) political appointees reinterpreted the Order, removing race as a factor in identifying and prioritizing populations. This move was criticized by activists and two reports by EPA's own Office of Inspector General. President Obama's EPA Administrator named EJ as one of their top priorities but also faced criticism from activists for only paying lip service to environmental racism.

In an October 2017 proposed rule to repeal
restrictions on power plant pollution, the Trump EPA acknowledged that
``low-income and minority communities located in proximity to {[}power
plants{]} may have experienced an improvement in air quality as a result
of the emissions reductions.'' Because the Executive Order requires
attention to environmental justice and because the Obama EPA discussed
it when promulgating the rule, the environmental justice implications could not safely
be ignored. However, the Trump EPA contended that the Obama EPA ``did not
address lower household energy bills for low-income households {[}and
that{]} workers losing jobs in regions or occupations with weak labor
markets would have been most vulnerable'' (EPA 2017). Like the comments of the Southern Company and other regulated industry commenters, these statements frame the distribution of jobs and electricity costs as EJ issues in order to push back against policies that would equalize distributions of health impacts of pollution.

The major conflict over the role of race in EJ analyses is one of many conflicts that the environmental justice movement has caused to be fought somewhat on its terms. To illustrate how these definitional conflicts shape rules and rulemaking, the next section briefly reviews the decades-long policy fight over regulating Mercury pollution.\footnote{This case and other examples in this article were not selected as the most similar, most different, or a representative sample. They emerged from reading hundreds of rulemaking documents where agencies did and did not respond to comments raising EJ concerns. Their purpose is to assess whether the cases in the quantitative analysis are plausibly what they appear to be: that changes in rule text are, sometimes, causally related to public comments and that non-changes are cases of agencies disregarding comments, not some accident of the data or measures. Tracing a few rulemaking processes also helped to avoid analytic pitfalls. For example, one case where an agency did an EJ analysis and then appeared not to respond to a comment discussing EJ was, in fact, due to the fact that the commenter included an annotated version of the draft rule their comment, adding only ``no comment'' next to the 12898 section. To correct this, I removed text copied from the proposed rule from comments in pre-processing. The qualitative reading also confirmed other key assumptions, such as the fact that advocates do, in fact, use ``environmental justice'' when they raise distributional concerns, even on many rules that are not about issues traditionally considered ``environmental'' because of its power to give distributional justice claims legal purchase.}

\hypertarget{the-evolving-distributional-politics-of-mercury-pollution}{%
\subsubsection{The Evolving Distributional Politics of Mercury Pollution}\label{the-evolving-distributional-politics-of-mercury-pollution}}

Fundamental definitions of the public good and minority rights are
implicit in agency rules. The public comment process offers an
opportunity to protest these definitions. Protest is one way that
marginalized groups can communicate opinions on issues to government
officials \citep{Gillion2013}. In the case of the EPA's Mercury Rules, two
such issues were decisive. First, as with many forms of pollution,
mercury-emitting power plants are concentrated in low-income, often
non-White communities. Second, certain populations consume much more
locally-caught freshwater fish, a major vector of Mercury toxicity.
Studies inspired by the political controversy around the Mercury Rules
found high risk among certain communities, including ``Hispanic, Vietnamese, and
Laotian populations in California and Great Lakes tribal populations
(Chippewa and Ojibwe) active on ceded territories around the Great
Lakes'' (EPA 2012). Thus the standards that EPA chooses are fundamentally
dependent on whom the regulation aims to protect: the average citizen,
local residents, or fishing communities. This decision has disparate
effects based on race and class because of disparate effects based on
geography and different cultural practices. Such disparate impacts are
often called EJ issues.

In December 2000, when the EPA first announced its intention to regulate
Mercury from power plants, the notice published in the Federal Register
did not address EJ issues, such as the disparate
effects of mercury on certain populations. Risks were only discussed in
reference to ``the U.S. population'' (EPA 2000). When the first draft rule
was published, it only discussed the effects of the rule on regulated
entities, noting that

\begin{quote}
``Other types of entities not listed could also be
affected'' (EPA 2002).
\end{quote}

Commenting on this draft, Heather McCausland of
the Alaska Community Action on Toxics (ACAT) wrote:

\begin{quote}
``The amount of methyl-mercury and other bioaccumulative chemicals
consumed by Alaskans (especially Alaskan Natives) could potentially be
much higher than is assumed\ldots{} {[}and could increase{]} the Alaskan Native mortality rate for
babies, which according to the CDC is 70\% higher than the United States
average. Indigenous Arctic \& Alaskan Native populations are some of
the most polluted populations in the world.
Global transport \& old military sites contaminate us too.''
\end{quote}

After receiving hundreds of thousands of comments and pressure from
tribal governments and organizations, a revised proposed rule echoed McCausland's
comment noting that

\begin{quote}
``Some subpopulations in the U.S., such as Native
Americans, Southeast Asian Americans, and lower-income subsistence
fishers may rely on fish as a primary source of nutrition and/or for
cultural practices. Therefore, they consume larger amounts of fish than
the general population and may be at a greater risk of the adverse
health effects from Hg due to increased exposure'' (EPA 2004).
\end{quote}

After nearly a million additional public comments, a revised proposed
rule ultimately included five pages of analysis of the disparate impacts
on "vulnerable populations" including ``African Americans,'' ``Hispanic,''
``Native American,'' and ``Other and Multi-racial'' groups (EPA 2011). In the final rule, ``vulnerable populations'' was replaced
with ``minority, low income, and indigenous populations'' (EPA 2012). The EPA
had also conducted an analysis of sub-populations with particularly high
potential risks of exposure due to high rates of fish consumption as well
as additional analysis of the distribution of mortality risk by
race.

Of this second round of comments, over 200 unique comments explicitly raised
EJ issues. The Little River Band of Ottawa Indians
expressed the Tribe's

\begin{quote}
``frustration at trying to impress upon the EPA the
multiple and profound impacts of mercury contamination from a Tribal
perspective. Not to mention the obligations under treaties to
participate with tribes on a `Government to Government' basis. At
present, no such meetings have occurred in any meaningful manner with
EPA Region V, the EPA National American Indian Environmental Office, nor
the State of Michigan's Department of Environmental Quality.''
\end{quote}

They
conclude that ``Although EPA purported to consider environmental justice
as it developed its Clean Air Mercury Rule, it failed utterly. In this
rulemaking, EPA perpetuated, rather than ameliorated, a long history of
cultural discrimination against tribes and their members'' (Sprague
2011).

Did comments like these play a role in EPA's changed analysis of
whom Mercury limits should aim to protect?
Given the many potential sources of influence, it may be difficult to
attribute causal effects of particular comments on a given policy.
However, comments may serve as a good proxy for the general mobilization
of groups and individuals around an administrative process, and it is
not clear why the EPA would not address EJ in the first
draft of a rule and then add it to subsequent drafts in the absence of
activist pressure. Electoral politics does not offer an easy
explanation. The notice proposing the Mercury Rule was issued by the
Clinton administration, the same administration that issued the
Executive Order on Environmental Justice, and the subsequent drafts that
did address EJ issues were published by the Bush
administration, which had a more contentious relationship with
EJ advocates, while Republicans controlled both
houses of Congress. The expansion of the analysis from one draft to the
next seems to be in response to activist pressure.

\hypertarget{measuring-policy-change}{%
\section{Measuring Policy Change}\label{measuring-policy-change}}

Having shown how changes in rule text can be related to public comments, I assess the overall relationship between comments and policy change. I use two indicators of policy change to model the effect of public comments on policy: \emph{whether} a rule addresses EJ and \emph{how} it addresses EJ. Both measures represent a fairly low bar, indicating whether the agency explicitly paid any attention to EJ. This is appropriate given that prior research shows little to no effect of public comments from advocacy groups as well as little attention to EJ in particular.

\hypertarget{measure-1-adding-text-addressing-ej-to-final-rules}{%
\subsection{Measure 1: Adding Text Addressing EJ to Final Rules}\label{measure-1-adding-text-addressing-ej-to-final-rules}}

For the subset of draft rules that did not address EJ, I measure whether agencies added any mention of ``environmental justice'' in the final rule. Such additions usually take the form of an ``E.O. 12898'' section where the agency justifies its policy changes with respect to some concept(s) of environmental justice. The next most common addition occurs in the agency's response to comments, explaining how the rule did not have disparate effects or that they were insignificant.

Sometimes an agency will respond to a comment and add a 12898 section. For example, the EPA responded to several commenters, including Earthjustice, the Central Valley Air Quality Coalition, the Coalition for Clean Air, Central California Environmental Justice Network, and Central California Asthma Collaborative: ``EPA agrees it is important to consider environmental justice in our actions and we briefly addressed environmental justice principles in our proposal.'' EPA had, as the commenters noted, not in fact addressed environmental justice in the proposed rule, which approved California rules regulating particulate matter emissions from construction sites, unpaved roads, and disturbed soils in open and agricultural areas. EPA did add a fairly generic 12898 section to the final rule but did not substantively change its position.

Less frequently, an agency may explicitly dismiss a comment and decline to add a 12898 section. For example, EPA responded to a comment on another rule, ``One commenter stated that EPA failed to comply with Executive Order 12898 on Environmental Justice\ldots We do not believe that these amendments will have any adverse effects on\ldots minority and low-income populations\ldots Owners or operators are still required to develop SSM plans to address emissions\ldots The only difference from current regulations is that the source is not required to follow the plan'' (71 FR 20445). As these examples illustrate, agencies may add text addressing environmental justice that would in no way satisfy critics. This measure merely indicates whether the agency engaged with the claims.

\hypertarget{measure-2-changing-text-addressing-ej-in-final-rules}{%
\subsection{Measure 2: Changing Text Addressing EJ in Final Rules}\label{measure-2-changing-text-addressing-ej-in-final-rules}}

Where draft rules did address EJ, I measure whether a rule changed \emph{how} it discussed ``environmental justice'' between its draft and final publication.\footnote{Occasionally, there is more than one version of a Proposed or Final rule on a rulemaking docket. Here I opt for an inclusive measure of change that counts change from \emph{any} proposed to \emph{any} final rule. This means that if the change occurred between the first and second draft of a proposed rule, it is counted as a change. This best captures the concept of rule change. However, estimates are similar if we only count cases where a change occurred between \emph{every} version of the rule.}
When an agency addresses EJ in the draft rule, it is almost always in a section about how it addressed E.O.12898. In many cases, much of the text of final rules, including 12898 sections where they exist, remain exactly the same between draft and final versions.
To measure change, I parse draft and final rules into sentences and identify sentences containing the phrase ``environmental justice.'' If these sentences are identical, this indicates the agency did not engage with comments raising EJ concerns.\footnote{An alternative approach would be to parse documents by section and assess whether E.O.12898 sections are identical. Parsing by sentences has three advantages: it is computationally faster, it avoids problems with section numbering and other frustrations with section matching, and it captures attention to EJ outside of this section, especially in the section responding to comments. If an agency is paying attention to EJ issues, sentence matching will be most likely to detect it. However, other measures, such as the percent of EJ sentences changed, the percent of words in a 12898 section that changed, or the change in topic proportions \citep{Judge-Lord2017}, could be useful in future work.}

\hypertarget{results}{%
\section{Results}\label{results}}

\hypertarget{are-final-rules-more-likely-to-address-environmental-justice-after-comments-do-so}{%
\subsection{Are final rules more likely to address environmental justice after comments do so?}\label{are-final-rules-more-likely-to-address-environmental-justice-after-comments-do-so}}

This subsection presents results from an analysis of draft
rules, comments, and final rules. Descriptively, figure
\ref{fig:ej-PR-winrate-1}
shows that, in general, where environmental
justice is not addressed in the draft rule, it is also not addressed in the final, but a higher percent of rules do add EJ language when comments raise EJ concerns. There is a large difference in the rate of addressing EJ between rules where commenters and did (33\%) and did not raise EJ concerns (4\%). However, in most cases, agencies did not respond at all to these commenters' concerns.

\begin{figure}

{\centering \includegraphics[width=0.6\linewidth]{/Users/devin/dissertation/Figs/ej-PR-winrate-1} 

}

\caption{Proposed Rules that Did Not Address Environmental Justice}\label{fig:ej-PR-winrate-1}
\end{figure}

Figure \ref{fig:ejejPR-winrate-2} shows that overall rates of adding EJ in rules without EJ comments decreased over time, leveling out at 3\% during the Obama and Trump presidencies. The rates of adding EJ when commenters did raise these concerns is consistently much higher, but it also decreases over time, from 57\% under G.W. Bush to 36\% under Trump.
Finally, looking at selected agencies, Figure \ref{fig:ej-PR-winrate-2} shows that most rules that addressed EJ in the draft were published by the EPA. EPA had a relatively high rate of baseline change, which increased when comments raised EJ concerns. Other agencies had too few rules to make strong inferences, but many responded to comments or changed how they discussed EJ 100\% of the time when comments raised EJ concerns, while inconsistently doing so when comments did not.

\begin{figure}

{\centering \includegraphics[width=1\linewidth]{/Users/devin/dissertation/Figs/ej-PR-winrate-president-1} \includegraphics[width=1\linewidth]{/Users/devin/dissertation/Figs/ej-PR-winrate-agency-1} 

}

\caption{Percent of Rules that Changed in How they Addressed Environmental Justice by President and by Agency}\label{fig:ej-PR-winrate-2}
\end{figure}

\hypertarget{estimating-whether-environmental-justice-is-added-to-final-rule}{%
\subsubsection{\texorpdfstring{Estimating \emph{Whether} ``Environmental Justice'' is Added to Final Rule}{Estimating Whether ``Environmental Justice'' is Added to Final Rule}}\label{estimating-whether-environmental-justice-is-added-to-final-rule}}

For this analysis, I estimate a logit regression where the outcome is whether environmental justice was addressed in the final rule. The predictors are
whether EJ was addressed in the comments, the number of unique comments addressing EJ, the total number of comments, and the interaction between whether EJ was raised and the total number of comments received. I also include fixed effects for presidential administration and, in models 2 and 4, for agency as well. Thus, estimates in Models 1 and 3 include variation \emph{across} agencies, while estimates in models 2 and 4 rely on variation \emph{within} agencies. All estimates rely on variation within the presidential administration.
Model 3 is identical except for the dependent variable. Predicted probabilities shown below are for models with agency fixed effects, 2 and 4.

\begin{table}[H]
\centering
\caption{\label{tab:tables}Logit Regression Models Predicting Change in Final Rule Text}

\begin{tabular}[t]{lcccc}
\toprule
Model  & 1 & 2 & 3 & 4\\
(Dependent Variable)  & (EJ Added) & (EJ Added) & (EJ Changed) & (EJ Changed)\\
\midrule
EJ comment = TRUE & 3.348*** & 2.284*** & 0.672*** & 0.731***\\
 & (0.219) & (0.233) & (0.250) & (0.255)\\
log(comments + 1) & 0.140*** & 0.231*** & -0.093*** & -0.128***\\
 & (0.022) & (0.034) & (0.029) & (0.031)\\
Unique EJ comments & 0.005 & 0.124** & 0.036** & 0.044***\\
 & (0.006) & (0.051) & (0.015) & (0.016)\\
EJ comment x log(comments + 1) & -0.309*** & -0.194*** & 0.051 & 0.061\\
 & (0.049) & (0.066) & (0.049) & (0.051)\\
\midrule
Num.Obs. & 12234 & 12234 & 2034 & 2034\\
President FE & X & X & X & X\\
Agency FE &  & X &  & X\\
AIC & 4263.5 & 3330.9 & 2279.7 & 2232.6\\
BIC & 4322.8 & 3671.8 & 2324.6 & 2395.5\\
Log.Lik. & -2123.728 & -1619.440 & -1131.840 & -1087.306\\
\bottomrule
\multicolumn{5}{l}{\textsuperscript{} * p $<$ 0.1, ** p $<$ 0.05, *** p $<$ 0.01}\\
\multicolumn{5}{l}{\textsuperscript{} Full Table in the Online Appendix}\\
\end{tabular}
\end{table}

\hypertarget{predicted-probability-of-added-text}{%
\subsubsection{Predicted Probability of Added Text}\label{predicted-probability-of-added-text}}

As logit coefficients are not easily interpretable, I present
predicted probabilities for the types of rules of interest.
Figure \ref{fig:ej-m-PR-comments-agencyFE} shows the predicted probability of a final rule addressing environmental justice when the draft rule did not with a varying number of comments (with other variables at their modal values: President Obama, the EPA, and zero additional EJ comments).\footnote{All predicted probability plots below also use President Obama, the EPA, 0 additional EJ comments, and the median number of total comments as the baseline values, unless otherwise specified.}
At low numbers of total comments, environmental justice being raised in any one comment does have a
statistically significant and substantively large effect. For rules with less than ten comments (most rules), one comment mentioning EJ is associated with a 30\% increase in the probability that EJ will be addressed in the final rule. This supports the \emph{Distributive Information Hypothesis}. However, as in figure \ref{fig:ej-PR-winrate-1}, the resulting rate is still below 50\%---even when comments raise EJ concerns, agencies do not address them. As the number of comments increases, the probability that a rule will add text addressing EJ increases. At the same time, there is a small but negative interaction effect between the number of comments and EJ comments--the more comments, the smaller the relationship between the comments raising EJ and EJ being addressed in the rule. In the small-portion of highly salient rules with 10,000 or more, the presence of comments raising EJ concerns no longer has a significant relationship with EJ being added to the text. With or without EJ comments, these rules have about the same probability of change as those with just one EJ comment, just under 50\%. This is evidence against the \emph{Conditional Pressure Hypothesis}---the number of comments matters (i.e., the scale of public attention) matters regardless of whether these comments explicitly raise EJ concerns. However, as shown in Figure \ref{fig:ej-comments}, very few rules with 10,000 or more comments do not have at least one comment mentioning EJ, so there is a great deal of uncertainty about estimates of the impact of EJ comments with high levels of public attention.

The probability of
``environmental justice'' appearing in the final rule also increases with the number of unique comments that mention ``environmental justice'' in models 2, 3, and 4, but it does not have a significantly positive relationship in Model 1. Overall this supports the \emph{Repeated Information Hypothesis}.

\begin{figure}

{\centering \includegraphics[width=0.75\linewidth]{/Users/devin/dissertation/Figs/ej-m-PR-comments-agencyFE-1} 

}

\caption{Proposed Rules Not Addressing Environmental Justice}\label{fig:ej-m-PR-comments-agencyFE}
\end{figure}

Figure \ref{fig:ej-m-PR-agency-top} shows estimated variation in rates of adding EJ to final rules.
Agencies with the largest average rates adding EJ language are the agencies we would expect to be more receptive to EJ claims. While many agencies make what could be called ``environmental policy'' and all policy decisions have distributive consequences, the Environmental Protection Agency have the Department of Transportation (which includes the Federal Railroad Administration (FRA), Department of Transportation, Federal Motor Carrier Safety Administration (FMCSA), and the Federal Highway Administration (FHWA)) the most prominent internal guidance on EJ in rulemaking. However, differences among agencies are fairly uncertain due to the small number of rules where EJ was added at most agencies. Thus, there is more support for the **Policy Receptivity Hypothesis than against it, but differences between agencies with different missions and institutional practices regarding EJ is not clear cut.

\begin{figure}

{\centering \includegraphics[width=0.99\linewidth]{/Users/devin/dissertation/Figs/ej-m-PR-agency-top-1} 

}

\caption{Proposed Rules Not Addressing Environmental Justice}\label{fig:ej-m-PR-agency-top}
\end{figure}

The Forest Service (FS) has the highest predicted baseline rate of adding EJ to their rules. This may be because the forest service is mainly in
the business of managing forests, leasing timber rights, and controlling
wildfires. These types of decisions may have acute distributional
effects that may not be the initial focus of the agency. Forest Service rule-writers may also simply have a practice of addressing E.O.12898 in final rules but not draft rules.

Similarly, the Federal
Railroad Administration, Department of Transportation, Federal Highway
Administration, Federal Motor Carrier Safety Administration all have large baseline rates of adding EJ to final rules. These agencies are
making decisions about infrastructure projects with implications for
neighborhood environments and air quality. Environmental justice may
often come up, but there may be a lot of variation in whether the agency
then decides if they are relevant to transportation policies and
projects that are primarily about neither environmental nor justice
concerns.

Research agencies, including the Nuclear Regulatory Commission (NRC), National
Oceanographic and Atmospheric Administration (NOAA) also have a statistically significant but small baseline rate of adding EJ to final rules, indicating a large difference but rare event. Indeed, as we saw in Figure \ref{fig:ejejPR-winrate-1}, there are only three NRC proposed rules in these data that received EJ comments, and the agency added EJ language to two of them. Thirty-one draft NOAA rules received EJ comments, five of which added EJ language.

\hypertarget{are-rules-more-likely-to-change-how-they-address-environmental-justice-when-comments-mention-it}{%
\subsection{Are rules more likely to change how they address environmental justice when comments mention it?}\label{are-rules-more-likely-to-change-how-they-address-environmental-justice-when-comments-mention-it}}

Turning to rules that do address EJ in the draft, we also see responsiveness to comments raising EJ concerns, now measured as whether any sentences containing ``environmental justice'' changed between draft and final rule.
Figure \ref{fig:ejejPR-winrate-1} shows that a higher percent of rules change when comments raise EJ concerns. Overall rates of change in rules without EJ comments are fairly consistent across presidencies.

\begin{figure}

{\centering \includegraphics[width=0.6\linewidth]{/Users/devin/dissertation/Figs/ejejPR-winrate-1} 

}

\caption{Percent of Rules that Changed in How they Addressed Environmental Justice}\label{fig:ejejPR-winrate-1}
\end{figure}

Finally, looking at selected agencies, Figure \ref{fig:ejejPR-winrate-2} that most rules that addressed EJ in the draft were published by the EPA, which had a high rate of baseline change, which increased when comments raised EJ concerns. Other agencies had too few rules to make strong inferences, but many changed how they discussed EJ 100\% of the time when comments raised it, while inconsistently doing so when comments did not.

\begin{figure}

{\centering \includegraphics[width=1\linewidth]{/Users/devin/dissertation/Figs/ejejPR-winrate-agency-1} \includegraphics[width=1\linewidth]{/Users/devin/dissertation/Figs/ejejPR-winrate-president-1} 

}

\caption{Percent of Rules that Changed in How they Addressed Environmental Justice by Agency}\label{fig:ejejPR-winrate-2}
\end{figure}

\hypertarget{estimating-change-in-how-environmental-justice-is-addressed}{%
\subsubsection{\texorpdfstring{Estimating Change in \emph{How} "Environmental Justice is Addressed}{Estimating Change in How "Environmental Justice is Addressed}}\label{estimating-change-in-how-environmental-justice-is-addressed}}

Models 3 and 4 in Table \ref{tab:tables} are the same as Models 1 and 2, except that the dependent variable is now whether any sentences mentioning EJ changed between the draft and final rule.

\hypertarget{predicted-probability-of-changed-text}{%
\subsubsection{Predicted Probability of Changed Text}\label{predicted-probability-of-changed-text}}

Controlling for average rates of change per agency and the number of comments, Figure \ref{fig:ej-mejPR-president-mean-1} shows there are no significant differences in baseline rates of adding EJ language across the Bush, Obama, and Trump presidencies. All are significantly lower than the rate in the Clinton administration, which could be a result of Clinton's Executive Order or simply an artifact of the limited sample of rules posted to regulations.gov before the mid-2000s.

\begin{figure}

{\centering \includegraphics[width=0.75\linewidth]{/Users/devin/dissertation/Figs/ej-mejPR-president-mean-1} 

}

\caption{Predicted Change in How Environmental Justice is Addressed Between Draft and Final Rules by Number of Comments}\label{fig:ej-mejPR-president-mean-1}
\end{figure}

The number of comments matters, but in the opposite direction posited by the \emph{General Pressure Hypothesis}. The logged total number of comments has a significantly negative relationship with the probability the final rule text changes. The more comments there are on a proposed rule, the less likely it is to change. Rules are more likely to change when they receive \emph{fewer} comments. The total number of comments thus has the opposite relationship to \emph{how} rules that already addressed EJ changed as it did to \emph{whether} rules added any EJ text. While the \emph{General Pressure Hypothesis} held for adding EJ text, the opposite is true for changing a text that already addressed EJ. Instead, this result supports the competing intuition that more salient rules may be harder to change because the agency has anticipated public scrutiny. Their position set forth in the draft is more likely to be the position of the final rule.

As shown in Figure \ref{fig:ej-mejPR-comments}, EJ comments have a small but discernable relationship to the probability of rule change at typical (low) numbers of comments. However, the relationship between the number of comments and the probability of rule change is different when some comments mentioned environmental justice. Specifically, the interaction term is positive; more comments overall means that environmental justice comments have a larger effect. When the total number of comments is larger, the positive interaction between the number of comments and comments mentioning EJ means that the predicted probability of change in how a rule addresses EJ is larger when the agency receives comments mentioning EJ.

\begin{figure}

{\centering \includegraphics[width=0.7\linewidth]{/Users/devin/dissertation/Figs/ej-mejPR-comments-1} 

}

\caption{Predicted Change in How Environmental Justice is Addressed Between Draft and Final Rules by Number of Comments}\label{fig:ej-mejPR-comments}
\end{figure}

\hypertarget{conclusion}{%
\section{Conclusion}\label{conclusion}}

This analysis has illustrated the importance of ideas in policymaking and challenges in assessing their impact. The results suggest that when issue frames
like environmental justice are raised, there is a higher probability that
policymakers consider the effects on marginalized populations. However, baseline rates of addressing environmental justice in rulemaking are so low that even as the probability that agencies will at least mention environmental justice increases when commenters raise these issues, in most rules, even those where commenters raise EJ concerns, there is no explicit attention to EJ. This holds across presidents G.W. Bush, Obama, and Trump. Indeed, there are surprisingly small differences across administrations in both baseline rates of considering EJ and the relationship between comments and change in rule text.
Importantly, there is a great deal of variation across agencies, suggesting that predispositions, policy receptiveness, and responsiveness to comments conditional on an
institutional being predisposed to such an analysis.

Furthermore, it is essential to note that the policy outcomes suggested
by environmental justice analysis depend on how minority populations are
defined. In some cases, those raising environmental justice concerns
present it as an issue of wealth or income inequality, leading policy to
account for disparate impacts on low-income populations. In other cases,
groups raise claims rooted in cultural practices, such as fish
consumption among certain tribes. As occurred in the Mercury Rule, the
analysis in subsequent drafts of the policy used evaluative criteria
specific to these communities.

The ability of a frame like environmental justice to construct certain
populations as deserving of consideration means that policy outcomes
will depend on the specific environmental justice concerns raised. In
this respect, second-order representation may become important. National
advocacy organizations may frequently request that regulators protect
``all people'' or even ``low-income communities of color.'' However, this
more generic advocacy may not lead to the same outcomes as groups that
present specific local environmental justice grievances that are unique
to a community. In between generic progressive advocacy organizations
and community-based organizations are organizations like the Sierra Club and Earthjustice, who, despite their national focus, frequently engaged in
community-specific litigation or place-based and thus raise these local concerns in
national policymaking. Given the importance of federal policy for local environmental outcomes, and advocacy organizations' potential to draw policymakers' attention to environmental justice issues, future research should examine the quality of partnerships between frontline communities and national advocacy organizations.

The examination of which groups raise environmental justice concerns and
second-order participation in these organizations' advocacy decisions validates some of the skepticism about who is able to
participate and make their voice heard. Elite groups dominate policy lobbying, even for
an issue like environmental justice.

In the end, the above analysis offers some clarity on a poorly
understood but important mechanism of U.S. policymaking. It offers
some hope that citizen opinions may be heard through direct democracy
institutions built into bureaucratic policymaking. At the same time, it highlights how disparate impacts are explicitly considered in a tiny percentage of policies.
% --- PAGE: endnotes -----------------------
% --- PAGE: refs -----------------------
\newpage
\singlespacing 
          \bibliography{/Users/devin/dissertation/assets/dissertation.bib} 
   

\end{document}


% \section{Field experiments}


\section{Conclusion}


% CONCLUSION
%This research will add to our understanding of how  bureaucratic policymaking fits with the practice of democracy.
If input solicited from ordinary people has little effect on policy outcomes, directly or indirectly, it may be best understood as providing a veneer of democratic legitimacy on an essentially technocratic and/or elite-driven process.
The legitimacy of bureaucratic policymaking is said to depend on the premise that rulemaking provides an outlet for public voice \citep{Croley2003, Rosenbloom2003}. Yet, it is not just the opportunity to engage, but actual engagement that matters \citep{Herz2018}, and we lack an empirical base necessary to evaluate whether any legitimacy the public comment process may provide is deserved, even if people believe that their comments matter \citep{Yackee2014JPART}.
If mass engagement does shape agency decisions, a new research program will be needed to investigate who exactly these campaigns mobilize and represent.



\newpage
\section{Appendix}

\begin{figure}[h!]
    \centering
        \caption{Rules ranked by number of comments posted to regulations.gov}
    \includegraphics[width = 6.5in]{Figs/rules-ranked-comments-per-year-1.png}
    \label{fig:rules-ranked}
\end{figure}

\begin{figure}[p!]
    \centering
        \caption{Major and non-major rules on regulations.gov}
    \includegraphics[width = 7in]{Figs/major-comments-density-1.png}
    \label{fig:rules-major}
\end{figure}

\begin{figure}[p!]
    \centering
        \caption{Rules on regulations.gov by priority level}
    \includegraphics[width = 7in]{Figs/priority-comment-density-1.png}
    \label{fig:rules-priority}
\end{figure}
\clearpage 

% Survey
% % \include{Survey.tex}

% Recruiting handout 

%\theendnotes
\singlespace
\small
\bibliographystyle{apsr.bst} 
\bibliography{mendeley.bib}
\end{document}

