In addition to affecting whether principals engage, mass comment campaigns may affect the positions they take when they do engage. 

\begin{hyp} \label{hyp:principals}
 Elected officials are more likely to support positions supported by the majority of comments and less likely to engage in opposition to the majority of comments.
\end{hyp}

\textit{Pr(Principal Comment = Position $i$ $|$ Comment Share Position $i >$ .5) $>$ Pr(Principal Comment = Position $i$ $|$ No mass comments)}

\textit{Pr(Principal Comment $\neq$ Position $i$ $|$ Comment Share Position $i >$ .5) $<$ Pr(Principal Comment = Position $i$  $|$ No mass comments)}

1.1 Groups mobilize when they perceive attentive public is on their side.
- public interest group coalitions mobilize more often than business-driven coalitions
1.2 Success of mobilization effort is moderated by public opinion. 
- 1.2.1 Number of comments 
- 1.2.2 Effort per comment 

2.1 Elected officials' advocacy is moderated by actual engagement





%This dissertation will examine the role of mass mobilization in agency policymaking. 
The rise of the administrative state as the main site of policymaking in the United States requires new thinking and study of the practice of democracy. In what ways do agency policy processes advance or undermine democratic ideals? While political scientists focus on accountability through elected officials, administrative procedures also create rights to petition and participate directly in policymaking.  Yet we lack systematic evidence on how civic engagement such as protests, petitions, or large numbers of public comments shape the political environment in which government agencies make policy decisions. Such questions of group influence are central to understanding democracy. While some speculate that mobilizing a large number of people can shift policy, the studies that come closest to addressing this question in the context of agency rulemaking find no such evidence. Theories of bureaucratic policymaking seem to point in both directions and specific theories about how mass-engagement might matter are lacking. Furthermore, the normative appeal of institutions like public comment periods, rooted in ideas of direct democracy, depends on who participates, who is influential, and why---empirical questions that require a study on whether---and if so, why---civic engagement matters. I aim to provide this study.

First, I offer an understanding of agency policymaking focusing on the political information available to policymakers. Occasionally, such information arises from contentious debate and civic mobilization. I draw on scholarship in political science, law, sociology, and public administration and data on hundreds of thousands of rules made in the past 40 years to identify what distinguishes the inputs and outputs of the few policy processes where large numbers of people are engaged.

Second, I aim to identify mechanisms, both direct and indirect, by which civic mobilization may shape an agency's decisionmaking environment. I assess direct mechanisms by measuring the attention and meanings that agency staff give to protests and the results of mass-comment campaigns and indirect mechanisms by measuring variation in other kinds of external attention to rulemaking processes, especially from elected officials.

[Potentially] Third, I develop new methods to study the causal effects of civic mobilization through experimental manipulation of advocacy strategies. This involves partnering with organizations to randomly assign strategies and surveying participating organizations before and after a campaign to compare expert priors with outcomes. 
% Public comment campaigns are not the only way to signal to agencies that large numbers of people are paying attention, but it is one signal that cannot be missed. Activist campaigns may also shape media attention and public discourse. 
%In this design, the number and text of public comments are the treatment and the text of bureaucrats' discussion of public comments and changes in policy are the response. 
Tracing the effects of activist campaigns on agencies' political environments and policy decisions will contribute to our understanding of the roles of advocacy organizations and the people they engage in bureaucratic policymaking. 


% ELLIE'S NOTES MID NOV

what do you mean by mass engagment 
- more than 1000 comments or 100 identical comments 
- empirical facts 

DIFF AND SIMILAR 
- how is it different from letter-writting campaigns 
- what are the theories of congressional engagment - what would they say

How is this different than public opinion 
- vs intensity of opinion 
- 

can we control for underlying public opinion? 
- correlated with public opinion? 
- when is it when is it not 

can we control for the poularity of the rule 
- media coverage

Shlasman verba brady 
- predicting engagment 
- why people get involved in different causes 

How does this form of participation differ or simiar

why are the interest groups doing this? 
- moveon does both 

What does the interest groups lit say at when people engage 
- 
- 

What if it is just a proxi for public opinion? 
- yes
- no, public opinion does not create....
- BOTH! 

Can you alter the signal
- policymakers don't perfectly observe public opinion (only on some issues) 
- hypothesis, we can move people more in a low information environment


Make some causal arrow diograms 
- which of the possible pathways can you show 
- a lot of pathways 
- be more precice about what you can and cannot show


Activist strategies and oportunities 
- in part 


Are members of congress and the president 

If everyone in the agency has a strong view, does congress matter 
- 
- 

If principals are divided
- 
- 

But under divided
- 
- 
QUESTIONS 
more rules under dividied govent
more mobilization under divided government (controling for number of rules?) 
- but what about agency resoruces 

In general, what are the altrantive explainations 
- do the empirical tests rule out alt explainations 




Adjudicting 
- The  
- 
On the other hand, scholars suggest the it does matter under limit.

My theoritical contributions from others 
- 
- 

Contributions: 
- 1 2 3 
- it is a resources 

I can rule out the alternative 
- 
- 
How would people uncheritibly interpret your results 
- The

can you measure unmeasured lobbying that could be explaining your results 
rule out most uncheritbile critics 
- intereviewing people 
- 

secret lobbying 
- are there groups who don't have resources to lobby? do we still have results? 
- 

lobbying data
- 

lobbying disclosure data 
- look at org 
- outside strategy 

how can control for inside lobbying 


Committment 
- Weekly updates 
- End of the semester
- 


% ELLIE's Notes:
% For future: twitter replaction
% how mode of engamgent affects ____
% 
% how to distinguish mechanism
 % evidence for oversight = contacts 
 % evidence not for oversight = no contacts 

% DAVIDS NOTE 
% characterizaions of orgs + issues 
% emph that those 
% catagories formally 
% a new resource
% 
% new avenu of politics 
% 
% cost of the mobilizate (affected by popuarity)
% 
% silent losers - are not payint atenttion - but they go bizzerk - challenge - 
% potentially mobilizable interests 
% example - catistrophic care - book Richard Himmalphar - catistrophc politics 
% example 2: poll tax in england 


% FIX MCCUBINS AND SCHWARTZ 


% MODEL - regulator is oncertian about backlash 
% org may reveil 
% people react to losses more than gains 
% 262 in 6th eddition - useful thing in identifying diffuse interests - though they may not opposed before - they may mobilize after 



\section{Expectations and Contribution}
This dissertation will address several questions: why do social movements mobilize around seemingly scientific or technocratic decisions made by bureaucrats? Does it work? If so, why? And why do agencies sometimes go rogue? I argue that the reasons that agencies diverge from the priorities of elected leaders are the reasons that activists find opportunities to influence agency policy. First agencies make meaningful policy decisions. Second, they do so in a rich political context in which ideas of the public good and legitimate knowledge shape policy through a diverse range of policy processes that are more or less, often less, driven or constrained by elected officials. Many escape the attention of elected officials all together, but understanding any rule as insignificant or politically “neutral” would be a mistake.

It would also 


%%%%%%%%%%% OLD 




% Who succeeds in shaping policy? How do policies reshape politics? This dissertation addresses these central questions in political science by using new methods in text analysis to understand the politics of bureaucratic policymaking. 



 %, and three corresponding causal questions: does policy shape who participates, does policy shape lobbying coalitions, and does policy shape who wins in future policymaking? These causal questions belong to a growing literature on ``policy feedback'' dynamics that suggests the answer is \textit{yes}, policies shape politics, and proposes causal mechanisms by which this occurs. I will test these propositions in a new domain to which they ought to apply: administrative policymaking. While the subjects of policy feedback scholarship often involve administrative policymaking, scholars have thus far focused on legislative policymaking. Some policy feedback mechanisms such as those that focus on public opinion and electoral drivers of policy change, may not travel. Others, such as those that focus on group mobilization, organization, and power, should travel. I aim to find out.


% I aim to parse, identify, and measure policy ideas across political texts over time. To do this, I develop new methods in text analysis that combine text reuse and topic modeling approaches. 






%With systematic data on who participates, what commenters ask for, and whether they get it in rulemaking, I hope to test policy feedback hypotheses. When we observe mobilization, reorganization, or success in rulemaking, does it correspond to elements of previous policy design through the causal mechanisms proposed by policy feedback scholars? While replete rich case studies tracing policy processes, policy feedback scholarship tends to focus on narrow domains because systematically observing mobilization, coalition advocacy, and policy change across multiple policy areas is difficult. Data on administrative rulemaking and text analysis methods offer a opportunity to observe participation and policymaking systematically over time and across a wide variety of policy areas (Figure 2). To the extent that administrative rules can be systematically coded for the elements of policy design that are theorized to cause positive and negative feedback effects, data on rulemaking can be used to test these theories.\footnote{Even if rules cannot be systematically coded for elements of policy designs in ways required for credible theory testing, the descriptive undertaking here will provide foundational data for case studies and may highlight cases that are ripe for causal analysis.} 

%If which groups participate, what they ask for, and who succeeds is a function of past policy, may be the a more central units of political science than groups. I join scholars who call for a more policy focused political science. 




% feedback $
%\begin{figure}[h!]
%\caption{Policy Feedback in Rulemaking Over Time}
%\label{dynamic}
%\begin{tabular}{@{\extracolsep{5pt}}ccccccccccc} 
%\fbox{NPRM$_{t}$}& $\longrightarrow$& \fbox{Rule$_{t}$}& $\longrightarrow$& \fbox{NPRM$_{t+1}$} & $\longrightarrow$ &\fbox{Rule$_{t+1}$}&$\longrightarrow$\\
%  &$\uparrow$  &$\downarrow $& & & $\uparrow $ &$\downarrow $&  &\\
%  &\fbox{Comments$_t$} & &  &  & \fbox{Comments$_{t+1}$} & \\
%& Participation$_t$ & $\longrightarrow$&  &  & Participation$_{t+1}$ &$\longrightarrow$ \\
%&Organization$_t$ & $\longrightarrow$ &  &  & Organization$_{t+1}$ &$\longrightarrow$ \\
% &Power$_t$ & $\longrightarrow$ &  &  & Power$_{t+1}$ &$\longrightarrow$
%\end{tabular}
%\end{figure}












%\subsection{Administrative Rulemaking as a Policymaking Domain and Force in Politics}



%If we hope to understand whose ideas win out and why in the making of policy, understanding this context---a policy's legal foundations, political history, prospects for future reversal, and impact of future policymaking and politics---cannot be ignored. Only systematic study of many lawmaking processes across venues and over time will allow general conclusions about what the rise of the administrative state means for whose ideas and interests become law.

% \section{Uniting Policy Feedback and Bureaucracy Scholarship}
% Political scientists who study the ``policy feedback'' dynamics that create interest group coalitions and power have called attention to the importance of policy design and of locating politics in time. Yet scholarship on policy feedback tends to overlook bureaucratic policymaking as the site where policy design becomes concrete and also where we see the most revision of policy over time. Conversely, bureaucracy scholars have well documented sources of power and constraint in administrative policymaking. Yet scholarship on bureaucratic politics tend to collapse the time dimension and rarely address the role of policy itself in mobilizing, organizing, and empowering groups. I aim to integrate the strengths of each of these literatures to address these oversights. Doing so points to a new kind of study of bureaucratic policymaking.

%Compared to legislative policymaking, we know little about bureaucratic policymaking. 



% Such scholarship includes how regulatory capture entrenches interest groups and how agency spending affects electoral politics. 

%Recent exceptions to this trend include research on the effects of policy design on public support and of agency reputation on policymaker, expert, and interest-group support (Carpenter 2014). 







%My hypotheses about who succeeds in rulemaking over time thus draw on what we know about the formation and effectiveness of lobbying coalitions, the foundations of agency reputations, the threat of judicial review, and the factors that generate and condition policy feedback effects. 



% complex
%Unimaginable to the country's founders (Robertson 2005), the modern bureaucracy blurs lines between executive, legislative, and judicial functions (Rosenbloom 1983). While much of the bureaucracy is hierarchically organized under the president, there are practical limits on presidential control and normative reasons why imperfect control may be desirable (Whittington and Carpenter 2003). Members of Congress also speak directly to agencies: through grants of and restrictions on agency authority, through budgeting, through hearings and reports, and through direct policy and constituent requests. Additionally, courts review policies and set policymaking deadlines, a process in which the president's solicitor general may or may not participate. In addition to (or perhaps because of) this multitude of relationships, agencies have achieved varying degrees of autonomy as legitimate policymaking institutions in their own right (Carpenter 2001, 2010). 

% review
%Existing empirical work on rulemaking suggests that some broader theories of policymaking are more relevant than others. Kerwin's (2003), \textit{Rulemaking: How Government Agencies Write Law and Make Policy}, is the reference text in this area and this understanding of agencies as policymaking venues has guided most scholarship since. This includes Nelson and Yackee's (2012) study of lobbying coalitions, Yackee and Yackee's (2006) study of the influence of business and non-business groups, Haeder and Yackee's (2016) study of lobbying at the Office of Management and Budget. This scholarship finds that business groups are more successful in rulemaking, in part because they are often the only ones lobbying on obscure rules and in part because they send consistent messages to policymakers by lobbing together.% While these studies focus on a few agencies and a fairly simple `more government vs. less government' coding of group interest and policy change, they offer central propositions about the politics of rulemaking to test at broader scales and with more nuanced measures of influence. 

% With improved measurement of interest group coalitions and the influence of actors with respect to each rule across many rulemaking processes, I add more powerful tests of existing theories of rulemaking. Specifically, I aim to test the findings that business groups are most influential, that those who lobby frequently are most influential, and that larger and more diverse coalitions are more influential.

\subsection{Broader Theories of Policy Change and Policy Feedback}

%To this theoretical work on lobbying coalitions and rulemaking, I aim to integrate what policy scholars have recently called ``policy feedback.'' 

In addition to advancing underlying theories that center agencies as policymaking venues, I aim to better connect these theories to recent scholarship on policymaking more broadly. To do so, I draw both on theories of bureaucratic politics and on broader theories of why lobbying coalitions form and how they drive policy change or stability. While many aspects of legislative policymaking, such as partisan and electoral drivers, may be less relevant, others, such as interest group lobbying and policy design, are highly relevant to rulemaking (Furlong and Kerwin 2005; Kerwin 2003; West 2004, Yackee and Yackee 2006).

Scholars who study lobbying during rulemaking have often drawn on theories of policy change developed in legislative studies, most commonly citing work by Hall and Wayman (1990) and Baumgartner et al. (2009) on how interest groups shape policy agendas and by Hula (1999) on how groups form coalitions. 

Recently, many of the most relevant theories, including work on interest group mobilization and policy design, have been integrated in an rapidly growing literature on policy feedback dynamics (e.g. Mettler 1998, Hacker 2001, Baumgartner 2002, Mettler 2005, Soss et al. 2007, Mettler et al. 2012, Skocpol et al. 2012). The growing focus on policy feedback has improved our understanding of who succeeds and why in iterative policymaking in political contexts that are already thick with layers of existing policy. 

Yet despite the fact that the domains of many policy feedback theories precisely describe most bureaucratic policymaking, perhaps even better than legislative policymaking, these theories have largely overlooked agencies as policymaking venues. Not only do bureaucrats expect to ``repeat endlessly'' the policymaking process (Lindblom 1967, 1979), they may also anticipate that their actions will reshape their political environment. 


Policy feedback scholarship recognizes that policy is made in a context rich with existing policy and thus policy itself shapes the political terrain (for reviews, see Campbell 2012, Pierson and Skocpol 2002, Thelen 1999). For example, by creating classes of users, regulated entities, and scientific justifications, regulations create and empower lobbying coalitions and lay out the terms of future policy debates. This literature integrates theories of interest group power (a key focus scholarship on bureaucracy) with theories of mass participation (a phenomenon that has received little attention in scholarship on bureaucracy, but may affect policymaking (Coglianese 2001)). Literature on policy feedback suggests that those who succeed in rulemaking are those empowered, organized, and mobilized by the perceived material and symbolic impacts of policy design, with past policy shaping who participates, the coalitions they form, and the likely direction policy is likely to move as a result. 

This scholarship traces back to the seminal work of Schattschneider (1935) on the ways tariff policy created organized political interests. Lowi (1964) and Wilson (1973) began to categorize different types of policy with different effects on political mobilization. Scholars since proposed a number of mechanisms by which policies shape politics, including path dependence, increasing returns, and self-sustaining processes. Policies can empower actors to entrench or expand policies (Skocpol 1992, Pierson 1994), or disorganize and disempower political groups (Patashnik 2003). 

% the observation that administration was deeply embedded in power and politics played a significant role in modern scholarship on interest groups and policy feedback between policy design and politics. This tradition begins with Schattschneider's observation that ``new policies create politics'' (Schattschneider 1935). 

While administrative actions are often key to their analysis, the broad scope often including mass politics and group politics over long time periods has led policy feedback scholars to see government as government. Paul Pierson's (1993) essay ``When Cause Becomes Effect'' and book \textit{Politics in Time: History, Institutions, and Social Analysis} built on Lindblom and others for a historical institutional perspective on welfare state politics. Current research in this area asks about how things like the visibility of public programs affects public support and how policies can re-shape interest groups and policy coalitions (Mettler 2011). 

Organized groups are critical to scholarship on policy feedback, but this vein of research departs significantly from the interest group pluralism perspective. Most importantly, policy feedback focuses on how policy affects organized groups (by restructuring their institutional incentives) as well as how groups affect policy (Thelen 1999). Policy feedback literature also tends to see ideas, movements, and public sentiments as causes (as well as effects) of policy change. Perceptions of policy are key. For example, in addition to which policies are formally linked, political support for a policy may depend on actors' experiences with previous policies and their perceived relationship to the policy in question (Weir 1989). 


%\includegraphics[width = 14cm]{Pictures/bureaucracylit.png}

Yet beliefs about what agencies ought and ought not to do affect both the dependent and independent variables that scholars select, and policy feedback scholars has focused on legislative policymaking. %The understanding of the bureaucracy as agents of the president and Congress makes the demands of these actors a dominant set of independent variables.

Existing work on policy feedback focuses on how legislated entitlements or regulations affect either public support or lobbying coalitions and thus shape future legislation. Some findings from this literature are more applicable to bureaucratic policymaking than others. For example, Suzanne Mettler's (Mettler 2011) work on program implementation and the visibility of benefits is likely highly relevant. Indeed, her independent variable--the visibility of costs and benefits--is a function of policy decisions in both Congress and agencies. On the other hand, if bureaucrats are less sensitive to public opinion than members of Congress, Mettler's dependent variables--individual support for specific policies and government in general--may be less relevant. Research on how policies give rise to and affect organized groups and lobbying coalitions (e.g. Mettler 1998, Hacker 2001, Baumgartner 2002, Mettler 2005, Soss, et al. 2007, Mettler, et al. 2012, Skocpol, et al. 2012) is highly relevant to bureaucratic policymaking. For example, when the Environmental Protection Agency establishes classes of power plants subject to different pollution standards, it creates distinct (and perhaps oppositional) groups of utilities and corporations with common interests who may lobby differently in future rulemaking processes as a result. 

Thus, some theories of policymaking developed in the legislative domain ought to travel better to bureaucratic policymaking than others. For those that ought to travel, however, bureaucratic policymaking offers a vast underutilized domain for theory testing. Data on rulemaking is especially well suited for testing theories regarding interest group lobbying and policymaking in political contexts that are rich with layers of existing law. Rulemaking data include who participated, what they wanted, and which previous policies are motivating or constraining policy process. 


In order to systematically test policy feedback hypotheses, I must code each rule %, perhaps even specific contested sections of rules 
for features of policy design described in the policy feedback literature. %(I may not be able to systematically measure all of these aspects of policy design and will rely on case studies for some of this analysis). 
This includes how policies generate defined constituencies with incentives to protect their benefits, which includes giving groups politically relevant resources (including financial benefits and rights). It also includes ``interpretive effects'' which alter the capacities and interests of affected groups (Pierson 1994). Policies can shape political agendas, identities, interests, and beliefs about what is possible, desirable, and normal (Soss and Schram 2007).

If policy feedback theories are correct, when positive feedback mechanisms are of the type that create or grow constituencies, successful coalitions should grow in size, diversity, and/or capacity over time. When policies create or reorganize classes of beneficiaries or losers, future rulemaking processes will reflect these divisions. %The next section formalizes these statements.\footnote{Additionally, when rules establish certain legitimate or legally required types of information, comments that include these types of information will be more likely to have influence, but this hypothesis is not given a formal statement below.}
%Thus, to lay a theoretical foundation for predicting who succeeds in shaping the law over time, I attempt to 

In sum, empirical studies of bureaucratic politics have largely ignored policy feedback dynamics and policy feedback scholars have glossed over bureaucratic policymaking, treating it as derivative of legislative politics. I remedy this by testing policy feedback theories in the domain of bureaucratic policymaking, which I argue is fertile ground for this task. By leveraging the large volume and quality of data on rulemaking and improved methods of relating policy demands to policy change using text analysis, I hope to offer the most systematic testing of policy feedback theories to date.




%\begin{document}


















\section{Research Design}

This project makes two core contributions. First, I introduce new methods to 
%test theories about
measure the formation of lobbying coalitions, their demands, and whether they got what they asked for. 
% and of specific actors within coalitions
Second, using these new measures, I test policy feedback theories (i.e. the political conditions and elements of policy design hypothesized to generate and condition policy feedback effects) in the domain of bureaucratic policymaking. By integrating and testing theories of bureaucratic policymaking and policy feedback, I aim to better identify whose ideas end up in policy and why.

After brief conceptual notes, 
%about the meaning of influence and the historically contingent nature of policymaking, 
this section discusses some of the uses of textual data, where I obtain my data, the methods I employ to answer my three descriptive questions, and how I go about testing theories.











\begin{table}[h!]
\begin{tabular}{@{\extracolsep{19pt}}lll} 
\underline{Question} & \underline{Approach} & \underline{Estimates distribution of}\\
What topics are discussed? & LDA Topic Model & Words over $T$ topics\\
Who is discussing what? & $\drsh$Structural Topic Model & $J$ document types over $T$ topics\\
How change relate to a suggestions? &    $  \ \ \   \drsh$Relational Topic Model & Topic change $j_1|j_1'$ vs. topics in $j_2$\\
What is copied from where? & Text Reuse & Tokens in document $j$ copied from $j'$
\end{tabular}
\end{table}




\section{Testing Theories of Policy Feedback}

The methods of identifying participation, coalition structure, and who gets what they ask for in rulemaking provide much of the data necessary to test policy feedback hypothesis. Specifically it they allow me to identify changes in participation, coalition membership, and success over time. Policy feedback theories suggest these are shaped by past policy. By coding rules for their expected feedback effects, I can then test the relationship between elements of policy design and observed participation, coalition structure, and success over time. I do not know if this is possible, but the descriptive findings will at least point to interesting case studies. 

Policy feedback effects (i.e. elements of policy design that distribute resources and organize ideas) condition how policy victories relates to future success for individuals and coalitions. \begin{align*}
Success_{j, r} \sim & Success_{j, t-1}*FeedbackEffects_{j, r}\\
\sum_{j \in c} Success_{j, r} =  & CoalitionSuccess_{c, r} \sim CoalitionSuccess_{i, t-1}*FeedbackEffects_{c, r}
\end{align*}

More specifically, coalition success is conditioned by policy feedback mechanisms that affect the size, cohesion, and power in rulemaking, accounting for alignment of coalition comments with Presidential, Congressional, and Judicial preferences: 

\begin{align*}
CoalitionSuccess_{c, r} \sim &\\
& Resources_{c, r}  \\
& + Size_{c, r}  \\
& + Cohesion_{c, r} \\
& + Mobilization_{c, r}  \\
& + PresidentalPriorities_{c, t}  \\
& + CongressionalPriorities_{c, t}  \\
& + JudicialPriorities_{c, t}\\
Where: \\
Resources_{c, r} \sim & Resources_{c, t-1}* ResourceFeedback_{c, r}\\
Size_{c, r} \sim & Size_{c, t-1}* SizeFeedback_{c, r}\\
Cohesion_{c, r}  \sim & Cohesion_{c, t-1}* CohesionFeedback_{c, r}\\
Mobilization_{c, r}  \sim & Mobilization_{c, t-1}*MobilizationFeedback_{c, r}
\end{align*}

Size is number of participants, mobilization is mass citizen participation (commenting in rulemaking, but perhaps also protest or other measures external to rulemaking in some cases), and cohesion is the relatively similarity of comment texts. Coalition size, cohesion, and mobilization are functions of how actors were empowered by previous policy design and mobilized by the current draft policy. 

Policy feedback scholarship suggests that coalition membership is a function of how classes were defined in past policies and are proposed to be defined in the current draft. Coalitions are a function of existing policy and the proposed policy (Baumgartner 2006). Coalitions are empowered, organized, and mobilized by the perceived material and symbolic impacts of policy design. In addition to common interests because of their type (e.g. business or nonprofit), policy creates coalitions with common interests and links them to certain scientific, moral, and representational ideas (Carpenter 2001, 2010) which will be reflected in the text of comments.% I expect the coefficient of $Same Type$ to be greater for subsets where more $i_1$ and $i_2$ pairs are business interests. 

If policy feedback hypotheses are correct, we should see significant effects of empowering and disempowering policy designs in previous periods, mobilizing and demobilizing designs in the current period, and organizing policy features in both periods.\footnote{As noted, I am not yet sure how I am going to systematically code these aspects of policy design. I also need to confront the problem of groups being empowered and commenting when they did not previously comment and groups being disempowered and failing to comment in future time periods, perhaps by assigning 0 rates of success for non-commenting groups when they do not comment.}

\section{Conclusion}

Words give meaning to political ideas. The methods advanced here offer scholars tools to model relationships across policy texts to and thus across politics and policy over time. 

Through this empirical project, I connect two major theoretical advances political science has made on the topic of public policy. One is that bureaucrats make policy and that they do so in a rich political context. The other is that policies shape the politics of future policymaking. If policies affect politics, then the massive amount of policymaking that takes place in the bureaucracy likely plays a large, if underappreciated, role in shaping the U.S. political landscape.%, especially future policymaking in each agency. Yet scholarship on bureaucratic politics tends to see agency policy as an effect rather than a cause, and policy feedback scholarship underappreciates the distinctiveness of bureaucratic policymaking.





























































































%\footnotesize
%\newpage
\section{References}
%\begin{center} %\removelastskip 
%\textbf{References}\end{center}
%\removelastskip

\hangindent=0.7cm
\flushleft
 %\begin{hangparas}{.25in}{1}

Baumgartner, F. R., and B. D. Jones.  1991.  ``Agenda Dynamics and Policy Subsystems.''  \textit{Journal of Politics 53.} 1044-74.

Baumgartner, F. R.  2002.  ``Positive and Negative Feedback in Politics.''  in \textit{Policy Dynamics}, eds.  F. R. Baumgartner and B. D. Jones.  Chicago: University of Chicago Press.

%Bendor, Jonathan.  1995.  "A Model of Muddling Through."  American political science review 89: 819-40.

%Bendor, Jonathan, Amihai Glazer, and Thomas Hammond.  2001.  "Theories of Delegation."  Annual review of political science 4: 235-69.

Benoit, Kenneth and Alexander Herzog. 2015. “Text Analysis: Estimating Policy Preferences From Written and Spoken Words.”

Blei, D, AY Ng and MI Jordan. 2003. “Latent dirichlet allocation.” \textit{Journal of Machine Learning Research.} 3(1):993–1022.


% Brady, Michael C., Jacob Robert Neihesel, and Kevin Richard Stout.  2016.  "Stimulating Presidential Support: The American Recovery and Reinvestment Act, Presidential Pork, and Vote-Buying in Congress." Paper presented at the Midwest Political Science Association 74th Annual Conference, Chicago, IL.

% Bueno de Mesquita, Ethan and Matthew C. Stephenson.  2007.  "Regulatory Quality under Imperfect Oversight."  American political science review 101: 605-20.

Campbell, Andrea L. 2012. "Policy Makes Mass Politics." \textit{Annual Review of Political Science.} 15: 333-51.

Carringan, Christopher, and Stuart Krazdin. 2015. ``Using Complexity to Secure Agency Autonomy in the Rulemaking Process.'' Presented at the Annual Meeting of the Midwest Political Science Association.

Carpenter, Daniel P. 2010. \textit{Reputation and Power: Organizational Image and Pharmaceutical Regulation at the FDA.} Princeton, NJ: Princeton University Press.

-----. 2001. \textit{The Forging of Bureaucratic Autonomy: Reputations, Networks, and Policy Innovation in Executive Agencies, 1862-1928.} Princeton, NJ: Princeton University Press.

Carpenter, Dan, and David Moss. (Eds.). 2013. \textit{Preventing regulatory capture.} Cambridge University Press.

Cashore, Benjamin, and Michael Howlett.  2007.  ``Punctuating Which Equilibrium? Understanding Thermostatic Policy Dynamics in Pacific Northwest Forestry.''  \textit{American Journal of Political Science} 51: 532-51.

Clinton, Joshua D., Anthony M. Bertelli, Christan R. Grose, David E. Lewis and David C. Nixon. 2012. ``Separated Powers in the United States: The Ideology of Agencies, Presidents, and Congress.'' \textit{American Journal of Political Science} 56(2):341 – 354.

Clinton, Joshua D., David E. Lewis and Jennifer L. Selin. 2014.``Influencing the Bureaucracy: The Irony of Congressional Oversight.'' \textit{American Journal of Political Science} 58(2):387– 401.

Clinton, Joshua D. and David E. Lewis. 2008. ``Expert Opinion, Agency Characteristics, and Agency Preferences.'' \textit{Political Analysis} 16(1):3 – 20.

Coglianese, Cary. 2006. ``Citizen participation in rulemaking: Past, present, and future.'' \textit{Duke Law Journal} 55:943–68.

-----. 2001. ``Social movements, law, and society: The institutionalization of the environmental movement.'' \textit{University of Pensylvania Law Review} 150(4): 1255-1360

-----. 1997. ``Assessing consensus: The promise and performance of negotiated rulemaking.'' \textit{Duke Law Journal} 46(6): 1255-1349.

Cohen, Michael D, James G March, and Johan P Olsen.  1972.  ``A Garbage Can Model of Organizational Choice.''  \textit{Administrative Science Quarterly} 1-25.

Dunleavy, Patrick. 2013. \textit{Democracy, bureaucracy, and public choice.} Routledge.

%Ethridge, Marcus E. 1982. "The policy impact of citizen participation procedures: A comparative state study." \textit{American Politics Quarterly} 10:489–509.

Farhang, Sean. and Miranda Yaver. 2015. ``Divided Government and the Fragmentation of American Law'' \textit{American Journal of Political Science.} 108(7): 2643-2650

Furlong, Scott R., and Cornelius M. Kerwin. 2005. I``nterest Group Participation in Rulemaking: A Decade of Change.'' \textit{Journal of Public Administration and Research} 15: 353–70.


%Hollibaugh, Gary E., Gabriel Horton and David E. Lewis. 2014. Presidents and Patronage. American Journal of Political Science 58(4):1024–1042.

Grimmer, Justin. 2013. \textit{Representational Style in Congress.} Cambridge University Press.

Grimmer, Justin, and Brandon M. Stewart. 2013. "Text as Data: The Promise and Pitfalls of Automatic Content Analysis Methods for Political Texts." \textit{Political Analysis} 21 (3): 267–97.

Grimmer, Justin, and Gary King. 2011. ``General Purpose Computer-Assisted Clustering and Conceptualization.'' \textit{Proceedings of the National Academy of Sciences} 108(7): 2643-2650.

Hacker, Jacob.  2001.  \textit{The Divided Welfare State.}  New York: Cambridge University Press.

Haeder, Simon, and Susan Webb Yackee. 2015. ``Influence and the Administrative Process: Lobbying the U.S. President's Office of Management and Budget.'' \textit{American Political Science Review} 109(3): 507-522

%Hall, Richard L., and Frank Wayman. 1990. ``Buying Time: Moneyed Interests and the Mobilization of Bias in Congres- sional Committees.'' \textit{American Political Science Review} 84(3): 797–820.

Hula, Kevin W. 1999. \textit{Lobbying Together.} Georgetown University Press. 

%Javeline, Debra. 2014. "The most important topic political scientists are not studying: adapting to climate change." \textit{Perspectives on Politics} 12.02: 420-434.

%Judge-Lord, D., I. Scher, and B. Cashore. 2014. "Non-domestic Sources of the Canadian Boreal Forest Policy: Integrating Theories of Internationalization and Pathways of Policy Change." In \textit{Forests Under Pressure: Local Responses to Global Issues.} Austria. IUFRO World Series.

%Judge-Lord, D. and J.R. Cochran. 2011. "Putting Ecosystems to Work: Institutional Changes Needed to Implement Ecosystem-based Plans." Oregon Planners Journal 28(1): 7.

Kerwin, Cornelius, and Scott R. Furlong. 2011. \textit{Rulemaking: How government agencies write law and make policy.} 4th ed. Washington, DC: Congressional Quarterly.

Kingdon, John W.  1995.  \textit{Agendas, Alternatives, and Public Policies.}  2nd ed.  New York: HarperCollins College Publishers.

Kl\"uver, Heike, and Christine Mahoney. 2015. ``Measuring interest group framing strategies in public policy debates.'' \textit{Journal of Public Policy} 35(2): 1-22.

%FIX ABOVE !!!!!!!!!!!!!!!!!
%Kwon, Namhee, Stuart W. Shulman, and Eduard Hovy. 2006. "Multidimensional text-analysis for eRulemaking." Proceedings of the 2006 international conference on Digital government research. Digital Government Society of North America.

Lauderdale, Benjamin, and Tom Clark. 2014. ``Scaling Politically Meaningful Dimensions Using Texts and Votes.'' \textit{American Journal of Political Science}: 58(3): 754-771.

Lewis, David E. 2008. \textit{The Politics of Presidential Appointments: Political Control and Bureaucratic Performance.} Princeton, NJ: Princeton University Press.

Lewis, David E and Abby K. Wood. ``Agency Responsiveness: A Federal Foia Experiment.'' Paper presented at the Midwest Political Science Association 2016 Annual Meeting, Chicago, IL.

Powell, Eleanor Neff and Justin Grimmer. 2016. “Money in Exile: Campaign Contributions and Committee Access.” \textit{The Journal of Politics.} 78(4):974–988.

%MacDonald, Jason A. 2010. Limitation Riders and Congressional Influence over Bureaucratic Policy Decisions. American Political Science Review 104(4):766 – 782.

%McKay, Amy Melissa, and Susan Webb Yackee. 2007. Interest Group Competition on Federal Agency Rules. American PoliticsResearch 35 (3): 336–57.

%Obama, Barack. 2013. "Executive Order -- Preparing the United States for the Impacts of Climate Change"

Rossi, Jim. 2001. ``Bargaining in the Shadow of Administrative Procedure: The Public Interest in Rulemaking Settlement.''  \textit{Duke Law Journal} 51(3): 1051-1058

%Shor, Boris, and Nolan McCarty. 2011. "The ideological mapping of American legislatures." \textit{American Political Science Review} 105.03: 530-551.

%Stramp, Nicholas, and John Wilkerson. 2015. "Legislative Explorer: Data-Driven Discovery of Lawmaking." \textit{PS: Political Science and Politics}, 48(01), 115-119.

West, William F., and Connor Raso. 2013. ``Who Shapes the Rulemaking Agenda?'' \textit{Journal of Public Administration Research and Theory} 23(3): 495–519.

Wilkerson, John, David Smith, and Nicholas Stramp. 2015. ``Tracing the Flow of Policy Ideas in Legislatures: A Text Reuse Approach.'' \textit{American Journal of Political Science}, 59(4): 943–956

Woods, Neal. 2013. ``Regulatory Democracy Reconsidered: The Policy Impact of Public Participation Requirements.'' \textit{Journal of Public Administration Research and Theory.} 25: 571–596.



Yackee, J. W., and S. W. Yackee.  2010.  ``Administrative Procedures and Bureaucratic Performance: Is Federal Rule-Making Ossified?.''  \textit{Journal of Public Administration Research and Theory} 20: 261-82.

-----.  2009a. ``Divided Government and Us Federal Rulemaking.''  \textit{Regulation and Governance} 3: 128-44.

-----.  2009b.  ``Is the Bush Bureaucracy Any Different? A Macro-Empirical Examination of Notice and Comment Rulemaking under `43.'''  in \textit{President George W. Bush's Influence over Bureaucracy and Policy: Extraordinary Times, Extraordinary Powers.}

-----. 2006. ``Sweet Talking the Fourth Branch: The Influence of Interest Group Comments on Federal Agency Rulemaking.'' \textit{Journal of Public Administration and Theory}: 16(1): 103-24.

Yackee, Jason Webb, and Susan Webb Yackee. 2006. ``A Bias towards Business? Assessing Interest Group Influence on the U.S. Bureaucracy.'' \textit{Journal of Politics} 68(1): 128–39.


%Grimmer. The Downside of Deadlines (with Dan Carpenter).


%Baumgartner, Frank R., Bryan D. Jones, John Wilkerson and E. Scott Adler. 2016. ``Policy Agendas Project.'' Online database distributed by the Department of Government at the University of Texas at Austin, \text{http://www.comparativeagendas.net/.}

%Carmines, Edward G., Michael J. Ensley, and Michael W. Wagner. 2012. ``Political Ideology in American Politics: One, Two, or None?'' \textit{The Forum}, 10 (4): 1-18.

%Carmines and Stimson. 1989. \textit{Issue Evolution: Race and the Transformation of American Politics.} Princeton University Press.

%Carpenter, Dan, and David Moss. (Eds.). 2013. \textit{Preventing regulatory capture.} Cambridge University Press. 

%Claggett, William J.M.  and Byron E. Shafer, \textit{The American Public Mind: The Issue Structure of Mass Politics in the Postwar United States.} New York: Cambridge University Press.

%Cox, Gary C. and Mathew D. McCubbins. 2005. \textit{Setting the Agenda: Responsible Party Government in the U.S. House of Representatives.} New York: Cambridge University Press.

%Farhang, Sean. and Miranda Yaver. 2015. "Divided Government and the Fragmentation of American Law" \textit{American Journal of Political Science.} 108(7): 2643-2650

%Gerring, John. 1998. \textit{Party Ideologies in America, 1828-1996} New York: Cambridge University Press.

%Huntington, Samuel P. 1981. \textit{American Politics: The Promise of Disharmony} Cambridge, MA: Harvard University Press.

%Kerwin, Cornelius, and Scott R. Furlong. 2011. \textit{Rulemaking: How government agencies write law and make policy.} 4th ed. Washington, DC: Congressional Quarterly.

%Key, V.O. 1993. \textit{A Theory of Critical Elections.} Ivington Publishers. 

%Lewis, David E. 2008. The Politics of Presidential Appointments: Political Control and Bureaucratic Performance. Princeton, NJ: Princeton University Press.

%Manza, Jeff and Clem Brooks. 1999. \textit{Social Cleavages and Political Change: Voter Alignments and U.S. Party Coalitions.} Oxford: Oxford University Press.

%Mayhew, David R. 1991. \textit{Divided We Govern: Party Control, Lawmaking, and Investigations, 1946-1990.} New Haven: Yale University Press.

%Obama, Barack. 2013. "Executive Order -- Preparing the United States for the Impacts of Climate Change"

%Potter, Rachel. Forthcoming. \textit{Journal of Politics}.

%Schlozman, Daniel. 2012. \textit{When Movements Anchor Parties}. 2015. Princeton University Press. 

%Shafer, Byron E. 2016. \textit{The American Political Pattern}. University Press of Kansas. 

%Shafer, Byron E. and Richard H. Spady. 2014. \textit{The American Political Landscape}. Cambridge, MA: Harvard University Press.

%Sundquist, James L. 1983. \textit{Dynamics of the Party System: Alignment and Realignment of Political Parties in the United States}. Washington, DC: Brookings Institution. 

%West, William F., and Connor Raso. 2013. "Who Shapes the Rulemaking Agenda?." \textit{Journal of Public Administration Research and Theory} 23(3): 495–519.

%Wilson, James Q.  1967.  "The Bureaucracy Problem."  \textit{The Public Interest}: 3.

%%%%%%%%%%%%%%%%%%%%%%%%%%%%%%%%%%%%%%%%





%Lieberman, Robert C.  2002.  "Ideas, Institutions, and Political Order: Explaining Political Change."  American political science review 96: 697-712.

Lindblom, Charles E.  1979.  "Still Muddling, Not yet Through."  \textit{Public Administration Review.} 39: 517-26.

Lindblom, Charles Edward.  1980.  The Policy-Making Process.  Prentice-Hall Foundations of Modern Political Science Series.  Englewood Cliffs, N.J.: Prentice-Hall.

%Long, N. E.  1949.  "Power and Administration."  Public Administration Review 9: 257-64.

%March, James G., and Johan P. Olsen.  1984.  "The New Institutionalism: Organizational Factors in Political Life."  American political science review 78.

%McCubbins, Mathew D, Roger G Noll, and Barry R Weingast.  1987.  "Administrative Procedures as Instruments of Political Control."  Journal of Law, Economics, & Organization 3: 243-77.

%McCubbins, Mathew D., and Thomas Schwartz.  1984.  "Congressional Oversight Overlooked: Police Patrols Versus Fire Alarms."  American Journal of Political Science 28.

Mettler, Suzanne.  1998.  Dividing Citizens : Gender and Federalism in New Deal Public Policy.  Ithaca, NY ; London: Cornell University Press.

———.  2005.  \textit{Soldiers to Citizens: The G.I. Bill and the Making of the Greatest Generation.} Oxford ; New York: Oxford University Press.

———.  2011.  \textit{The Submerged State: How Invisible Government Policies Undermine American Democracy.} Chicago Studies in American Politics.  Chicago: University of Chicago Press.

Mettler, Suzanne, and Joe Soss. 2004. "The Consequences of Public Policy for Democratic Citizenship: Bridging Policy Studies and Mass Politics."  \textit{Perspectives on Politics.} 2: 55-73.

%Mills, Russel W., Jennifer L. Selin.  n.d.  "Behind "Enemy" Lines?: Congressional Detailes as an Indocator of Congressional Committee Capacity." Paper presented at the Midwest Political Science Assocation 2016 Meeting, Chicago, IL.

%Mintrom, Michael.  1997.  "Policy Entrepreneurs and the Diffusion of Innovation."  American Journal of Political Science 41: 738-70.

%Nelson, Douglas.  1988.  "Endogenous Tariff Theory: A Critical Survey."  American Journal of Political Science 32.

%Niskanen, William A.  1975.  "Bureaucrats and Politicians."  The Journal of Law & Economics 18: 617-43.

%Ortiz, Stephen R., and Suzanne Mettler.  2012.  "Veterans' Policies, Veterans' Politics New Perspectives on Veterans in the Modern United States."  Gainesville: University Press of Florida.

Pierson, Paul.  1996.  ``The New Politics of the Welfare State.'' \textit{ World Politics.} 48: 143-79.

———.  2000. ``Not Just What, but When:  Timing and Sequence in Political Processes.''  \textit{Studies in American Political Development.} 14: 72-92.

———.  2004.  \textit{Politics in Time : History, Institutions, and Social Analysis.}  Princeton, N.J. ; Oxford: Princeton University Press.

———.  1993.  ``When Effect Becomes Cause: Policy Feedback and Political Change.''  \textit{World Politics} 45: 595-628.

%Rhodes, R. A. W.  1996.  "The New Governance: Governing without Government."  Political Studies 44: 652-67.

Robertson, D. B.  2005.  ``Madison's Opponents and Constitutional Design.''  \textit{American Political Science Review.} 99: 225-43.

%Rosenbloom, David H.  1983.  "Public Administrative Theory and the Separation of Powers."  Public Administration Review: 219-27.

Schattschneider, E.E.  1935.  \textit{Politics, Pressures and the Tariff. } New York: Prentice-Hall.

%Simon, Herbert A.  1957.  Administrative Behavior: A Study of Decision-Making Processes in Administrative Organization.  New York: MacMillan.

Skocpol, Theda, Larry M. Bartels, Mickey Edwards, Suzanne Mettler, and ebrary Inc.  2012.  ``Obama and America's Political Future.''  In Alexis de Tocqueville lectures on American politics.  Cambridge, MA: Harvard University Press.

%Soss, Joe, Jacob S. Hacker, and Suzanne Mettler.  2007.  Remaking America : Democracy and Public Policy in an Age of Inequality.  New York: Russell Sage Foundation.

%Soss, Joe, Sanford F Schram, Thomas P Vartanian, and Erin O'Brien.  2001.  "Setting the Terms of Relief: Explaining State Policy Choices in the Devolution Revolution."  American Journal of Political Science. 378-95.

Thelen, Kathleen.  1999.  ``Historical Institutionalism in Comparative Politics.''  \textit{Annual review of Political Science.} 2: 369-404.

%Verba, Sidney, and Norman H Nie.  1972.  Participation in America: Harper & Row.

%Waggoner, Philip Daniel.  n.d.  "Informal Politics: The Role of Congressional Letters in Inter-Branch Communications between Congress and the President." Paper presented at the Midwest Political Science Association 2016 Annual Meeting.

%Weir, Margaret.  1989.  "Ideas and Politics: The Acceptance of Keynesianism in Britain and the United States."  In The Political Power of Economic Ideas, ed.  Peter A. Hall.  Princeton, New Jersey: Princeton University Press.  53-86.

%Whittington, Keith E, and Daniel P Carpenter.  2003.  "Executive Power in American Institutional Development."  Perspectives on Politics 1: 495-513.

%Wildavsky, Aaron B.  1964.  Politics of the Budgetary Process.  Boston, MA: Little Brown.

Wilson, James Q.  1967.  ``The Bureaucracy Problem.'' \textit\textit{The Public Interest:} 3.

%Wilson, James Q.  1994.  "Reinventing Public Administration."  PS: Political Science and Politics 27: 667-74.

Yackee, J W and S W Yackee.  2006.  ``A Bias Towards Business? Assessing Interest Group Influence on the Us Bureaucracy.''  Journal of Politics 68: 128-39.

———.  2009. ``Divided government and US federal rulemaking.'' \textit{Regulation and Governance.} 3(2):128–144.

———.  2012.  "Testing the Ossification Thesis: An Empirical Examination of Federal Regulatory Volume and Speed, 1950-1990."  George Washington Law Review 80: 1414-92.

Yackee, S. W.  2014.  ``Reconsidering Agency Capture During Regulatory Policymaking.''  Preventing Regulatory Capture: Special Interest Influence and How to Limit It: 292-325.



%To what extent is bureaucratic policymaking driven or influenced by presidential transitions, Executive Orders, new legislation, new science, corporate lobbying, and judicial review? Lacking measures that allow us to compare influence across such diverse potential drivers of policy change, political scientists do not know. 

%\end{hangparas}




Assumptions about what bureaucrats ought to do has thus far hindered scholars in integrating these two ideas--that half a century after Norton E. Long, Charles Lindblom, James Q. Wilson, and others called our attention to bureaucratic politics, we have yet to fully appreciate bureaucrats as political actors. This chapter aims to (1) identify how assumptions about what the bureaucracy ought to do have limited the scope of scholarship, (2) identify lessons learned from broader literature on policymaking, and (3) suggest hypotheses about coalition mobilization and influence in rulemaking, given what we now know about how bureaucratic policy is made and how policy feedback mechanisms operate. %I also examine the conditions under which policy feedback effects are likely to result in incremental, paradigmatic (i.e. ``punctuated''), or thermostatic policy change. 
The result is a set of hypotheses that includes common predictors of policy influence but also recasts factors often seen as exogenous determinants of bureaucratic policy as endogenous relationships with expectations for the magnitude and direction of their effects on policy over time. 



\subsection{The Bureaucracy Problem in Political Science}

Early studies of public administration in the late nineteenth and early twentieth centuries assumed that problems could be solved rationally. Administration was a science with objective methods to design and carry out administrative tasks. The principles of good administration were discoverable, generalizable, and neutral. In the mid-twentieth century, a wave of scholarship responded to the contrary that ``the lifeblood of administration is power'' and that bureaucratic decision-making is saturated with ``forces on whose support, acquiescence, or temporary impotence the power to act depends'' (Long 1949, pg. 1). 


James Q. Wilson (1967). Wilson argues that the bureaucracy is expected to optimize a variety of conflicting goals--each the concern of different constituents and, as I argue subsection 2.3, scholars. He calls these goals Accountability, Efficiency, Responsiveness, Equity, and Fiscal Integrity. This subsection reviews the assumptions and scholarly concerns associated with each goal except the last, which is characterized by broad general agreement but specific definitions of fiscal integrity fully depend on one's orientation to the other goals. 

\subsubsection{Accountability}

Accountability for Wilson means serving the national interest as defined by the president. If we take Congress as the body responsible for legislating the national interest, accountability to Congress may be equally important, implying multiple political principals. Principal-agent scholarship tends to focus on Congress and the president (e.g. Wildavsky 1964, Niskanen 1975, McCubbins and Schwartz 1984, McCubbins, et al. 1987). However, courts also arbitrate between congressional and presidential control by reviewing bureaucratic policy for congressional intent and adding additional procedural requirements to policymaking (Bueno de Mesquita 2007). 

The assumption that bureaucrats are to be accountable leads to questions about control. Under what conditions do agencies do what Congress or the president demand? Under what conditions do different structural incentives and enforcement mechanisms operate? For example, McCubbins and Schwartz (1984) argue that measures of overt control fail to capture ``fire alarm'' means of control that only kick in when agencies diverge from principals' preferences. Indeed there is empirical support that bureaucratic decision-making varies more from president to president than the use of direct mechanisms of control would suggest (Yackee and Yackee 2009b).

%Recently, and especially relevant to the present exercise, scholarship has begun to explore the effects of ``presidential pork''--bureaucratic spending as political currency--(Brady 2016) and direct requests made by Congress (Mills n.d., Waggoner n.d.).

\subsubsection{Efficiency}

Efficiency--``minimizing expenditures for a given output'' (Wilson 1967, pg. 409)--is prioritized by economists and proponents of ``rational'' forms of policymaking like cost-benefit analysis. This concern is raised most pointedly by certain professional organizations, and think tanks. When paired with accountability it is often a certain elected officials. As will be shown in the next section, separate strands of research tend to see accountability and efficiency as either linked or in opposition. Though much more sophisticated than studies of public administration from the 19th century, scholars focusing on efficiency tend to share the view that the role of politics is to set goals and the role of administration is to mechanically optimize how goals are achieved. For example, in his influential work on public administration, Simon (1957) argues that administrative decisions ought to be judged by how well they achieved their stated objectives and the efficiency these results were obtained. 

A focus on efficiency leads to questions about performance, cost, and output. Scholars view the public as consumers and want to know what the public is getting for its money and if the same results could be achieved at lower cost. Politics and power are distortionary and institutions are to be designed to optimize policy and policy output regardless of political agreement. For example, Bendor (1995) develops a formal model aimed at testing the ``effectiveness of muddling through''--deriving results suggesting that it is not the most effective way to make policy in the presence of conflict. Yet contrary to claims that agency policymaking is ``ossified,'' Yackee and Yackee (Yackee and Yackee 2010) find that that procedural constraints aimed at accountability and characterizing muddling through may actually speed up the promulgation of rules. Regardless of the empirical debate about the relationship between accountability and efficiency, in principle, mechanisms by which political a principal's control agencies may be good or bad for efficiency. A focus on efficiency simply asks what forms of accountability properly align agency incentives with maximizing output and minimizing cost and which kinds of accountability distort efficient production and delivery of public goods. 

Identity of public servants loyal to an overriding mission and 
 
I argue that efforts to assign agencies partisan orientations are misguided and that capture theory is inadequate to explain why agencies go rogue.
 
One explanation for why agencies go rogue is that some agencies are more liberal, and some more conservative, others fairly neutral. When liberals are in power, conservative agencies resist and vice versa. Divide government presents opportunities to avoid political control from partisan enemies by working with partisan allies.
 
I argue that agencies are not best understood as partisan actors and that attempts to assign ideology scores are better seen as capturing how well the current issue alignment of parties maps onto a much more stable and issue-specific missions of agencies. Agency missions are better understood as akin to Lowi and North’s public ideology. Mission-orientation is a very different kind of ideology a collection of roll call votes across many issues. Unlike legislators, agencies are not taking positions on a wide range of issues. Instead, orientation toward a few issues is constituted in an agency’s authorizing statutes that survive partisan transitions and become orienting philosophies.  
 
 
Another explanation is that agencies become “captured” by powerful actors, for example a certain group of businesses that stand to profit from policy change or stasis. While there is ample evidence that certain actors like businesses often have more influence, true “capture” would require that the agency be acting against both the preferences of political principals and its mission. Because actors with power in agency policymaking also likely hold power with elected officials and may have helped to shape the mission of the agency, influence is like much more common than capture. Furthermore, while perhaps sufficient,  external influence is not a necessary condition for going rogue. Agencies may also go rogue when their internal organizational logic and elected principals conflict. Capture theory is poorly suited to explain these cases. For example, if career policymakers in the National Park Service see climate change as a grave threat to the National Park System, but the president would prefer to keep climate change off the agenda, we would not say that NPS was “captured” by climate scientists or environmental activists. Instead, we would say that they are acting on their mission as they understand it and that this drive is more powerful than presidential preferences.
 
 
 
% Attempts to place agencies on an ideological space reflect a misunderstanding.
 
 
 
If agencies were simple partisans or agents, social movements would have no reason to target them. They would target elected principles or party leaders. If agencies had stable policy positions, and the policies they make were a function of veto players, why would activists mobilize citizens to write letters to bureaucrats? One possibility is that they are signaling potential political costs. Another is that these policies serve a site for recruitment and mobilization for other ends or internal organizational imperatives. I suspect that activists sincerely aim to influence policy and that they are not naive in doing so—i.e. that they have a chance of influencing policy because it is not fully determined by a constellation of political actors but negotiated and constructed among a number of constituencies, each with distinct claims to representation and knowledge of science, the law, and the public good. 


To say that agencies are significantly mission driven does not require one to be a progressive Utopian. Indeed, agencies may be founded on conceptions of the public good that most citizens and intellectuals would reject. They may, and often are, founded with contradictions, baked in political compromises that may be intractable within the institution. Institutions may be wholly inadequate to fulfill their mission and, in rich institutional environments like the united states and complex issues like climate change, formal authority over many parts of public life may be overlapping or contested. Certain narrow conceptions of the public good may contradict others such that different parts of government are working at cross-purposes. 

Political appointees may very well have partisan loyalties  (Bonica et al 2015).

Within the institution oriented toward broad goals, there are no doubt more progressive and conservative policies. Indeed, many policies change with partisan control. 

It may also be that party partisans become aligned such that most of an agency's activities are 

Because programs change and because the very goals, reputations, and truths on which policies are based are contested, social movements as well as businesses and other organized groups have opportunities to significantly shape the law by participating agency policymaking. 

"Efficiency," "Accountability," only make sense as "goals" if the substantive ends are clear and we are interested in how to measure appropriate adherence to those goals. If the end is contested, it is not clear what efficiency means because doing one thing efficiently may be less desirable for many than doing it inefficiently. Similarly accountability only makes sense we agree on to whom. Often accountability to one potential principal may pull in the opposite direction as accountability to another. 

% It

\subsubsection{Knowledge}
\subsubsection{Representation}
\subsubsection{Responsiveness} 

Responsiveness takes several different meanings and ``rarely has been defined precisely'' (Verba and Nie 1972). First, it is used by principal-agent scholars to describe faithful compliance with direct orders (e.g. Chaney and Saltzstein 1998). 

Second, responsiveness is sometimes used with respect to policymaking as broader version of accountability--not only to elected officials, but also to public opinion and constituent groups. This definition has led to research questions regarding the timeliness of response to citizen requests (e.g. Lewis n.d.).
Third, an agency may be responsive to unique specific requests, perhaps from members of Congress, but more commonly from the members of the public. Wilson (1967) uses such a definition--``to meet, with alacrity and compassion, those cases which can never be brought under a single national rule and which, by common human standards of justice or benevolence, seem to require that an exception be made or a rule stretched.'' Wilson notes that responsiveness to individual requests and accountability to principals may conflict and Lewis and Wood (n.d.) find empirical support for this proposition. Individual responsiveness may also stand in sharp contrast with some definitions of equity. Yet courts may still support agencies when they bend the law in individual cases if they agree with the bureaucrat's concept of justice.

All three versions of responsiveness are concerns of scholars who see the proper role of agencies as faithful and neutral managers.  The first type desires responsiveness to the goals of political principals. The second type assumes there exists a discernible public interest, either from public mood or through group advocacy and representation. The third presumes universal standards of justice and proper relationships between citizens and government that can be applied to individual cases and that these standards of conduct must sometimes trump, or at least shape the application of formal policy. The subfield of representative bureaucracy is based on the assumption that bureaucracy should mirror and engage the population it serves. 
\subsubsection{Claims to equity}
\subsection{Historical}

\subsubsection{Equity}

James Q. Wilson describes equity as the goal to ``treat like cases alike and on the basis of clear rules known in advance'' (Wilson 1967 pg. 409). Equity is often the concern of groups or segments of the public who feel bureaucratic policy or its implementation is biased. Demands for equity come both from those who feel discriminated against (for example due to race) and from those who resent policies they view as serving ``undeserving'' groups (Mettler 1998, Soss, et al. 2007, Mettler 2011, Cramer 2016). Both kinds of demands (and thus scholarship) are rooted in conceptions of fairness and broader notions of what kinds of difference require different treatment under the law. Enforcement of equity comes largely from the courts on standards that policy should not discriminate nor be arbitrary and capricious. 
The frame of equity focuses scholarship on who is systematically advantaged or disadvantaged by bureaucratic processes as well as non-systematic distortions of an equitable system. Scholarship on representative bureaucracy and disparities in service delivery and policing are two lines of research using this frame.
The next section identifies four lines of research, each presuming that bureaucrats ought to pursue different combinations of the above goals. 

\subsection{Four Silos}

%I argue that the conceptual frames used to study bureaucratic policymaking (i.e. viewing agencies as agents, managers, or gatekeepers) has led scholars to focus on the effects of other actors on agencies rather than the effect of agency policy decisions on other political actors. The dominant normative idea that unelected bureaucrats lack a legitimate basis to exert independent agency has led most scholarship to focus on constraints, ``objective'' goals, and mechanisms of control. Scholarship on bureaucratic autonomy focuses more on the ways agencies avoid control than on how they shape the political and policy terrain around them. Of this latter vein of scholarship, Carpenter (2001, 2010) offers the most complete treatment of agency policy as both effect and cause of policy and politics elsewhere, but this type of scholarship is rare and has yet to be fully integrated with macro-theories of policy feedback. 

Four main lines of research emerged from the early writings on the bureaucratic politics (mapped in Figure 1). Each is explicitly or implicitly rooted in different assumptions about what bureaucracy ought to do and appropriate kinds of interaction with other political actors. Scholars in the law and economics tradition build on rational choice theory. Scholars of organizational behavior take a more sociological approach, focusing on how bureaucracy structures decision-making. This group sees balancing group pressures as key to what bureaucracy does and they implicitly view autonomy from principals as a neutral if not positive development. A third line, what I call the pluralist approach, is related to the second but more focused on the structure of group politics outside government and exogenous shocks to the policy process. Here agencies are venues where bureaucrats are seen more stakeholders than policymakers. Even more than the sociological approach, this approach sees agencies more as a site of conflict than as actors. Finally, unlike the above three (especially principal agent and organizational behavior research) the policy feedback literature, has, since Lindblom, tended to lump bureaucracy and other governmental institutions. This section reviews the development of each approach and its organizing questions. The next section pulls out key variables identified by these literatures. 

Accountability has been a primary concern of law and economics scholars who have developed and tested models of agencies as agents of the president or Congress that may be ``captured'' by interest groups. Efficiency is a shared concern of both economic and pluralist-oriented scholars despite very different ideas of the source of these goals. Law and economics see policy goals defined by government hierarchy whereas pluralists look for group coalitions and consensus.  While scholars from all camps claim to study responsiveness, scholars of organizational behavior use the concept most closely to the way Wilson did. Equity is a primary concern of scholars of social policy that have most advanced theories of policy feedback but their hypotheses tend not to be formulated specifically with respect to bureaucratic policymaking. Fiscal integrity, as Wilson notes, is one goal on which nearly everyone can agree and a motivating force behind scholarship on regulatory capture among a diverse array of scholars. However, beyond obvious corruption, definitions of fiscal integrity depend on one's other assumptions about what bureaucrats ought to be doing. 

\subsubsection{Law and Economics}

The law and economics camp has focused on formal models to explore the implications of institutional choice, transaction costs (Horn 1995), delegation, and the separation of powers. A primary concern is the principal agent problem--that bureaucrats will act in their own policy interests rather than those articulated by elected representatives. Wildavsky (1964) proposed a model of the budgeting process relating to how much agencies ask for and how much Congress allocates, and the strategies actors employ to get what they want. More rigorous modeling in the law and economics tradition takes seriously that bureaucrats may have political objectives as well as resource objectives. McCubbins et al. (1987) offer a framework with two general types of control--oversight and administrative procedures. Bendor, et al. 2001) suggest rational principals will delegate to agents with similar goals, repeated interactions, and when they are able to overcome commitment and information problems.

This camp continues to develop principal agent theory. It has gone beyond budgets to include regulatory authority, integrating observations about the power of interest groups through models of ``regulatory capture.'' Current work continues to ask how institutional design and incentives can increase political control over administrative institutions. 

\subsubsection{Organizational Behavior and New Institutionalism}

Scholars of organizational behavior, building on the work of Long, Lindblom, Allison, Wilson, as well as many sociologists and management scholars, helped give rise to the New Institutionalism. These scholars explore logics of decision-making, and institutional factors that affect norms and processes around things like problem-definitions, collaboration, and learning (March and Olsen 1984). Unlike some in the law and economics and pluralist camps, they do not tend to believe that it is possible to fully clarify public values in the context of public administration. It may, however, be possible for decision-makers to agree on policy without agreeing on means or ends (Lindblom 1980).

Lindblom claims that administrators cannot make policy rationally and must restrict their attention to a limited number of values and alternatives at a time. Building on this, Cohen et al. (1972) introduced a model where policymaker aims are ambiguous and bureaucrats are under pressure to produce solutions quickly. Sometimes policy ideas linger until some problem arises to which they might be supplied as a ready-made solution. In sharp contrast to principal-agent models, this ``garbage can'' model suggests that problems, solutions, and choices are collected non-strategically by administrators or policymakers. This work informed Kingdon's (1995) ideas of ``policy streams'' and ``policy windows.'' Howlett et al. (2015) articulate a research agenda based on the trend of more complex policy design where multiple policy tools are combined into baskets that address multiple goals. 

New institutionalist scholars critiqued the assumptions of rationality in pluralist, elite, and principal-agent models, suggesting that either a self-oriented ``logic of consequences'' or a other-oriented reasoning--either altruism or a ``logic of appropriateness''--alone were each insufficient to explain policy decisions and must be considered in tandem. These logics operate under different conditions and stages of policy development, adoption, and entrenchment (Bernstein and Cashore 2000).  

The organizational behavior tradition has spun off two lines of research--studies of bureaucratic autonomy and theories of delegation. Bureaucratic autonomy scholarship asks to what extent agencies have independent power and from where it arises (usually historically). For example, Carpenter (2001, 2010) examines the politics by which agencies secure independent political power and thus autonomy. Rooted in organizational theory, but relevant for scholarship on interest groups and principal-agent dynamics, Carpenter concludes that bureaucracies form symbiotic relationships with constituent groups. %Interestingly, similar to Long's analysis, Carpenter finds that agencies, with the help of their constituents, often drive congressional policymaking. Delegation theory is a key part of scholarship on federalism and empirically, delegation has implications for which kinds of ideas are implemented by government (Elmore 1978).

\subsubsection{Pluralism}

Pluralist scholars and their descendants see interest group coalitions, more than organizational factors, as the core of politics. For these scholars, policy tends to emerge from groups outside the agency. Bureaucrats may play a gate-keeping role or take sides in broader political coalition-making. This line of scholarship is focused on groups and coalitions of groups, which may ``include'' agencies or ``capture'' agencies depending on a scholar's normative orientation (Carpenter and Moss 2014).
Lowi (1967) suggests that the theory of pluralism arose from a need to justify the New Deal and rise of the administrative state, which lacked legitimacy due to the abdication of power from Congress to the President. Interest group pluralism offers a justification for deferring and delegating. With the assumption that groups are representative, the consensus of interest group coalitions can be a legitimate basis of government policy (Lowi 1967).

Studies of agency capture and influence often rest on implicit notions of the public interest (Carpenter and Moss 2014). In a pluralist world where group advocacy is not just acceptable but necessary, defining capture is a tricky line-drawing exercise between desirable and non-desirable groups and appropriate and inappropriate influence. This often leads to estimation of relative influence among groups and stated intent of policymaking procedures. For example, Yackee (2014) defines capture as ``control'' rather than ``influence'' and Yackee and Yackee (2006) investigate procedural mechanisms intended to make agencies responsive to the public and finds that primarily coalitions of regulated businesses, not public interest groups, are able to affect change through this process. 

Some pluralist work focuses on the role of individuals in organizing group action and affecting policy change. However, because the impulse for change is, often implicitly, assumed to come from outside the agency, scholarship has focused on elites outside agencies as policy entrepreneurs and their advocacy in legislatures (e.g. Mintrom 1997) rather than on bureaucrats or direct advocacy. 

For forty years, scholars of policymaking rested on Lindblom's theory of incrementalism. More recently, scholars have focused on the mechanics of agenda-setting, the likelihood of non-incremental policy change, and how policymaking processes vary in different institutions, issues, or political contexts  (Gormley 2007). In the 1990s, punctuated equilibrium theory (Baumgartner and Jones 1991) launched research agendas focused on explaining policy stability and rapid change. It expanded on scholarship of incrementalism, but did not replace it. Indeed many scholars, including Cashore and Howlett (Cashore and Howlett 2007) showed that incrementalism was not inconsistent with punctuated equilibrium theory and indeed incremental policy decisions could themselves lead to paradigmatic policy change over time. 

\subsubsection{Historical Institutionalism and Policy Feedback}

Finally, the observation that administration was deeply embedded in power and politics played a significant role in modern scholarship on interest groups and policy feedback between policy design and politics. This tradition begins with Schattschneider's observation that ``new policies create politics'' (Schattschneider 1935). 

While administrative actions are often key to their analysis, the broad scope often including mass politics and group politics over long time periods has led policy feedback scholars to see government as government. Paul Pierson's (1993) essay ``When Cause Becomes Effect'' and book \textit{Politics in Time: History, Institutions, and Social Analysis} built on Lindblom and others for a historical institutional perspective on welfare state politics. Current research in this area asks about how things like the visibility of public programs and their administration affects public support and how policies can re-shape interest groups and policy coalitions. 

Organized groups are critical to scholarship on policy feedback, but this vein of research departs significantly from the interest group pluralism perspective. Most importantly, policy feedback focuses on how policy affects organized groups (by restructuring their institutional incentives) as well as how groups affect policy (Thelen 1999). Policy feedback literature also tends to see ideas, movements, and public sentiments as causes (as well as effects) of policy change. Perceptions of policy are key. For example, in addition to which policies are formally linked, political support for a policy may depend on actors' experiences with previous policies and their perceived relationship to the policy in question (Weir 1989). 


\includegraphics[width = 14cm]{Pictures/bureaucracylit.png}

Beliefs about what agencies ought and ought not to do affect both the dependent and independent variables that scholars select. The understanding of the bureaucracy as agents of the president and Congress makes the demands of these actors a dominant set of independent variables. Another set emerges from pluralist understandings of agency policymaking as a balancing of organized interests as well as concerns that some groups are too influential and that some agencies are captured. Following Lindblom (1979), a significant body of work also investigates agency policy as endogenous to previous agency decisions.


\subsubsection{Bureaucratic Policy Feedback: Scope and Mechanism}

This subsection briefly reviews some of the important developments in literature on policy feedback effects. %To ease the exercise of integrating this research with research on bureaucratic policymaking, I distinguish among the scale and mechanism of effects. Some scales of feedback effects may be more relevant than others. For example, research that focuses on how policy empowers, disempowers, or mobilizes groups is more relevant than that focused on effects of individual opinions. 

Theories of policy feedback have emerged from studies of historical institutionalism and path dependency. Much of this scholarship is indebted to Schattschneider's work on how tariff policy created vested interests that helped to maintain it (Schattschneider 1935). Not only did tariff policy create new and different types of coalitions from whole cloth but ``a policy that is so hospitable and catholic as the protective tariff disorganizes the opposition'' (Schattschneider 1935). This has led to both economic theories of ``endogenous'' dynamics of policy demand and supply (Nelson 1988) as well as historical instructional accounts focused on how policy shapes the ``political terrain'' for future policymaking (e.g. Pierson 2000, 2004). 

It should be noted that policy feedback generally means something distinct from path dependency. While path dependency implies that a certain path is the inevitable result of certain decisions at key junctures, feedback is more about that nature of a dynamic among multiple political actors. As noted, Lindblom and scholars of organizational behavior have long recognized forms of path dependency, but they focus on the more direct possibilities and constraints of past actions rather than more circuitous dynamics involving other political actors. Policy feedback is explicitly focused on the way policies shape politics. 

Most policy feedback scholarship involves the study of bureaucratic decisions, but rarely addresses how the distinctive nature of bureaucratic policymaking may affect policy feedback dynamics. Weir (1989) focuses on how administrative arrangements and decisions shaped the policy ideas that politicians considered and less on the distinctive role of bureaucrats. Pierson, in is influential review essay ``When Cause Becomes Effect'' (1993), identifies how policy affects ``government elites, interest groups, and mass public''--and proceeding scholarship has tended to focus on either interest groups or mass publics and lump government actors together as being similarly affected by groups and the public. 

Policies can empower or mobilize some groups, disempower or disorganize others. Mettler and Soss (2004) built on the work of Pierson (1996) to further illustrate how policies play a critical role in politics--both on restructuring interest group power and mobilization and public opinion.

Existing work on policy feedback focuses on how legislated entitlements or regulations affect either public support or lobbying coalitions and thus shape future legislation. Some findings from this literature are more applicable to bureaucratic policymaking than others. For example, Suzanne Mettler's (Mettler 2011) work on program implementation and the visibility of benefits is likely highly relevant. Indeed, her independent variable--the visibility of costs and benefits--is a function of policy decisions in both Congress and agencies. On the other hand, if bureaucrats are less sensitive to public opinion than members of Congress, Mettler's dependent variables--individual support for specific policies and government in general--may be less relevant. Research on how policies give rise to and affect organized groups and lobbying coalitions (e.g. Mettler 1998, Hacker 2001, Baumgartner 2002, Mettler 2005, Soss, et al. 2007, Mettler, et al. 2012, Skocpol, et al. 2012) is highly relevant to bureaucratic policymaking. For example, when the Environmental Protection Agency establishes classes of power plants subject to different pollution standards, it creates distinct (and perhaps oppositional) groups of utilities and corporations with common interests who may lobby differently in future rulemaking processes as a result. 

\subsection{Legal Scholarship}

``the prospect of settlement is a traditional component of any strategy to influence agency decisions-making'' (Rossi 2001).


\subsection{Hypotheses}

\subsubsection{Deductive}

\textbf{Pre-rulemaking:}
$H_{1.1}$: The decision to make a new rule has four main causes: 1. new legislation or other direction from Congress (Yackee 2009a); (2) presidential priorities (Yackee and Yackee 2009b); (3) thermostatic response to changing conditions (Cashore and Howlett 2007); (4) regularly scheduled updates (e.g. 5 year review of the effectiveness of pollution standards). 

$H_{1.2}$:The content of proposed rules are a function of existing policy (the status quo is the default), presidential priorities, organized interests pressures, public demand for problems to be solved, new science, and... 


\textbf{Rulemaking Coalitions:}

$H_{2.1}$: Coalitions are a function of existing policy and the proposed policy (Baumgartner 2006). Coalitions are empowered, organized, and mobilized by the perceived material and symbolic impacts of policy design (see policy feedback scholarship). Policy creates coalitions with common material interests and links them to certain scientific, moral, and representational claims (Carpenter 2001, 2010). 

$H_{2.2}$: Concentrated material interests (usually businesses or interest groups they fund) are $H_{2.2a}$ always present, $H_{2.2b}$ often the only participants, and $H_{2.2c}$ often lobbying together (Nelson and Yackee 2012, Yackee and Yackee 2006).

\textbf{Rulemaking Influence:}

$H_{3.1}$: Lobbying in rulemaking is more successful when the commenter is $H_{2.2a}$ a repeat commenter, $H_{2.2b}$ uncontested, or $H_{2.2c}$ in a larger coalition (Nelson and Yackee 2012).
%
$H_{3.2}$: Lobbying in rulemaking is more successful when pushing in the direction preferred by the $H_{3.2a}$ President, $H_{3.2b}$ Congressional Majority, and $H_{3.2c}$ Courts (McCubbins and Schwartz 1984, Yackee and Yackee 2009b, Potter 2017). 

$H_{3.4}$: Conditional upon 3.1 and 3.2, businesses are still more successful than public interest groups (Yackee and Yackee 2006).

$H_{3.5}$: Success in rulemaking is correlated with the resources and experience to succeed in court. Past court victories against the agency may influence especially likely.

$H_{3.6}$: Those who succeed in rulemaking tend to be those with close relationships with the agency, conditional upon (and because of) how those relationships support the agency's reputation for expertise, competence, and representativeness (Carpenter 2001, 2010). For example, agencies that cultivate reputations for scientific competence more than representativeness will be more influenced by scientific truth claims than mass participation supporting the same position.

$H_{3.7}$: Mass participation can be influential (Coglianese 2001) conditional upon $H_{3.7a}$ alignment with scientific claims (because bureaucrats see themselves as competent experts and relatively objective public servants (Carpenter 2010)), $H_{3.7b}$ plausibly representing latent majority opinion, and $H_{3.7c}$ alignment with the president (no citations and uncertainty here because I am not aware of any empirical work on this).

%$H_{2.4}$: Political power, perceived legitimacy of scientific claims, and likelihood of succeeding upon judicial review 

\subsubsection{Inductive:}

A key advantage of text-analysis methods is that they can identify new dimensions of variation the researcher had no prior reason to suspect, suggesting new hypotheses:

$H_{4.1}$: Topics, priorities, arguments, or issue frames on which comments cluster also predict variation in influence, i.e. the dimension(s) on which the text of public comments cluster reflect potential coalitions and predict the direction(s) of policy change.

$H_{4.2}$: Dimensions that predict success in rulemaking, predict $H_{4.2a}$ success in future rulemaking when policy design has positive feedback characteristics and $H_{4.2b}$ failure when policy design has negative feedback characteristics (I may not be able to systematically measure these aspects of policy design and would have rely on case studies).

As noted, Nelson and Yackee's (2012) find businesses having larger effect when they are more unified and agencies are less likely to move against the president, Congress, and courts (Potter 2017, Yackee and Yackee 2009b). Similarly, science should have a larger effect when it is more unified and aligned with political ``principals,'' especially the president. Mass participation (i.e thousands of public comments) should also matter more when they are all in the same direction and align with presidential priorities. What we do not know, is what happens when the president's agenda, business interests, science, and/or mass participation push in different directions. Carpenter (2001, 2010) would suggest that this will vary by agency, but existing scholarship has yet to give me clear intuitions. I hope to find a satisfying answer. 



\newpage






\tableofcontents
\newpage
\section{Introduction}
\subsection{What do we mean by influence?}
\subsubsection{Different ideas about how government ought to work}
\subsubsection{Integrating mechanisms of policy feedback into theories of bureaucratic policymaking}
\subsection{Scope and case selection}
\subsubsection{Rulemaking}






\subsubsection{Thematic Examples: Adapting to Climate Change}
Because there has been little federal environmental legislation in recent decades, environmental issues have been dominated by executive policymaking. In particular, the issue of climate change has been characterized by partisan gridlock. There is rapidly growing scholarship in political science regarding the politics of regulating green house gas emissions. By now, we have seen a decade of policy proposals. The partisan lines are clearly drawn and interest groups on both sides are entrenched. If we are going to see policy regulating GHGs, we have a pretty good idea of what it will look like. We can also imagine a world where no action is taken at the federal level.

In contrast to the question of how to stop causing climate change, the question of what to do about the changes we have already caused is much less clear. Adaptation could take wide variety of forms from large infrastructure investment, to insurance (e.g. crop and flood insurance), to planned resettlement, to massive ecological restoration. Even as some politicians instruct bureaucrats not to discuss the causes of climate change, they are also instructing them to increase preparedness for floods, fires, droughts, hurricanes, and other effects of climate change. These latter things are difficult to avoid and can have real political consequences. 

While this study uses data on all significant regulations since 1980, I give special attention in each chapter to recent policymaking regarding climate adaptation for both its significance and its utility. First, adapting to a changing climate is an important but understudied issue in political science---it has been called `the most important topic political scientists are not studying' (Javeline 2014)--and it will only grow in importance. Second, partially because it is an emerging issue, it has the rare attribute of both cutting across a wide range of agencies and political constituencies while still being relatively easy to bound. While almost all government decisions affect the ability of society to adapt to a changing climate, I only dive deep into rules that are explicitly aimed at addressing this emerging issue.

The wide array of policies related to climate adaptation highlight why rulemaking is important. Rulemaking is the primary way government is responding to climate change. Citizens and interest groups have mobilized to influence agency decisions that affect their access to shifting natural resources. President Obama ordered 30 agencies, from the Army Corps of Engineers to the Department of Agriculture, to make `changes to policies, programs, and regulations...to manage climate risks' (Obama 2013). Political science has explored congressional inaction, but few have studied who influences what government \textit{is} doing about climate change. My dissertation will address this empirical gap by analyzing climate-related rulemaking. For example, should we expect farmers, fishermen, or cities to have more influence on regulations affecting water allocation in times of drought? %Existing theories may miss variables that are key to answering such questions. 

Without methods to analyze thousands of pages of rule and comment texts, we cannot know who influences climate policy or other areas where agency policy is decisive. 

 The next chapter introduces several innovations in how we measure influence across large numbers of political texts.

\section{Measuring Influence Across Political Texts}




This rest of this chapter develops 

[cut]

 Topic and cluster alignment scores provide Bayesian priors for document level inference.

\subsection{Identifying issues and frames}
\subsubsection{Identifying issues with Structural Topic Models}
I proposed a two stage-process to identify how issue or clusters of issues differ between input and outcome texts. First a basic topic model is run on the final version of the text of interest, ideally with a researcher selecting the most sensible number of topics. Next, structural topic model is run on the full corpus with an algorithm that selects the number of topics that that results in the best match to the topics identified in the first stage plus however many additional topics are present. An input text is considered most aligned (perhaps indicating influence) if its scores highly the topics contained in the outcome text and low on topics not present in the outcome text.
\subsubsection{Dimensions of a debate and policy location: TLAB}
\subsubsection{Sentiment on issues}

\subsection{Specific influence}
\subsubsection{Priors: Issues alignment and past influence}
\subsubsection{Text reuse}
Smith-Waterman alignment scores identify specific sections of text where strings of words match. These scores indicate the relative amount of a policy or policy change copied from various texts in a corpus and thus provide a measure of relative influence. With priors based on the alignment of issue frames and/or how influential authoring groups have been previously, we can estimate levels of certainty that the outcome (e.g. a change to a rule) text was drawn from any given input text (e.g. a comment)
\subsubsection{Attribution}
\subsubsection{Sentiment on attributable text}

\section{Presidential Power}

\subsection{Presidential transitions: Midnight rulemaking, deregulation, and reregulation}
How responsive is rulemaking to presidential transitions?
\subsubsection{Case: DOJ Paroll Rules and the Private Prison Industry}
Bush, Obama, and Trump have each reversed course on the same rulemaking with no new legislation. Example of administrative policy creating an industry that lobbied for policy. These policies also shaped the agenda of social movements
\subsection{Executive Orders}
How much rulemaking is driven by EOs? 
\subsubsection{Case: How did the Obama Climate Adaption EO affect agency policymaking?}
8 completed rules, 26 notices explicitly motivated by climate adaptation EO.
\subsubsection{How are Trump EOs affecting agency policymaking}

\section{Social Movements and Public Opinion}

\subsection{Do regulations cause protests? Do protests affect regulations?}

\subsection{Measuring policy-specific opinion}
FCC Open Internet Regulations
\subsection{Broad trends in rulemaking since 1980}
\subsubsection{Focusing events}
\subsubsection{Public opinion}
\subsubsection{Party control and new legislation}

\section{Lobbying and Interest Groups}

\subsection{Money}
\subsubsection{Rules cause to corporations agree or disagree on future rules}
Coalition lobbing, contested, and uncontested rule.

(Haeder and Yackee 2015, Yackee 2006, Yackee 2011, Nelson 2012, Hula 1999)

\subsection{Scientists and Public Interest Groups}
\subsubsection{When scientists agree and disagree}
Coalition lobbing, contested, and uncontested rule.

\subsubsection{When does new science influence policy?}
Illustrative examples: Compare HUD Disaster Preparedness Rulemaking, DOT Major Capital Investments Rulemaking and EPA Coastal Flood-zone Mapping
\subsubsection{Activist scientists and political attention}
Illustrative examples: geoengineering
\subsubsection{When public interests groups agree and disagree}
Coalition lobbing, contested, and uncontested rule.

Illustrative examples: Compare 2014 ESA Sage-grouse Rule, 2016 BLM Resource Management Planning Rule and 2010 USDA NRCS Healthy Forest Reserve Program Rule

\section{Lawsuits and Judicial Review}

\subsection{How courts evaluate agency policy}
\subsubsection{Is deference the result of policy process or substance?}
\subsubsection{Are briefs filed by the winners or losers in the rulemaking process more influential in judicial review?}
\subsection{Making Regulations in the Shadow of Judicial Review}
\subsubsection{Do commenters with a higher ability to bring litigation have more influence in rulemaking?}
\subsubsection{Mechanisms of Influence: Do policy texts differ based on the probability to judicial review?}
\subsubsection{Patterns of Influence: Do hostile courts constrain policymaking?}



\section{Conclusion: How this improves understanding of bureaucratic policymaking and theories of policy change}







What do winning ideas have in common? Are there broadly observable trends in the text that finds its way into the law or who is suggeting these changes? 

who is mobilized, who is together, who is winning? what role does policy play in mobilizing and organizing coalitions? When we see change in mobilization and organization, can we attribute it to a change in policy? 

1) we can do a better job of identifying winning and losing coalitions  by looking at text
2) what role does policy play in reoganizing and empowering coalitions 
- identify the growth of a advocacy coalition - why? policy?
- identify the reorganization of politics - why? policy?

same people asking for the same things. when this changes, why? 



% SPECIFIC INFLUENCE

\subsection{The Growing Text-As-Data Arsenal}

\subsubsection{Word Frequency}

One dimensional scaling: Wordscores

Multidimensional scaling: TLAB

Topic Models

Structural Topic Models

\subsubsection{Text Reuse}

\subsubsection{Sentiment}



\section{Modeling Rulemaking}

In this section, I formalize my two core contributions: a measure of influence and a model of influence over time that tests policy feedback hypotheses regarding elements of policy design that empower or disempower, organize or disorganize, and mobilize or demobilize groups. I model the influence of commenter $i$ and coalition $c_i$ on rule $r$ at date $t_r$:\footnote{For simplicity, I denote definitional equations with = and estimated quantities with $\sim$, omitting coefficient symbols.}\footnote{Individual and Coalition influence are estimated with a hierarchical model with common perimeters for each level of the organizational hierarchy as reflected in Office of Management and Budget agency codes, i.e. common rule influence parameters for each sub-agency, agency, and department: \\
$Influence[i, rule] \sim Influence[i, sub agency]$ \\
$Influence[i, sub agency] \sim Influence[i, agency]$\\
	Alternately, I may use the Policy Agendas Project's policy area codes to estimate influence by policy area\\
$Influence[i, rule] \sim Influence[i, sub code]$ \\ 
$Influence[i, sub code] \sim Influence[i, policy area]$}\footnote{I discuss assessing change between a draft (Notice of Proposed Rule Making) and final rule. However, the same method will be applied between drafts when there are multiple NPRMs, called Advanced or Supplemental NPRMs depending on whether the notice announces the intent to draft a rule or a revised draft rule. The change from an ANPRM to an NPRM will be especially interesting as commenters have the opportunity to suggest the specific content of the draft.}


\subsubsection{Measuring Success}

A relational topic modeling approach could be used model the distribution of issues over comments compared to the change from draft to final rule. Text reuse methods could also detect specific changes in rule text attributable to certain comments. The topic distribution of modified or copied rule t



\removelastskip
\begin{align}Influence[i, r] &= TextReuse[i, r] + Citations[i, r_{response}]*Sentiment[i, r] \\&+ CoalitionInfluence[c_i, r]
\\
CoalitionInfluence[c, r] &= TopicSimilarity[c, r] + LatentDimensionDirection[c, r]
\end{align}

Topic Similarity is Structural Topic Model coefficients comparing the comments 
%(possibly minus the distribution of words in the draft, $r^{\prime}$) 
and the change ($r-r^{\prime}$) and the direction of movement on latent dimensions is measured with TLAB (a multidimensional version of Wordfish/Wordscores). $r_response$ is the text accompanying the final rule in the federal register responding to comments, sometimes citing specific commenters. Coalition membership $c_i$ may be defined (1) a priori, (2) with a clustering method assuming unique membership, or (3) with a clustering method assuming a distribution of membership.

Finally, individual influence scores are calculated with respect to the judicial branch for when the same groups participate in a court case $j$:
\removelastskip\begin{align}JudicialInfluence[i, j] = TextReuse[i, j]*Sentiment[i, j] + Citations[i, r_{response}]\end{align}


\subsection{Testing Policy Feedback Mechanisms: }

Individual influence is a function of one's past influence, the influence of one's coalition, and the credible risk of a lawsuit:

\begin{align}Success[j, r] \sim \sum^{1:n} Success[i, t-1:n] + CoalitionInfluence[c_i, r] + LitigationThreat[i, r]\end{align}

\begin{align}
Success[j, r] \sim \sum^{1:n} Success[i, t-1:n]\\
\sum_{j \in c} Success[j, r] = CoalitionSuccess[c, r] \sim \sum^{1:n} CoalitionSuccess[i, t-1:n]
\end{align}

As an alternative, less sophisticated but more interpretable DV to my influence score, I can estimate the model with certain components of the score, for example, the citation count:

\begin{align}Citations[i, r]*Sentiment[i, r] \sim &\sum^{1:n} Influence[i, t-1:n] \\&+ CoalitionInfluence[c_i, r] + LitigationThreat[i, r]\end{align}

Summing the influence in previous periods aligned with Nelson and Yackee's (2012) finding that frequent commenters have more influence. %Assuming 0 influence for groups that do not participate, it also captures the idea that one is more influential in uncontested policy processes 
Litigation Threat is a combination of how often and how recently the agency has appeared in a DC Circuit Court case $j$ and the judicial influence of the commenter. 

\begin{align}LitigationThreat[i, r] &= \sum^{1:n} CourtCases[agency_r, t-1:n]*(1+JudicialInfluence[i, t-1:n])\end{align}

Policy feedback effects (i.e. elements of policy design that distribute resources and organize ideas) condition how policy victories relates to future success for individuals and coalitions. \begin{align}
Success[j, r] \simSuccess[j, t-1]*FeedbackEffects_{j, r}\\
\sum_{j \in c} Success[j, r] = CoalitionSuccess[c, r] \sim CoalitionSuccess[i, t-1]*FeedbackEffects_{c, r}
\end{align}

More specifically, coalition success is conditioned by policy feedback mechanisms that affect the size, cohesion, and power in rulemaking, accounting for alignment of coalition comments with Presidential, Congressional, and Judicial preferences: 

\begin{align}
CoalitionSuccess[c, r] \sim &\\
& Resources[c,r] + \\
& Size[c,r] + \\
& Cohesion[c,r]+ \\
& Mobilization[c, r] +  \\
& PresidentalPriorities[c,t] + CongressionalPriorities[c,t] + JudicialPriorities[c,t]\\
Where: \\
Resources[c,r] \sim & Resources[c, t-1]* ResourceFeedback_{c, r}\\
Size[c,r] \sim & Size[c, t-1]* SizeFeedback_{c, r}\\
Cohesion[c,r] \sim & Cohesion[c, t-1]* CohesionFeedback_{c, r}\\
Mobilization[c, r] \sim & Mobilization[c, t-1]*MobilizationFeedback_{c, r}
\end{align}

Size is number of groups, mobilization is mass citizen participation (commenting in rulemaking, but perhaps also protest in a few cases), and cohesion is the relatively similarity of comment texts. Coalition size, cohesion, and mobilization are functions of how actors were empowered by previous policy design and mobilized by the current draft policy:

\begin{align}
Size[c,r] \sim &
\sum^{1:n} EmpoweringPolicy[i\in c, t-n] -  DisempoweringPolicy[i\in c, t-n] \\&+ MobilizingPolicy[i\in c, t] - DemobilizingPolicy[i\in c, t],
\\
\\
Cohesion[c,r] \sim &
\sum^{1:n} EmpoweringPolicy[i\in c, t-1:n] - DisempoweringPolicy[i\in c, t-1:n] \\&+ MobilizingPolicy[i\in c, r_t^{\prime}] - DemobilizingPolicy[i\in c, r_t^{\prime}]
\end{align}

Coalition membership is a function of how classes were defined in past policies and are proposed to be defined in the current draft:

\begin{align}
P(c_{i_1} | c_{i_2}, t) \sim &SameType[i_1, i_2] + \sum^{1:n}OrganizingPolicy[i_1,i_2, t-n] \\&- \sum^{1:n}DisorganizingPolicy[i_1, i_2, t-1:n] \\&+ OrganizingPolicy[i_1,i_2, r^{\prime}] - DisorganizingPolicy[i_1, i_2, r^{\prime}]
\end{align} 
Coalitions are a function of existing policy and the proposed policy (Baumgartner 2006). Coalitions are empowered, organized, and mobilized by the perceived material and symbolic impacts of policy design (see policy feedback scholarship). In addition to common interests because of their type (e.g. business or nonprofit), policy creates coalitions with common interests and links them to certain scientific, moral, and representational ideas (Carpenter 2001, 2010) which will be reflected in the text of comments. I expect the coefficient of $Same Type$ to be greater for subsets where more $i_1$ and $i_2$ pairs are business interests. 

If policy feedback hypotheses are correct, we should see significant effects of Empowering and Disempowering policy designs in previous periods, mobilizing and demobilizing designs in the current period (and perhaps past periods, but this is not currently included), and organizing policy features in both periods. \textit{NOTE: I am not yet sure how I am going to systematically code these aspects of policy design. I also need to confront the problem of groups being empowered and commenting when they did not previously comment and groups being disempowered and failing to comment in future time periods, perhaps by assigning 0 influence for non-commenting groups when they do not comment.}

\textbf{Descriptive Predictions:}

Following Nelson and Yackee (2012), I expect business interests to be present in the most rulemaking processes, present in the most rulemaking processes where only one commenter type comments, and the most influential among all commenter types. I also expect that businesses are more often in the same coalition than different coalitions than other types of commenters.

%\subsection{Peripheral (Pre-rulemaking) Theory Testing (which I may not pursue):}

%The decision to make a new rule has four main causes: 1. new legislation or other direction from Congress (Yackee 2009a); (2) presidential priorities (Yackee and Yackee 2009b); (3) thermostatic response to changing conditions (Cashore and Howlett 2007); (4) regularly scheduled updates (e.g. 5 year review of the effectiveness of pollution standards). 

%\begin{align}P(r^{\prime} | t) = NewLegislation(t-1) + PresidentialPriorities(t) + \Delta Conditions(t:t-1) + ScheduledReview(t)\end{align}

%\begin{align}P(r | t) = NewLegislation(t-1) + PresidentialPriorities(t) + LegislativeDeadline(t) + CourtDeadline(t)\end{align}

%The content of a proposed rule is a function of existing policy (the status quo is the default), presidential priorities, organized interests pressures, public demand for problems to be solved, new science, and... 

%\begin{align}Text(r^{\prime}) \sim &Text(r_{t-1} + NewLegislationText(cited) + PresidentialPriorities(t)\\&+ \sum^{1:i}CommentText[i, t-1]*Coalitioninfluence[c_i, t-1] \\&+ PublicOpinionText[t-1] + ScienceTexts[t-1]\end{align}

%Cited legislation appears in the ``authorities'' section of the proposed rule (I am not sure how far back to go, perhaps proposed rule texts become less likely to be similar to authorizing legislation with temporal distance). Public opinion here elite public discourse, for example tweets or news articles that include the name of the agency. Scientific articles could be selected if they include the agency name and/or keywords. All of these are likely endogenous. 




OUTLINE 

Introduction 

Rulemaking is important 

Can help us test theories

Theory 
Intro
Bureaucracy folks don’t see agencies as political venue--as a cause of polics
Feedback folks dont see agencies as a policymaking venue 
I am to fix this with a policy feedback model of bureaucratic policymaking

Methods 
Data
Textual data 
Topic models 
Text reuse 
Text reuse and topic models 

My approach
descriptive 
Causal 
Mechanisms and Cases 
Policy empowers groups
Policy organizes and reorganizes groups 
Policy affects groups’ abilities to get what they want in future policymaking


NOTES FROM ELLIE AND SUSAN: 


David Brockman 
- partners with left groups
- 



LETTERS:
I am writing up my dissertation proposal and a large part is revolving around being able to measure when activist campaigns succeeded in influencing agency action, especially when that action may diverge from what the president and Congress want (or at least conforms less than could). My motivation is that a just transition is going to require that seemingly technocratic policymaking processes are responsive to mass movements and EJ concerns, especially when there is opposition from above. 

There is an opportunity for me to be useful in real time if I could partner with groups like SC who are running these campaigns. Specifically, I am interested in actions around the notice-and-comment rulemaking process, for example, comments are now being collected through the #cleancars campaign. I would love to help design some experiments, for example on the kinds of frames and policy interventions that are most effective, and put my expertise in measuring policy change to good use.

Do you know who at SC I might talk to about this kind of a partnership? I know there are people who track proposed rule changes and decide which ones to target with campaigns. Perhaps there are also folks that work on measuring campaign effectiveness? 






Source:
In 2010, the EPA moved forward on regulation of motor vehicles with a suite of proposals, in concert with the National Highway Traffic Safety Administration (NHTSA). One proposal issued standards for reducing emissions from heavy duty vehicles — trucks, vans, tractors, etc. That regulation is currently under challenge in U.S. District Court in the case Delta Construction v. EPA. Another proposal regulates light duty vehicles for model year 2012 through 2016. Finally, a third rule sets standards for light weight vehicles for the years 2017 and beyond. In 2013, the Supreme Court declined to hear a petition challenging the first set of light weight vehicle regulations.

The third set of rules deal with standards for emissions from power plants, and by far is capturing the most public attention as measured by submitted comments — 4.2 million and counting. In April 2012, the EPA issued proposed standards for carbon emissions for new electric power plants. After receiving a whopping 2.5 million comments, the agency issued a new version of the proposal this past January. Comments were due by March. Then, earlier this month, Obama made a major announcement that he would move to cut emissions from existing power plants. Comments on this proposal are due in October.



Susan notes: 

Matching is hard
- within rule
- across states - easy ask

Wisconsin must publish scope statement 
- heads up that rule is coming


Simplist form of test
- truth vs representation 

Does it mobilize 
Does it effect 

What is the sufficient N 

------------------------------------------------------------------------------------------------------------

Text - the 
- matching 
- 

IRB 





Lara Chausow 
- yale phd - CRS 
- dissertation on lobbying - survey of lobbies tis 
- lobbying firms - in house - 
- look at appendix 

CAN I CONTACT PEOPLE based on contact info 
FEC - survey of donors is OK

Sampling 
- 

in the Weeds 
- more context 

Abstract

gap in knowledge
- what is the 

More broadly than comments
- protest 
- comments
- twitter 
- article in NYT 

All the things that might matter 
- 


Clair Abernathy 
- in new jersey 
- surveyed congressional offices about const communication 
- how they processed communication 
- only res 

Arnod - anticipate 
- what are consequences of backlash 
 - but 

Congress 
- letters 
- 1 minute speeches - dataset
- constituent newsletter -  dataset 
- 


Outline + timeline (backwards) 
intro
macro discriptives and theory 
theory and methods
1 - macro descriptives 
2 - 3 issues - 
3 - deep dive into ej and climate adaptation 

Talk to Scott 
- 

patinaship vs. identity 
- CBC
- Women 
- Personal experiences 
- personal roots of representation 
- PROFESSIONAL ROOTS 

Reputation - good and bad

bureaucrats have a goal for maintainng 
- 

organizational identity - reputation for x, y, z 
- 3 areas 

nonpartisan implementation


3 methods 
- fix, reverse framing. 1. answers question. 2 to do so, introduce methods 







% ERATA 
% Is it any more than a highly biased survey constructed by advocacy groups and ignored by policymakers? 

% Conceptual clarity and answers to these empirical questions will have normative and practical import. Legal scholars are actively debating what to make of mass commenting. Agencies advertise commenting as an opportunity for voice but also assert that it is not a vote. Advocates who organize campaigns are unsure about the tactic. Ironically, the only people to confidently claim that mass commenting mattered have been agency employees. 








 

% \section*{Summary of Conceptual Approach}

  
% \paragraph{Empirical Question:}
% This dissertation aims to understand the effects of public attention on executive-branch policymaking. Cross-sectionally: How are agency decision-making processes that are targeted by civic mobilization campaigns different from those that are not? Causally: Which of these differences can be attributed to activist organizations’ mobilization campaigns and the subsequent decisions of individuals to voice their opinions? 
% By civic mobilization, I mean that a large number of ordinary people (i.e. not professional policy influencers) engage in a specific policy decision.
Political scientists often define civic engagement as writing to government officials, signing petitions, attending hearings, attending protests, or donate to a political campaign. While donating is more common in electoral politics, activists often attempt to influence agency policymaking through letter-writing, petitions, hearings, and protests. Does it work? If so, by what mechanisms?

% \paragraph{Motivation:} By one estimate upward of 90\% of legally binding U.S. federal policy is now written by agencies \citep{West2013WhoControl}. 
%To understand the practice of democracy in the united states we must understand the politics of bureaucratic policymaking.
% How does this square with various democratic ideals? 
% What does this mean for the practice of democracy?
% Do mechanisms for public input like public comment periods in rulemaking serve their intended function? Do protests have their intended effect? 
% What drives agency policymaking? 
% Congressional behavior can be modeled as if seeking election, what are the proximate causes of agency decisions? 
% \footnote{Agencies advertise public comment periods as an opportunity for a voice in government decisions. Commenters often describe commenting on proposed agency rules ``an important part of democracy'' (WSJ 2017), the ``purest example of participatory democracy in actual American governance'' \citep{Herz2016E-RulemakingsPotential}.}

% Both theory and empirical scholarship suggest skepticism that the input of ordinary people matters. Empirical scholarship finds rulemaking to be dominated by the powerful \citep{Crow2015EvaluatingAnalysis, Wagner2011RulemakingRegulations, West2009InsideBox, Yackee2006b, Yackee2006, Yackee2012a}. Perhaps this is unsurprising.
% From a strategic perspective, agency officials are not directly accountable to voters. And even if organized groups are able to supplement congressional and judicial checks on executive power, the groups that participate in rulemaking represent only certain citizens and may not represent them well \citep{Seifter2016Second-OrderLaw}. Early optimism among legal scholars that the internet “change everything” \citep{Johnson1998TheInternet} and that ``cyberdemocracy''  would enable more deliberative rulemaking has faded.  Here, the prediction that the internet would merely facilitate engagement among the like-minded \citep{Sunstein2001Republic.com} has largely been correct. From a science-based policy perspective, average citizens signing form letters provide no new information to policymakers. 
% Mass comment campaigns are thus often called ``spam'' \citep{Balla2018Wheres2} and dismissed as epiphenomenal to spatial bargaining with principals or interest groups. So why do they occur at all?


% \paragraph{Why mobilize?} Before attempting to identify the effects of activist campaigns, it is necessary to understand their aims. Given the technocratic nature of agency decision-making and lack of evidence that it is an effective tactic, why would an organization spend resources to campaign on a rule? And why some on rules and not others? I suggest organizations invest in mass campaigns for two reasons: First, a campaign may result from a need to perform to supporters’ expectations and build its base of support.  I do not expect such campaigns to affect the target policy.\footnote{Failure of previous research to parse cases where policy change is the goal from those where it is not may contribute to lack of evidence for the influence of mass mobilization}.  Second, a campaign may result from a perceived political opportunity\footnote{What social movement scholars call a ``political opportunity'' \citep{Mcadam2017}, policy process scholars may identify as an opportunity to align politics with certain identified problems and solutions to create a ``window'' for policy change \citep{Kingdon1984}. To some extent all rulemaking processes create opportunities, however small, to shape the new status quo, loosely bounded by the problems the process was initiated to solve, a set of policy solutions considered legitimate, and a constellation of political forces.} to affect policy by changing decisionmakers' beliefs or incentives. 


% \paragraph{Variation:}Rulemaking is rarely a mechanical translation of legislation, and variation in rulemaking may have political causes and consequences. My unit of analysis is a lobbying effort on an agency decision. Strategies may include mobilizing public comments, protests, and media attention. The number of people engaged and mode of engagement are the key features of interest. Often, only businesses participate. Other rulemakings include broader interest groups. A few engage thousands of people. Some agency decisions, especially those targeted by activist campaigns, attract the attention of Members of Congress and the White House.  

% Different inputs may yield different results. Agencies may or may not change draft policies or may speed up or delay finalizing them. They write lengthy justifications of their decisions in response to some demands but not others. 

% \paragraph{Mechanisms: How might mobilization matter?} A lobbying effort can generate new information, re-frame information, or reshape the political context of a decision. Agency staff may update their beliefs in response to new information or framing. Activists can also reshape agency policymakers’ strategic environment by drawing in or scaring off other actors, especially elected officials. 


\begin{table}[!htbp] \centering 
  \caption{How Information May Influence Agency Policymakers} 
  \label{2x2} 
\begin{tabular}{@{\extracolsep{5pt}} lcc} 
 & Normative Beliefs & Strategic Incentives  \\ 
\hline \\
Direct    & Select ``public'' opinion & Scientific or legal information \\
& (public service informed by direct democracy) & \\
 \\
 \hline \\
Indirect & Elected official opinion  & Retribution and reward \\ 
& (accountability through representative democracy) & (budgets, careers, support) \\
\\
\hline 
\end{tabular}
\end{table}

% Agency rulemaking, like all policymaking, is an excise in judgment by those vested with the power to write rules with the force of law. Such decisions have consequences, including possible rewards and sanctions from other decisionmakers. 
% Policymakers process information and make judgments about the appropriate or strategic action for someone in their position. Personal career ambitions, concepts of their institution’s mission, and the agency’s political context (e.g. the relative support and power of congressional committees or the White House) may all be in play.

% \paragraph{Direct influence by providing consequential information:} 
% Several kinds of information are key to the policy outcomes of rulemaking, including scientific\footnote{Physical facts may include toxicity levels or economic costs and benefits that are contested among researchers. Agency staff often face pressure to recognize certain claims because doing may set in motion a new policy process or enhance the power of a group in a given process. Claims about legal authority and its limits are ultimately adjudicated in court, but the shadow of litigation hangs over rulemaking (Coglianese 1997; Rossi 2001), making claims about legal standards and precedent consequential.} or legal information and information about the political context \citep{Yackee2012a}.
% Both types of claims are often contested. 

%Appreciating the political aspects of rulemaking leads to consideration of political information. 
% New scientific or legal information allows revision of rational hard-nosed calculations about cost and benefits and the likelihood of being reversed in court. New political information allows revision of beliefs about levels of support among certain populations or their elected representatives as well as the potential political consequences of a decision.

% \paragraph{Direct influence through political information:} 
Mass commenting and protests communicate politically-relevant information.
%Campaigns signal two kinds of resolve. First, they show the mobilizers' willingness to commit resources to the issue. Second, they show the intensity of opinions among a segment of the public. The number of people engaged by a campaign is not strictly proportional to an organizations investment. The less people care, the more it costs to mobilize them. If agency staff do not trust organizations' representational claims, engaging actual people may be one of the few credible signals of a broad base of support. 

% Importantly, rule-specific campaigns inform agencies about the distribution and intensity of opinions that are often too nuanced to estimate a priori.\footnote{\citet{Rauch2016Two-TrackE-Commenting} suggests that agencies revise the public comment process to include opinion polls. I build from a similar intuition that mass comment campaigns currently function much like a poll or, more accurately, a petition, measuring the intensity of preferences among a segment of the population--i.e. how many people are willing to take the time to engage. } Indeed, most members of the public and their elected representatives may only learn about the issue in response to a campaign.

% Because rulemaking is an exercise in relating information about the world to certain norms and policy agendas, how decisions are framed may be decisive. 
% Thus, both new political information and how information is framed may influence policy decisions. The source, number, and content of comments all provide political information. Comments from different sides may offer competing frames for interpreting these facts and others.

% For example, the number, geographic distribution, size, and proportion of businesses who lobby against a rule, may provide information about how much money and which of their political principals may be invested in attacking a rule. 
% Agency officials may update their understanding of the constellations of interests, their intensity, and the power and resources of each coalition.
% Additionally, whether businesses are seen as united or divided may shape how officials think about the appropriate course of action. 
% Whether the result of strategic calculation or perceived obligation, business unity predicts policy change (Yackee and Yackee 2006). 
% Additionally, divisions among elites may offer opportunities for activists (Tarrow 1991). 

% Similarly, the number of people who engage in a rulemaking and the intensity of engagement may provide information about how much support or scrutiny an agency is likely to receive from certain political principals. 
% As activist campaigns may be less predictable than business lobbying, civic mobilization may provide even more information about constellations of support or opposition and the intensity of these policy demanders. 

% If the opinions of letter-writers, petition-signers, and protesters are framed as the opinion of the public or as expressing valid public interests, such a frame may shape how officials think about the appropriate course of action for a public servant.  

% Even, perhaps especially, when positions expressed through petitions and protests are not majoritarian, these tactics may communicate political information and demands that are not represented in electoral politics. 
% Evidence that minority petitions and protests affect rulemaking would support theories that focus on policymakers’ limited attention, finite agenda, and satisficing rather than strategic decision making.
% Petitions and protests may also frame minority groups as deserving of special attention and protection. 



% \paragraph{Indirect influence through elected officials:} 
% Campaigns do more than reveal latent political information; they mobilize both members of the public and elected officials to take positions on issues they may have never previously considered, thus creating new relevant political information for policymakers. To the extent that the expressed will of elected officials guides agency decisionmaking--i.e. to the extent that agency decisions are shaped by norms of representative democracy--campaigns may be influential by inspiring elected officials to offer new political information. When elected officials take a position publicly or in a private letter to an agency, such political information may have normative force beyond simply simple strategic calculations.  %Movements help to shape the political space in which they operate’ (Gamson and Meyer 1996, p. 289).


% \paragraph{Indirect influence through changing the decision environment:} 
% Agency officials care about the consequences of their actions, both for themselves for their agency’s mission. Their success and power depend on the support of a political coalition that includes elected officials (Carpenter 2001). West (2004) theorizes that the primary mechanism by which mass-commenting matters is to alert political principals. Members of Congress, especially, may often be unaware of rulemaking (Nou and Stiglitz 2016). Conversely, campaigns may ``scare off'' elected officials who otherwise would have weighed in or threatened consequences (personal communication with former agency director).



% Who are the targets? 
% Rules are written by teams of experts, usually led by career public servants trained as a lawyer. They act in a highly institutionalized environment. 
% Political appointees are rarely heavily involved rulemakings. 

% Most rules address long-defined problems. They are next steps advancing a policy agenda (West and Raso 2012) or the first steps in a new policy direction. 

% Why does mass mobilization in rulemaking occur? 
% Advocacy groups perceive a political opportunity.
% Sometimes the political benefits are largely the long-run benefits of movement building, but these campaigns often also aim to win policy battles. 
% Given the momentum of political agendas and the fact that much is determined before draft rules are made public, changes are often on the margins. But such marginal victories are also the aim of business groups and other interest groups. 


%How democratic is this? 
% \citet{Seifter2016Second-OrderLaw} shows that organizations’ claims to representation are often tenuous
 
% \paragraph{Challenges:} Observational studies of policy decisions are almost always frustrated by the fact that decisionmakers rationally anticipate the actions of those who would influence them, rendering this influence difficult to observe. Thus I expect to observe larger effects in cases where mobilization or the level of engagement achieved was not anticipated by agency staff. I also hope to leverage small random manipulations in, for example, the specific policy provisions targeted by activist campaigns. 



% Precisely identifying who participates, how, and the dimensions of disagreement over time is key to any study attempting to discover whose ideas end up in policy. These are descriptive questions but they are not easy ones. In the empirical section, I address three descriptive questions: who participates, who lobbies together, and who wins\footnote{By \textit{who wins?} I mean whose ideas end up in policy. This is distinct from measuring \textit{influence} with respect to a counterfactuals or constellations of ideal points. I measure what people say they want and whether they get it. For example, I measure whether rules where commenters requested consideration of environmental justice issues were more likely to address environmental justice issues in the final draft}. Here, \textit{who wins?} is descriptive rather than causal. While insufficient to infer specific causal influence, policy moving in one's preferred direction may indicate that one is aligned with those who have power in that policy process. %There are many potential causes for policy outcomes matching certain policy demands, and I proposed field experiments to test several of them.



% The remainder of this paper presents a case study and an empirical test of whether comments influences rulemaking.



% [Elaborate on org behavior]

% \subsection{Advancing Theories of Bureaucratic Policymaking}

% Quantitative studies of bureaucratic policymaking in political science tend to collapse the time dimension and rarely consider the historical context in which each rule is made. For example, scholarship exploring how political context affects timing and delay in rulemaking models rules as if they are independent of each other and independent of the date they began (see Potter 2017). These studies also tend to focus on the degree to which agency policymaking reflects presidential, congressional priorities and, occasionally, interest group priorities. Political scientists most often ask if agencies are doing what the president wants, what congress wants, or something else. They find significant amounts of ``something else,'' but theory inconsistent on what it is and where it comes from. 

