

\section{Methods}
This project makes two core contributions. First, I introduce new methods to 
%test theories about
measure the formation of lobbying coalitions, their demands, and whether they got what they asked for. 
% and of specific actors within coalitions
Second, I employ field experiments to test mechanisms by which mass mobilization may influence bureaucratic policymaking. 
%Rulemaking gives specific meaning to legislation and thus governmental force to political ideas. Like other policies, regulations also shape the terrain for future politics. 





% % DATA

\subsection{Data}
I focus on bureaucratic policymaking because, due to its sheer volume, it is both rich in opportunities to see different types of political mobilization, organization, and power at work and incompletely understood by political scientists. Specifically, I focus on agency rulemaking, a key part of U.S. policymaking that offers analytical leverage. Rulemaking is a process where agencies must solicit and respond to public comments on regulations (rules) before they carry the force of law (see Figure \ref{inputs}. Draft rules are published in a Notice of Proposed Rulemaking (NPRM).\footnote{Occasionally, comments are also solicited before the draft rule is published through an Advanced Notice of Proposed Rulemaking (ANPRM).} %Intriguingly, while this process originally aimed to promote direct democracy and citizen voice, it is now generally seen as a mechanism to engage expertise \citep{Coglianese2006}. Furthermore, research finds that the comment process actually favors business interests \citep{Yackee2006JOP}. %Despite a large number of case studies, largely from legal scholars, our systematic understanding of the politics of rulemaking is thin.

% diogram of rulemaking  and commenting 
\begin{figure}[h!]
\label{inputs}
\caption{The Textual Record of Agency Rulemaking}
%\begin{table}
\begin{tabular}{@{\extracolsep{5pt}}cccccc}
 &  & &  \\
 & &\multicolumn{3}{c}{(Public Comments)}\\
 & & &$ \downarrow $& \\
\fbox{Inputs} & $\longrightarrow$ & \fbox{Proposal Text} &$\longrightarrow$ & \fbox{Outcome Text}\\
 & & & \\
List of Statutory Authorities &  & Proposed Rule & & Final Rule\\
(Advanced Notice)  &   &  & &   (and Response to Comments)\\
(Comments)  &   &  & &   \\
\end{tabular}
%\end{table}
\end{figure}
 
Automated text analysis allows me to leverage thousands of rules and over 9 million comments posted on regulations.gov, for parts of this study.\footnote{Regulations.gov is used by 90\% of agencies. I also capture comments from agencies that maintain their own systems, such as the Federal Trade Commission (CommentWorks) and the Federal Communications Commission (fjallfoss.fcc.gov/ecfs).} When hand-coding is required, I limit my sample to all rules receiving more than 1000 comments or 100 identical comments and a comparable matched sample (e.g. on agencies, date, economic impact) of remaining rules. % Assessing indirect mechanisms is limited by data availability. I use textual data on congressional interventions since 2007 and attempt to collect political appointee interventions for rules in the above-limited sample. I compliment this broad analysis with case studies of rules related to E.O. 12898 on environmental justice and contemporary rules where I am able to survey participating groups (see appendix for a draft survey). 



% Rich data on several decades of rulemaking are available but have yet to be fully utilized by scholars.  Agencies publish draft rules, and comments received by interest groups, experts, and citizens. This offers leverage to identify the players, winners, and losers and to track those participating in the policy process over time. Rulemaking records often also cite the statutes, executive orders, and court cases that form a rule's historical institutional context. Some of this information, along with draft and final rule publication or withdrawal dates, is summarized since 1981 in the Unified Agenda of Regulatory and Deregulatory Actions (reginfo.gov). From 1994 onward, the text of most proposed draft rules, final rules, and summaries of comments received are published in the Federal Register (federalregister.gov). The result is the text of more than 70 thousand rules. Finally, I collected text of over 9 million comments from 2002 onward via regulations.gov's API.

% With the text of over 70 thousand regulations published since 1981 and over 9 million of the public comments on regulations since 2002, the second chapter of this dissertation will sketch the broad outlines of rulemaking in the American political context: who participates, how often rules are contested, whose ideas and interests are reflected in the text of rules, and who wins with different patterns of mobilization and contestation (or non-contestation). 

% To make the project reasonable, the remaining chapters focus on [three] policy areas that have seen the highest levels of mass mobilization: [environmental, financial services, and communications technology]. To identify environmental rules, I select all rules made by the environmental protection agency and rules made by other agencies that cite president Clinton's executive order on environmental justice or president Obama's executive order on climate adaption. This allows me to consider how the same environmental problems may be addressed by different agencies. Financial services regulations are those that cite the Dodd-Frank act. Communications technology regulations are those proposed by the Federal Communications Commission.
