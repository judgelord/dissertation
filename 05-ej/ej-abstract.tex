I explore the role of public comments in rulemaking by focusing on their role in the environmental justice movement. Environmental justice concerns focus on unequal access to healthy environments and protection from harms caused by things like pollution and climate change. The ways in which agencies consider environmental justice highlights how rulemaking has distributive consequences, how the public comment process creates a political community, and how claims raised by activists are addressed. Examining thousands of rulemaking processes at agencies known to address environmental justice concerns, I find that when public comments raise environmental justice concerns, these concerns are more likely to be addressed in the final rule. However, baseline rates of addressing environmental justice in rulemaking are so low that even as the probability that agencies will address environmental justice significantly increases when commenters raise these issues, in most rules, even those where commenters raise environmental justice concerns, there is no explicit attention to environmental justice. Furthermore, even when agencies do address environmental justice concerns, they often do not make the substantive policy changes that activists demand. While the number of comments raising environmental justice concerns is positively correlated with change in policy texts, the effect of the general level of public attention is mixed. Rules with more comments are more likely to address environmental justice when they did not address it in the draft rule, but rules with more comments are less likely to change how they addressed environmental justice if they did address it in the draft rule. These results suggest that the politics of rulemaking differs when there is more public attention. Patterns also vary across agencies, possibly due to the alignment of environmental justice aims with agency missions. 












