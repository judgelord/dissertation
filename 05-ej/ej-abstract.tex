Political movements advance new rights and social values by shaping policy agendas and reframing policy debates. Likewise, lack of movement pressure is a leading explanation for the failure of policy efforts. To examine how movements affect policymaking, I assess the aggregate impact of environmental justice advocacy on federal policy from 1993-2020 using a new dataset of 13,179 draft and final rule pairs from 40 U.S. federal agencies and 42 million public comments on these draft policies, I find that when public comments and petition campaigns raise distributive justice concerns, these concerns are more likely to be addressed in the final policy document. Supporting theories about how institutions shape receptivity to issue frames, I find that agencies with institutional processes for addressing environmental justice concerns are more responsive to movement pressure. The scale of mobilization also matters. When more groups and individuals raise environmental justice concerns, policy texts are more likely to change. However, within the movement, some organizations are much more successful than others. Furthermore, baseline rates of addressing environmental justice are so low that, even when groups raise environmental justice concerns, most policy documents pay no explicit attention to them---a pattern that persists across agencies and throughout the G.W. Bush, Obama, and Trump administrations. These findings suggest that who makes distributive justice claims, their alignment with institutional cultures, and levels of public pressure all systematically affect policymaking.