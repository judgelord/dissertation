I explore the role of public comments in rulemaking by focusing on their role in the environmental justice movement. Environmental justice concerns focus on the unequal access to healthy environments and protection from harms caused by things like pollution and climate change. The ways in which agencies consider environmental justice highlights how rulemaking has distributive consequences, how the public comment process creates a political community, and how claims raised by activists are addressed. Examining thousands of rulemaking processes at agencies known to address environmental justice concerns, I find that when public comments raise environmental justice concerns, these concerns are more likely to be addressed in the final rule. Effects vary across agencies, possibly due to the alignment of environmental justice aims with agency missions. While we cannot infer that agencies addressing environmental justice concerns is caused by the public comments themselves, comments may be a good proxy for mobilization in general. Furthermore, the correlation between mobilization and policy changes is largest and most significant in agencies with missions focused on environmental and distributional policy, i.e. the kinds of agencies we may expect to be most responsive to environmental justice concerns.
