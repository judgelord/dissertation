Social movements play a critical role in advancing landmark statutes that recognize new rights and social values. Likewise, lack of movement pressure is a leading explanation for the failure of policy efforts. Yet, we have little systematic evidence about the impact of social movements on policy. To what extent do movements shape the thousands of policies that governments make every year? I examine how social movements affect policymaking by assessing the environmental justice movement's impact on 25 thousand policy documents from 40 U.S. federal agencies.  Leveraging a new dataset of 42 million public comments on these policies, I find that when public comments raise environmental justice concerns, these concerns are more likely to be addressed in the final rule. Effect sizes vary across agencies, possibly due to the alignment of environmental justice aims with agency missions.
The magnitude of public pressure also matters. When more groups and individuals raise environmental justice concerns, policy texts are more likely to change, even when controlling for overall levels of public attention. These findings suggest that distributive justice claims, levels of public attention, and levels of public pressure all systematically affect policymaking.