Political movements advance new rights and social values by shaping policy agendas and reframing policy debates. Likewise, lack of movement pressure is a leading explanation for the failure of policy efforts. Yet, we have little systematic evidence about the impact of movement pressure on policymaking, especially policies made in the bureaucracy. I examine how movements affect policymaking by assessing the environmental justice movement's impact on 25 thousand draft and final policy documents from 40 U.S. federal agencies. Leveraging a new dataset of 42 million public comments on these draft policies, I find that when public comments and petition campaigns raise distributive justice concerns, these concerns are more likely to be addressed in the final policy document. Effect sizes vary across agencies; agencies with institutional processes for addressing environmental justice concerns are more responsive to movement pressure. The magnitude of pressure also matters. When more groups and individuals raise environmental justice concerns, policy texts are more likely to change. However, within the movement, some organizations are much more successful than others. Furthermore, baseline rates of addressing environmental justice are so low that, even when groups raise environmental justice concerns, most policy documents pay no explicit attention to them---a pattern that persists across agencies and throughout the G.W. Bush, Obama, and Trump administrations. These findings suggest that who makes distributive justice claims, their alignment with institutional cultures, and levels of public pressure all systematically affect policymaking.