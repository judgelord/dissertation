Social movements play a critical role in advancing landmark statutes that recognize new rights and social values. Likewise, lack of movement pressure is a leading explanation for the failure of policy efforts. Yet, we have little systematic evidence about the impact of social movement pressure on policymaking. I examine how social movements affect policymaking by assessing the environmental justice movement's impact on 25 thousand draft and final policy documents from 40 U.S. federal agencies. Leveraging a new dataset of 42 million public comments on these draft policies, I find that when public comments and petition campaigns raise environmental justice concerns, these concerns are more likely to be addressed in the final policy document. Effect sizes vary across agencies, possibly due to organizational identities, missions, and institutional processes for addressing environmental justice concerns. The magnitude of public pressure also matters. When more groups and individuals raise environmental justice concerns, policy texts are more likely to change, even when controlling for overall levels of public attention. These findings suggest that distributive justice claims, their alignment with institutional cultures, levels of public attention, and levels of public pressure all systematically affect policymaking. However, baseline rates of addressing environmental justice are so low that, even when activists raise EJ concerns, most policy documents pay no explicit attention to it---a pattern that persists across agencies and throughout the G.W. Bush, Obama, and Trump administrations.