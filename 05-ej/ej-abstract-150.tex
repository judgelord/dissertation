Political movements shape policy agendas and reframe policy debates.  To systematically examine how movements affect policymaking, I assess the aggregate impact of environmental justice advocacy on US federal policy from 1993 to 2020 using a new dataset of 13,179 draft and final rule pairs from 40 agencies and 42 million public comments. I find that when advocates raise distributive justice concerns, final rules more often address these concerns. Supporting theories about how institutions shape receptivity to issue frames, agencies with institutional processes for addressing environmental justice concerns are more responsive to movement pressure. The scale of mobilization also matters. When more groups and individuals raise environmental justice concerns, policy texts are more likely to change. However, within the movement, some organizations are much more successful than others. These findings suggest that who makes distributive justice claims, their alignment with institutional cultures, and levels of public pressure all systematically affect policymaking.