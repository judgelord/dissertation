\documentclass[
      12pt,
        ]{article}






% --- type and typeface? -----------------------

% input
\usepackage[utf8]{inputenc}

% typography
\usepackage{microtype}


\usepackage[T1]{fontenc}


% text block
\usepackage{setspace}
\usepackage[ 
              left = 1in,top = 1in,right = 1in,bottom = 1in 
            ]{geometry}

\usepackage{enumitem}
  \setlist{noitemsep}



% decimal numbering for appendix figs and tabs


% Deletes section counters
% \setcounter{secnumdepth}{0}







  \usepackage{longtable, booktabs}









  \usepackage{natbib}
  \bibliographystyle{/Users/devin/dissertation/assets/apsr.bst}
  % protect underscores in most circumstances
  \usepackage[strings]{underscore} 


% 

% \newtheorem{hypothesis}{Hypothesis}

\makeatletter
  \@ifpackageloaded{hyperref}{}{%
    \ifxetex
      % page size defined by xetex
      % unicode breaks when used with xetex
      \PassOptionsToPackage{hyphens}{url}\usepackage[setpagesize = false, 
                                                     unicode = false, 
                                                     xetex]{hyperref}
    \else
      \PassOptionsToPackage{hyphens}{url}\usepackage[unicode = true]{hyperref}
    \fi
  }

  \@ifpackageloaded{color}{
    \PassOptionsToPackage{usenames,dvipsnames}{color}
  }{
    \usepackage[usenames,dvipsnames]{color}
  }
\makeatother

\hypersetup{breaklinks = true,
            bookmarks = true,
            pdfauthor = {Devin Judge-Lord ()},
             pdfkeywords  =  {},  
            pdftitle = {The Environmental Justice Movement and Technocratic Policymaking},
            colorlinks = true,
            citecolor = black,
            urlcolor = blue,
            linkcolor = magenta,
            pdfborder = {0 0 0}}

% \urlstyle{same}  % don't use monospace font for urls


% set default figure placement to htbp
\makeatletter
  \def\fps@figure{hbtp}
\makeatother


% optional footnotes as endnotes


% ----- Pandoc wants this tightlist command ----------
\providecommand{\tightlist}{
  \setlength{\itemsep}{0pt}
  \setlength{\parskip}{0pt}
}





% --- title & section styles -----------------------


% title, author, date
  \title{The Environmental Justice Movement and Technocratic Policymaking}
 

  \author{ % author, option footnote, optional affiliation
            Devin Judge-Lord\footnote{University of Wisconsin-Madison, \href{mailto:judgelord@wisc.edu}{\nolinkurl{judgelord@wisc.edu}}. Slides and the most recent draft are available at \url{https://judgelord.github.io/research/whymail}} 
            }

% auto-format date?
  \date{2020-12-17}


% abstract
\usepackage{abstract}
  \renewcommand{\abstractname}{}    % clear the title
  \renewcommand{\absnamepos}{empty} % originally center

  \newcommand*{\authorfont}{\sffamily\selectfont}


% section titles
\usepackage[small, bf, sc]{titlesec}
  % \titleformat*{\subsection}{\itshape}
  \titleformat*{\subsubsection}{\itshape} 
  \titleformat*{\paragraph}{\itshape} 
  \titleformat*{\subparagraph}{\itshape}



%\usepackage{float}
%\floatstyle{plaintop}
%\restylefloat{table}
\usepackage{floatrow}
\floatsetup[figure]{capposition=top}
\floatsetup[table]{capposition=top}
\usepackage{multirow}
\usepackage{rotating} 
\usepackage{caption}






















\begin{document}
 

% --- PAGE: title and abstract -----------------------

  \maketitle

% \pagenumbering{gobble}
  \pagenumbering{gobble}



  \begin{abstract}
    \noindent THIS DRAFT WAS PREPARED FOR THE 2018 ASSOCIATION FOR ENVIRONMENTAL STUDIES AND SCIENCES CONFERENCE.

SLIDES AND THE MOST RECENT DRAFT ARE \href{https://judgelord.github.io/research/ej/}{HERE}

\bigskip

This chapter explores the role of public comments in rulemaking by focusing on their role in the environmental justice movement. Environmental justice concerns focus on the unequal access to healthy environments and protection from harms caused by things like pollution and climate change. The ways in which agencies consider environmental justice highlights how rulemaking has distributive consequences, how the public comment process creates a political community, and how claims raised by activists are addressed. Examining over 20,000 rulemaking processes at agencies known to address environmental justice concerns, I find that when public comments raise environmental justice concerns, these concerns are more likely to be addressed in the final rule. Effects vary across agencies, possibly due to the alignment of environmental justice aims with agency missions. While we cannot infer that agencies addressing environmental justice concerns is caused by the public comments themselves, comments may be a good proxy for mobilization in general. Furthermore, the correlation between mobilization and policy changes is largest and most significant in agencies with missions focused on environmental and distributional policy, i.e.~the kinds of agencies we may expect to be most responsive to environmental justice concerns. 

    

  \end{abstract}



% --- PAGE: contents -----------------------





% --- PAGE: body -----------------------


  \newpage
  \pagenumbering{arabic}

\noindent 
 
   
   
\hypertarget{introduction}{%
\section{Introduction}\label{introduction}}

I explore the role of public comments in rulemaking by focusing on their
role in the environmental justice movement. Environmental justice
concerns focus on the unequal access to healthy environments and
protection from harms caused by things like pollution and climate
change. The ways in which agencies consider environmental justice
highlights how rulemaking has distributive consequences, how the public
comment process creates temporary political communities, and how claims
raised by activists are addressed.

Examining over 20,000 rulemaking processes at agencies known to address
environmental justice concerns, I find that when public comments raise
environmental justice concerns, these concerns are more likely to be
addressed in the final rule. In this preliminary analysis, however, the
number of comments mobilized is not related to success. While we cannot
infer that agencies addressing environmental justice concerns is caused
by the public comments themselves, comments may be a good proxy for
lobbying in general. Furthermore, the correlation between raising
environmental justice and policy changes is largest and most significant
in agencies with missions focused on environmental and distributional
policy, i.e.~the kinds of bureaucrats who we may expect to have
institutional and cognitive processes primed to be most responsive to
environmental justice concerns.

This case illustrates the way that activists use public comments to
inject ideas directly into the rulemaking process. I focus on the
environmental justice movement because it offers a broad but tractable
scope for analysis and shows what is at stake in the politics of
rulemaking. How rules consider environmental justice issues illustrates
how rulemaking constructs a political community of ``relevant''
stakeholders and ``appropriate'' criteria to evaluate policy consequences.
Thus, the idea of environmental justice is an example of how social
movements can mobilize norms and evaluative frameworks that are
connected to organizational identities, mission, and reputations and
that have implications for bureaucratic decisions.

The use of an environmental justice frame does not always imply the same
communities of concern. Environmental justice emerged out of movements
against environmental racism, especially the disposal of toxic
substances in communities of color \citep{Bullard1993}. However, the term
quickly took on a wider array of meanings, encompassing any marginalized
group. In 1994 the Bill Clinton signed an Executive order on
Environmental Justice that required all parts of the federal government
to make ``addressing disproportionately high and adverse human health or
environmental effects of programs, policies, and activities on minority
populations and low-income populations'' a core aspect of their mission.
This meant considering disproportionate effects during rulemaking.

Fundamental definitions of the public good and minority rights are
implicit in agency rules. The public comment process offers an
opportunity to protest these definitions. Protest is one way that
marginalized groups can communicate opinions on issues to government
officials \citep{Gillion2013}. In the case of the EPA's Mercury Rules, two
such issues were decisive. First, as with many forms of pollution,
mercury-emitting power plants are concentrated in low-income, often
non-White communities. Second, certain populations consume much more
locally-caught freshwater fish, a major vector of Mercury toxicity.
Studies inspired by the political controversy around the Mercury Rules
found high risk among communities included ``Hispanic, Vietnamese, and
Laotian populations in California and Great Lakes tribal populations
(Chippewa and Ojibwe) active on ceded territories around the Great
Lakes'' (EPA 2012). Thus the standards that EPA chooses are fundamentally
dependent on whom the regulation aims to protect: the average citizen,
local residents, or fishing communities. This decision has disparate
effects based on race and class because of disparate effects based on
geography and different cultural practices. Such disparate impacts are
often called environmental justice issues.

In December 2000, when the EPA first announced its intention to regulate
Mercury from power plants, the notice published in the Federal Register
did not address environmental justice issues such as the disparate
effects of mercury on certain populations. Risks were only discussed in
reference to ``the U.S. population'' (EPA 2000). When the first draft rule
was published, it only discussed the effects of the rule on regulated
entities, noting that ``Other types of entities not listed could also be
affected'' (EPA 2002). Commenting on this draft, Heather McCausland of
the Alaska Community Action on Toxics (ACAT) wrote:

\begin{quote}
``The amount of methyl-mercury and other bioaccumulative chemicals
consumed by Alaskans (especially Alaskan Natives) could potentially be
much higher than is assumed...The Alaska Native mortality rate for
babies which according to the CDC is 70\% higher than the United States
average. Indigenous Arctic \& Alaskan Native populations are some of
the most polluted populations in the world (\url{http://www.amap.no/}).
Global transport \& old military sites contaminate us too''
\end{quote}

After receiving hundreds of thousands of comments and pressure from
tribal organizations, a revised proposed rule echoed McCausland's
comment noting that ``Some subpopulations in the U.S., such as Native
Americans, Southeast Asian Americans, and lower-income subsistence
fishers, may rely on fish as a primary source of nutrition and/or for
cultural practices. Therefore, they consume larger amounts of fish than
the general population and may be at a greater risk of the adverse
health effects from Hg due to increased exposure'' (EPA 2004).

After nearly a million additional public comments, a revised proposed
rule ultimately included five pages of analysis of the disparate impacts
on "vulnerable populations" including ``African Americans,'' ``Hispanic,''
``Native American,'' and ``Other and Multi-racial'' groups (EPA 2011). In
the final rule, the language of ``vulnerable populations'' was replaced
with ``minority, low income, and indigenous populations'' (EPA 2012). EPA
had also conducted an analysis of sub-populations with particularly high
potential risks exposure due to high rates of fish consumption as well
as an additional analysis of the distribution of mortality risk by to
race.

Of this second round of comments, over 200 explicitly raised the
environmental justice issues. The Little River Band of Ottawa Indians
expressed the Tribe's ``frustration at trying to impress upon the EPA the
multiple and profound impacts of mercury contamination from a Tribal
perspective. Not to mention the obligations under treaties to
participate with tribes on a''Government to Government" basis. At
present, no such meetings have occurred in any meaningful manner with
EPA Region V, the EPA National American Indian Environmental Office, nor
the State of Michigan's Department of Environmental Quality." They
conclude that "Although EPA purported to consider environmental justice
as it developed its ``Clean Air Mercury Rule,'' it failed utterly. In this
rulemaking, EPA perpetuated, rather than ameliorated, a long history of
cultural discrimination against tribes and their members" (Sprague
2011). Did comments like these play a role in EPA's changed analysis of
who Mercury limits should aim to protect?

Given the many potential sources of influence, it may be difficult to
attribute causal effects of particular comments on a given policy.
However, comments may serve as a good proxy for the general mobilization
of groups and individuals around an administrative process, and it is
not clear why EPA would not address environmental justice in the first
draft of a rule and then add it to subsequent drafts in the absence of
activists mobilization. Electoral politics does not offer an easy
explanation. The notice proposing the Mercury Rule was issued by the
Clinton administration, the same administration that issued the
Executive Order on Environmental Justice, and the subsequent drafts that
did address environmental justice issues were published by the Bush
administration, which had a more contentious relationship with
environmental justice advocates, while Republicans controlled both
houses of Congress. The expansion of the analysis from one draft to the
next seems to be in response to activist pressure.

Mobilization around ideas like environmental justice may even affect
policy discourse when agency administrators are explicitly hostile to
the cause. For example, in an October 2017 proposed rule to repeal
restrictions on power plant pollution, the Trump EPA acknowledges that
``low-income and minority communities located in proximity to {[}power
plants{]} may have experienced an improvement in air quality as a result
of the emissions reductions.'' Because the Executive Order requires
attention to environmental justice and because the Obama EPA discussed
it when promulgating the rule, the environmental justice cannot safely
be ignored. However, the Trump EPA contends, the Obama EPA ``did not
address lower household energy bills for low-income households {[}and
that{]} workers losing jobs in regions or occupations with weak labor
markets would have been most vulnerable'' (EPA 2017). As of , this
proposed rule has received over 150,000 public comments.

Tracing ideas like environmental justice through the rulemaking record
may offer one way to study the mechanisms by which social movements do
or do not influence bureaucratic policymaking. Specifically, if rules
are proposed without attention to environmental justice concerns, but
environmental justice concerns are raised in the public comments and
then appear in the final policy, this may be evidence that mobilization
mattered.

Environmental justice is certainly not the only way to observe such
effects, but it has some convenient properties. First, policies framed
as ``environmental'' issues are, unlike issues like issues like civil
rights and immigration, inconsistently racialized and, unlike issues
like taxes and spending, inconsistently focus on \emph{distributions} of
costs and benefits. Despite almost always having disparate impacts, an
environmental frame often creates a human-environment distinction and
shifts attention to non-human objects such as air, water, food, or
landscapes and away from the distribution of access to these objects or
protection from them when they are contaminated. Second, compared to
other ideas around which people mobilize, ``environmental justice'' is a
fairly distinctive phrase. Most people who use this phrase share a
general definitional foundation. Third, this phrase is fairly common
when the idea is being discussed, i.e.~there are not many synonyms and
groups raising equity concerns on ``environmental'' issues commonly refer
to environmental justice. Many who use the narrower, related term
``environmental racism'' also use ``environmental justice'' in their
advocacy. Finally, the term is relevant to rulemaking records in
particular because of an Executive Order issued in 1994 by President
Clinton "Federal Actions to Address Environmental Justice in Minority
Populations and Low-Income Populations" which required all agency
actions and policies to consider environmental justice implications.
This does not mean that all draft or final rules do so, but when they
do, they tend to cite the executive order and explicitly discuss
environmental justice. For the same reason, commenters, especially
sophisticated ones, who critique draft rules also use the cite this
executive order and use this language.

\hypertarget{data}{%
\section{Data}\label{data}}

In order to examine whether the environmental justice movement's mass
mobilization of letter-writing influences the discourse around policies,
I use the text of draft rules, public comments, and final rules
retrieved from regulations.gov. Figure
\ref{fig:ej} compares
the use of the term ``environmental justice'' in draft policies, public
comments on these drafts, and the final versions of the policies. I
collected all documents from the website regulations.gov and selected
58,789 that use the phrase ``environmental justice.'' This includes 5,109
proposed rules, 17,539 public comments on these proposed rules, and
10,418 final rules. I then added all draft and final rules from all 35
agencies that have published at least one rule addressing environmental
justice, an additional 40,096 documents.\footnote{This may be an over-inclusive sample and in future work, I may
  attempt to refine this sample to rules that plausibly relate to
  environmental justice issues.}

Notably, more than twice as many final rules as proposed rules contain
the phrase ``environmental justice.'' This suggests a systematic element
in how agency policymakers are revising draft rules and responding to
public comment. Below, I investigate the extent that this change from
the draft to final policy is related to environmental justice issues
being raised in the public comments.

\begin{figure}

{\centering \includegraphics[width=0.49\linewidth]{/Users/devin/dissertation/Figs/eandj-hist} \includegraphics[width=0.49\linewidth]{/Users/devin/dissertation/Figs/eandj-comments} 

}

\caption{Number of Rules where Environmental Justice Appears in the Record (Left) and Number of Comments per Notice or Proposed Rule (Right).}\label{fig:ej}
\end{figure}

Figure \ref{fig:ej}
shows that the number of rules and comments increasing over time. This
may reflect increased salience of this concept, but it may primarily be
the result of the increasing prevalence of searchable texts. Similarly,
the increased number of rules where comments mention environmental
justice may reflect growth in the movement but also may reflect more
overall comments as technology has made commenting easier. Testing the
hypotheses that comments raising environmental justice concerns are
related to specific rules where environmental justice is addressed in
the final but not in the draft requires rule-specific analysis.

What we can say from Figure \ref{fig:ej} is that each year, more final rules directly address
environmental justice when their draft did not. Additionally, there are
many rules where environmental justice is mentioned in the public
comments and not in the draft. Recently, there are also many rules where
environmental justice is raised in the comments but does not make it
into the final draft.\footnote{Note that, because not all comments are machine-readable, this is an
  underestimate of the number of comments mentioning environmental
  justice, so we cannot conclude that before 2010, rules were
  mentioning environmental justice when the comments had not.
  Additionally, this does not include comments like those of the
  Bishops who raise justice issues but do not use the phrase
  ``environmental justice.''}

\hypertarget{second-order-representation}{%
\section{Second-order Representation}\label{second-order-representation}}

Before analysis of whether comments matter, I briefly describe who these
commenters are. This is what \citet{Seifter2016UCLA} calls ``second-order''
representation. It is insufficient to know which groups participate. We
also need to know who these groups represent.

I investigate who is raising environmental justice concerns in two ways.
First, I identify the top organizational commenters such as tribes,
businesses, and nonprofits that are using environmental justice language
and investigate who these groups represent. Second, for comments where a
citizen signed their name, I compare surnames to their racial and ethnic
identity propensities with respect to the U.S. census. Together these
two pieces of information allow me to comment on ``second order''
representation, i.e.~not just the extent to which public comments relate
to government policy, but the extent to which public comments are
representative of the public and of the groups they claim to represent.

\begin{longtable}[]{@{}llll@{}}
\caption{\label{tab:orgs} Organizations mobilizing mass comment campaigns mentioning ``environmental justice''}\tabularnewline
\toprule
& Organization & Comments & Rules\tabularnewline
\midrule
\endfirsthead
\toprule
& Organization & Comments & Rules\tabularnewline
\midrule
\endhead
1 & Earthjustice & 1114782 & 28\tabularnewline
2 & Natural Resources Defense Council & 340554 & 8\tabularnewline
3 & Sierra Club & 349841 & 5\tabularnewline
4 & Alliance for Climate Protection & 253867 & 5\tabularnewline
5 & WE ACT for Environmental Justice & 2402 & 3\tabularnewline
6 & CREDO & 112879 & 2\tabularnewline
7 & Union of Concerned Scientists & 43559 & 2\tabularnewline
8 & Earthworks & 308 & 2\tabularnewline
9 & Communities for a Better Environment & 21 & 2\tabularnewline
10 & Southern Company & 8 & 2\tabularnewline
11 & Move On & 165948 & 1\tabularnewline
12 & Care2 & 70450 & 1\tabularnewline
13 & The Pew Charitable Trusts & 63769 & 1\tabularnewline
14 & Hudson-Environmental Action & 35000 & 1\tabularnewline
15 & Democracy for America & 4426 & 1\tabularnewline
\bottomrule
\end{longtable}

Table \ref{tab:orgs} shows the top 15 organizational commenters who
used the phrase ``environmental justice'' in their comments, including all
organizations who did so on more than one rule or mobilized more than
100,000 such comments. The six organizations responsible for mobilizing
more than 100,000 comments and several others on the list are national
nonprofit advocacy groups. We Act and Communities for a Better
Environment are both more community-based groups focusing primarily on
environmental justice issues. Southern Company is the only corporation
on the list

The top mobilizer, Earthjustice, is primarily engaged in litigation on
behalf environmental causes. Their website boasts 2.2 million
supporters, but it is not clear who they are or if they play any role in
the advocacy strategy. A search on the website returns 360 results for
"Environmental Justice," with the top results from staff biographies
who work on more local or targeted work such as environmental conditions
for the incarcerated, but the environmental justice language used on the
main page is relatively mild. For example, ``We are fighting for a future
where children can breathe clean air, no matter where they live''
\citep{Earthjustice2017}. The website does contain Spanish language content.

The Natural Resources Defense Council is similar to Earthjustice--a
national nonprofit funded by donations and focused on litigation--but
they also lobby. CREDO Action and MoveOn are more generic progressive
mobilizers who lack a systematic focus on environmental justice issues,
but occasionally leverage their very large membership lists to support
campaigns environmental justice campaigns led by others
\citep{MoveOn.org2017, CREDO2017}. The Alliance for Climate Protection is a
more of an elite political group founded by former Vice President Al
Gore.

We Act and Communities for a Better Environment both have environmental
justice in their central mission statement. We Act was founded by
community leaders in Harlem, NY, to fight environmental racisms and
advocate for better air quality \citep{WEACT2017}. Communities for a Better
Environment has projects throughout California but is particularly
active in Oakland \citep{CBECAL2017CommunitiesEnvironment}. Much of the
content of their website is in both English and Spanish. Both
organizations focus primarily on ``low-income communities of color'' and
thus frame their work with respect to race and class. While both
organizations participated in national policymaking We Act is more
focused on communities in Harlem and New York whereas Communities for a
Better Environment casts a wider frame: "CBE's vision of environmental
justice is global--that's why the organization continues to participate
in such international efforts as the Indigenous Environmental Network
and the Global Week of Action for Climate Justice" \citep{CBECAL2017}.

The Southern Company comments are too few to count as mass mobilization.
Companies do sometimes fund mobilization campaigns, but all of 8 of
these comments were submitted by the Southern Company. Interestingly,
the company repeatedly raises research into environmental justice
concerns in order to frame these issues as a legitimate but unresolved
scientific debate that is not yet conclusive enough to base regulations
on: "People with lower SES are exposed to almost an order of magnitude
more traffic near their homes (Reynolds et al., 2001), and live closer
to large industrial sites and are exposed to more industrial air
pollution (Jerrett et al., 2001). Legitimate health concerns must be
addressed. But adopting standards with a scientific basis so uncertain
that health improvement cannot be assured is not sound public health
policy." Like many companies they claim to represent their customers as
"electric generating companies and their customers are expected to bear
much of the burden" of regulations \citep{Hobson2004}.

With respect to second-order representation, it appears that the groups
most often using the language of environmental justice may do so
sincerely but do not themselves represent affected communities. Several
groups representing local communities and led by community leaders have
participated, but not nearly as often or with the same intensity as the
``big greens.'' This highlights the importance of resources as a condition
for mobilizing. Not all groups who may benefit from political
information are able to leverage it because they lack the resources to
invest in a campaign. However, it may be the case that smaller, more
member-driven groups join coalitions with groups with more resources who
mobilize on their behalf. More work is needed to identify coalitions to
assess this possibility.

Finally, a third class of commenter raises environmental justice issues
as a way to re-frame them as ongoing debates and thus undermine their
urgency. I call this reason for engaging as ``breaking a perceived
consensus.'' In a way, the fact that an energy company felt compelled to
acknowledge and question environmental justice concerns suggests their
importance for policy outcomes.

Next, I attempt to estimate the racial distribution of those who comment
using environmental justice language. This can only be done for
individuals who commented separately from mobilizing organizations and
signed their full name on their comment. Figure
\ref{fig:ejcommentsbyrace} shows two ways of estimating the racial
distribution of commenters who raise ``environmental justice'' concerns in
their comments. Both methods use the reported racial identities
associated with surnames as recorded in the 2010 census\footnote{I recode ``Hispanic'' as ``Latinx'' in both cases because the
  prediction method assumes a forced choice that includes ``Hispanic''
  as a primary racial category}
and are
based on a limited sample of 327 commenters who signed their name with a
surname matching census records. The first is based on the proportion of
people with a given surname that identified as belonging to each racial
category (from this limited set of options). The estimated proportion of
each race for this sample is simply the average of proportions
identified with each surname. This is likely the most accurate way to
represent the racial distribution of a set of surnames, but it does not
assign specific individuals to racial categories. The second method
does. It predicts the race of each individual in the sample based on
their surname given the distribution of racial categories reported by
people with that surname and the proportion of each race in the U.S.
population. Thus, while a surname may be more common among people who
identify as black rather than as white, there may still be more White
people with that surname and this method would predict that the person
is White. For this reason, the portion of individuals predicted to be
White (right) is higher than in the probabilistic distribution (left).

\begin{figure}

{\centering \includegraphics[width=0.49\linewidth]{/Users/devin/dissertation/Figs/race-prob} \includegraphics[width=0.49\linewidth]{/Users/devin/dissertation/Figs/race-pred} 

}

\caption{Probabilistic (Left) and Predicted (Right) Race from Census Surnames}\label{fig:ejcommentsbyrace}
\end{figure}

Compared to estimates from the 2010 census, this sample of commenters
appears to be disproportionately Black and less than proportionately
Latinx or Asian, with just slightly fewer Whites relative to the
national population. This makes sense given that environmental justice
theorizing and activism have been led by African American's
\citep{Bullard1993}.

Figure \ref{fig:ejwordsbyrace} shows the most common words used in comments
with respect to the predicted race of each commenter in the sample. As
there are very few predicted non-White commenters in the sample, it is
unwise to infer too much from this figure.

\begin{figure}

{\centering \includegraphics[width=0.32\linewidth]{/Users/devin/dissertation/Figs/ej-white-words} \includegraphics[width=0.32\linewidth]{/Users/devin/dissertation/Figs/ej-latinx-words} \includegraphics[width=0.32\linewidth]{/Users/devin/dissertation/Figs/ej-black-words} 

}

\caption{Most frequent words used in comments mentioning "environmental justice" by predicted race (From Left: White, Latinx, Black)}\label{fig:ejwordsbyrace}
\end{figure}

\hypertarget{results-are-final-rules-more-likely-to-address-environmental-justice-after-comments-do-so}{%
\section{Results: Are final rules more likely to address environmental justice after comments do so?}\label{results-are-final-rules-more-likely-to-address-environmental-justice-after-comments-do-so}}

This subsection presents preliminary results from an analysis of draft
rules, comments, and final rules. Descriptively, figure
\ref{fig:ejwinrate}
shows that in general, most rules that do not address environmental
justice in the draft but these issues are raised in the comments, do not
end up addressing them in the final version. It appears that it may have
been the case in 2006 and 2007 but since then the number of rules
receiving comments raising environmental justice concerns has grown
while the number of rules that end up adding it has remained the same.
Since 2015, there has been a decline in both the number of rules adding
environmental justice and the number of rules where commenters demanded
it, especially in 2017. One way to interpret figure
\ref{fig:ejwinrate}
would be to say that commenters saw a potential ally in President Obama
and increased their demands for environmental justice, but that these
increased demands had little effect. However, a better approach would be
to estimate a statistical model of the effect of comments on the change
from draft to final rules. This is what I do.

\begin{figure}

{\centering \includegraphics[width=6.5in]{/Users/devin/dissertation/Figs/ej-win-rate} 

}

\caption{Rules With Comments Addressing EJ on a Draft That Did Not}\label{fig:ejwinrate}
\end{figure}

For this preliminary analysis, I estimate a logit regression where the
outcome is whether environmental justice was addressed in the final rule
and the predictors are whether it was addressed in the draft rule,
whether it was addressed in the comments, and the total number of
comments received.

\[ \hat{EJ in Final} = \Bigg\{ \begin{array}{ll}
1 & if \beta_0 + \beta_1 EJ in Draft + \beta_2 EJ in Comments + \beta_3 Total Comments  + \epsilon > 0\\
0 & otherwise 
  \end{array}\]

As logit coefficients are not easily interpretable, I calculate
predicted probabilities for the types of rules of interest, i.e.~rules
where environmental justice was not raised in the draft, \emph{EJinDraft}=0.
Figure \ref{fig:ejpredicted} shows the predicted probability of a final rule
addressing environmental justice when the draft rule did not for all
agencies that have ever published a rule addressing environmental
justice (left) and the EPA alone (right). The EPA accounts for nearly
two-thirds of the cases where environmental justice is raised in the
comments on a draft rule that did not address it. The total number of
comments mentioning ``environmental justice'' had a substantively small
and statistically insignificant effect on policy. As the flat lines on
both figures show, the predicted probability of adding the phrase
``environmental justice'' does not increase with the number of comments.
Environmental justice being raised in any one comment does have a
statistically significant and substantively large effect.

Overall the probability across all agencies of adding environmental
justice increases from under 2\% to about 9\%. At the EPA, the probability
triples from about 6\% to about 18\%.

\begin{figure}

{\centering \includegraphics[width=0.49\linewidth]{/Users/devin/dissertation/Figs/ej-prob-env-nprms} \includegraphics[width=0.49\linewidth]{/Users/devin/dissertation/Figs/ej-prob-epa-nprms} 

}

\caption{Proposed Rules Not Addressing Environmental Justice (Left: All agencies. Right: Environmental Protection Agency}\label{fig:ejpredicted}
\end{figure}

To examine the degree to which this generalizes across agencies, figure
\ref{ejlogitagencies} presents predicted probabilities modeled
for each agency, showing that the range of predicted probabilities is
systematically higher when environmental justice, but with varying
degrees of confidence. There is considerable variation among agencies.
Point estimates are only shown for agencies where confidence intervals
do not overlap. Only a few agencies have statistically significant
different estimates, but the agencies where effects are largest are
exactly the agencies we would expect to be influenced by comments
raising environmental justice concerns, i.e.~agencies that deal with
environmental issues with distributive consequences.

\begin{figure}

{\centering \includegraphics[width=6.5in]{/Users/devin/dissertation/Figs/ej-prprob-by-agency} 

}

\caption{Proposed Rules Not Addressing Environmental Justice}\label{fig:ejlogitagencies}
\end{figure}

As the EPA accounts for the largest portion of the data, we are most
confident in this result, but the EPA is not the agency with the largest
predicted effects. The Forest Service has a predicted 30\% difference
between rules that do and do not receive comments raising environmental
justices concerns. This may be because the forest service is mainly in
the business of managing forests, leasing timber rights, and controlling
wildfires. These types of decisions may have acute distributional
effects that may not be the initial focus of the agency. Once raised,
however, addressing such effects fits squarely in their mission. Though
not statically significant the Department of Agriculture and Bureau of
Land Management, who make similar kinds of decisions, are also
susceptible to environmental justice claims. Similarly, the Federal
Railroad Administration, Department of Transportation, Federal Highway
Administration, Federal Motor Carrier Safety Administration, and the
Federal Transportation Administration (which aids local transportation
projects) all have large probability distributions. These agencies are
making decisions about infrastructure projects with implications for
neighborhood environments and air quality. Environmental justice may
often come up, but there may be a lot of variation in whether the agency
then decides if they are relevant to transportation policies and
projects that are primarily about neither environmental nor justice
concerns.

Research agencies, including the National Research Council, National
Oceanographic and Atmospheric Administration and Fish and Wildlife
Service all had statistically significant but small spreads. In these
cases, we can be confident of the correlation but understand that it is
a rare occurrence, which makes sense if most research does not have
direct environmental justice consequences, but agencies are open to
adding analysis of these issues if when they are raised.

\hypertarget{conclusion}{%
\section{Conclusion}\label{conclusion}}

This analysis has illustrated the importance of ideas in policymaking
and cross-sectional statistical results suggest that when issue frames
like environmental justice are raised there is a higher probability that
policymakers consider the effects on marginalized populations.
Importantly this relationship seems to be conditional on an
institutional environment that is predisposed to such an analysis.
Furthermore, it is important to note that the policy outcomes suggested
by environmental justice analysis depend on how minority populations are
defined. In some cases, those raising environmental justice concerns
present it as an issue of wealth or income inequality, leading policy to
account for disparate impacts on low-income populations. In other cases,
groups raise claims rooted in cultural practices, such as fish
consumption among certain tribes. As occurred in the Mercury Rule, the
analysis in subsequent drafts of the policy used evaluative criteria
specific to these communities.

The ability of a frame like environmental justice to construct certain
populations as deserving of consideration means that policy outcomes
will depend on the specific environmental justice concerns raised. In
this respect, second-order representation may become important. National
advocacy organizations may frequently request that regulators protect
``all people'' or even ``low-income communities of color.'' However, this
more generic advocacy may not lead to the same outcomes as groups that
present specific local environmental justice grievances that are unique
to a community. In between generic progressive advocacy organizations
and community-based organizations are organizations like Earthjustice
who, despite their national focus, frequently engaged in
community-specific litigation and thus raise these local concerns in
national policymaking.

In the end, the above analysis offers some clarity on a poorly
understood but important mechanism of American policymaking. It offers
some hope that citizen opinions may be heard through direct democracy
institutions built into bureaucratic policymaking. The examination of
second-order participation validates some of the skepticism about who
participates. It is elite groups who participate, even with respect to
an issue like environmental justice. However, government responsiveness
does not seem to depend on mass mobilization or elite support. Compared
to cases where environmental justice issues were not raised, when they
are, we see a significantly higher probability that they will be
addressed by policymakers.
% --- PAGE: endnotes -----------------------
% --- PAGE: refs -----------------------
\newpage
\singlespacing 
          \bibliography{/Users/devin/dissertation/assets/dissertation.bib} 
   

\end{document}
