\documentclass[
      12pt,
        ]{article}






% --- type and typeface? -----------------------

% input
\usepackage[utf8]{inputenc}

% typography
\usepackage{microtype}


\usepackage[T1]{fontenc}


% text block
\usepackage{setspace}
\usepackage[ 
              left = 1in,top = 1in,right = 1in,bottom = 1in 
            ]{geometry}

\usepackage{enumitem}
  \setlist{noitemsep}



% decimal numbering for appendix figs and tabs


% Deletes section counters
% \setcounter{secnumdepth}{0}







  \usepackage{longtable, booktabs}









  \usepackage{natbib}
  \bibliographystyle{/Users/devin/dissertation/assets/apsr.bst}
  % protect underscores in most circumstances
  \usepackage[strings]{underscore} 


% 

% \newtheorem{hypothesis}{Hypothesis}

\makeatletter
  \@ifpackageloaded{hyperref}{}{%
    \ifxetex
      % page size defined by xetex
      % unicode breaks when used with xetex
      \PassOptionsToPackage{hyphens}{url}\usepackage[setpagesize = false, 
                                                     unicode = false, 
                                                     xetex]{hyperref}
    \else
      \PassOptionsToPackage{hyphens}{url}\usepackage[unicode = true]{hyperref}
    \fi
  }

  \@ifpackageloaded{color}{
    \PassOptionsToPackage{usenames,dvipsnames}{color}
  }{
    \usepackage[usenames,dvipsnames]{color}
  }
\makeatother

\hypersetup{breaklinks = true,
            bookmarks = true,
            pdfauthor = {Devin Judge-Lord ()},
             pdfkeywords  =  {},  
            pdftitle = {The Environmental Justice Movement's Impact on Technocratic Policymaking},
            colorlinks = true,
            citecolor = black,
            urlcolor = blue,
            linkcolor = magenta,
            pdfborder = {0 0 0}}

% \urlstyle{same}  % don't use monospace font for urls


% set default figure placement to htbp
\makeatletter
  \def\fps@figure{hbtp}
\makeatother


% optional footnotes as endnotes


% ----- Pandoc wants this tightlist command ----------
\providecommand{\tightlist}{
  \setlength{\itemsep}{0pt}
  \setlength{\parskip}{0pt}
}





% --- title & section styles -----------------------


% title, author, date
  \title{The Environmental Justice Movement's Impact on Technocratic Policymaking}
 

  \author{ % author, option footnote, optional affiliation
            Devin Judge-Lord\footnote{University of Wisconsin-Madison, \href{mailto:judgelord@wisc.edu}{\nolinkurl{judgelord@wisc.edu}}. Slides and the most recent draft are available at \url{https://judgelord.github.io/research/ej}. Additional tables, figures, and replication code are available at \url{https://judgelord.github.io/dissertation/ej-appendix.html}} 
            }

% auto-format date?
  \date{2021-02-13}


% abstract
\usepackage{abstract}
  \renewcommand{\abstractname}{}    % clear the title
  \renewcommand{\absnamepos}{empty} % originally center

  \newcommand*{\authorfont}{\sffamily\selectfont}


% section titles
\usepackage[small, bf, sc]{titlesec}
  % \titleformat*{\subsection}{\itshape}
  \titleformat*{\subsubsection}{\itshape} 
  \titleformat*{\paragraph}{\itshape} 
  \titleformat*{\subparagraph}{\itshape}



%\usepackage{float}
%\floatstyle{plaintop}
%\restylefloat{table}
\usepackage{floatrow}
\floatsetup[figure]{capposition=top}
\floatsetup[table]{capposition=top}
\usepackage{multirow}
\usepackage{rotating} 
\usepackage{caption}






















\begin{document}
 

% --- PAGE: title and abstract -----------------------

  \maketitle

% \pagenumbering{gobble}
  \pagenumbering{gobble}



  \begin{abstract}
    \noindent THIS DRAFT WAS PREPARED FOR THE AMERICAN POLITICS WORKSHOP

THE MOST RECENT DRAFT IS \href{https://judgelord.github.io/research/ej/}{HERE}

\bigskip

I explore the role of public comments in rulemaking by focusing on their role in the environmental justice movement. Environmental justice concerns focus on unequal access to healthy environments and protection from harms caused by things like pollution and climate change. The ways in which agencies consider environmental justice highlights how rulemaking has distributive consequences, how the public comment process creates a political community, and how claims raised by activists are addressed. Examining thousands of rulemaking processes at agencies known to address environmental justice concerns, I find that when public comments raise environmental justice concerns, these concerns are more likely to be addressed in the final rule. However, baseline rates of addressing environmental justice in rulemaking are so low that even as the probability that agencies will address environmental justice significantly increases when commenters raise these issues, in most rules, even those where commenters raise environmental justice concerns, there is no explicit attention to environmental justice. Furthermore, even when agencies do address environmental justice concerns, they often do not make the substantive policy changes that activists demand. While the number of comments raising environmental justice concerns is positively correlated with change in policy texts, the effect of the general level of public attention is mixed. Rules with more comments are more likely to address environmental justice when they did not address it in the draft rule, but rules with more comments are less likely to change how they addressed environmental justice if they did address it in the draft rule. These results suggest that the politics of rulemaking differs when there is more public attention. Patterns also vary across agencies, possibly due to the alignment of environmental justice aims with agency missions. 

    

  \end{abstract}



% --- PAGE: contents -----------------------





% --- PAGE: body -----------------------


  \newpage
  \pagenumbering{arabic}

\noindent 
 
   
    \onehalfspacing 
   
   
\thispagestyle{empty}

\singlespacing

\centerline{\textbf{\underline{NOTE TO READER}}}

The following chapter is part of a dissertation exploring the effects of public pressure on agency rulemaking, a technocratic policy process where ``public participation'' is usually limited to sophisticated lobbying but occasionally includes millions of people mobilized by public pressure campaigns. Public comment periods on proposed policies purport to provide democratic accountability. Yet theories of bureaucratic policymaking largely ignore the occasional bursts of civic engagement that generate the vast majority of public comments on proposed rules. To fill this gap, I build and test theories about the role of public pressure in policymaking. I collect and analyze millions of public comments to develop the first systematic measures of civic engagement and influence in bureaucratic policymaking. The outline of the dissertation is as follows:

\textbf{Chapter 1 ``Agency Rulemaking in American Politics''} situates agency rulemaking in the context of American politics. Tracing broad trends over the past 40 years, I show that rulemaking has become a major site of policymaking and political conflict.

\textbf{Chapter 2 ``Why Do Agencies (Sometimes) Get So Much Mail?''} addresses who participates in public pressure campaigns and why. Are public pressure campaigns, like other lobbying tactics, primarily used by well-resourced groups to create an ``astroturf'' impression of public support? Or are they better understood as conflict expansion tactics used by less-resourced ``grassroots'' groups? I find that mass comment campaigns are almost always a conflict expansion tactic. Furthermore, I find no evidence of negativity bias in public comments. Indeed, from 2005 to 2017, most comments supported proposed rules. This is because public comments tend to support Democratic policies and oppose Republican policies, reflecting the asymmetry in mobilizing groups.

\textbf{Chapter 3 ``Do Public Pressure Campaigns Influence Congressional Oversight?''} examines the effect of public pressure campaigns on whether legislators are more likely to engage in rulemaking. This involves collecting and coding thousands of comments from Members of Congress on proposed rules with and without public pressure campaigns. These data also allow me to assess congressional oversight as a mediator in policy influence, i.e., the extent to which public pressure campaigns affect policy indirectly through their effects on legislators' oversight behaviors.

\textbf{Chapter 4 ``Do Public Pressure Campaigns Influence Policy?''} leverages a mix of hand-coding and computational text analysis methods to assess whether public pressure campaigns increase lobbying success. To measure lobbying success, I develop computational methods to identify lobbying coalitions and estimate their effect on each rule posted for comment on regulations.gov. I then validate these methods against a random sample of 100 rules with a mass-comment campaign and 100 rules without a mass-comment campaign, hand-coded for whether each coalition got the policy outcome they sought. Finally, I assess potential mechanisms by which mass public engagement may affect policy.

\textbf{Chapter 5 ``The Environmental Justice Movement and Technocratic Policymaking''} examines the discursive effects of environmental justice claims both qualitatively and quantitatively. I write about the role of Native activists and environmental groups in shaping federal environmental regulations. Looking across over 20,000 draft regulations that failed to address environmental justice issues, I find that agencies are more likely to add language addressing environmental justice in their final rules when public comments raise environmental justice concerns.

\newpage

\onehalfspacing

\setcounter{page}{1}

\hypertarget{introduction}{%
\section{Introduction}\label{introduction}}

When do activist movements change public policy? Over the last year, few topics have dominated politics more than political protest. Yet, we lack systematic evidence about the impact of social movements on modern policymaking. I examine how social movements affect policymaking by assessing the environmental justice movement's impact on policymaking in 40 U.S. federal agencies from 1992 to 2020. Environmental justice
concerns focus on unequal access to healthy environments and
protection from harms caused by things like pollution and climate
change. How government agencies consider (or fail to consider) environmental justice in policy documents allows empirical tests of theories about when institutions will address claims raised by activists.

The environmental justice movement's successes and failures in rulemaking illustrate how activists attempt to
inject ideas directly into the policymaking process. I focus on the
environmental justice movement because it offers a broad but tractable
scope for analysis and illuminates what is at stake in the politics of
rulemaking. Executive-branch policymaking has distributive consequences. How policy documents address distributive issues highlights policy processes construct communities of ``relevant'' stakeholders and ``appropriate'' criteria to evaluate policy consequences.
Mobilizing for environmental justice is an example of how social
movements mobilize norms and evaluative frameworks that are
connected to organizational identities, mission, and reputations and
thus have implications for policy decisions \citep{Carpenter2001}.

Tracing ideas like environmental justice through policy processes like rulemaking offers one way to study the mechanisms by which social movements succeed or fail to influence policy. If draft rules
do not mention EJ concerns, but
EJ concerns are raised in the public comments and
then appear in the final policy, this may be evidence that public pressure mattered. Likewise, when proposed rules do include an EJ analysis, if groups comment on it and the agency changes its discussion of EJ in the final rule, this may be evidence that public pressure mattered.

Tracing the evolution of environmental justice (EJ) analyses through several of these rulemaking processes shows that the concept is hotly contested and rarely addressed by agencies in ways that activists find acceptable. Activist pressure affected how rules address EJ in some cases but failed to affect others.

Examining all rules published by 40 agencies to regulations.gov between 1992 and 2020, I find that Activist mobilization affected policy discourse even under administrations that were explicitly hostile to their cause. When public comments raise
EJ concerns, these concerns are more likely to be
addressed in policy documents. Specifically, the number of comments mobilized (both overall and by EJ advocates specifically) is related to agencies adding language addressing EJ to final rules where the draft did not. In contrast, change in draft rules that did address EJ is inversely related to the overall level of public attention while still positively correlated with comments raising EJ concerns.
While we cannot infer that public comments cause agencies to address EJ concerns, comments may be a good proxy for activist pressure in general. Furthermore, the correlation between EJ activist mobilization and policy changes is largest and most significant in agencies with missions focused on ``environmental'' and distributional policy---the kinds of policymakers who we may expect to have
institutional and cognitive processes primed to be most responsive to
EJ concerns.

\hypertarget{environmental-justice-as-a-boundary-drawing-tool}{%
\subsection{``Environmental Justice'' as a Boundary-drawing Tool}\label{environmental-justice-as-a-boundary-drawing-tool}}

The politics of environmental justice has several convenient properties for studying the policy impact of social movements. First, discourse around policies framed
as ``environmental'' issues are, unlike issues like civil
rights and immigration, inconsistently racialized and, unlike issues
like taxes and spending, inconsistently focused on \emph{distributions} of
costs and benefits. This means that policies may or may not be framed in environmental justice terms. Despite policy almost always having disparate impacts, an ``environmental'' frame often creates a human-environment distinction and
shifts attention to non-human objects such as air, water, food, or
landscapes and away from the distribution of access to these objects or
protection from them when they are contaminated. By focusing on distributions of costs and benefits, fights over EJ analyses differ from more traditional utilitarian or preservationist analyses.

Second, compared to
other ideas around which people mobilize, ``environmental justice'' is a
fairly distinctive phrase. Most people who use this phrase share a
general definitional foundation. Even attempts to reframe the term (e.g., to focus on class rather than race or jobs rather than health) come about as dialectical moves related to the term's historical uses. Thus, when ``environmental justice'' appears in a text, it is rarely a coincidence of words; its appearance is a result of the movement or reactions to it.

Third, this phrase appears frequently
when the idea is discussed, i.e., there are few synonyms. Groups raising equity concerns on ``environmental'' issues commonly use the phrase ``environmental justice.'' Those who use narrower, related terms--including the older concept of
``environmental racism'' and the newer concept of ``climate justice''--almost always use ``environmental justice'' in their advocacy as well.

Finally, the term is relevant to rulemaking records in
particular because Executive Order 12898 issued in 1994 by President
Clinton ``Federal Actions to Address Environmental Justice in Minority
Populations and Low-Income Populations'' requires all agency
actions and policies to consider EJ implications. Executive Orders from Presidents Obama and Biden and statements from agency heads in every administration have since interpreted and reinterpreted parts of this Order, all with direct implications for rulemaking.
This does not mean that all draft or final rules address EJ, but when they
do, they tend to cite Executive Order 12898 and explicitly discuss
environmental justice. For the same reason, commenters who critique draft rules also cite this
Executive Order and use this language.

Advocates may even sue agencies for failing to satisfy the procedural requirements of E.O. 12898, giving agencies an incentive to explain how their policies and justifications address the Executive Order and the EJ concerns of commenters.
Courts may strike down rules for failing to comply with procedural requirements of the Administrative Procedures Act (APA) if the agency fails to ``examine the relevant data'' or ``consider an important aspect of the problem'' (\emph{Motor Vehicle Mfrs. Ass'n v. State Farm Mut. Auto. Ins. Co.}, 1983). This can include an agency's 12898 EJ analysis: ``environmental justice analysis can be reviewed under NEPA and the APA'' (\emph{Communities Against Runway Expansion, Inc.~v. FAA}, 2004).

\hypertarget{theory}{%
\section{Theory}\label{theory}}

Participatory processes like public comment periods, where government
agencies must solicit public input on draft policies, are said to provide democratic legitimacy \citep{Croley2003, Rosenbloom2003}, new technical information \citep{Yackee2006JPART, Nelson2012}, and political oversight opportunities \citep{Balla1998, Mccubbins1984}. While recent scholarship on agency policymaking has shed light on sophisticated lobbying by businesses, we know surprisingly little about the vast majority of public comments, which are submitted as part of public pressure campaigns.\footnote{
  As shown in \citet{judgelord2019SPSA}, most comments submitted to
  regulations.gov are part of organized campaigns, more akin to petition signatures than ``deliberative'' participation or sophisticated lobbying. Indeed, approximately 40 million out of
  50 million (80\%) of these public comments on rulemaking dockets between 2004 and 2019 were mobilized by just 100
  advocacy organizations.}
Activists frequently target agency policymaking with letter-writing campaigns, petitions, protests,
and mobilizing people to attend hearings, all classic examples of ``civic engagement'' \citep{Verba1987}. Yet civic engagement remains poorly understood in the context of bureaucratic policymaking.
While practitioners and administrative law scholars have long pondered
what to make of activists' mass comment campaigns, political scientists have had
surprisingly little to say about this kind of civic participation.

\hypertarget{technical-information-as-the-currency-of-lobbying}{%
\subsection{Technical Information as the Currency of Lobbying}\label{technical-information-as-the-currency-of-lobbying}}

Dominant theories of bureaucratic policymaking focus on how agencies learn about policy problems \citep{Kerwin2011}. Leading formal models are information-based models where comments matter by revealing information to the agency \citep{Gailmard2017, Libgober2018}, and empirical studies support the conclusion that information is the currency of lobbying in rulemaking \citep{Yackee2012, Cook2017, Gordon2018, Walters2019}.

Rulemaking is a highly technocratic and legalistic form of policymaking that explicitly privileges scientific and legal facts as the appropriate basis for decisions. Procedural requirements to consider relevant information create incentives for lobbying groups to overwhelm agencies with complex technical information, making rulemaking obscure to all but the most well-informed insiders \citep{Wagner2010}.
As \citet{Yackee2019} notes:

\begin{quote}
``to be influential during rulemaking,
commenters may require resources and technical expertise.
As Epstein, Heidt, and Farina (2014) suggest, agency rule-writers--who are often chosen because
of their technical or policy-specific expertise--privilege the type of data-driven
arguments and reasoning that are not common to citizen comments.'' (p.~10)
\end{quote}

The result is that rulemaking is dominated by sophisticated and well-resourced interest groups capable of providing new technical or legal information. Empirical scholarship finds that economic elites and business groups
dominate American politics in general \citetext{\citealp{Jacobs2005}; \citealp{Soss2007}; \citealp[Hertel-Fernandez2019;][]{Hacker2003}; \citealp{Gilens2014}} and rulemaking in
particular. While some are optimistic that requirements for agencies to
solicit and respond to public comments on proposed rules allow ``civil
society'' to provide public oversight \citep{Michaels2015, Metzger2010}, most
studies find that participants in rulemaking often represent elites and
business interests \citep{Seifter2016UCLA, Crow2015, Wagner2011, West2009, Yackee2006JOP, Yackee2006JPART, Golden1998, Haeder2015, Cook2017, LibgoberCarpenter2018}. To the extent that scholars address public pressure campaigns, both
existing theory and empirical scholarship suggest skepticism that it
matters \citep{Balla2018}.

\hypertarget{political-information}{%
\subsection{Political Information}\label{political-information}}

\citet{Nelson2012}
identify political information as a potentially influential result of
lobbying by different business coalitions. While they focus on
mobilizing experts, \citet{Nelson2012} describe a dynamic that can be extended
to mobilizing public pressure:

\begin{quote}
``strategic recruitment, we theorize, mobilizes new actors to
participate in the policymaking process, bringing with them novel
technical and political information. In other words, when an expanded
strategy is employed, leaders activate individuals and organizations
to participate in the policymaking process who, without the
coordinating efforts of the leaders, would otherwise not lobby. This
activation is important because it implies that coalition lobbying can
generate new information and new actors---beyond simply the `usual
suspects' ---relevant to policy decisionmakers.''
\end{quote}

I argue that, concerning political information, this logic extends to
non-experts in at least two ways.

\textbf{1. Information about a policy's disparate effects:}
First, while specific \emph{data} on disparate impacts of policy may require expertise, anyone can highlight a community of concern and potential distributional effects of a policy. Just as \citet{Nelson2012} found regarding the mobilizing of diverse experts, mobilizing diverse communities affected by a policy may introduce new claims from new actors about how the communities benefited or harmed by a policy should be constructed. Instead of bolstering
\emph{scientific} claims, such comments bolster \emph{political}
claims about who counts and even \emph{who exists} as a distinct, potentially affected group that deserves policymakers' attention. While bureaucratic policymaking in the United States is dominated by cost-benefit analysis that must abstract away from the distribution of costs and benefits, agencies have many reasons to consider the distributional effects of policy and often do.

\textbf{2. Public pressure as a political resource: }
Second, The number of supporters may
matter because it indicates support among relevant communities or the broader public. Again, instead of bolstering
\emph{scientific} claims, perceived public support bolsters \emph{political}
claims.

Like other forms of political participation, such as protests and letter-writing campaigns,
public pressure campaigns provide no new technical information.
Nor do they wield any formal authority to reward or sanction bureaucrats.
The number on each side, be it ten or ten million, has no legal import for an agency's response.

However, an organization's ability to expand the scope of conflict by mobilizing
a large number of people can be a valuable political resource \citep{Schattschneider1975}. \citet{Furlong1997} and \citet{Kerwin2011}
identify mobilization as a tactic. The organizations they surveyed
believed that forming coalitions and mobilizing large numbers of people
were among the most effective lobbying tactics. While \citet{Furlong1997} and \citet{Kerwin2011} focused on how
organizations mobilize their members, I expand on this understanding of mobilization as a lobbying tactic to include a campaign's broader audience, more akin to the concept of
an attentive public \citep{Key1961} or issue public \citep{Converse1964}.

Regardless of the specific claims of commenters, expanding the scope of conflict by mobilizing public attention to rulemaking may shift policymakers' attention away from the technical information provided by the ``usual suspects'' and toward the distributive effects of policy.

\hypertarget{hypotheses}{%
\subsection{Hypotheses}\label{hypotheses}}

The existing literature on bureaucratic policymaking in general---and EJ advocacy in particular---presents competing intuitions about the effect of EJ activists and the broader public in rulemaking. Below, I posit hypotheses in the direction that these advocacy groups do affect rulemaking while also noting equally plausible intuitions for the opposite conclusions. Because of the general skepticism and empirical work that has found that advocacy groups and public pressure campaigns have little to no effect on rulemaking, I set the empirical bar low: do EJ advocates and public pressure campaigns have \emph{any} effect at all on policy documents.

\hypertarget{information}{%
\subsubsection{Information}\label{information}}

\begin{quote}
\emph{Distributive Information Hypothesis}: Policymakers are more likely to change whether or how policies address distributional justice when commenters raise distributional justice concerns.
\end{quote}

As discussed above, agency policymakers have incentives to address distributive concerns, especially environmental justice, due to judicial review of compliance with the Administrative Procedures Act and E.O. 12898. By raising EJ concerns, commenters draw attention to the distribution of policy impacts--who a policy may affect. Asserting definitions and categories of stakeholders and affected groups is one type of policy-relevant information.

\begin{quote}
\emph{Repeated Information Hypothesis}: Policymakers are more likely to change whether or how policies address concerns when more commenters raise them.
\end{quote}

Scholarship on lobbying in rulemaking emphasizes the value of repeated information and coalition size \citep{Nelson2012}. This implies that the more unique comments raising EJ concerns, the more likely the coalition will influence policy.

Competing intuitions and other prior studies oppose both the \emph{Distributive and Repeated Information Hypotheses}. Scholarship on lobbying in rulemaking emphasizes the importance of novel science and technical information--things unknown to agency experts \citep{Wagner2010}. Furthermore, scholarship finds business commenters are influential, and public interest groups are not \citep[\citet{Haeder2015}]{Yackee2006JOP}. Furthermore, policymakers may be more likely to anticipate EJ concerns when they are more salient to interest groups and the public. This would mean that rules where commenters raise EJ concerns may be the \emph{least} likely to change whether or how EJ is addressed because policymakers are more likely to have considered the issues and stated their final position in the draft rule.

\begin{quote}
\emph{Policy Receptivity Hypothesis}: Policymakers that more frequently address concerns like environmental justice will be more responsive to commenters raising those concerns.
\end{quote}

Bureaucracies are specialized institutions built to make and implement certain kinds of policies based on certain goals and facts. Each agency's distinct norms and epistemic community determine whether policymakers see issues as ``environmental'' and whether they have disparate impacts that demand consideration of distributive ``justice.'' Some policymakers may see their policy area as more related to environmental justice than others and thus be more receptive to commenter concerns.

The competing intuition to the \emph{Policy Receptivity Hypothesis} is that policymakers familiar with EJ concerns are the \emph{least} likely to respond to EJ concerns because they anticipate these concerns--they are not novel. If so, agencies that rarely consider EJ may be more easily influenced by commenters who present somewhat novel information and concerns. These agencies may be less likely to have preempted critiques.

\hypertarget{public-pressure}{%
\subsubsection{Public Pressure}\label{public-pressure}}

\begin{quote}
\emph{General Pressure Hypothesis}: Policies are more likely to change when they receive more public attention (e.g., more public comments).
\end{quote}

The competing intuition against the \emph{General Pressure Hypothesis} is again that large numbers of comments indicate policy processes that were already salient before the public pressure campaign. Policymakers anticipate public scrutiny and are thus more likely to have stated their final position in the draft rule.

\begin{quote}
\emph{Specific Pressure Hypothesis}: Policies are more likely to address an issue when they receive more public attention (e.g., more public comments) \emph{and} some comments raise that issue.
\end{quote}

This hypothesis asserts that the overall level of public attention will condition policy responses to specific issues--it is the interaction between the number of total public comments and at least some of those comments raising EJ concerns that makes policy more likely to address EJ.

The competing intuition against the \emph{Specific Pressure Hypothesis} is again that large numbers of comments indicate high-salience rulemakings where the agency anticipates public scrutiny, including how it did or did not address specific issues like environmental justice. Policymakers anticipate public scrutiny and are thus more likely to have stated their final position in the draft rule.

\hypertarget{data}{%
\section{Data}\label{data}}

In order to examine whether EJ advocates or public pressure campaigns shape the discourse around policies,
I use the text of draft rules, public comments, and final rules retrieved from regulations.gov. I select rulemaking documents from agencies that published at least one rule explicitly addressing EJ from 1992 to 2020. This yields over 25,000 rulemaking dockets from 40 agencies.

Despite E.O. 12898, most rules do not address EJ. Figure \ref{fig:ej-data} shows that most draft and final rules (about 90\%) do not address EJ. Interestingly, the total number of final rules and the percent of the total addressing EJ have remained fairly stable for the time period where regulations.gov data are complete (after 2005). Every year from 2006 to 2020, these agencies published between 2000 and 3000 final rules, of which between 200 and 300 addressed EJ.

\begin{figure}

{\centering \includegraphics[width=0.99\linewidth]{/Users/devin/dissertation/Figs/ej-data-ejpr-1} \includegraphics[width=0.99\linewidth]{/Users/devin/dissertation/Figs/ej-data-ejfr-1} 

}

\caption{Number of Proposed and Final Rules Addressing Environmental Justice.}\label{fig:ej-data}
\end{figure}

Even at the Environmental Protection Agency (EPA), where most policies are clearly framed as ``environmental'' issues, a consistent minority of rules address EJ. Many agencies that almost never address EJ make policy with clear EJ effects, including Housing and Urban Development (HUD), the Nuclear Regulatory Commission (NRC), and the Office of Surface Mining (OSM). A majority of rules addressed EJ only in a few years at a few agencies that publish relatively few rules, such as the Council on Environmental Quality (CEQ), Federal Emergency Management Agency (FEMA), and several Department of Transportation (DOT) agencies (the Federal High Way Administration, Federal Railroad Administration (FRA)). Figure \ref{fig:ej-data-agencies100} shows the number of rulemaking projects at each agency over time by whether they ultimately addressed EJ.

\begin{figure}

{\centering \includegraphics[width=0.99\linewidth]{/Users/devin/dissertation/Figs/ej-data-agencies100-1} 

}

\caption{Number of Proposed and Final Rules Addressing Environmental Justice at the Environmental Protection Agency (EPA), Department of Housing and Urban Development (HUD), the Nuclear Regulatory Commission (NRC), Office of Surface Mining (OSM), Council on Environmental Quality (CEQ), Federal Emergency Management Agency (FEMA), Federal High Way Administration, Federal Railroad Administration (FRA), National Highway Transportation Saftey Administration (NHTSA), and Fish and Wildlife Service (FWS)}\label{fig:ej-data-agencies100}
\end{figure}

\hypertarget{comments}{%
\subsection{Comments}\label{comments}}

Figure \ref{fig:ej-comments} shows the number of comments on each proposed rule published between 1992 and 2020. Red circles indicate rules where no commenters raised EJ concerns. Blue Triangles indicate rules where they did. The bottom row of plots shows the subset of rules where ``environmental justice'' appeared in neither the draft nor the final rule. The middle row of plots show rules where ``environmental justice'' appeared in the final but not the draft. My first analysis compares these two rows. The top row of plots shows rules where ``environmental justice'' appeared in both the draft and final rule. My second analysis compares rules in this first row. Predictably, commenters most often raised EJ concerns on rules in the first row.

\begin{figure}

{\centering \includegraphics[width=1\linewidth]{/Users/devin/dissertation/Figs/ej-comments-1} 

}

\caption{Number of Comments on Proposed and Final Rules and Whether Comments Raised Environmental Justice Concerns}\label{fig:ej-comments}
\end{figure}

\hypertarget{interest-groups-and-second-order-representation}{%
\subsection{Interest Groups and Second-order Representation}\label{interest-groups-and-second-order-representation}}

When lobbying during rulemaking, groups often
make dubious claims to represent broad segments of the public \citep{Seifter2016UCLA}. It is thus insufficient to know which groups participate. We
also need to know who these groups claim to represent and whether those people are actually involved in the decisions of the organization.

I investigate who is raising EJ concerns in two ways.
First, I identify the top organizational commenters such as tribes,
businesses, and nonprofits that are using EJ language
and investigate whom these groups represent. Second, for comments where a
citizen signed their name, I compare surnames to their racial and ethnic
identity propensities with respect to the U.S. Census. Together these
two pieces of information allow me to comment on ``second-order'' representation, i.e., the extent to which public comments are
representative of the groups they claim to represent \citep{Seifter2016UCLA}.

\hypertarget{organizations-raising-ej-concerns-on-the-most-rulemaking-dockets}{%
\subsubsection{Organizations Raising EJ Concerns on the Most Rulemaking Dockets}\label{organizations-raising-ej-concerns-on-the-most-rulemaking-dockets}}

The top mobilizer of comments mentioning ``environmental justice'' between 1992 and 2020 was the Sierra Club, with over 340,000 comments on dozens of rules. While it is a membership organization whose members pay dues, elect the leaders of local chapters, and have some say in local advocacy efforts, its policy work is directed by a more traditional national advocacy organization funded by donations, including over \$174 million from Bloomberg Philanthropies that funded several of the public pressure campaigns in these data. Like other national advocacy groups, the Sierra Club advocates on behalf of ``EJ and Frontline'' communities, but those individuals have little formal say in the national organization's lobbying decisions. The Sierra Club does have a major program arm dedicated to Environmental Justice and many local programs. As a federated organization with many local efforts, it is difficult to generalize about second-order representation, which likely varies across its campaigns. The National Board of Directors adopted a statement on social justice in 1993 and principles on environmental justice in 2001. The national website does contain regular Spanish language content.

The second most prolific organizer of EJ comments was Earthjustice, with over 175,000 comments on many of the same rules that the Sierra Club lobbied on. Earthjustice is primarily engaged in litigation on
behalf of environmental causes. Their website boasts 2.2 million
supporters, but it is not clear who they are or if they play any role in
the advocacy strategy. A search on the website returns 360 results for
"Environmental Justice," with the top results from staff biographies
who work on more local or targeted campaigns, such as environmental conditions
for the incarcerated, but the EJ language used on the
main page is relatively vague. For example, ``We are fighting for a future
where children can breathe clean air, no matter where they live.''
\citep{Earthjustice2017}. The website does contain Spanish language content.

The Natural Resources Defense Council is similar to Earthjustice--a
national nonprofit funded by donations and focused on litigation--but
they also lobby and organize public pressure campaigns, including over 160,000 comments mentioning environmental justice.

CREDO Action and MoveOn are more generic progressive
mobilizers who lack a systematic focus on EJ issues,
but occasionally leverage their very large membership lists to support
EJ campaigns led by others
\citep{MoveOn.org2017, CREDO2017}.

The Alliance for Climate Protection is more of an elite political group founded by former Vice President Al
Gore.

We Act and Communities for a Better Environment both have environmental
justice in their central mission statement. We Act was founded by
community leaders in Harlem, NY, to fight environmental racism and
advocate for better air quality \citep{WEACT2017}. Communities for a Better Environment has projects throughout California but is particularly
active in Oakland \citep{CBECAL2017}. Much of the
content of their website is in both English and Spanish. Both
organizations focus primarily on ``low-income communities of color'' and
thus frame their work primarily in terms of race and class. While both
organizations participated in national policymaking We Act is more
focused on communities in Harlem and New York, whereas Communities for a
Better Environment casts a wider frame: "CBE's vision of environmental
justice is global--that's why the organization continues to participate
in such international efforts as the Indigenous Environmental Network
and the Global Week of Action for Climate Justice" \citep{CBECAL2017}.

While not a large portion of EJ comments, companies repeatedly raise research about the unequal impacts of policy in order to frame these issues as a legitimate but unresolved scientific debate that is not yet conclusive enough to base regulations
on. For example, in one comment, the Southern Company wrote:

\begin{quote}
``People with lower SES are exposed to almost an order of magnitude
more traffic near their homes (Reynolds et al., 2001), and live closer
to large industrial sites and are exposed to more industrial air
pollution (Jerrett et al., 2001). Legitimate health concerns must be
addressed. But adopting standards with a scientific basis so uncertain
that health improvement cannot be assured is not sound public health
policy.''
\end{quote}

Like many companies, they claim to represent their customers:
"electric generating companies and their customers are expected to bear
much of the burden" of regulations \citep{Hobson2004}.

With respect to second-order representation, it appears that the groups
most often using the language of environmental justice may do so
sincerely but do not themselves represent affected communities. Several
groups representing local communities and led by community leaders have
participated, but not nearly as often or with the same intensity as the
``big greens.'' This highlights the importance of resources as a condition
for mobilizing. Not all groups who may benefit from political
information are able to leverage it because they lack the resources to
invest in a campaign. However, it may be the case that smaller, more
member-driven groups join coalitions with groups with more resources who
mobilize on their behalf.
Finally, a third, much less common type of commenter raises EJ issues
as a way to re-frame them as ongoing debates and thus undermine their
urgency. I call this reason for engaging as ``breaking a perceived
consensus.'' In a way, the fact that an energy company felt compelled to
acknowledge and question EJ concerns suggests their
importance for policy outcomes.

\hypertarget{commenter-race}{%
\subsubsection{Commenter Race}\label{commenter-race}}

To estimate the racial distribution of those who comment
using EJ language, I select commenters who
signed their full name on their comment with a
surname appearing in census records. Figure
\ref{fig:ejcommentsbyrace} shows a probabilistic racial
distribution of commenters who raise ``environmental justice'' concerns in
their comments based on the distribution of self-reported racial identities
associated with surnames as recorded in the 2010 census\footnote{I recode ``Hispanic'' as ``Latinx''}. This distribution is estimated using the proportion of
people with a given surname identified as belonging to each racial
category (from this limited set of options). It does not
assign specific individuals to racial categories, but instead represents each commenter as a set of probabilities adding up to 1 and the estimated racial distribution of the sample as a sum of individual probabilities.

\begin{figure}

{\centering \includegraphics[width=0.49\linewidth]{/Users/devin/dissertation/Figs/race-prob} 

}

\caption{Estimated Racial Distribution from Census Surnames of Commenters raising "Environmental Justice" Concerns in Rulemaking}\label{fig:ejcommentsbyrace}
\end{figure}

Compared to the overall distribution in the 2010 census, this sample of commenters
appears to be disproportionately Black and less than proportionately
Latinx or Asian, with just slightly fewer Whites relative to the
national population. This makes sense given that environmental justice
theorizing and activism have been led by African Americans
\citep{Bullard1993}.

\hypertarget{tracing-ideas-through-rulemaking}{%
\section{Tracing Ideas Through Rulemaking}\label{tracing-ideas-through-rulemaking}}

\hypertarget{environmental-justice-as-a-contested-concept}{%
\subsection{``Environmental Justice'' as a Contested Concept}\label{environmental-justice-as-a-contested-concept}}

The use of an environmental justice frame does not always imply the same
communities of concern. Environmental justice emerged out of movements
against environmental racism, especially the disposal of toxic
substances in predominantly-Black neighborhoods \citep{Bullard1993}. However, the term
quickly took on a wider array of meanings, encompassing various marginalized groups. President Clinton's 1994 Executive Order on
Environmental Justice required all parts of the federal government
to make ``addressing disproportionately high and adverse human health or
environmental effects of programs, policies, and activities on minority
populations and low-income populations'' a core aspect of their mission.
This meant considering disproportionate effects during rulemaking.

In 2005, Environmental Protection Agency (EPA) political appointees reinterpreted the Order, removing race as a factor in identifying and prioritizing populations. This move was criticized by activists and two reports by EPA's own Office of Inspector General. President Obama's EPA Administrator named EJ as one of their top priorities but also faced criticism from activists for only paying lip service to environmental racism.

In an October 2017 proposed rule to repeal
restrictions on power plant pollution, the Trump EPA acknowledged that
``low-income and minority communities located in proximity to {[}power
plants{]} may have experienced an improvement in air quality as a result
of the emissions reductions.'' Because the Executive Order requires
attention to environmental justice and because the Obama EPA discussed
it when promulgating the rule, the environmental justice implications could not safely
be ignored. However, the Trump EPA contended that the Obama EPA ``did not
address lower household energy bills for low-income households {[}and
that{]} workers losing jobs in regions or occupations with weak labor
markets would have been most vulnerable'' (EPA 2017). Like the comments of the Southern Company and other regulated industry commenters, these statements frame the distribution of jobs and electricity costs as EJ issues in order to push back against policies that would equalize distributions of health impacts of pollution.

The major conflict over the role of race in EJ analyses is one of many conflicts that the environmental justice movement has caused to be fought somewhat on its terms. To illustrate how these definitional conflicts shape rules and rulemaking, the next section briefly reviews the decades-long policy fight over regulating Mercury pollution.\footnote{This case and other examples in this article were not selected as the most similar, most different, or a representative sample. They emerged from reading hundreds of rulemaking documents where agencies did and did not respond to comments raising EJ concerns. Their purpose is to assess whether the cases in the quantitative analysis are plausibly what they appear to be: that changes in rule text are, sometimes, causally related to public comments and that non-changes are cases of agencies disregarding comments, not some accident of the data or measures. Tracing a few rulemaking processes also helped to avoid analytic pitfalls. For example, one case where an agency did an EJ analysis and then appeared not to respond to a comment discussing EJ was, in fact, due to the fact that the commenter included an annotated version of the draft rule their comment, adding only ``no comment'' next to the 12898 section. To correct this, I removed text copied from the proposed rule from comments in pre-processing. The qualitative reading also confirmed other key assumptions, such as the fact that advocates do, in fact, use ``environmental justice'' when they raise distributional concerns, even on many rules that are not about issues traditionally considered ``environmental'' because of its power to give distributional justice claims legal purchase.}

\hypertarget{the-evolving-distributional-politics-of-mercury-pollution}{%
\subsubsection{The Evolving Distributional Politics of Mercury Pollution}\label{the-evolving-distributional-politics-of-mercury-pollution}}

Fundamental definitions of the public good and minority rights are
implicit in agency rules. The public comment process offers an
opportunity to protest these definitions. Protest is one way that
marginalized groups can communicate opinions on issues to government
officials \citep{Gillion2013}. In the case of the EPA's Mercury Rules, two
such issues were decisive. First, as with many forms of pollution,
mercury-emitting power plants are concentrated in low-income, often
non-White communities. Second, certain populations consume much more
locally-caught freshwater fish, a major vector of Mercury toxicity.
Studies inspired by the political controversy around the Mercury Rules
found high risk among certain communities, including ``Hispanic, Vietnamese, and
Laotian populations in California and Great Lakes tribal populations
(Chippewa and Ojibwe) active on ceded territories around the Great
Lakes'' (EPA 2012). Thus the standards that EPA chooses are fundamentally
dependent on whom the regulation aims to protect: the average citizen,
local residents, or fishing communities. This decision has disparate
effects based on race and class because of disparate effects based on
geography and different cultural practices. Such disparate impacts are
often called EJ issues.

In December 2000, when the EPA first announced its intention to regulate
Mercury from power plants, the notice published in the Federal Register
did not address EJ issues, such as the disparate
effects of mercury on certain populations. Risks were only discussed in
reference to ``the U.S. population'' (EPA 2000). When the first draft rule
was published, it only discussed the effects of the rule on regulated
entities, noting that

\begin{quote}
``Other types of entities not listed could also be
affected'' (EPA 2002).
\end{quote}

Commenting on this draft, Heather McCausland of
the Alaska Community Action on Toxics (ACAT) wrote:

\begin{quote}
``The amount of methyl-mercury and other bioaccumulative chemicals
consumed by Alaskans (especially Alaskan Natives) could potentially be
much higher than is assumed\ldots{} {[}and could increase{]} the Alaskan Native mortality rate for
babies, which according to the CDC is 70\% higher than the United States
average. Indigenous Arctic \& Alaskan Native populations are some of
the most polluted populations in the world.
Global transport \& old military sites contaminate us too.''
\end{quote}

After receiving hundreds of thousands of comments and pressure from
tribal governments and organizations, a revised proposed rule echoed McCausland's
comment noting that

\begin{quote}
``Some subpopulations in the U.S., such as Native
Americans, Southeast Asian Americans, and lower-income subsistence
fishers may rely on fish as a primary source of nutrition and/or for
cultural practices. Therefore, they consume larger amounts of fish than
the general population and may be at a greater risk of the adverse
health effects from Hg due to increased exposure'' (EPA 2004).
\end{quote}

After nearly a million additional public comments, a revised proposed
rule ultimately included five pages of analysis of the disparate impacts
on "vulnerable populations" including ``African Americans,'' ``Hispanic,''
``Native American,'' and ``Other and Multi-racial'' groups (EPA 2011). In the final rule, ``vulnerable populations'' was replaced
with ``minority, low income, and indigenous populations'' (EPA 2012). The EPA
had also conducted an analysis of sub-populations with particularly high
potential risks of exposure due to high rates of fish consumption as well
as additional analysis of the distribution of mortality risk by
race.

Of this second round of comments, over 200 unique comments explicitly raised
EJ issues. The Little River Band of Ottawa Indians
expressed the Tribe's

\begin{quote}
``frustration at trying to impress upon the EPA the
multiple and profound impacts of mercury contamination from a Tribal
perspective. Not to mention the obligations under treaties to
participate with tribes on a `Government to Government' basis. At
present, no such meetings have occurred in any meaningful manner with
EPA Region V, the EPA National American Indian Environmental Office, nor
the State of Michigan's Department of Environmental Quality.''
\end{quote}

They
conclude that ``Although EPA purported to consider environmental justice
as it developed its Clean Air Mercury Rule, it failed utterly. In this
rulemaking, EPA perpetuated, rather than ameliorated, a long history of
cultural discrimination against tribes and their members'' (Sprague
2011).

Did comments like these play a role in EPA's changed analysis of
whom Mercury limits should aim to protect?
Given the many potential sources of influence, it may be difficult to
attribute causal effects of particular comments on a given policy.
However, comments may serve as a good proxy for the general mobilization
of groups and individuals around an administrative process, and it is
not clear why the EPA would not address EJ in the first
draft of a rule and then add it to subsequent drafts in the absence of
activist pressure. Electoral politics does not offer an easy
explanation. The notice proposing the Mercury Rule was issued by the
Clinton administration, the same administration that issued the
Executive Order on Environmental Justice, and the subsequent drafts that
did address EJ issues were published by the Bush
administration, which had a more contentious relationship with
EJ advocates, while Republicans controlled both
houses of Congress. The expansion of the analysis from one draft to the
next seems to be in response to activist pressure.

\hypertarget{measuring-policy-change}{%
\section{Measuring Policy Change}\label{measuring-policy-change}}

Having shown how changes in rule text can be related to public comments, I assess the overall relationship between comments and policy change. I use two indicators of policy change to model the effect of public comments on policy: \emph{whether} a rule addresses EJ and \emph{how} it addresses EJ. Both measures represent a fairly low bar, indicating whether the agency explicitly paid any attention to EJ. This is appropriate given that prior research shows little to no effect of public comments from advocacy groups as well as little attention to EJ in particular.

\hypertarget{measure-1-adding-text-addressing-ej-to-final-rules}{%
\subsection{Measure 1: Adding Text Addressing EJ to Final Rules}\label{measure-1-adding-text-addressing-ej-to-final-rules}}

For the subset of draft rules that did not address EJ, I measure whether agencies added any mention of ``environmental justice'' in the final rule. Such additions usually take the form of an ``E.O. 12898'' section where the agency justifies its policy changes with respect to some concept(s) of environmental justice. The next most common addition occurs in the agency's response to comments, explaining how the rule did not have disparate effects or that they were insignificant.

Sometimes an agency will respond to a comment and add a 12898 section. For example, the EPA responded to several commenters, including Earthjustice, the Central Valley Air Quality Coalition, the Coalition for Clean Air, Central California Environmental Justice Network, and Central California Asthma Collaborative: ``EPA agrees it is important to consider environmental justice in our actions and we briefly addressed environmental justice principles in our proposal.'' EPA had, as the commenters noted, not in fact addressed environmental justice in the proposed rule, which approved California rules regulating particulate matter emissions from construction sites, unpaved roads, and disturbed soils in open and agricultural areas. EPA did add a fairly generic 12898 section to the final rule but did not substantively change its position.

Less frequently, an agency may explicitly dismiss a comment and decline to add a 12898 section. For example, EPA responded to a comment on another rule, ``One commenter stated that EPA failed to comply with Executive Order 12898 on Environmental Justice\ldots We do not believe that these amendments will have any adverse effects on\ldots minority and low-income populations\ldots Owners or operators are still required to develop SSM plans to address emissions\ldots The only difference from current regulations is that the source is not required to follow the plan'' (71 FR 20445). As these examples illustrate, agencies may add text addressing environmental justice that would in no way satisfy critics. This measure merely indicates whether the agency engaged with the claims.

\hypertarget{measure-2-changing-text-addressing-ej-in-final-rules}{%
\subsection{Measure 2: Changing Text Addressing EJ in Final Rules}\label{measure-2-changing-text-addressing-ej-in-final-rules}}

Where draft rules did address EJ, I measure whether a rule changed \emph{how} it discussed ``environmental justice'' between its draft and final publication.\footnote{Occasionally, there is more than one version of a Proposed or Final rule on a rulemaking docket. Here I opt for an inclusive measure of change that counts change from \emph{any} proposed to \emph{any} final rule. This means that if the change occurred between the first and second draft of a proposed rule, it is counted as a change. This best captures the concept of rule change. However, estimates are similar if we only count cases where a change occurred between \emph{every} version of the rule.}
When an agency addresses EJ in the draft rule, it is almost always in a section about how it addressed E.O.12898. In many cases, much of the text of final rules, including 12898 sections where they exist, remain exactly the same between draft and final versions.
To measure change, I parse draft and final rules into sentences and identify sentences containing the phrase ``environmental justice.'' If these sentences are identical, this indicates the agency did not engage with comments raising EJ concerns.\footnote{An alternative approach would be to parse documents by section and assess whether E.O.12898 sections are identical. Parsing by sentences has three advantages: it is computationally faster, it avoids problems with section numbering and other frustrations with section matching, and it captures attention to EJ outside of this section, especially in the section responding to comments. If an agency is paying attention to EJ issues, sentence matching will be most likely to detect it. However, other measures, such as the percent of EJ sentences changed, the percent of words in a 12898 section that changed, or the change in topic proportions \citep{Judge-Lord2017}, could be useful in future work.}

\hypertarget{results}{%
\section{Results}\label{results}}

\hypertarget{are-final-rules-more-likely-to-address-environmental-justice-after-comments-do-so}{%
\subsection{Are final rules more likely to address environmental justice after comments do so?}\label{are-final-rules-more-likely-to-address-environmental-justice-after-comments-do-so}}

This subsection presents results from an analysis of draft
rules, comments, and final rules. Descriptively, figure
\ref{fig:ej-PR-winrate-1}
shows that, in general, where environmental
justice is not addressed in the draft rule, it is also not addressed in the final, but a higher percent of rules do add EJ language when comments raise EJ concerns. There is a large difference in the rate of addressing EJ between rules where commenters and did (33\%) and did not raise EJ concerns (4\%). However, in most cases, agencies did not respond at all to these commenters' concerns.

\begin{figure}

{\centering \includegraphics[width=0.6\linewidth]{/Users/devin/dissertation/Figs/ej-PR-winrate-1} 

}

\caption{Proposed Rules that Did Not Address Environmental Justice}\label{fig:ej-PR-winrate-1}
\end{figure}

Figure \ref{fig:ejejPR-winrate-2} shows that overall rates of adding EJ in rules without EJ comments decreased over time, leveling out at 3\% during the Obama and Trump presidencies. The rates of adding EJ when commenters did raise these concerns is consistently much higher, but it also decreases over time, from 57\% under G.W. Bush to 36\% under Trump.
Finally, looking at selected agencies, Figure \ref{fig:ej-PR-winrate-2} shows that most rules that addressed EJ in the draft were published by the EPA. EPA had a relatively high rate of baseline change, which increased when comments raised EJ concerns. Other agencies had too few rules to make strong inferences, but many responded to comments or changed how they discussed EJ 100\% of the time when comments raised EJ concerns, while inconsistently doing so when comments did not.

\begin{figure}

{\centering \includegraphics[width=1\linewidth]{/Users/devin/dissertation/Figs/ej-PR-winrate-president-1} \includegraphics[width=1\linewidth]{/Users/devin/dissertation/Figs/ej-PR-winrate-agency-1} 

}

\caption{Percent of Rules that Changed in How they Addressed Environmental Justice by President and by Agency}\label{fig:ej-PR-winrate-2}
\end{figure}

\hypertarget{estimating-whether-environmental-justice-is-added-to-final-rule}{%
\subsubsection{\texorpdfstring{Estimating \emph{Whether} ``Environmental Justice'' is Added to Final Rule}{Estimating Whether ``Environmental Justice'' is Added to Final Rule}}\label{estimating-whether-environmental-justice-is-added-to-final-rule}}

For this analysis, I estimate a logit regression where the outcome is whether environmental justice was addressed in the final rule. The predictors are
whether EJ was addressed in the comments, the number of unique comments addressing EJ, the total number of comments, and the interaction between whether EJ was raised and the total number of comments received. I also include fixed effects for presidential administration and, in models 2 and 4, for agency as well. Thus, estimates in Models 1 and 3 include variation \emph{across} agencies, while estimates in models 2 and 4 rely on variation \emph{within} agencies. All estimates rely on variation within the presidential administration.
Model 3 is identical except for the dependent variable. Predicted probabilities shown below are for models with agency fixed effects, 2 and 4.

\begin{table}[H]
\centering
\caption{\label{tab:tables}Logit Regression Models Predicting Change in Final Rule Text}

\begin{tabular}[t]{lcccc}
\toprule
Model  & 1 & 2 & 3 & 4\\
(Dependent Variable)  & (EJ Added) & (EJ Added) & (EJ Changed) & (EJ Changed)\\
\midrule
EJ comment = TRUE & 3.348*** & 2.284*** & 0.672*** & 0.731***\\
 & (0.219) & (0.233) & (0.250) & (0.255)\\
log(comments + 1) & 0.140*** & 0.231*** & -0.093*** & -0.128***\\
 & (0.022) & (0.034) & (0.029) & (0.031)\\
Unique EJ comments & 0.005 & 0.124** & 0.036** & 0.044***\\
 & (0.006) & (0.051) & (0.015) & (0.016)\\
EJ comment x log(comments + 1) & -0.309*** & -0.194*** & 0.051 & 0.061\\
 & (0.049) & (0.066) & (0.049) & (0.051)\\
\midrule
Num.Obs. & 12234 & 12234 & 2034 & 2034\\
President FE & X & X & X & X\\
Agency FE &  & X &  & X\\
AIC & 4263.5 & 3330.9 & 2279.7 & 2232.6\\
BIC & 4322.8 & 3671.8 & 2324.6 & 2395.5\\
Log.Lik. & -2123.728 & -1619.440 & -1131.840 & -1087.306\\
\bottomrule
\multicolumn{5}{l}{\textsuperscript{} * p $<$ 0.1, ** p $<$ 0.05, *** p $<$ 0.01}\\
\multicolumn{5}{l}{\textsuperscript{} Full Table in the Online Appendix}\\
\end{tabular}
\end{table}

\hypertarget{predicted-probability-of-added-text}{%
\subsubsection{Predicted Probability of Added Text}\label{predicted-probability-of-added-text}}

As logit coefficients are not easily interpretable, I present
predicted probabilities for the types of rules of interest.
Figure \ref{fig:ej-m-PR-comments-agencyFE} shows the predicted probability of a final rule addressing environmental justice when the draft rule did not with a varying number of comments (with other variables at their modal values: President Obama, the EPA, and zero additional EJ comments).\footnote{All predicted probability plots below also use President Obama, the EPA, 0 additional EJ comments, and the median number of total comments as the baseline values, unless otherwise specified.}
At low numbers of total comments, environmental justice being raised in any one comment does have a
statistically significant and substantively large effect. For rules with less than ten comments (most rules), one comment mentioning EJ is associated with a 30\% increase in the probability that EJ will be addressed in the final rule. This supports the \emph{Distributive Information Hypothesis}. However, as in figure \ref{fig:ej-PR-winrate-1}, the resulting rate is still below 50\%---even when comments raise EJ concerns, agencies do not address them. As the number of comments increases, the probability that a rule will add text addressing EJ increases. At the same time, there is a small but negative interaction effect between the number of comments and EJ comments--the more comments, the smaller the relationship between the comments raising EJ and EJ being addressed in the rule. In the small-portion of highly salient rules with 10,000 or more, the presence of comments raising EJ concerns no longer has a significant relationship with EJ being added to the text. With or without EJ comments, these rules have about the same probability of change as those with just one EJ comment, just under 50\%. This is evidence against the \emph{Conditional Pressure Hypothesis}---the number of comments matters (i.e., the scale of public attention) matters regardless of whether these comments explicitly raise EJ concerns. However, as shown in Figure \ref{fig:ej-comments}, very few rules with 10,000 or more comments do not have at least one comment mentioning EJ, so there is a great deal of uncertainty about estimates of the impact of EJ comments with high levels of public attention.

The probability of
``environmental justice'' appearing in the final rule also increases with the number of unique comments that mention ``environmental justice'' in models 2, 3, and 4, but it does not have a significantly positive relationship in Model 1. Overall this supports the \emph{Repeated Information Hypothesis}.

\begin{figure}

{\centering \includegraphics[width=0.75\linewidth]{/Users/devin/dissertation/Figs/ej-m-PR-comments-agencyFE-1} 

}

\caption{Proposed Rules Not Addressing Environmental Justice}\label{fig:ej-m-PR-comments-agencyFE}
\end{figure}

Figure \ref{fig:ej-m-PR-agency-top} shows estimated variation in rates of adding EJ to final rules.
Agencies with the largest average rates adding EJ language are the agencies we would expect to be more receptive to EJ claims. While many agencies make what could be called ``environmental policy'' and all policy decisions have distributive consequences, the Environmental Protection Agency have the Department of Transportation (which includes the Federal Railroad Administration (FRA), Department of Transportation, Federal Motor Carrier Safety Administration (FMCSA), and the Federal Highway Administration (FHWA)) the most prominent internal guidance on EJ in rulemaking. However, differences among agencies are fairly uncertain due to the small number of rules where EJ was added at most agencies. Thus, there is more support for the **Policy Receptivity Hypothesis than against it, but differences between agencies with different missions and institutional practices regarding EJ is not clear cut.

\begin{figure}

{\centering \includegraphics[width=0.99\linewidth]{/Users/devin/dissertation/Figs/ej-m-PR-agency-top-1} 

}

\caption{Proposed Rules Not Addressing Environmental Justice}\label{fig:ej-m-PR-agency-top}
\end{figure}

The Forest Service (FS) has the highest predicted baseline rate of adding EJ to their rules. This may be because the forest service is mainly in
the business of managing forests, leasing timber rights, and controlling
wildfires. These types of decisions may have acute distributional
effects that may not be the initial focus of the agency. Forest Service rule-writers may also simply have a practice of addressing E.O.12898 in final rules but not draft rules.

Similarly, the Federal
Railroad Administration, Department of Transportation, Federal Highway
Administration, Federal Motor Carrier Safety Administration all have large baseline rates of adding EJ to final rules. These agencies are
making decisions about infrastructure projects with implications for
neighborhood environments and air quality. Environmental justice may
often come up, but there may be a lot of variation in whether the agency
then decides if they are relevant to transportation policies and
projects that are primarily about neither environmental nor justice
concerns.

Research agencies, including the Nuclear Regulatory Commission (NRC), National
Oceanographic and Atmospheric Administration (NOAA) also have a statistically significant but small baseline rate of adding EJ to final rules, indicating a large difference but rare event. Indeed, as we saw in Figure \ref{fig:ejejPR-winrate-1}, there are only three NRC proposed rules in these data that received EJ comments, and the agency added EJ language to two of them. Thirty-one draft NOAA rules received EJ comments, five of which added EJ language.

\hypertarget{are-rules-more-likely-to-change-how-they-address-environmental-justice-when-comments-mention-it}{%
\subsection{Are rules more likely to change how they address environmental justice when comments mention it?}\label{are-rules-more-likely-to-change-how-they-address-environmental-justice-when-comments-mention-it}}

Turning to rules that do address EJ in the draft, we also see responsiveness to comments raising EJ concerns, now measured as whether any sentences containing ``environmental justice'' changed between draft and final rule.
Figure \ref{fig:ejejPR-winrate-1} shows that a higher percent of rules change when comments raise EJ concerns. Overall rates of change in rules without EJ comments are fairly consistent across presidencies.

\begin{figure}

{\centering \includegraphics[width=0.6\linewidth]{/Users/devin/dissertation/Figs/ejejPR-winrate-1} 

}

\caption{Percent of Rules that Changed in How they Addressed Environmental Justice}\label{fig:ejejPR-winrate-1}
\end{figure}

Finally, looking at selected agencies, Figure \ref{fig:ejejPR-winrate-2} that most rules that addressed EJ in the draft were published by the EPA, which had a high rate of baseline change, which increased when comments raised EJ concerns. Other agencies had too few rules to make strong inferences, but many changed how they discussed EJ 100\% of the time when comments raised it, while inconsistently doing so when comments did not.

\begin{figure}

{\centering \includegraphics[width=1\linewidth]{/Users/devin/dissertation/Figs/ejejPR-winrate-agency-1} \includegraphics[width=1\linewidth]{/Users/devin/dissertation/Figs/ejejPR-winrate-president-1} 

}

\caption{Percent of Rules that Changed in How they Addressed Environmental Justice by Agency}\label{fig:ejejPR-winrate-2}
\end{figure}

\hypertarget{estimating-change-in-how-environmental-justice-is-addressed}{%
\subsubsection{\texorpdfstring{Estimating Change in \emph{How} "Environmental Justice is Addressed}{Estimating Change in How "Environmental Justice is Addressed}}\label{estimating-change-in-how-environmental-justice-is-addressed}}

Models 3 and 4 in Table \ref{tab:tables} are the same as Models 1 and 2, except that the dependent variable is now whether any sentences mentioning EJ changed between the draft and final rule.

\hypertarget{predicted-probability-of-changed-text}{%
\subsubsection{Predicted Probability of Changed Text}\label{predicted-probability-of-changed-text}}

Controlling for average rates of change per agency and the number of comments, Figure \ref{fig:ej-mejPR-president-mean-1} shows there are no significant differences in baseline rates of adding EJ language across the Bush, Obama, and Trump presidencies. All are significantly lower than the rate in the Clinton administration, which could be a result of Clinton's Executive Order or simply an artifact of the limited sample of rules posted to regulations.gov before the mid-2000s.

\begin{figure}

{\centering \includegraphics[width=0.75\linewidth]{/Users/devin/dissertation/Figs/ej-mejPR-president-mean-1} 

}

\caption{Predicted Change in How Environmental Justice is Addressed Between Draft and Final Rules by Number of Comments}\label{fig:ej-mejPR-president-mean-1}
\end{figure}

The number of comments matters, but in the opposite direction posited by the \emph{General Pressure Hypothesis}. The logged total number of comments has a significantly negative relationship with the probability the final rule text changes. The more comments there are on a proposed rule, the less likely it is to change. Rules are more likely to change when they receive \emph{fewer} comments. The total number of comments thus has the opposite relationship to \emph{how} rules that already addressed EJ changed as it did to \emph{whether} rules added any EJ text. While the \emph{General Pressure Hypothesis} held for adding EJ text, the opposite is true for changing a text that already addressed EJ. Instead, this result supports the competing intuition that more salient rules may be harder to change because the agency has anticipated public scrutiny. Their position set forth in the draft is more likely to be the position of the final rule.

As shown in Figure \ref{fig:ej-mejPR-comments}, EJ comments have a small but discernable relationship to the probability of rule change at typical (low) numbers of comments. However, the relationship between the number of comments and the probability of rule change is different when some comments mentioned environmental justice. Specifically, the interaction term is positive; more comments overall means that environmental justice comments have a larger effect. When the total number of comments is larger, the positive interaction between the number of comments and comments mentioning EJ means that the predicted probability of change in how a rule addresses EJ is larger when the agency receives comments mentioning EJ.

\begin{figure}

{\centering \includegraphics[width=0.7\linewidth]{/Users/devin/dissertation/Figs/ej-mejPR-comments-1} 

}

\caption{Predicted Change in How Environmental Justice is Addressed Between Draft and Final Rules by Number of Comments}\label{fig:ej-mejPR-comments}
\end{figure}

\hypertarget{conclusion}{%
\section{Conclusion}\label{conclusion}}

This analysis has illustrated the importance of ideas in policymaking and challenges in assessing their impact. The results suggest that when issue frames
like environmental justice are raised, there is a higher probability that
policymakers consider the effects on marginalized populations. However, baseline rates of addressing environmental justice in rulemaking are so low that even as the probability that agencies will at least mention environmental justice increases when commenters raise these issues, in most rules, even those where commenters raise EJ concerns, there is no explicit attention to EJ. This holds across presidents G.W. Bush, Obama, and Trump. Indeed, there are surprisingly small differences across administrations in both baseline rates of considering EJ and the relationship between comments and change in rule text.
Importantly, there is a great deal of variation across agencies, suggesting that predispositions, policy receptiveness, and responsiveness to comments conditional on an
institutional being predisposed to such an analysis.

Furthermore, it is essential to note that the policy outcomes suggested
by environmental justice analysis depend on how minority populations are
defined. In some cases, those raising environmental justice concerns
present it as an issue of wealth or income inequality, leading policy to
account for disparate impacts on low-income populations. In other cases,
groups raise claims rooted in cultural practices, such as fish
consumption among certain tribes. As occurred in the Mercury Rule, the
analysis in subsequent drafts of the policy used evaluative criteria
specific to these communities.

The ability of a frame like environmental justice to construct certain
populations as deserving of consideration means that policy outcomes
will depend on the specific environmental justice concerns raised. In
this respect, second-order representation may become important. National
advocacy organizations may frequently request that regulators protect
``all people'' or even ``low-income communities of color.'' However, this
more generic advocacy may not lead to the same outcomes as groups that
present specific local environmental justice grievances that are unique
to a community. In between generic progressive advocacy organizations
and community-based organizations are organizations like the Sierra Club and Earthjustice, who, despite their national focus, frequently engaged in
community-specific litigation or place-based and thus raise these local concerns in
national policymaking. Given the importance of federal policy for local environmental outcomes, and advocacy organizations' potential to draw policymakers' attention to environmental justice issues, future research should examine the quality of partnerships between frontline communities and national advocacy organizations.

The examination of which groups raise environmental justice concerns and
second-order participation in these organizations' advocacy decisions validates some of the skepticism about who is able to
participate and make their voice heard. Elite groups dominate policy lobbying, even for
an issue like environmental justice.

In the end, the above analysis offers some clarity on a poorly
understood but important mechanism of U.S. policymaking. It offers
some hope that citizen opinions may be heard through direct democracy
institutions built into bureaucratic policymaking. At the same time, it highlights how disparate impacts are explicitly considered in a tiny percentage of policies.
% --- PAGE: endnotes -----------------------
% --- PAGE: refs -----------------------
\newpage
\singlespacing 
          \bibliography{/Users/devin/dissertation/assets/dissertation.bib} 
   

\end{document}
