% intro do not call story 
When George W. Bush replaced Bill Clinton as president, career bureaucrats at the Federal Trade Commission knew that this meant a change in policy priorities. Many rulemaking projects initiated under the Clinton administration were likely to be withdrawn or put on hold. They also knew that the new administration wanted to be perceived as advancing a new policy agenda, not merely undoing Clinton-era regulations. Entrepreneurs within the agency saw a political window of opportunity to initiate a new regulatory agenda aimed at curbing a growing volume of telemarketing calls. This initiative seemed likely to be popular with voters but, even with a supportive president, would be difficult to advance over the objections of the telemarketing industry whose campaign donations had earned them many powerful allies in Congress. Agency officials report being pessimistic about the FTC's telemarketing effort succeeding over opposition from Congress.

When the draft ``Telemarketing Sales Rule'' (also known as the ``Do Not Call'' rule) was published, however, public support and engagement were overwhelming. The rule received thousands of supportive comments from frustrated members of the public who were encouraged to comment by advocacy groups like the Consumer Federation of America. Agency officials report that the volume of public response not only encouraged the agency and the administration but, more importantly, ``scared off'' Members of Congress who the industry was relying on to kill or reverse the rule. Once it became clear that the public was paying attention and sufficiently mobilized to act on the issue, elected officials became much less willing to take unpopular positions supporting industry donors. Instead, Congress ended up codifying the agency's authority to implement the Do Not Call regulations with legislation the following year.

The story of the Do Not Call rule suggests that public engagement in rulemaking may occasionally be influential because it affects the behavior of elected officials who have the power to provide key support or opposition to a proposed rule.


% oversight affects policy outcomes 
\paragraph{Political oversight.}
Political oversight of bureaucracies has long concerned both practitioners and theorists \citep{Wilson1989, Epstein1999, Huber2002, McCubbins1984, Wilson1989, Potter2016, Lowande2018}. Political scientists often model the relationship between elected officials and bureaucrats as a principal-agent problem. For example, an agency may have a preferred policy but, upon observing the preferences of principals, may change the rule or delay its publication to avoid being reversed \citep{Potter2016}. While it is widely accepted that agency officials must take the positions of their principals into account, the mechanisms by which this occurs and the empirical conditions for political control are debated.

 %add Meier and O’Toole 2006; West 1995; Wood and Waterman 1994
% both branches of government can influence regulatory decision making (Moe 1985, Wood & Waterman 1991, Furlong 1997).
%  transparency, participation, and rational analysis, procedures may actually serve to insulate regulators from ad hoc pressures by individual members of Congress on behalf of their constituents (Croley 2008)
% see review "The Politics of Regulation: From New Institutionalism to New Governance" Carrigan and Coglianese 2011

% classic model 
Because I focus on influence in the period between publication of draft and final rules, I focus on information about principals' preferences revealed to the agency in this period. Oversight during rulemaking is a form of ex-post control \citep{Epstein1994}, in this case after the proposed rule is published. Figure \ref{fig:causal-classic-principals} shows a version of the classic model of principal influence in rulemaking. Upon learning of content of a draft rule, an official with power over the agency may choose to signal their demands to the agency. There is an ongoing debate among scholars over how political oversight operates--i.e. how the observable behaviors of principals inform agency decisions. 

\begin{figure}[h!]
    \centering
    \caption{The Classic Model of Principal-Agent Oversight in Bureaucratic Policymaking}
    \label{fig:causal-classic-principals}
\tiny
\begin{tikzpicture}[%
    node distance=1.2cm,
    auto,
    text width=1.5cm,
dnode/.style={diamond, align=center, aspect=2, fill=green!5,draw=green!60, very thick, minimum size=2cm},
squarednode/.style={rectangle, align=center, aspect=1, draw=red!60, fill=red!5, very thick, minimum size=1cm},
pnode/.style={ellipse, align=center, aspect=1, draw=black!60, fill=black!5, very thick, minimum size=1cm},
title/.style={rectangle, align=center, aspect=1, minimum size=2cm},
]
% Draft 
\node[dnode]      (draft)                     {Draft Policy};


% principal Nodes
\node[pnode]        (principaldemands) [right=of draft] {Principal Demands};

\node[dnode]      (principaldecides) [right=of principaldemands] {?};

\node[squarednode]      (principalinfo) [right=of principaldecides] {Perceived Political Consequences};


% \node[squarednode]      (principalinfo2) [below=of principalinfo] {Perceived Principal Opinion};


% principal Lines
\draw[->] (draft.east) -- (principaldemands.west);
\draw[->] (principaldemands.east) -- (principaldecides.west);
\draw[->] (principaldecides.east) -- (principalinfo.west);

% policy 
\node[dnode]      (policy)       [right=of principalinfo] {Policy Response};
\draw[->] (principalinfo.east) -- (policy.west);


% Titles
% \node[title]      (1) [above=of draft] {Policy};
%\node[title]      (2) [above=of principaldemands] {Preferences};
%\node[title]      (4) [above=of principalinfo] {Information/ Signal};
%\node[title]      (3) [above=of principaldecides] {Observed Behavior};
% \node[title]      (5) [above=of policy] {Policy'};

\end{tikzpicture}
\end{figure}
\normalsize

\citet{McCubbins1987} suggest two oversight mechanisms. Principals may proactively attend to agency activities, like a ``police patrol'' or they may rely on bureaucrats' fear of sanction when attentive interest groups alert principals about agency activities, more like a ``fire alarm.'' Administrative procedures like notice and Comment remaking thus offer opportunities for direct oversight and to be alerted to oversight opportunities.


\subsubsection{Incorporating political information into models of political oversight.}
In addition to interest groups directly alerting elected officials to oversight opportunities as in the ``fire alarm'' model,
the political information signaled by mass engagement may serve as a serve as ``warning sign,''--altering elected officials to political risks
% More precisely, political information may alert elected officials of opportunities to rein in an agency (the concept of ``fire alarm'' oversight discussed by \citep{Mccubbins1984}) 
\emph{or} to, conversely, to encourage the agency to hold course (what might better be described as a ``beacon'' attracting positive attention and credit claiming opportunities. In the case of the FTC's ``Do Not Call'' rule and subsequent legislation, mass engagement functioned more as a ``warning'' and a ``beacon,'' effectively enabling and empowering rather than restraining the agency as classical ``fire alarm'' concept suggests.

Mass engagement in bureaucratic policymaking may affect the behavior of an agency's principals because the shadow of public sanction hangs over elected officials \citep{Arnold1979, Mayhew2000}. \citet{Moore2018} finds that agencies that recieve more comments per rule are subject to more congressional hearings. When the public is more attentive, it is more important for officials to take popular positions and avoid unpopular ones.
Thus, when a coalition goes public, especially if it generates a perceived consensus in expressed public sentiments, elected officials ought to be more likely to intervene on their behalf and less likely to intervene against them.  

\begin{hyp} \label{hyp:principals}
Elected officials' advocacy is moderated by mass engagement. Elected officials are more likely to engage in rulemaking in support of positions supported by the majority of comments and less likely to engage in opposition to the majority of comments.
\end{hyp}

This suggests an addendum to Hall and Miler's (2008) finding that legislators are more likely to engage in rulemaking when they have been lobbied by a like-minded interest group.
When interest groups lobby elected officials to engage in rulemaking, they may be more likely to engage when aligned with the majority of commenters than when opposed to them (Hypothesis \ref{hyp:principals}.
If elected officials learn from political information, they will be even more likely to engage when lobbied by a coalition that includes public interest groups running a mass-comment campaign, and less likely when lobbied by a coalition dominated by private interests opposed by a large mass comment campaign.\footnote{Of course, if Members of Congress receive signals about the distribution of comments from their districts, the distribution of opinions in their district constituency may be more important. Figure \ref{fig:sierra}, shows that the Sierra Club requires zip code information from commenters, so mass-mobilizers may often send such signals to elected officials.}

\begin{figure}
    \centering
    \caption{Incorporating Mass Engagement and Political Information into Models of Political Oversight}
    \label{fig:causal-principals}
\tiny
\begin{tikzpicture}[%
    node distance=1.2cm,
    auto,
    text width=1.5cm,
dnode/.style={diamond, align=center, aspect=2, fill=green!5,draw=green!60, very thick, minimum size=2cm},
squarednode/.style={rectangle, align=center, aspect=1, draw=red!60, fill=red!5, very thick, minimum size=1cm},
pnode/.style={ellipse, align=center, aspect=1, draw=black!60, fill=black!5, very thick, minimum size=1cm},
title/.style={rectangle, align=center, aspect=1, minimum size=2cm}
]
% Draft 
\node[dnode]      (draft)                     {Draft Policy};



% \draw[->] (publicinfo.east) -- (policy.west);
% \draw[->] (principalinfo.east) -- (policy.south);
% \draw[->] (principalinfo2.east) -- (policy.south);

% Titles
% \node[title]      (1) [above=of draft] {Policy};
%\node[title]      (2) [above=of groupdemands] {Preferences};
%\node[title]      (4) [above=of groupinfo] {Information/ Signal};
%\node[title]      (3) [above=of groupdecides] {Observed Behavior};
% \node[title]      (5) [above=of policy] {Policy'};

% principal Nodes

\node[dnode]      (principaldecides) [right=of principaldemands] {Principal Comments};
\node[pnode]        (principaldemands) [right=of draft] {Principal Demands};


% public Nodes
\node[pnode]        (publicdemands) [above=of principaldemands] {Latent Public Demands};
\node[dnode]      (publicdecides) [right=of publicdemands] {Mass\\ Engagement};

% political info
\node[text centered]      (mobilization) [above=of publicdecides] {};

\node[text centered]      (mobilization2) [right=of mobilization] {};

\node[rectangle, minimum width =3cm, minimum height = 6.5cm, draw=red!60, fill=red!5, very thick]      (politicalinfo) [below=of mobilization2] {};

\node[text centered]      (politicalinfotext) [below=of mobilization2] {Political Information};

% info
\node[squarednode]      (publicinfo) [right=of publicdecides] {Perceived Public Opinion};
\node[squarednode]      (principalinfo) [right=of principaldecides] {Perceived Political Consequences};
\node[squarednode]      (principalinfo2) [below=of principalinfo] {Perceived Principal Opinion};


%\draw[->] (mobilization.south) -- (publicdecides.north);


% public Lines
% \draw[->] (draft.east) -- (publicdemands.west);
\draw[->] (publicdemands.east) -- (publicdecides.west);
\draw[->] (publicdecides.east) -- (publicinfo.west);




% principal Lines
\draw[->] (draft.east) -- (principaldemands.west);
\draw[->] (principaldemands.east) -- (principaldecides.west);
\draw[->] (publicinfo.south west) -- (principaldecides.north east);
\draw[->] (principaldecides.east) -- (principalinfo.west);
\draw[->] (principaldecides.south east) -- (principalinfo2.north west);
\draw[->] (publicdemands.south east) -- (principaldecides.north west);

% policy 
\node[dnode]      (policy)       [right=of principalinfo] {Policy Response};
\draw[->] (politicalinfo.east) -- (policy.west);

\end{tikzpicture}
\end{figure}
\normalsize

Figure \ref{fig:causal-principals} builds on the classic model of political oversight in two ways. First, it suggests that the public and private comments by elected officials are a particularly relevant oversight behavior and a mechanism by which bureaucrats learn and update beliefs about their principals demands. Second, it suggests that such oversight behaviors may be affected by mass engagement because of the impressions of public opinion (i.e. the political information) it creates.