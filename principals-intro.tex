% intro do not call story 
When George W. Bush replaced Bill Clinton as president, career bureaucrats at the Federal Trade Commission knew that this meant a change in policy priorities. Many rulemaking projects initiated under the Clinton administration were likely to be withdrawn or put on hold. They also knew that the new administration wanted to be perceived as advancing a new policy agenda, not merely undoing Clinton-era regulations. Entrepreneurs within the agency saw a political window of opportunity to initiate a new regulatory agenda aimed at curbing a growing volume of telemarketing calls. This initiative would seemed likely to be popular with voters but, even with a supportive president, would be difficult to advance over the objections of the telemarketing industry whose campaign donations had earned them many powerful allies in Congress. Agency officials report being pessimistic about the FTC's telemarketing effort succeeding over opposition from Congress.

When the draft "Telemarketing Sales Rule" (also known as the "Do Not Call" rule) was published, however, public support and engagement was overwhelming. The rule received thousands of supportive comments from frustrated members of the public encouraged to comment by advocacy groups like the Consumer Federation of America. Agency officials report that the volume of public response not only encouraged the agency and administration, but, more importantly "scared off" Members of Congress that the industry was relying on to kill or reverse the rule. Once it became apparent that the public was paying attention and sufficiently mobilized to act on the issue, elected officials became much less willing to take unpopular positions supporting industry donors. Instead, Congress ended up codifying the agency's authority to implement the rule with legislation the following year.

The story of the Do-not-call rule suggests that public engagement in rulemaking may occasionally be influential because it affects the behavior of elected officials who have the power to provide key support or opposition to a proposed rule.


% oversight affects policy outcomes 
\paragraph{Political oversight.}
Political oversight of bureaucracies has long concerned both practitioners and theorists \citep{Wilson1989}. Political scientists often model the relationship between elected officials and bureaucrats as a principal-agent problem. For example, an agency may have a preferred policy but, upon observing the preferences of principals, may change the policy text or delay publication to avoid being reversed. While it is widely accepted that agency officials must take the positions of their principals into account, the mechanisms by which this occurs and the empirical conditions for political control are debated.

% classic model 
Because I investigate influence in the period between publication of  draft and final rules, I focus on information about principals preferences revealed to the agency in this period. \ref{causal-classic-principals} shows a version of the classic version of the classic model of principal influence in rulemaking. The question mark reflects the ongoing debate among scholars over how oversight operates--i.e. how the actions of principals informs agency decisions. 

\begin{figure}
    \centering
    \caption{The Classic Model of Principal-Agent Oversight in Bureaucratic Policymaking}
    \label{fig:causal-classic-principals}
\tiny
\begin{tikzpicture}[%
    node distance=1.2cm,
    auto,
    text width=1.5cm,
dnode/.style={diamond, align=center, aspect=2, fill=green!5,draw=green!60, very thick, minimum size=2cm},
squarednode/.style={rectangle, align=center, aspect=1, draw=red!60, fill=red!5, very thick, minimum size=1cm},
pnode/.style={ellipse, align=center, aspect=1, draw=black!60, fill=black!5, very thick, minimum size=1cm},
title/.style={rectangle, align=center, aspect=1, minimum size=2cm},
]
% Draft 
\node[dnode]      (draft)                     {Draft Policy};


% principal Nodes
\node[pnode]        (principaldemands) [right=of draft] {Principal Demands};

\node[dnode]      (principaldecides) [right=of principaldemands] {?};

\node[squarednode]      (principalinfo) [right=of principaldecides] {Perceived Political Consequences};


% \node[squarednode]      (principalinfo2) [below=of principalinfo] {Perceived Principal Opinion};


% principal Lines
\draw[->] (draft.east) -- (principaldemands.west);
\draw[->] (principaldemands.east) -- (principaldecides.west);
\draw[->] (principaldecides.east) -- (principalinfo.west);

% policy 
\node[dnode]      (policy)       [right=of principalinfo] {Policy Response};
\draw[->] (principalinfo.east) -- (policy.west);


% Titles
% \node[title]      (1) [above=of draft] {Policy};
%\node[title]      (2) [above=of principaldemands] {Preferences};
%\node[title]      (4) [above=of principalinfo] {Information/ Signal};
%\node[title]      (3) [above=of principaldecides] {Observed Behavior};
% \node[title]      (5) [above=of policy] {Policy'};

\end{tikzpicture}
\end{figure}
\normalsize

\citet{McCubbins1987} suggest two oversight mechanisms. Principals may proactively attend to agency activities, like a "police patrol," or they may rely on bureaucrats' fear of sanction when attentive interest groups alert principals about agency activities, more like a "fire alarm." 



\paragraph{Incorporating mass engagement and political information into models of political oversight.}

The political information signaled by mass engagement may serve as a ``fire alarm''---altering elected officials to oversight opportunities to rein in an agency. It may also serve as ``warning sign''---altering them to political risks.
% More precisely, political information may alert electedto rein in an agency (the concept of ``fire alarm'' oversight discussed by \citep{Mccubbins1984}) 
\emph{or} to encourage the agency (what might better be described as a ``beacon'' attracting positive attention). In the case of the FTC's "Do Not Call" rule and subsequent legislation, mass engagement functioned more as a "warning" and a "beacon," effectively enabling and empowering rather than restraining the agency as classical "fire alarm" concept suggests.

Mass engagement in bureaucratic policymaking may affect the behavior of an agency's principals because the shadow of public sanction hangs over elected officials \citep{Arnold1979, Mayhew2000}. When the public is more attentive, it is more important for officials to take popular positions and avoid unpopular ones.
Thus, when a coalition goes public, especially if it generates a perceived consensus in expressed public sentiments, elected officials ought to be more likely to intervene on their behalf and less likely to intervene against them.  

This suggests an addendum to Hall and Miler's (2008) finding that legislators are more likely to engage in rulemaking when they have been lobbied by a like-minded interest group.
When interest groups lobby elected officials to engage in rulemaking, they may be more likely to engage when aligned with most commenters than when opposed.
If politicians learn from political information, they will be even more likely to engage when lobbied by a coalition that includes public interest groups running a mass-comment campaign, and less likely when lobbied by a coalition dominated by private interests opposed by a large mass comment campaign. 

\begin{figure}
    \centering
    \caption{Incorporating Mass Engagement and Political Information into Models of Political Oversight}
    \label{fig:causal-principals}
\tiny
\begin{tikzpicture}[%
    node distance=1.2cm,
    auto,
    text width=1.5cm,
dnode/.style={diamond, align=center, aspect=2, fill=green!5,draw=green!60, very thick, minimum size=2cm},
squarednode/.style={rectangle, align=center, aspect=1, draw=red!60, fill=red!5, very thick, minimum size=1cm},
pnode/.style={ellipse, align=center, aspect=1, draw=black!60, fill=black!5, very thick, minimum size=1cm},
title/.style={rectangle, align=center, aspect=1, minimum size=2cm}
]
% Draft 
\node[dnode]      (draft)                     {Draft Policy};



% \draw[->] (publicinfo.east) -- (policy.west);
% \draw[->] (principalinfo.east) -- (policy.south);
% \draw[->] (principalinfo2.east) -- (policy.south);

% Titles
% \node[title]      (1) [above=of draft] {Policy};
%\node[title]      (2) [above=of groupdemands] {Preferences};
%\node[title]      (4) [above=of groupinfo] {Information/ Signal};
%\node[title]      (3) [above=of groupdecides] {Observed Behavior};
% \node[title]      (5) [above=of policy] {Policy'};

% principal Nodes

\node[dnode]      (principaldecides) [right=of principaldemands] {Principal Comments};
\node[pnode]        (principaldemands) [right=of draft] {Principal Demands};


% public Nodes
\node[pnode]        (publicdemands) [above=of principaldemands] {Latent Public Demands};
\node[dnode]      (publicdecides) [right=of publicdemands] {Mass\\ Engagement};

% political info
\node[text centered]      (mobilization) [above=of publicdecides] {};

\node[text centered]      (mobilization2) [right=of mobilization] {};

\node[rectangle, minimum width =3cm, minimum height = 6.5cm, draw=red!60, fill=red!5, very thick]      (politicalinfo) [below=of mobilization2] {};

\node[text centered]      (politicalinfotext) [below=of mobilization2] {Political Information};

% info
\node[squarednode]      (publicinfo) [right=of publicdecides] {Perceived Public Opinion};
\node[squarednode]      (principalinfo) [right=of principaldecides] {Perceived Political Consequences};
\node[squarednode]      (principalinfo2) [below=of principalinfo] {Perceived Principal Opinion};


%\draw[->] (mobilization.south) -- (publicdecides.north);


% public Lines
% \draw[->] (draft.east) -- (publicdemands.west);
\draw[->] (publicdemands.east) -- (publicdecides.west);
\draw[->] (publicdecides.east) -- (publicinfo.west);




% principal Lines
\draw[->] (draft.east) -- (principaldemands.west);
\draw[->] (principaldemands.east) -- (principaldecides.west);
\draw[->] (publicinfo.south west) -- (principaldecides.north east);
\draw[->] (principaldecides.east) -- (principalinfo.west);
\draw[->] (principaldecides.south east) -- (principalinfo2.north west);
\draw[->] (publicdemands.south east) -- (principaldecides.north west);

% policy 
\node[dnode]      (policy)       [right=of principalinfo] {Policy Response};
\draw[->] (politicalinfo.east) -- (policy.west);

\end{tikzpicture}
\end{figure}
\normalsize




