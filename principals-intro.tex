
\paragraph{Step 2: Does mass engagement bureaucratic policymaking affect elected officials' engagement?}
% QUESTION 2 \textbf{Puzzle:} 
The political information signaled by mass engagement may serve as ``fire alarms''---altering elected officials to oversight opportunities---or ``warning sign''---altering them to political risks.\footnote{
More precisely, political information may alert elected officials of oversight opportunities to rein in an agency (the concept of ``fire alarm'' oversight discussed by \citep{Mccubbins1984}) \emph{or} to encourage the agency (what might better be described as a ``beacon'' attracting positive attention).
The shadow of public sanction hangs over elected officials \citep{Arnold1979, Mayhew2000}. When the public is more attentive, it is more important for officials to take popular positions and avoid unpopular ones.
}
Thus, when a coalition goes public, especially if it generates a perceived consensus in expressed public sentiments, elected officials ought to be more likely to intervene on their behalf and less likely to intervene against them.  
% This suggests an addendum to Hall and Miler's (2008) finding that members are more likely to engage in rulemaking when they have been lobbied by a like-minded interest group.
% When interest groups lobby elected officials to engage in rulemaking, they may be more likely to engage when aligned with most commenters than when opposed.
% If politicians learn from political information, they will be even more likely to engage when lobbied by a coalition that includes a public interest group's with a large mass-comment campaign, and less likely when lobbied by a coalition dominated by private interests opposed by a mass comment campaign. 
% MEASUREMENT  2
To assess these hypotheses, I count the number of times Members of Congress engage the agency\footnote{
By engaging the agency, I mean that Members of Congress raise a rule in hearings, committee reports, and personal letters that members send to the agency.
}
across rules and before, during, and after comment periods on rules where lobbying organizations did and did not go public and use text analysis to compare legislators' sentiment and rhetoric to that used by each coalition.
% Similarly, I asses the involvement of presidential appointees and the President's Office of Management and Budget before and after public comment, again comparing rules that were and were not targeted by a campaign (a difference-in-difference). 
% As a validity check, I also look for remarks by elected officials and judges on the level of public engagement.\footnote{
% I expect courts to be more likely to cite the procedural legitimacy of notice comment rulemaking when ruling in favor of public interest group that went public, and less likely to do so when ruling against them, compared to cases where rules received few comments. For example, citing the procedural legitimacy of rulemaking in Vermont Yankee v. NRDC (1978), Justice Rehnquist noted ``More than 40 individuals and organizations representing a wide variety of interests submitted written comments.'' I have collected data, including mentions of public comments, on all Supreme Court cases reviewing agency rules since 1984 and will do the same for a sample of D.C. circuit cases. While I focus on elected officials because they are more likely to respond to mass engagement, courts are also important political principals who explicitly review the legitimacy of rulemaking processes.
1) Comments from Members of Congress on the rule (total, those mentioning mass comments, and those mentioning organizations in the coalition), All  $\sim$ zero-inflated negative binomial. 
2) Share of mentions supporting the coalition,  $\sim$  beta. 
3) Rhetorical similarity between comments from the coalition and Members of Congress. 

DVs 1-2 are one observation per coalition per rule. Model 3 is one observation per comment from a Member of Congress. Explanatory variables are similar to step 1 as well as the DVs from step 1. (how many and what types of comments).% In addition to cross sectional analysis, I use a difference-in-difference design within members on rules where groups do and do not go public.

%I examine the relationship between mass engagement and another key variable in agency decisions, political oversight. % other key features of agencies' decisionmaking environments. 
% Do mass comment campaigns indicate that elected officials will be more involved in a rulemaking? 
% Do they indicate a greater chance of a rule being challenged or overturned in court?
% Dependent variables include political principals' attention, positions, and rhetoric, which I measure several ways across rules and within policy areas before and after mobilization campaigns.
% THEORY 2
% Accountability to Congress, the president, and courts have long been central concerns for bureaucracy scholars \citep{Wilson1989}. 
 % Elected officials, political appointees, and judges may also see it as their job to hold agencies accountable to the public will. On the other hand, elected officials often serve private interests,  such as campaign donors, especially when there is little risk of being held publicly accountable themselves.




