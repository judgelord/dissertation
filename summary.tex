
Thank you so much for reading this draft dissertation chapter. For context, this is the first chapter developing a measure of public engagement in agency rulemaking that will be used in the following empirical chapters. In the dissertation, I argue that mass engagement results from interest groups' strategic choices. When lobbying organizations have an opportunity to shape policy, resources to mobilize, and broader support than their opposition, outside lobbying (``going public'') may produce valuable, politically-relevant information. Depending on how agencies process political information, outside lobbying may %be a plausible strategy for organizations to 
influence policy, both directly and indirectly.
For example, those lobbying in rulemaking often make suspect claims to represent broad segments of the public. Mobilizing a large number of people may directly support such claims.
Indirectly, it may alert elected officials to political risks and opportunities, affecting oversight behavior. % and thus shifting bureaucrats' relationships with their political principals. % It remains to be seen if conditions under which this is plausible ever occur and, if so, if mass engagement does indeed influence policy. 
\section*{Outline of the broader project}

Does mass engagement in bureaucratic policymaking affect policy? This question drives my project. However, two questions must be answered first: (1) Why does it occur? and (2) How does it affect the oversight behaviors of agencies' political principals? These questions drive two initial empirical chapters.
I then use my new measures of the political information that lobbying coalitions create by going public to test whether mass engagement explains variation agency rulemaking and rules.% But first, I must develop a measure of ``going public.'' % and why it occurs.

\paragraph{Part 1. Why do agencies (occasionally) get so much mail?} %: Lobbying coalitions, mass comments, and political information in bureaucratic policymaking
% Scholars of bureaucratic policymaking have focused on the sophisticated lobbying efforts of powerful interest groups. Yet agencies occasionally receive thousands or even millions of comments from ordinary people. Why? Why do individuals comment when they seemingly have no new information to offer and no power to influence decisions? Who inspires them and to what end? How, if at all, should scholars incorporate mass commenting into models of bureaucratic policymaking? I argue that mass commenting produces political information about the coalition that mobilized it. 
% QUESTION 1\textbf{Puzzle:} 
Why do people comment on draft policies when they seem to have no new information to offer and no power to influence decisions? Who inspires them and to what end? 
% THEORY AND METHODS 1
Answering these questions requires a theory explaining variation in mass engagement and method to link comments to the lobbying coalitions that mobilized them.  
To link individual comments to the more sophisticated lobbying efforts they support, I use text reuse and Bayesian classifiers to identify clusters of similar comments, reflecting formal and informal coalitions.
%Using new measures of public engagement in agency rulemaking, I identify the conditions under which it occurs and produces different politically-relevant information. 
% The dependent variable is the number of people engaged.
I argue that activists' resources, opportunities, and public support explain variation in mass engagement %, which I measure in several ways. 
and that it will fit one of three patterns:
(1) Coalitions will ``go public'' when they are disadvantaged in insider politics but have more support than opposing coalitions. More public support yields more engagement, more effort per comment, and contagion beyond those mobilized directly. (2) Coalitions with less support may ``counter-mobilize'' with smaller effects. (3) Finally, coalitions may mobilize for reasons unrelated to the policy at hand, yielding similar mass engagement but with little sophisticated lobbying. 
Measures of mass engagement include 
%(1) total public comments, % $\sim$ zero-inflated negative binomial; 
(1) comments per coalition, % $\sim$ negative binomial; 
(2) effort per comment, % $\sim$ truncated normal; 
(3) share of comments per coalition mobilized indirectly (i.e. the potential for conflict spread).
Next, I test whether variation in engagement explains variation in oversight behavior (step 2) and policy outcomes (step 3).
% (4) type of campaign. % $\sim$ multinomial. 
%Model 1 is one observation per rule. Models 2-4 are one observation per coalition per rule. Explanatory variables include agency alignment with Congress and the president (models 1-4), coalition alignment and unity (models 2-4), whether a coalition is driven more by public or private interests (models 2-3).%, part of the DV in model 4).

%\paragraph{Step 2. Are elected officials more or less likely to engage after mass public engagement?} 
\paragraph{Part 2. Does mass engagement affect political oversight?} The political information signaled by mass engagement may serve as a ``fire alarm,'' altering principals to oversight opportunities or a ``warning signs'' altering them to political risks.
When a coalition goes public, %especially if it generates a perceived consensus in expressed public sentiments, 
principals ought to be more likely to engage on their behalf and less likely to engage against them.
% This suggests an addendum to Hall and Miler's (2008) finding that members are more likely to engage in rulemaking when they have been lobbied by a like-minded interest group.
% When interest groups lobby elected officials to engage in rulemaking, they may be more likely to engage when aligned with most commenters than when opposed.
% If politicians learn from political information, they will be even more likely to engage when lobbied by a coalition that includes a public interest group's with a large mass-comment campaign, and less likely when lobbied by a coalition dominated by private interests opposed by a mass comment campaign. 
% MEASUREMENT  2
To assess these hypotheses, I count the number of times Members of Congress engage the agency before, during, and after comment periods on rules where lobbying organizations did and did not go public. I then use text analysis to compare legislators' sentiments and rhetoric to that used by each coalition.
% Similarly, I asses the involvement of presidential appointees and the President's Office of Management and Budget before and after public comment, again comparing rules that were and were not targeted by a campaign (a difference-in-difference). 
% As a validity check, I also look for remarks by elected officials and judges on the level of public engagement.
Dependent variables include 
(1) the number of comments from Members of Congress on the rule %(total, those mentioning mass comments, and those mentioning organizations in the coalition), %All  $\sim$ zero-inflated negative binomial. 
(2) the share of supportive congressional comments, %  $\sim$  beta. 
(3) the similarity of words in comments from the coalition and Members of Congress. 

\paragraph{Part 3. Does mass engagement affect rulemaking and rules?} 
I theorize that the effects of political information on policy depend on the extent to which the strategic environment allows change and how political information is processed, both directly within agencies and indirectly through other actors (e.g. Members of Congress) whose appraisals matter to bureaucrats.
The main dependent variable is change in the rule text.
%Different inputs may yield different results: 
I systemically identify changes between draft and final rules, parse these differences to identify meaningful policy changes, and compare them to demands raised in comments to measure which coalition got their way. However, assessing policy change is difficult. Thus, I also use other measures of agency responses to lobbying efforts. 