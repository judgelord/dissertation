The broader project aims to better understand the role of ordinary people in bureaucratic policymaking. 
I develop theories of why mass engagement occurs and how it may affect policy. To assess these theories, I tackle three related empirical questions: (1) Why does it occur?; (2) How does it affect the oversight behaviors of agencies' political principals?; and %These questions drive two initial empirical chapters.
(3)  Does mass engagement in bureaucratic policymaking affect policy?
% I then use my new measures of the political information that lobbying coalitions create by going public to test whether mass engagement explains variation in agency rulemaking and rules.% But first, I must develop a measure of ``going public.'' % and why it occurs.

\paragraph{Part 1. Why do agencies (occasionally) get so much mail?} %: Lobbying coalitions, mass comments, and political information in bureaucratic policymaking
% Scholars of bureaucratic policymaking have focused on the sophisticated lobbying efforts of powerful interest groups. Yet agencies occasionally receive thousands or even millions of comments from ordinary people. Why? Why do individuals comment when they seemingly have no new information to offer and no power to influence decisions? Who inspires them and to what end? How, if at all, should scholars incorporate mass commenting into models of bureaucratic policymaking? I argue that mass commenting produces political information about the coalition that mobilized it. 
% QUESTION 1\textbf{Puzzle:} 
Why do some rules receive many comments from ordinary people and some do not?
% Why do people comment on draft policies when they seem to have no new information to offer and no power to influence decisions? Who inspires them and to what end? 
% THEORY AND METHODS 1
% Answering this question requires a theory explaining variation in mass engagement. 
The literature suggests two possible explanations for variation in mass engagement; groups may leverage public support as a lobbying resources (``grass roots'' mobilization) or groups with more resources may leverage those resources into impression of public support (sometimes called ``astro-turf'').
%Using new measures of public engagement in agency rulemaking, I identify the conditions under which it occurs and produces different politically-relevant information. 
% The dependent variable is the number of people engaged.
I theorize that public support explains variation in mass engagement. That is, unlike other forms of lobbying, it is not primarily driven by groups with more resources. Specifically, I anticipate three patterns:
(1) When a coalition is disadvantaged in insider politics but as more public support than opposing coalitions, they are more likely to ``go public'' to bolster their lobbying effort (assuming they have the resources to do so). More public support yields more engagement, more effort per participant, and contagion beyond those mobilized directly. (2) Coalitions with less support may ``counter-mobilize'' with proportionally smaller results. That is, groups with more resources but less public support only mobilize when their opponents do so. (3) Finally, coalitions may mobilize for reasons unrelated to the policy at hand, yielding significant mass engagement but without a corresponding insider lobbying effort. 
Because the vast majority of comments are inspired by interest-group campaigns, finding their cause requires a method to link comments to the lobbying coalitions that mobilized them.  
To link individual comments to the more sophisticated lobbying efforts they support, I use text reuse and clustering methods to capture formal and informal coalitions.
Measures of mass engagement include 
%(1) total public comments, % $\sim$ zero-inflated negative binomial; 
(1) comments per coalition, % $\sim$ negative binomial; 
(2) effort per comment, % $\sim$ truncated normal; 
(3) share of comments per coalition mobilized indirectly (i.e. the potential for conflict spread).
With measures, I test whether variation in engagement explains variation in oversight behavior (part 2) and policy outcomes (part 3).
% (4) type of campaign. % $\sim$ multinomial. 
%Model 1 is one observation per rule. Models 2-4 are one observation per coalition per rule. Explanatory variables include agency alignment with Congress and the president (models 1-4), coalition alignment and unity (models 2-4), whether a coalition is driven more by public or private interests (models 2-3).%, part of the DV in model 4).

%\paragraph{Step 2. Are elected officials more or less likely to engage after mass public engagement?} 
\paragraph{Part 2. Does mass engagement affect political oversight?} The political information signaled by mass engagement may serve as a ``fire alarm,'' altering principals to oversight opportunities or ``warning signs'' altering them to political risks.
When a coalition mobilizes successfully, %especially if it generates a perceived consensus in expressed public sentiments, 
elected officials ought to be more likely to engage on their behalf and less likely to engage against them.
% This suggests an addendum to Hall and Miler's (2008) finding that members are more likely to engage in rulemaking when they have been lobbied by a like-minded interest group.
% When interest groups lobby elected officials to engage in rulemaking, they may be more likely to engage when aligned with most commenters than when opposed.
% If politicians learn from political information, they will be even more likely to engage when lobbied by a coalition that includes a public interest group's with a large mass-comment campaign, and less likely when lobbied by a coalition dominated by private interests opposed by a mass comment campaign. 
% MEASUREMENT  2
To assess these hypotheses, I count the number of times Members of Congress engage the agency before, during, and after comment periods on rules where lobbying organizations did and did not go public. I then use text analysis to compare legislators' sentiments and rhetoric to that used by each coalition.
% Similarly, I asses the involvement of presidential appointees and the President's Office of Management and Budget before and after public comment, again comparing rules that were and were not targeted by a campaign (a difference-in-difference). 
% As a validity check, I also look for remarks by elected officials and judges on the level of public engagement.
Dependent variables include 
(1) the number of comments from Members of Congress on the rule %(total, those mentioning mass comments, and those mentioning organizations in the coalition), %All  $\sim$ zero-inflated negative binomial. 
(2) the share of supportive congressional comments, %  $\sim$  beta. 
(3) the similarity of words in comments from the coalition and Members of Congress. 

\paragraph{Part 3. Does mass engagement affect rulemaking and rules?} 
I theorize that the effects of political information on policy depend on the extent to which the strategic environment allows change and how political information is processed, both directly within agencies and indirectly through other actors (e.g. Members of Congress) whose appraisals matter to bureaucrats.
The main dependent variable is change in the rule text.
%Different inputs may yield different results: 
I systematically identify changes between draft and final rules, parse these differences to identify meaningful policy changes, and compare them to demands raised in comments to measure which coalition got their desired outcomes.% However, assessing policy change is difficult. Thus, I also use other measures of agency responses to lobbying efforts. 