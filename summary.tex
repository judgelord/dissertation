
\paragraph{Summary:} This dissertation is about ordinary people's input on policies made by bureaucrats. 
% People may believe that their voices matter, but it is unclear if they do. % or ought to. 
I analyze millions of public comments on thousands of agency rules to develop the first systematic measures of mass engagement in bureaucratic policymaking. 
I theorize that mass engagement may, in limited circumstances, influence bureaucrats by shifting their incentives or evoking powerful norms. Using my new measures to assess these mechanisms, 
I show how various parts of the U.S. government respond to public input.  %aims to understand the effects of public attention on executive-branch policymaking.

\paragraph{Motivation:} 



Leading models of bureaucratic policymaking focus on how agencies either learn about policy problems, negotiate or avoid accountability to various principals, or balance interest-group demands.
The contentious politics that inspire ordinary people to engage have no place in these models and have largely been ignored by political scientists, leaving a weak empirical base for normative and prescriptive work. 
%Like most forms of political participation, 
Mass public comments on draft agency rules provide no new technical information. 
They lack the authority of elected officials' opinions. 
And the number on each side has no legal import for an agency's response.
Policymakers may very well pay no attention to them. 
Instead, scholars focus on the sophisticated lobbying efforts of powerful interest groups, whose role in shaping policy has been theoretically developed and empirically tested.
Yet agencies occasionally receive thousands or even millions of comments from ordinary people. %Why? 
How, if at all, should scholars incorporate mass engagement into models of bureaucratic policymaking? 

I argue that mass engagement produces political information about the coalition that mobilized it and thus, depending on how agencies process political information, ``going public'' may occasionally be an effective strategy for organizations to influence policy, both directly and indirectly.
For example, those lobbying in rulemaking often make suspect claims to represent broad segments of the public. Mobilizing a large number of people may support such claims.
Indirectly, it may alert elected officials to political risks and opportunities, thus reshaping an agency's strategic environment.

Does mass engagement in bureaucratic policymaking affect policy? This question drives my project. However, two questions must be answered first: (1) Why does it occur? and (2) How does it affect agencies' political principals? These questions drive two initial empirical chapters.
Using new measures of the political information that lobbying coalitions create by going public from step 1, I assess the effect of mass mobilization on elected officials' attention (step 2) and on agency responses and policy outcomes (step 3).% But first, I must develop a measure of ``going public.'' % and why it occurs.

\paragraph{Step 1: Why do agencies (occasionally) get so much mail?} 
%: Lobbying coalitions, mass comments, and political information in bureaucratic policymaking

% Scholars of bureaucratic policymaking have focused on the sophisticated lobbying efforts of powerful interest groups. Yet agencies occasionally receive thousands or even millions of comments from ordinary people. Why? Why do individuals comment when they seemingly have no new information to offer and no power to influence decisions? Who inspires them and to what end? How, if at all, should scholars incorporate mass commenting into models of bureaucratic policymaking? I argue that mass commenting produces political information about the coalition that mobilized it. 


% QUESTION 1\textbf{Puzzle:} 
Why do people comment on draft policies when they seem to have no new information to offer and no power to influence decisions? Who inspires them and to what end? 
% THEORY AND METHODS 1
Answering these questions requires a method to link comments to coalitions and a theory explaining variation in mass engagement.  
To link individual comments to the more sophisticated lobbying efforts they support, I use text reuse and topic models to identify clusters of similar comments, reflecting formal and informal coalitions.
%Using new measures of public engagement in agency rulemaking, I identify the conditions under which it occurs and produces different politically-relevant information. 
% The dependent variable is the number of people engaged.
I argue that activists' opportunities and strategies explain variation in engagement. %, which I measure in several ways. 
I then use this variation in engagement as an explanation for variation in policymaker behavior (step 2) and policy outcomes (step 3). 
Dependent variables include: 
1) Total comments, % $\sim$ zero-inflated negative binomial; 
2) comments per coalition, % $\sim$ negative binomial; 
3) effort per comment, % $\sim$ truncated normal; 
4) type of campaign. % $\sim$ multinomial. 
Model 1 is one observation per rule. Models 2-4 are one observation per coalition per rule, with DVs built using text reuse and topic models. Explanatory variables include agency alignment with Congress and the president (models 1-4), coalition unity and alignment (models 2-4), and coding coalitions as driven more by public or private interests (models 2-3).%, part of the DV in model 4).


\paragraph{Step 2: Does mass engagement bureaucratic policymaking affect elected officials' engagement?}
% QUESTION 2 \textbf{Puzzle:} 
The political information signaled by mass engagement may serve as ``fire alarms''---altering elected officials to oversight opportunities---or ``warning sign''---altering them to political risks.
Thus, when a coalition goes public, especially if it generates a perceived consensus in expressed public sentiments, elected officials ought to be more likely to intervene on their behalf and less likely to intervene against them.  
% This suggests an addendum to Hall and Miler's (2008) finding that members are more likely to engage in rulemaking when they have been lobbied by a like-minded interest group.
% When interest groups lobby elected officials to engage in rulemaking, they may be more likely to engage when aligned with most commenters than when opposed.
% If politicians learn from political information, they will be even more likely to engage when lobbied by a coalition that includes a public interest group's with a large mass-comment campaign, and less likely when lobbied by a coalition dominated by private interests opposed by a mass comment campaign. 
% MEASUREMENT  2
To assess these hypotheses, I count the number of times Members of Congress engage the agency across rules and before, during, and after comment periods on rules where lobbying organizations did and did not go public and use text analysis to compare legislators' sentiment and rhetoric to that used by each coalition.
% Similarly, I asses the involvement of presidential appointees and the President's Office of Management and Budget before and after public comment, again comparing rules that were and were not targeted by a campaign (a difference-in-difference). 
% As a validity check, I also look for remarks by elected officials and judges on the level of public engagement.
Dependent variables include 1) Comments from Members of Congress on the rule (total, those mentioning mass comments, and those mentioning organizations in the coalition), %All  $\sim$ zero-inflated negative binomial. 
2) Share of mentions supporting the coalition, %  $\sim$  beta. 
3) Rhetorical similarity between comments from the coalition and Members of Congress. 
Models 1 and 2 are one observation per coalition per rule. Model 3 is one observation per comment from a Member of Congress. Explanatory variables of interest are the DVs from step 1 (how many and what types of comments--i.e. variation in political information).% In addition to cross sectional analysis, I use a difference-in-difference design within members on rules where groups do and do not go public.

%I examine the relationship between mass engagement and another key variable in agency decisions, political oversight. % other key features of agencies' decisionmaking environments. 
% Do mass comment campaigns indicate that elected officials will be more involved in a rulemaking? 
% Do they indicate a greater chance of a rule being challenged or overturned in court?
% Dependent variables include political principals' attention, positions, and rhetoric, which I measure several ways across rules and within policy areas before and after mobilization campaigns.
% THEORY 2
% Accountability to Congress, the president, and courts have long been central concerns for bureaucracy scholars \citep{Wilson1989}. 
 % Elected officials, political appointees, and judges may also see it as their job to hold agencies accountable to the public will. On the other hand, elected officials often serve private interests,  such as campaign donors, especially when there is little risk of being held publicly accountable themselves.






% QUESTION 3 
\paragraph{Step 3: Does mass engagement in bureaucratic policymaking affect policy?} 

% THEORY 3
I theorize that the effects of political information on policy depend on the extent to which the strategic environment allows change, and how political information is processed, both directly within agencies and indirectly through other actors (e.g. Members of Congress) whose appraisals matter to bureaucrats.

The main dependent variable is changes in rule text.
%Different inputs may yield different results: 
I aim to use automated methods to systemically identify changes between draft and final rules, parse these textual differences to identify meaningful policy changes, and compare them to demands raised in comments to measure which coalition got their way. However, assessing policy change is difficult. Thus, I also use other measures of agency responses to lobbying efforts: 
Agencies may speed up or delay finalizing rules. They write lengthy justifications of their decisions in response to some demands but not others. They may or may not extend the comment period.


\paragraph{Causal mechanisms:} How might mass engagement matter?
% A lobbying effort can generate new information, re-frame information, or reshape the political context of a decision. Agency staff may update their beliefs in response to new information or framing. Activists can also reshape agency policymakers’ strategic environment by drawing in or scaring off other actors, especially elected officials. 

\textbf{Strategic calculations:} 
New information may affect agency strategy directly or indirectly. New scientific or legal information spurs revision of calculations about cost and benefits or the likelihood of being reversed in court. New political information spurs bureaucrats to update their beliefs about levels of support among certain populations or their elected representatives and thus the likely political consequences of a decision.
Reshaping strategic incentives may shift how rulewriters weigh commenter demands.

% NORMATIVE FRAMING / INFO PROCESSING 
\textbf{Information processing and normative evaluations:} 
In addition to strategic calculations, mass engagement may shift how information is processed and evaluated, both institutionally and cognitively.
Institutionally, higher comment volume may engage a larger and more politically-oriented set of staff and consultants. Cognitively, expanding the scope of conflict highlights the political aspects of a decision, perhaps mobilizing cognition focused more on norms of public service or partisan ideology than on strategic or technical rationality. In both cases, campaigns re-frame decisions as political and provide information that is especially relevant if processed through such a frame.
The effects of political information on bureaucrats' normative evaluations may be
direct---the weight that norms of direct democracy give to limited public input---or 
indirect---the weight that norms of accountability give to elected officials' input.

%\textbf{Indirect influence through elected officials:} 
%Campaigns  do more than reveal latent political information; they mobilize both members of the public and elected officials to take positions on issues they may have never previously considered, thus creating new relevant political information for bureaucrats. 
%Movements help to shape the political space in which they operate’ (Gamson and Meyer 1996, p. 289).
%The result of thinking differently about a decision may be a shift in how the agency evaluates or weights commenter demands.

\textbf{Assessing causal mechanisms:}
While it may be impossible to causally identify or attribute effects to normative or strategic mechanisms, 
a focus on political information suggests places to look for influence in rulemaking. For example, if Members of Congress are not more likely to voice support for a coalition that goes public, this would be evidence against that indirect mechanism.

To supplement the methods outlined above, the last two chapters explore historical and experimental case studies. My historical case is the environmental justice movement, relying on all rules where ``environmental justice'' is raised in the comments and quantitative and qualitative assessment of agency responses. I find that responsiveness varies with with agency missions. My experimental cases will be rules selected by organizations that have agreed to randomly assign specific targets of mass comment campaigns.

\paragraph{Conclusion:} This research will add to our understanding of how  bureaucratic policymaking fits with the practice of democracy.
If input solicited from ordinary people has little effect on policy outcomes, directly or indirectly, it may be best understood as providing a veneer of democratic legitimacy on an essentially technocratic and/or elite-driven process.
If public input does shape agency decisions, a new research program will be needed to investigate who exactly these campaigns mobilize and represent.