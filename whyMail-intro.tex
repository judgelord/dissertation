%: Lobbying coalitions, mass comments, and political information in bureaucratic policymaking

Scholars of bureaucratic policymaking have focused on the sophisticated lobbying efforts of powerful interest groups. 

[INSERT - INFORMATION IS THE CURRENCY OF INTEREST GROUP LOBBYING]

\begin{figure}
    \centering
    \caption{The Classic Model of Interest Group Lobbying in Bureaucratic Policymaking}
    \label{fig:causal-classic-lobbying}
\tiny
\begin{tikzpicture}[%
    node distance=1.2cm,
    auto,
    text width=1.5cm,
dnode/.style={diamond, align=center, aspect=2, fill=green!5,draw=green!60, very thick, minimum size=2cm},
squarednode/.style={rectangle, align=center, aspect=1, draw=red!60, fill=red!5, very thick, minimum size=1cm},
pnode/.style={ellipse, align=center, aspect=1, draw=black!60, fill=black!5, very thick, minimum size=1cm},
title/.style={rectangle, align=center, aspect=1, minimum size=2cm},
]
% Draft 
\node[dnode]      (draft)                     {Draft Policy};



% Group Nodes
\node[pnode]      (groupdemands) [right=of draft] {Group Demands};
\node[dnode]        (groupdecides) [right=of groupdemands] {Lobbying};
\node[squarednode]      (groupinfo) [right=of groupdecides] {Technical Information};

% policy 
\node[dnode]      (policy)       [right=of groupinfo] {Policy Response};
\draw[->] (groupinfo.east) -- (policy.west);
% \draw[->] (publicinfo.east) -- (policy.west);
% \draw[->] (principalinfo.east) -- (policy.south);
% \draw[->] (principalinfo2.east) -- (policy.south);

% Group Lines
\draw[->] (draft.east) -- (groupdemands.west);
\draw[->] (groupdemands.east) -- (groupdecides.west);
\draw[->] (groupdecides.east) -- (groupinfo.west);

% Titles
% \node[title]      (1) [above=of draft] {Policy};
%\node[title]      (2) [above=of groupdemands] {Preferences};
%\node[title]      (4) [above=of groupinfo] {Information/ Signal};
%\node[title]      (3) [above=of groupdecides] {Observed Behavior};


\end{tikzpicture}
\end{figure}
\normalsize

Yet agencies occasionally receive thousands or even millions of comments from ordinary people. 
%Why? Why do individuals comment when they seemingly have no new information to offer and no power to influence decisions? Who inspires them and to what end? How, if at all, should scholars incorporate mass commenting into models of bureaucratic policymaking? I argue that mass commenting produces political information about the coalition that mobilized it. 
% QUESTION 1
% subsubsection{Puzzle} 
Why do people comment on draft policies when they seem to have no new information to offer and no power to influence decisions? Who inspires them and to what end? 
% THEORY AND METHODS 1
Answering these questions requires a method to link comments to coalitions and a theory explaining variation in mass engagement.  

This section offers a definition of mass engagement, an argument that it is best understood as an outside lobbying tactic, and theory predicting different patterns of engagement depending which of three reasons organizations may have for launching a mobilization campaign. In the next section, I develop methods to measure these patterns.



% DEFINITION
\subsubsection{Defining mass engagement}
Political scientists often define civic engagement as writing to government officials, signing petitions, attending hearings, attending protests, or donate to a political campaign. While donating is more common in electoral politics, activists frequently attempt to influence agency policymaking through letter-writing, petitions, hearings, and protests. 
% I suspect that mass commenting is driven by the same privileged populations known to engage in other civic activities. 
% Does it work? If so, by what mechanisms?

Following the conventional terms ``mass comment campaign'' and ``public engagement,'' I call the general phenomenon ``mass engagement'' resulting from ``mass mobilization'' in order to distinguish the magnitude of civic engagement.
By mass engagement, I mean that thousands of people beyond professional policy influencers engage. Specifically, I define mass engagement as more than 1000 public comments or 100 identical comments, plausibly reflecting a mobilization effort.  Contrary to the common assumption that this emerges organically, it is almost always mobilized by an organization that also engages in sophisticated lobbying. %\footnote{
As \citet{SantAmbrogio2018} conclude ``The `mass comments' occasionally submitted in great volume in highly salient
rulemakings are one of the more vexing challenges facing agencies in recent years. These comments are typically the result of orchestrated campaigns by advocacy groups to persuade members or other like-minded individuals to express support or opposition for an agency's proposed rule.'' 

We lack systematic analysis of public comments. Political scientists have thus far focused on sophisticated lobbying efforts. However, as \citet{Cuellar2005} finds in his study of several rules, ``contrary to conventional wisdom, comments from the lay public make up the vast majority of total comments about some regulations. This shows at least some potential demand among the mass public for a seat at the table in the regulatory process.'' Having collected over 70 million comments on over 300,000 proposed rules, I am able to offer a much more systematic analysis. As figure \ref{fig:mass-comments} shows, not only do comments from ordinary people make up the vast majority of comments across all rule, most comments are, in fact mobilized by mass commenting campaigns. 

\begin{figure}
    \centering
    \includegraphics{}
    \caption{Unique vs Form-letter Comments Posted to Regulations.gov 2006-2018}
    \label{fig:mass-comments}
\end{figure}


\subsubsection{Understanding mass mobilization as a tactic}
% MY THEORY 
% Representation 
When lobbying during rulemaking, groups often make suspect claims to represent broad segments of the public \citep{Seifter2016UCLA}. Mobilizing a large number of people may support such claims.
% public private interests
Appeals to government are almost always couched in the language of public interest, even when true motivations obviously private \citep{Schattschneider1975}. Theorists may debate whether effectively signing a petition of support without having a role in crafting the appeal is meaningful voice and whether petitions effectively channel public interests, but, at a minimum, engaging a large number of supporters helps distinguish narrower interests from broader ones. It suggests the organization is not ``memberless'' \citep{Skocpol2003} in the sense that they are able to demonstrate some public support.

% tactic
Mass mobilization is a strategy. When successful, mass engagement is the result. An organization's ability to expand the scope of conflict by mobilizing members of the public is a political resource. 
In contrast to scholars who focus on the deliberative potential of public comment processes, I focus on public engagement as a tactic aimed at gaining power, either by leveraging powerful ideas or engaging actors with the institutional power to shape decisions.
Scholars who do understand mobilization as a tactic \citep{Furlong1997, Kerwin2011} have thus far focused organizations mobilizing their membership. %In contrast, 
I expand this to include a campaign's broader audience and its potential to grow, more akin to the concept of an attentive public \citep{Key1961} or issue public \citep{Converse1964}. If organizations claim to represent people beyond their official members, 
reforms requiring groups to disclose information about their funding and membership \citep{Seifter2016UCLA} only go part way to assess groups' claims to represent these broader segments of the public. Indeed, if advocacy group decisions are largely made by D.C. professionals, these advocates themselves may be unsure how broadly their claims resonate until potentially-attentive publics are actually engaged.

% three insights 
Here I build on three insights. First, \citet{Kerwin2011} and \citet{Furlong1997} identify mobilization as a tactic. In their survey, organizations report that forming coalitions and mobilizing large numbers of people are among the most effective lobbying tactics. Second, \citet{Nelson2012} identify political information a potentially influential result of lobbying by different business coalitions. While they focus on mobilizing experts, \citet{Nelson2012} describe a dynamic that can be extended to mass commenting: 
``strategic recruitment, we theorize, mobilizes new actors to participate in the policymaking process, bringing with them novel technical and political information. In other words, when an expanded strategy is employed, leaders activate individuals and organizations to participate in the policymaking process who, without the coordinating efforts of the leaders, would otherwise not lobby. This activation is important because it implies that coalition lobbying can generate new information and new actors---beyond simply the `usual suspects'---relevant to policy decision makers. Thus, we theorize consensus, coalition size, and composition matter to policy change.'' 
I argue that, with respect to political information, this logic extends to non-experts. 
Third, \citet{Furlong1998}, \citet{Yackee2006JPART}, and others distinguish direct and indirect forms interest group influence in rulemaking. I argue that mass mobilization is a tactic aimed at producing political information that may have direct and indirect influence. 

% PUBLIC OPINION and INFORMATION 
\citet{Rauch2016} suggests that agencies reform the public comment process to include opinion polls. I build from a similar intuition that mass comment campaigns currently function like a poll or, more accurately, a petition, capturing the intensity of preferences among a segment of the public---i.e. how many people are willing to take the time to engage. Self-selection may not be ideal for representation, but opt-in participation---whether voting, attending a hearing, or writing a comment---still provides political information. 
Mobilizing citizens and generating new political information are key functions of interest groups in a democracy \citep{Mansbridge1992, Mahoney2007}. The information generated by mass mobilization campaigns is explicitly political and more complex than an opinion poll. Activists aim to convince people which issues are important and how to think about them---mapping new issues and debates to familiar ones, thereby shifting the political landscape. 

Importantly, rule-specific campaigns inform agencies about the distribution and intensity of opinions that are often too nuanced to estimate a priori. Many rules may lack analogous public opinion polling questions, making mass commenting a unique source of political information. Indeed, most members of the public and their elected representatives may only learn about the issue as a result of an mobilization campaign. I thus consider public demands to be a latent factor in my model of policymaking. Public demands shape the decisions of groups who lobby in rulemaking. If they beleive the attentive pubic is on their side, groups may attempt to reveal this political information to policymakers by launching a mass mobilization campaign. The public response to the campaign depends on extent that the attentive public is passionate about the issue.

\begin{figure}[h!]
    \centering
    \caption{Incorporating Mass Engagement and Political Information into Models of Lobbying}
    \label{fig:causal-whymail}
\tiny
\begin{tikzpicture}[%
    node distance=1.2cm,
    auto,
    text width=1.5cm,
dnode/.style={diamond, align=center, aspect=2, fill=green!5,draw=green!60, very thick, minimum size=2cm},
squarednode/.style={rectangle, align=center, aspect=1, draw=red!60, fill=red!5, very thick, minimum size=1cm},
pnode/.style={ellipse, align=center, aspect=1, draw=black!60, fill=black!5, very thick, minimum size=1cm},
title/.style={rectangle, align=center, aspect=1, minimum size=2cm},
]
% Draft 
\node[dnode]      (draft)                     {Draft Policy};



% Group Nodes
\node[pnode]      (groupdemands) [right=of draft] {Group Demands};
\node[dnode]        (groupdecides) [right=of groupdemands] {Lobbying Strategy};
\node[squarednode]      (groupinfo) [right=of groupdecides] {Technical Information};

% policy 
\node[dnode]      (policy)       [right=of groupinfo] {Policy Response};
\draw[->] (groupinfo.east) -- (policy.west);
% \draw[->] (publicinfo.east) -- (policy.west);
% \draw[->] (principalinfo.east) -- (policy.south);
% \draw[->] (principalinfo2.east) -- (policy.south);

% Group Lines
\draw[->] (draft.east) -- (groupdemands.west);
\draw[->] (groupdemands.east) -- (groupdecides.west);
\draw[->] (groupdecides.east) -- (groupinfo.west);

% Titles
% \node[title]      (1) [above=of draft] {Policy};
% \node[title]      (2) [above=of groupdemands] {Preferences};
% \node[title]      (4) [above=of groupinfo] {Information/ Signal};
% \node[title]      (3) [above=of groupdecides] {Observed Behavior};
% \node[title]      (5) [above=of policy] {Policy'};

% political info
\node[rectangle, minimum width =2cm, minimum height = 3cm, draw=red!60, fill=red!5, very thick]      (politicalinfo) [below=of groupinfo] {};
\node[text centered]      (politicalinfotext) [below=of groupinfo] {Political Information};
\node[text centered]      (mobilizing) [below=of groupdecides] {Mass\\ Mobilization};
\draw[->] (politicalinfo.north east) -- (policy.south west);

% public Nodes
\node[squarednode]      (publicinfo) [below=of politicalinfotext] {Perceived ``Public'' Opinion};
\node[dnode]      (publicdecides) [left=of publicinfo] {Mass\\ Engagement};
\node[pnode]        (publicdemands) [left=of publicdecides] {Latent Public Demands};

% public Lines
% \draw[->] (draft.east) -- (publicdemands.west);
\draw[->] (publicdemands.east) -- (publicdecides.west);
\draw[->] (publicdemands.north east) -- (groupdecides.south west);
\draw[-] (groupdecides.south) -- (mobilizing.north);
\draw[->] (mobilizing.south) -- (publicdecides.north);
\draw[->] (publicdecides.east) -- (publicinfo.west);



\end{tikzpicture}
\end{figure}
\normalsize






% TYPES OF CAMPAIGNS AND COALITIONS
\paragraph{Types of campaigns:} The mix of types of supporters depends, in part, on the aims of a campaign. Campaigns may have one of three distinct aims: (1) to win concessions by going public, (2) to disrupt a perceived consensus, or (3) to go down fighting. 

Coalitions ``go public'' when they believe that expanding the scope of conflict gives them an advantage.\footnote{
Going public (or an ``outside strategy'') is used by Presidents \citep{Kernell2007}, Members of Congress \citep{Malecha2012}, interest groups \citep{Walker1991, Dur2013}, Lawyers, and Judges (Davis 2011). 
% Sophisticated organizations also use phone banks, targeting strategies, and direct-mail techniques to drum-up and channel public support (see Cooper 1985:2036).
This strategy is likely to be used by those disadvantaged (those \citet{Schattschneider1975} calls the `losers') with less public attention.
Rulemaking with little public attention is the norm. Nearly all scholarship on rulemaking in political science thus focuses on interest-group and inter-branch bargaining, ignoring public opinion and social movements. 
}
As these are the coalitions that believe they have more intense public support, many people may be inspired indirectly and to engage with more effort. In these cases, mass engagement will likely skew heavily toward this side. This is important because a perceived consensus may be especially influential political information.\footnote{
For example, consensus among interest groups \citep{Golden1998, Yackee2006JPART}, especially business unity \citep{Yackee2006JOP, Haeder2015}, predicts policy change, though it is not clear if this is a result of strategic calculation, a perceived obligation due to the normative power of consensus (e.g. following a majoritarian \citep{Mendelson2011}), or simply that the information is easier to process.
}

Second, because the perception of consensus is powerful, when a coalition goes public, an opposing coalition may countermobilize. As this is likely a coalition with less intense public support and its aim is merely to break a perceived consensus, I expect such campaigns to engage fewer people, less effort per person, and yield a smaller portion of indirect engagement. 

Finally, campaigns may target supporters rather than policymakers. Sometimes organizations ``go down fighting'' to fulfill supporters' expectations.\footnote{
I use ``going down fighting'' as shorthand for campaigns aimed at only at fulfilling supporter (e.g. donor, membership) expectations and related logics that are internal to the organization (e.g. fundraising, member retention or recruitment, or satisfying a board of directors).} While such campaigns may engage many people, they are unlikely to affect policy or to inspire countermobilization. I expect such campaigns to occur on rules that have high partisan salience (e.g. rules following major legislation passed on a narrow vote), propose large shifts on policy issues dear to well-funded public interest groups, and occur after presidential transitions when executive-branch agendas shift more quickly than public opinion.


% \subsection{Mobilizing for Recruitment}
% A third possibility is that mobilization around bureaucratic decisions is unrelated to the possibility of affecting policy and primarily a way to recruit and engage members or raise the profile of the movement. If this is the case, behaviors like protesting and mass commenting on rules are largely epiphenomena to unrelated kinds of politics. Organizers may know that mobilization has minimal effects, but lead members to engage as means to other ends. Many of the mobilized themselves may doubt their efficacy but still take advantage of the opportunity to protest. 

While the coalitions may form around various material and ideological conflicts, those most likely to be advantaged by going public or going down fighting are public interest groups---organizations primarily serving an idea of the public good rather than the material interests of their members.\footnote{
One exception may be the few types of membership organizations that are both broad and focused on material outcomes such as labor unions.} Thus, I theorize that mass mobilization is most likely to occur in conflicts of public versus private interests or public versus public interests (i.e. between coalitions led by groups with distinct ideas of the public good), but only ones with sufficient resources to run a campaign.\footnote{
If true, one implication is that mass mobilization will systematically run counter to concentrated business interests where they conflict with the values of organized, privileged groups.
}
To assess these propositions, I classify coalitions as primarily driven by public or private interests and roughly estimate each coalition's resources. 



% We do not really know who engages in mass commenting. Some assume that people who engage in mass commenting belong to membership organizations. Others imply that they are people who happen to have an opinion. RAUCH discusses both "members" and  _______
% % The people who engage in mass commenting are often assumed to be 

% Engaging a broader audience and thus changing the scope of conflict is a basic political strategy. Presidents, supreme court justices, and others "go public" when doing so alters their opponents' calculations. 

% Which campaigns engage a broader audience and which do not? 


\paragraph{Types of engagement:} I classify supporters into three types using the texts of their comment to infer how they were mobilized.  Comments that are exact copies of a form letter are akin to petition signatures from supporters who were engaged by a campaign to comment with minimal effort. Commenters that repeat text but also take time to add their own text indicate more intense preferences. Finally, commenters who express solidarity in similar but distinct phrases indicate they were engaged indirectly
, perhaps by a news story or a social media post about the campaign, 
as campaign messages spread beyond those originally targeted. 

The size of each of each group thus offers political information to policymakers, including coalition resources, intensity of sentiment, and potential for conflict to spread.
The first two types signal two kinds of intensity or resolve. First, they show the mobilizers' willingness to commit resources to the issue. Second, costly actions show the intensity of opinions among the mobilized segment of the public \citep{Dunleavy1991}. The number of people engaged by a campaign is not strictly proportional to an organizations investment. The less people care, the more it costs to mobilize them. If agency staff do not trust organizations' representational claims, engaging actual people may be one of the few credible signals of a broad base of support. The third type indicates potential contagion. Indications that messages spread beyond those originally targeted be especially effective \citep{Kollman1998}. Information about organizational resolve, intensity of preference, and contagiousness are thus produced, but will only influence decisions if mass comments are processed in a way that captures this information and relays it to decisionmakers. These organizational processes may vary significantly across agencies.

